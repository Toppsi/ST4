%Figure references:
\newcommand{\figref}[1]{figur \ref{#1}}

%Figure references after full stop/period:
\newcommand{\Figref}[1]{Figur \ref{#1}}

%Table references:
\newcommand{\tabref}[1]{tabel \ref{#1}}

%Table references after full stop/period:
\newcommand{\Tabref}[1]{Tabel \ref{#1}}

%Appendix references:
\newcommand{\appref}[1]{appendix \ref{#1}}

%Appendix references after full stop/period:
\newcommand{\Appref}[1]{Appendix \ref{#1}}

%Section references:
\newcommand{\secref}[1]{afsnit \ref{#1}}

%Section references:
\newcommand{\Secref}[1]{Afsnit \ref{#1}}

%chapter references: 
\newcommand{\chapref}[1]{kapitel \ref{#1}}

%chapter references: 
\newcommand{\Chapref}[1]{Kapitel \ref{#1}}


%Units:
%inserting '\omit' before '{\put' prior ot final compile will fix allignment (and generate errors)
\newcommand{\unit}[1]{{\put(300,0){$\hfill\left[\: #1 \:\right]$}}}

%Text:
\newcommand{\tx}[1]{\text{#1}}

%Equation references:
%1 equation:
\renewcommand{\eqref}[1]{\textbf{ligning (\ref{#1})}}
%2 equations:
\newcommand{\eqrefTwo}[2]{\textbf{ligning (\ref{#1})} and \textbf{(\ref{#2})}}
%3 equations:
\newcommand{\eqrefThree}[3]{\textbf{ligning (\ref{#1})}, \textbf{(\ref{#2})} and \textbf{(\ref{#3})}}
%4 equations:
\newcommand{\eqrefFour}[4]{\textbf{ligning (\ref{#1})}, \textbf{(\ref{#2})}, \textbf{(\ref{#3})} and \textbf{(\ref{#4})}}
%5 equations:
\newcommand{\eqrefFive}[5]{\textbf{ligning (\ref{#1})}, \textbf{(\ref{#2})}, \textbf{(\ref{#3})}, \textbf{(\ref{#4})} and \textbf{(\ref{#5})}}
%5 equations:
\newcommand{\eqrefSix}[6]{\textbf{ligning (\ref{#1})}, \textbf{(\ref{#2})}, \textbf{(\ref{#3})}, \textbf{(\ref{#4})}, \textbf{(\ref{#5})} and \textbf{(\ref{#6})}}
%5 equations:
\newcommand{\eqrefSeven}[7]{\textbf{ligning (\ref{#1})}, \textbf{(\ref{#2})}, \textbf{(\ref{#3})}, \textbf{(\ref{#4})}, \textbf{(\ref{#5})}, \textbf{(\ref{#6})} and \textbf{(\ref{#7})}}

%Equation references after full stop/period:
%1 equation:
\newcommand{\Eqref}[1]{\textbf{Ligning (\ref{#1})}}
%2 equations:
\newcommand{\EqrefTwo}[2]{\textbf{Ligning (\ref{#1})} and \textbf{(\ref{#2})}}
%3 equations:
\newcommand{\EqrefThree}[3]{\textbf{Ligning (\ref{#1})}, \textbf{(\ref{#2})} and \textbf{(\ref{#3})}}
%4 equations:
\newcommand{\EqrefFour}[4]{\textbf{Ligning (\ref{#1})}, \textbf{(\ref{#2})}, \textbf{(\ref{#3})} and \textbf{(\ref{#4})}}
%5 equations:
\newcommand{\EqrefFive}[5]{\textbf{Ligning (\ref{#1})}, \textbf{(\ref{#2})}, \textbf{(\ref{#3})}, \textbf{(\ref{#4})} and \textbf{(\ref{#5})}}
%5 equations:
\newcommand{\EqrefSix}[6]{\textbf{Ligning (\ref{#1})}, \textbf{(\ref{#2})}, \textbf{(\ref{#3})}, \textbf{(\ref{#4})}, \textbf{(\ref{#5})} and \textbf{(\ref{#6})}}
%5 equations:
\newcommand{\EqrefSeven}[7]{\textbf{Ligning (\ref{#1})}, \textbf{(\ref{#2})}, \textbf{(\ref{#3})}, \textbf{(\ref{#4})}, \textbf{(\ref{#5})}, \textbf{(\ref{#6})} and \textbf{(\ref{#7})}}