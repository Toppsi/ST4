\section{Bevægelsesanalyse}
%Indhold:
%- Hvad er bevægelse
%	- definition og opståen 
%- Hvordan måles bevægelse
%- Karakteristika for forskellige bevægelser - hvordan adskiller de sig fra hinanden
%

\subsection{Løb}
%
%kilde: Adelaar
%Faser:
%- standfase (40\%)
%	- Højre hæl i jorden
%	- midt stand
%	- tær af - sætter af med tæerne --> leder op til næste fase 
%- svævefase (15\%)
%	- Begge ben er ikke-supportet 
%- svingfase (30\%)
%	- venstre hæl i jorden
%	- midt stand
%	- tær af - sætter af med tæerne --> leder op til næste fase 
%- svævefase (15\%)
%	- Begge ben er ikke-supportet 
%--> gentag
%NOTE: billede på side 498
%
%Den bevægelse foden udfører er altså start med hæl i, derefter ned med det midterste af foden, og slut med at hæve hælen og sætte af med tæerne.
%
%Når hastigheden øges, øges stresset på leddene: eks. for en person på 68 kg - stress på en fod: gang = 35 kg/m og løb = 110 ton.
%_________
%Kilde: Lee and Farley
%
%Massemidtpunkter:
%- for gang er det højest midt i standfasen
%- for løb er det lige modsat, det er mindst midt i standfasen
%	- hver gang fødderne rammer jorden under løb, bevæges massemidtpunktet nedad??
%______________
%Kilde: Novacheck
%
%Jo hurtigere man løber, jo kortere tid er fødderne på jorden.
%%______________________________________________________________________

Løb beskrives, ligesom gang, gennem forskellige faser. 
Løbecyklussen består af fire faser som vist på \figref{fig:loebecyklus}: standfasen, den første svævefase, svingfasen og den anden svævefase. \fxnote{\citep{Adelaar}}

***Billede*** \label{fig:loebecyklus} \fxnote{\citep{Adelaar}}

Den første fase, standfase, udgør 40\% af løbecyklussen og starter idet den højre hæl rammer jorden, derefter fortsættes foden til midt stand, og afslutningsvis afsættes der med tæerne, hvilket leder op til den næste fase, den første svævefase. Svævefasen, som går igen to gange i løbecyklussen, udgør hver 15\%, og er karakteriseret ved at begge ben er løftet fra jorden, hvorved de ikke er supportet. Mellem de to svævefaser, er svingfasen, som udgør 30\%. Denne fase initieres idet hælen hæves og knæet føres frem, hvorefter hælen igen sænkes, og svævefasen gentages, før en ny cyklus kan påbegyndes. \fxnote{\citep{Adelaar,Novacheck}}

Længden af disse faser varierer dog alt efter hvor hurtigt man løber, da svævefasen øges. 

Løb er karakteriseret ved, at kun én fod rør jorden ad gangen. Dette resulterer i at der er et større stress på leddene ved løb i forhold til gang. Eksempelvis vil en person på 68 kg have et stress på sin fod på 35 kg/m ved gang, mens det ved løb vil være et stress på 110 ton/m. 

 














