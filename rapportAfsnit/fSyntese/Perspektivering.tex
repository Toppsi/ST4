\section{Perspektivering}\label{sec:perspektivering}
\textit{I dette afsnit bliver systemets funktionelle aspekter sat i perspektiv. Systemets funktionalitet bliver vurderet i en grad, hvor mulige optimeringsmetoder bliver undersøgt og præsenteret. Desuden beskriver perspektiveringen eventuelle videreudviklinger af prototypen. Afslutningsvis vurderes prototypen i et samfundsmæssigt perspektiv.}

Det udviklede system er en prototype af en udgave, som opfylder alle succeskriterier for den optimale aktivitetsmåler til børn som beskrevet i \secref{succeskrav}. Prototypen har derfor en række hardware- og softwaremæssige elementer, som kan optimeres men henblik på en færdigudviklet aktivitetsmåler.

Prototypen kan eksempelvis opbygges mere kompakt i forhold til hardware, således aktivitetsmåleren fremkommer mere diskret. Det antages, at børn i den valgte aldersgruppe ikke ønsker, at systemet er iøjnefaldende for andre. Derfor vil en fordelagtig løsning være, at aktivitetsmåleren opbygges som et diskret bånd, der kan placeres på anklen uden det forekommer tydeligt for andre. For at dette kan lade sig gøre, skal systemet benytte en mindre MCU, der har påmonteret accelerometer, gyroskop og pulssensor i én lille enhed. Yderligere vil en kompakt udgave betyde, at pulssensoren skal være en del af båndet omkring anklen og ikke være placeret på øreflippen. Derved kræves der, at det samlede produkt har god kontakt til huden som resultat af dets implementerede pulssensor. Pulssensoreren er optisk og kræver dermed kontakt med huden ud for en blodåre, for at bestemme pulsen og heraf give en indikation omkring brugerens intensitetsniveau. Systemet anses mest optimalt placeret omkring anklen. Idet en ud fire forsøgspersoner i pilotforsøget ikke vurderede placering af aktivitetsmåleren omkring anklen som værende den mest komfortable, vurderes det at en sekunder placeringsmulighed vil være fordelagtig. Førend andre placeringer kan muliggøres, skal systemets algoritmer og tærskelværdier justeres.  

Algoritmen tilhørende detektering af cykling er implementeret i MATLAB grundet softwaremæssige komplikationer, som umuliggjorde implementering af algoritmen på MCUen. Et fremtidigt system bør derfor have en implementering, af algoritmen til cykling placeret i MCUens software. MCUen er, uanset opbygning, nødvendig til dataopsamling, databehandling og datakommunikation. Det er derfor fordelagtigt, hvis GAP peripheral besidder mere hukommelse, således data kan gemmes på enheden. Dette vil muliggøre en oplagring af data, således aktivitetsmåleren er uafhængig af dataoverførsel i realtid til en ekstern enhed, hvilket er tilfældet for prototypen. 

BLE er en trådløs kommunikationsform mellem elektroniske enheder, som dog har sine begrænsninger i forhold til eksempelvis rækkevidde. En optimal kommunikation med BLE kræver, at enhederne er forholdsvis tæt på hinanden. Dette er problematisk for prototypen, idet den centrale enhed består af en MCU tilkoblet en computer. En bruger kan derfor kun bevæge sig inden for en begrænset afstand herfra, grundet dataoverførsel i realtid mellem enhederne. Det vil derfor være fordelagtigt at udvikle en app til smartphones, således smartphonen bliver tilsvarende det udviklede GUI for prototypen. Ved at benytte en app, vil afstanden mellem enhederne være mindre antaget at smartphonen er placeret i brugerens bukselomme. Ydermere er aktivitetsmåleren mobil, idet enhederne vil have en acceptabel afstand mellem hinanden i forhold til den trådløse kommunikation. Endnu en løsning kunne være at implementere et mobilt netværksmodul i GAP peripheral, hvoraf data til enhver tid uploades til en online database. Herved vil afstand imellem enhederne ikke kunne forsage pakketab og er derfor ikke et problem. Uanset løsningen vil det ydermere være fordelagtigt at installere en alarm i form af enten en lysende LED eller lyde, som aktiveres i tilfælde af, at enhedernes kommunikation og dataoverførsel ikke er optimal og data derfor går tabt. 

Systemet har med en antagelse om et lineært spændingsforbrug ikke mulighed for at være funktionel over en hel dag. Dette har flere årsager, blandt andet systemtes IC-opsætning, algoritme og realtids dataoverførsel. Systemets IC er opsat til at opsamle og sende data fra gyroskopet hele tiden. Gyroskopet er det delelement af ICen, som har det største strømforbrug. En forbedring heraf ville være, at indstille ICen til kun at sende data, når nyt data er klar, i stedet for at sende konstant. ICen har et interrupt ben, som kan benyttes hertil. Mellem dataoverførsler skal gyroskopet indstilles til sleep mode, hvilket vil medføre et lavere strømforbrug. Systemets algoritmer bør undersøges i forhold til antallet af clock cycles, og hvorvidt det er muligt at optimere eksekveringstiden af koden. En reduktion af varigheden for eksekveringen af algoritmen vil medføre, at det samlede system kan gå i sleep mode hurtigere og over længere tid. Dette vil ligeledes medføre et lavere strømforbrug for aktivitetsmåleren. Som nævnt tidligere vil en optimering af prototypen være en forøgelse af MCUens hukommelse, hvilket kunne afhjælpe problematikken vedrørende realtids dataoverførsel. Ved opgradering af flash hukommelsen vil systemets resultater kunne lagres og sendes over BLE med større tidsintervaller. Resultatet heraf ville medføre, at systemets BLE modul kunne være i sleep mode i længere tid end tilfældet for prototypen. En forbedring af ovenstående vil antageligt medføre en besparelse i systemets totale strømforbrug, og batteriets levetid vil blive forøget. Derved kan en mere strømbesparende aktivitetsmåler opnås, hvilket kan gøre denne funktionel over en hel dag.

Det samlede system benytter sig af en GUI til visualisering af det fysiske aktivitetsniveau. Denne GUI er et element, som har en række optimeringsmuligheder, idet den kan udvikles mere brugervenligt og motiverende. Brugervenligheden kan eksempelvis optimeres ved tydelige knapper til start og stop af dataopsamling. Brugerinterfacet bør derfor indebære et motiverende element såsom spil, når målgruppen tages i betragtning. Det udviklede spil vil gøre det muligt at benytte antallet af optjente point fra fysisk aktivitet. Hvis et spil implementeres, antages dette at motivere børnene til et øget fysisk aktivitetsniveau. Dette understøttes af et amerikansk studie, hvilket konkluderer at aktive spil medfører en motivationsfaktor, men at aktive konsolspil ikke kan erstatte udendørs aktivitet \citep{Oestergaard2012}. Heraf kan det antages, at en aktivitetsmåler, som benytter udendørs aktivitet som et element i dets motiverende element, vil være fordelagtigt.

Yderligere vil det være fordelagtigt at installere en brugervenlig kalibreringsenhed, hvor eksempelvis tærskelværdierne mellem gang og løb kan indstilles til den individuelle bruger. Dette vil gøre systemet mere præcist, og pointværdierne for hver aktivitet vil udregnes mere korrekt. Hertil vil en mulig optimering være at udvide antallet af aktiviteter, således at flere af børns hverdagsaktiviteter kan detekteres og medregnes i deres totale aktiviteter.\fxnote{Heriblandt trappegang, hop, rulleskøjtening, ect.} Desuden vil det være muligt at lave et login til forældre, hvorigennem de kan følge deres børns aktivitetsvaner og progression. Derved bliver det muligt for forældrene at motivere børnene igennem konkurrencer og fælles aktiviteter. Denne logintype kan ligeledes benyttes i sammenspil med relevant sundhedspersonale, hvilket kan optimere grundlaget for øge barnets fysiske aktivitetsniveau.\fxnote{Hvis barnet skal tabe sig og eventuel skal i kontakt med sundhedspersonale.}

Rapporten sætter fokus på fysisk inaktivitet hos børn i aldersgruppen 9-12 år, og hvilke konsekvenser dette kan have for barnet. Det er med udgangspunkt i disse fysiologiske og psykologiske konsekvenser, at der er vurderet et behov for en ny og forbedret aktivitetsmåler. Væsentligt er især valget af målgruppe, hvor det er beskrevet, at inkorporationen af nye vaner kan være hensigtsmæssigt hos målgruppen. Ydermere bør børn motiveres ved spil og leg, hvoraf børnene i målgruppen bør blive påvirket herigennem. Børnene bør altså udsættes for en tilgang til sundhed igennem gode oplevelser, hvoraf en afhjælpning af inaktivitet og overvægt er mulig. Det er dog vigtigt, at de sundhedsmæssige vaner, som en aktivitetsmåler ligger op til, ikke bliver overdrevet hos børn. Det er hensigten at børn gennem spil og leg skal få en mere aktiv hverdag, som har en række fysiologiske fordele i form af indlæring, hukommelse og velvære. Det skal gøres til en god vane uden at blive en altafgørende faktor for hverdagen.