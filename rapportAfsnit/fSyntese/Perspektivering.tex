\section{Perspektivering}
\textit{I dette afsnit bliver systemets funktionalitet sat i perspektiv. Systemets funktionalitet bliver vurderet i en grad, hvor mulige optimeringsmetoder bliver undersøgt. Desuden beskrives der i perspektiveringen eventuelle videreudviklinger af det samlede system.}

Det nuværende system opfattes som en prototype, da opbygningen heraf ikke anses som værende optimal. Systemet kan eksempelvis opbygges mere kompakt i forhold til hardware, således det er mere diskret. Det antages, at børn i den valgte aldersgruppe ikke ønsker, at systemet er tydeligt for andre, da hensigten ved brugen heraf er at mindske inaktivitet. Derfor vil den umiddelbare mest optimale løsning være, hvis systemet opbygges som et diskret bånd, der kan sidde på anklen uden det forekommer tydeligt for andre. For at dette kan lade sig gøre, kan systemet eventuelt benytte en mindre MCU, der har påmonteret accelerometer, gyroskop og pulssensor i én lille enhed. Pulssensoren skal altså være en del af båndet omkring anklen og ikke sidde på øret. Derved kræves der, at det samlede produkt har god kontakt til huden som resultat af dets implementerede pulssensor. Pulssensoreren er optisk og kræver dermed kontakt med huden udfor en arterie for at bestemme pulsen og heraf give en indikation omkring brugerens intensitetsniveau. Systemet anses mest optimalt placeret omkring anklen i et bånd, men en sekundær placering kunne være mulig. Førend en sekundær placering kan inddrages, vil en justering af systemets algoritmer og tærskelværdier være nødvendig. Algoritmen tilhørende detektering af cykling bør ydermere implementeres på MCUen ligesom algoritmerne tilhørende gang og løb. Hvis dette bliver implementeret, er det samlede system ikke begrænset i nogen grad med henblik på en eventuel GAP central i form af anden platform. MCUen vil uanset opbygning være nødvendig, da der skal fortages dataopsamling, databehandling og datakommunikation. Det kunne derudover være fordelagtigt, hvis den perifere MCU besidder mere RAM, således mere data kan gemmes herpå og ikke er afhængig af konstant overførsel af data. \\
BLE er en kommunikationsmulighed imellem den perifere og centrale enhed, dog har denne også sine begrænsninger i forhold til eksempelvis afstand. For optimal kommunikation kræver BLE, at enhederne er forholdsvis tæt på hinanden. Dette kan være problematisk især for det nuværende system, da den centrale enhed består af en MCU tilkoblet en computer. En bruger kan derfor kun bevæge sig inden for en begrænset afstand herfra, da den perifere enhed sender data i realtid. En løsning heraf kunne enten være, at den perifere enhed som sagt indeholder flere RAM, således data kan gemmes i en periode førend det videresendes. En anden løsning kunne være, hvis den centrale enhed bliver en app på en telefon, således det samlede system bliver mere mobilt. Endnu en løsning kunne være at implementere et 3G netværk i GAP peripheral, hvoraf data til enhver tid uploades til en online database. Uanset løsningen vil det være fordelagtigt at installere en alarm i form af enten en lysende LED eller lyd, som aktiveres i tilfælde af, at enhedernes kommunikation og dataoverførsel ikke er optimal og pakker bliver tabt. 

Det samlede systems funktionalitet samt design giver et stort strømforbrug, der medfører, at systemt ikke kan opererer over en hel dag. Dette har flere årsager, bland andet systemtes IC opsætning, algoritme og konstant dataoverførsel. Systemets IC er opsat til at opsamle data fra gyroskopet konstant. Gyroskopet er det delelement af ICen, som har det største strømforbrug. En forbedring heraf vil være at indstille ICen til kun at sende data, når nyt data er klar, istedet for at sende konstant. ICen har et interrupt ben, som kan benyttes hertil. I mellemtiden skal gyroskopet indstilles til søvntilstand, hvilket vil medføre at dets strømforbrug bliver reduceret kraftigt. Systemets algoritmer bør undersøges i forhold til clock cycles, og hvorvidt det er muligt at optimere koden. En reduktion i køretiden vil medføre, at det samlede system kan gå i søvntilstand hurtigere og over længere tid, hvoraf strømforbruget ligeledes reduceres. Som nævnt tidligere ville en optimering af det samlede system bestå af en opgradering af dets RAM, hvilket kunne afhjælpe problematikken vedrørende konstant dataoverførsel. Ved opgradering af RAM vil systemets resultater kunne lagres, og sendes med længere mellemrum. Resultatet heraf ville medføre at systemets BLE modul kunne være i søvntilstand i længere tid end hidtil. En forbedring af ovenstående vil antageligt medføre en besparelse i systemets totale strømforbrug, og batteriets levetid vil blive forøget. Derved kan et mere strømbesparende system muligvis opnås.

Det samlede system benytter sig af et brugerinterface til illustration af dagens aktivitet. Dette brugerinterface er et element i denne prototype, som kræver en optimering. Det kan gøres mere brugervenligt og motiverende end den nuværende GUI. Herigennem kunne det være muligt at installere en brugervenlig kalibreringsenhed, hvor eksempelvis tærskelværdierne imellem gang og løb kan finindstilles til den individuelle bruger i stedet for at have en generel værdi. Dette vil gøre systemet mere præcist, og pointværdierne for hver aktivitet vil uddeles mere korrekt. Ydermere bør brugerinterfacet indebære et motiverende spilelement taget målgruppen i betragtning. Hvis et spil implementeres antages dette at motivere børnene til at være mere aktive end foruden. Desuden vil det være muligt at lave et forældrelogin, hvorigennem forældre kan følge deres børns aktivitetsvaner og progression. Derved bliver det muligt for forældrene at motivere børnene igennem konkurrencer og fælles aktiviteter. Denne logintype kunne ligeledes gøres tilgængeligt for relevant sundhedspersonale\fxnote{eks. læge, sundhedsplejeske mm.}, hvilket kunne optimere deres grundlag for afhjælpning af eventuel inaktivitet og eller overvægt.

% - Pulsdata samplet med informationerne omkring den pågældende aktivitet kunne samles i systemet i en speciel fil, som ens læge kan få adgang til. Pulsen siger meget om kroppens fysiologiske tilstand. (Det er ret langt ude, så har ikke skrevet om det. Det vil virke lidt mærkeligt tror jeg)