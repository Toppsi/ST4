\section{Perspektivering}
\textit{I dette afsnit beskrives en eventuel videreudvikling af systemet. Dette er med henblik på hardware, software samt brugerinterface.}

Det nuværende system opfattes som en prototype, da opbygningen af det ikke anses som værende helt optimal. Systemet kan for eksempel opbygges mere kompakt i forhold til hardware, således det er mere diskret. Det antages, at børn i den valgte aldersgruppe ikke ønsker, at systemet er tydeligt for andre, da hensigten ved brugen heraf er at mindske inaktivitet. Derfor ville det være optimalt, hvis systemet kan opbygges som et diskret bånd, der kan sidde på anklen uden det forekommer tydeligt for andre. For at dette kan lade sig gøre, kan systemet eventuelt benytte en mindre MCU, som har påmonteret accelerometer, gyroskop og pulssensor i én lille enhed, pulssensoren skal altså være en del af båndet omkring anklen og ikke sidde på øret. MCUen vil uanset opbygning være nødvendig, da der skal fortages dataopsamling, databehandling og BLE kommunikation. Det kunne derudover være fordelagtigt, hvis den perifere MCU besidder mere RAM, således mere data kan gemmes herpå og ikke er afhængig af konstant streaming af data. Derved kan et mere strømbesparende system muligvis opnås, da live visualisering ikke er af høj prioritet. 

BLE er en god kommunikationsmulighed imellem den perifere og centrale enhed men har også sine begrænsninger i forhold til eksempelvis afstand. For optimal kommunikation kræver BLE, at enhederne er forholdsvis tæt på hinanden, hvilket kan være problematisk især for det nuværende system, da den centrale enhed består af en MCU tilkoblet en computer. En bruger kan derfor kun bevæge sig inden for en begrænset afstand herfra, da den perifere enhed sender data live. En løsning heraf kunne enten være, at den perifere som sagt indeholder flere RAM eller at den centrale enhed bliver en App til en telefon i stedet, hvilket gør det samlede system mere mobilt. Uanset løsningen vil det være fordelagtigt at installere en alarm i form af enten en lysende LED eller lyd, som aktiveres i tilfælde af, at enhedernes kommunikation via BLE ikke er optimal og pakker bliver tabt. \\
Hvis en App bliver den centrale enhed, er der også mulighed for videreudvikling af brugerinterfacet. Denne kan gøres mere brugervenligt og motiverende end den nuværende GUI. Herigennem kunne det være muligt at installere en brugervenlig kalibreringsenhed, hvor eksempelvis tærskelværdierne imellem gang og løb kan finindstilles til den individuelle bruger i stedet for at have en generel værdi. Dette vil gøre systemet mere præcist og pointværdierne for hver aktivitet vil uddeles mere korrekt. Desuden vil det være muligt at lave et forældrelogin, hvorigennem forældre kan følge deres børn aktivitetsvaner og progression, det vil ligeledes være muligt for forældrene at motivere børnene igennem konkurrencer og fælles aktiviteter. Denne logintype kunne ligeledes gøres tilgængeligt for relevant sundhedspersonale\fxnote{eks. læge, sundhedsplejeske mm.}, hvilket kunne optimere deres grundlag for afhjælpning af eventuel overvægt. \\



% - Pulsdata samplet med informationerne omkring den pågældende aktivitet kunne samles i systemet i en speciel fil, som ens læge kan få adgang til. Pulsen siger meget om kroppens fysiologiske tilstand. (Det er ret langt ude, så har ikke skrevet om det. Det vil virke lidt mærkeligt tror jeg)