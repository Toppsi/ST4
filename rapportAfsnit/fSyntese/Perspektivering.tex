\section{Perspektivering}
\textit{I dette afsnit beskrives en eventuel videreudvikling af systemet. Dette er med henblik på hardware, software samt brugerinterface.}

Det nuværende system opfattes som midlertidig løsning, da opbygningen af det ikke anses for værende helt optimal. Systemet kunne for eksempel opbygges mere kompakt i forhold til hardware, således det opfattes mere diskret. Det antages, at børn i den valgte aldersgruppe ikke ønsker, at systemet er tydeligt for andre, da hensigten ved brugen heraf er at mindske inaktivitet. Derfor ville det være optimalt, hvis systemet kan opbygges som et diskret bånd, der kan sidde på anklen uden det forekommer tydeligt for andre. For at dette kan lade sig gøre, kan systemet eventuelt benytte en mindre MCU, som har påmonteret accelerometer, gyroskop og pulssensor i én lille enhed. MCUen vil uanset opbygning være nødvendig, da der skal fortages dataopsamling, databehandling og BLE kommunikation. Det kunne derudover være fordelagtigt, hvis den perifere MCU besidder mere RAM, således mere data kan gemmes herpå og ikke er afhængig af konstant streaming af data. Derved kan et mere strømbesparende system muligvis opnås, da live visualisering ikke er af høj prioritet. \\
BLE er en god kommunikationsmulighed imellem den perifere og centrale enhed men har også sine begrænsninger i forhold til eksempelvis afstand. For optimal kommunikation kræver BLE, at enhederne er forholdsvis tæt på hinanden, hvilket kan være problematisk især for det nuværende system, da den centrale enhed består af en MCU tilkoblet en computer. En bruger kan derfor kun bevæge sig inden for en begrænset afstand herfra, da den perifere enhed sender data live. En løsning heraf kunne enten være, at den perifere som sagt indeholder flere RAM eller den centrale enhed bliver en mobil i stedet, hvilket gør det samlede system mere mobilt. Uanset løsningen vil det være fordelagtigt at installere en alarm i form af enten en lysende LED eller lyd, som aktiveres i tilfælde af, at enhedernes kommunikation via BLE ikke er optimal og pakker bliver tabt. \\
Hvis en mobil bliver den centrale enhed, er der også mulighed for videreudvikling af brugerinterfacet. Herved kan der laves en app, som kan gøres mere brugervenligt og motiverende end den nuværende GUI. Herigennem kunne det være muligt at installere en brugervenlig kalibreringsenhed, hvor eksempelvis tærskelværdierne imellem gang og løb kan finindstilles til den individuelle bruger i stedet for at have en generel værdi. Dette vil gøre systemet mere præcist og pointværdierne for hver aktivitet vil uddeles mere korrekt.\\
%%%%%%%%%%%%%%%%%%%%%%%%%%%%%%%%%%%%
% Inhold:
%
% Hardware
% - Systemet skal gøres mindre og mere diskret, så børn ikke fravælger systemet fordi de er flove over at være inaktive.
% - Den perifære enhed, som samler og databehandler, skal have mere RAM hukommelse, således data kan gemmes herpå i måske et kvarter og derefter sendes. (batterisparende)
%
% Software
% - Sender via bluetooth, hvilket kræver at modtagerenheden skal være inden for rækkevidde. Ikke så praktisk - måske mistes data, hvis enhederne er for langt fra hinanden? Man kunne indtallere en form for advarsel, hvis de kommer uden for rækkevidde - en LED eller hyletone?
% Datavisualiseringen skal foregå på en mobil app istedet, og ikke på en computer igennem Matlabs GUI.
% - Mobil app, hvor man kan finindstille tærskelværdier til den bestemte person. Man kunne kalibrere igennem et opsætningssystem, således systemet selv kan justere (starter et kallibreringsprogram og går, stopper, starter, løber, stopper, starter, cykler, således den er præcist tilpasset personen)
% - Pulsdata samplet med informationerne omkring den pågældende aktivitet kunne samles i systemet i en speciel fil, som ens læge kan få adgang til. Pulsen siger meget om kroppens fysiologiske tilstand. (Det er ret langt ude, så har ikke skrevet om det. Det vil virke lidt mærkeligt tror jeg)