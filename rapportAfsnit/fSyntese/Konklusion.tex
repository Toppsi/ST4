\section{Konklusion}
\textit{Følgende afsnit beskriver og konkluderer på projektets essentielle problemstillinger samt resultater. Afsnittet sammenholder projektets resultater med problemformuleringen, hvoraf en besvarelse præsenteres.}

Projektet beskriver udviklingen af en aktivitetsmåler, som har potentialet til at reducere antallet af fysisk inaktive børn i aldersgruppen 9-12 år. Systemet er bestående af algoritmer til detektering af aktiviteterne gang, løb og cykling. Systemets algoritmer er designet med hensigten at registrere disse specifikke fysiske aktiviteter, hvorefter resultaterne heraf benyttes som en motiverende faktor ved at visualisere tid og point. Aktiviteterne detekteres af henholdsvis et accelerometer og et gyroskop. Disse sensorer bidrager til detekteringen af den samlede udførte tid af hver aktivitet samt GUIs point system.

Systemet optræder  mobilt i en grad, hvor trådløs dataoverførsel mellem GAP peripheral og GAP central er mulig. Ydermere er størstedelen af kravene til systemets delelementer opfyldt. Dette bidrager til, at det samlede system kan adskille og detektere aktiviteterne gang, løb og cykling ved benyttelse af de valgte sensorer. \\
Det samlede system kan tilnærmelsesvis automatisk detektere og adskille aktiviteterne. Under udførelsen af gang ved 4,8 km/t var alle forsøgspersonernes resultater tilnærmelsesvis ens, hvilket ligeledes afspejledes under løb samt cykling. Systemet overholder dermed kravet vedrørende en afvigelse på maksimalt 10\% i forhold til detektion af aktiviteterne. En tredjedel af de samlede samples detekteres ikke for samtlige forsøgspersoner og aktiviteter, hvorfor tidsenheden ikke angives korrekt. Dermed overholder systemet ikke det føromtalte krav med hensyn til afvigelse i varigheden. Dette medfører, at yderligere algoritme optimering er nødvendig, førend dette krav vil blive imødekommet til fulde.

Systemets funktionalitet bliver testet i GUI, hvor algoritmernes resultater bliver visualiseret i form af aktiviteternes pointfordeling og varighed. Pulssensoren skulle have bidraget som en justerbar variabel til pointfordelingen, således et højt intensitetsniveau kunne belønnes. Algoritmen hertil fungerer efter hensigten, men eftersom pulssensoren ikke leverer et korrekt output under udførsel af fysisk aktivitet, kan denne ikke accepteres. Pulssensoren skal derfor optimeres, førend en implementering i et endeligt system kan foregå. \\
Det samlede system har et gennemsnitligt spændingsforbrug på 0,0956~V per time med antagelsen om et lineært spændingsforbrug. Det samlede system vil med den benyttede spændingsforsyning antageligvis kunne operere i 8,3 timer, førend spændingsniveauet er under de tilladte 1,71~V. Det konkluderes hermed, at det samlede system ikke kan detektere og adskille aktiviteterne gang, løb og cykling i over 15 timer. 

På baggrund af testene af systemets delelementer samt testen af det samlede system kan det konkluderes, at systemet skal optimeres for at imødekomme alle funktionelle krav og herefter operere fuldstændigt efter hensigten. Hvis systemet bliver optimeret i en grad, hvor disse krav imødekommes, vil den udviklede aktivitetsmåler have potentialet til at kunne medvirke til en reduktion af fysisk inaktive børn i aldersgruppen 9-12 år. En optimering af systemet er en nødvendighed, hvormed der bør tages forbehold for at det udviklede system er en prototype. En videreudvikling af prototypen vil have potentialet til at indgå i et kommercielt samfundsmæssigt perspektiv.