\section{Konklusion}
\textit{Følgende afsnit beskriver og konkluderer på projektets essentielle resultater samt problemstillinger. Afsnittet sammenkobler projektets testresultater med problemformuleringen, hvoraf en besvarelse heraf bliver fremført.}

Igennem dette projekt blev der udviklet en aktivitetsmåler hvis formål var at reducere antallet af fysisk inaktive børn i aldersgruppen 9-12 år. Systemets funktionalitet er bestående af algoritmer til detektering af aktiviteterne gang, løb og cykling, som heraf benyttes til afhjælpning af projektets problemstilling. Aktiviteternes totale udbredelse igennem en hel dag bliver igennem GUI visualiseret, som et produkt af algoritmernes resultater. Systemets algoritmer er designet for at sikre, at al aktivitet udført af brugeren bliver registreret, hvorefter det er muligt benytte disse resultater igennem forskellige motiverende elementer. Aktiviteterne registreres af henholdsvis et accelerometer, et gyroskop og en pulssensor. Disse sensorer bidrager hver især til optællingen af den samlede udførte tid af hver aktivitet, samt GUIs point system. På baggrund af dette, problemanalysen, litteratur og et pilotforsøg blev der opstillet funktionelle og specifikke krav til det samlede system. Heraf undersøges det hvorvidt systemets funktionalitet overholder, og hermed af,- eller bekræfter disse krav. 

Systemet optræder  mobilt i en grad hvor trådløs dataoverførsel mellem GAP peripheral og GAP central er mulig. Ydermere bliver største delene af kravene til systemets delelementer opfyldt, hvilket bibrager til at det samlede system kan adskille og detektere aktiviteterne gang, løb og cykling ved benyttelse af de valgte sensorer. Det samlede system kan til en hvis grad automatisk detektere og adskille de føromtalte aktiviteter i en grad hvor der tilnærmelsesvis ses en lineær sammenhæng. Under udførelsen af gang ved 4,8 km/t var alle forsøgspersonernes resultater tilnærmelsesvis ens, hvilke ligeledes kunne afspejles under løb samt cykling. Systemets overholdte dog ikke dets krav vedrørende en afvigelse på 10\%, hvilket medfører at yderligere algoritme optimering er nødvendig, førend disse krav vil blive imødekommet. Systemets funktionalitet blev testet igennem GUI hvor algoritmernes resultater blev visualiseret hvoraf aktiviteternes pointfordeling samt varighed blev repræsenteret. Pulssensoreren skulle have bidraget som en justerbar variable til pointfordelingen, således et højt intensitetsniveau kunne belønnes. Dette fungerede efter hensigten, dog kunne pulssensoreren ikke leverer et output under udførelse af aktivitet som kunne accepteres, hvoraf denne skal optimeres førend en implementering i det endelige system kan foregå. Det samlede system har et gennemsnitligt spændingsforbrug på 0,0956 V per time, med antagelsen om at spændingsforsyningen opererer lineært i hele dets arbejdsomrpde. Resultatet af dette spændingsforbrug er MCUens databehandling, samt transmission af data. Det samlede system vil med den benyttede spændingsforsyning antageligvis kunne opperer 15,48 timer, førend spændingsniveauet er under det tilladte. Det konkluderes hermed at det samlede system med sikkerhed kan detektere og adskille aktiviteterne gang, løb og cykling igennem 15 timer. 

På baggrund af testene af systemets delelementer samt testen af det samlede system, kan det konkluderes at systemet skal optimeres for at imødekomme alle funktionelle krav, og herefter opererer fuldstændigt efter hensigten. Hvis systemet bliver optimeret i en grad hvor disse krav imødekommes vil den udviklede aktivitetsmåler have potentialet til at kunne medvirke til en reduktion af fysisk inaktive børn i aldersgruppen 9-12 år, og dets helbredsmæssige konsekvenser heraf. En optimering af systemet er en nødvendighed og dermed bør der tages forbehold for at det udviklede system er en prototype. Resultatet heraf medfører at yderligere videreudvikling og undersøgelser vil være omdrejningspunkt for at systemet udvikles til produkt fra prototype.