\section{Konklusion}
% Hvordan kan en aktivitetsmåler udvikles således, at den har potentialet til at reducere antallet af fysisk inaktive børn i aldersgruppen 9-12 år?
\textit{Følgende afsnit beskriver og konkluderer på projektets essentielle problemstillinger samt resultater. Afsnittet sammenkobler projektets testresultater med problemformuleringen, hvoraf en besvarelse heraf bliver fremført.}

Igennem projektet bliver en aktivitetsmåler udviklet således, at den har potentiale til at reducere antallet af fysisk inaktive børn i aldersgruppen 9-12 år. Systemet er bestående af algoritmer til detektering af aktiviteterne gang, løb og cykling, som benyttes til afhjælpning af projektets problemstilling. Systemets algoritmer er designet med hensigten at registrere disse specifikke fysiske aktiviteter, hvorefter resultaterne heraf benyttes som en motiverende faktor ved at visualisere tid og point for hver aktivitet samt aktiviteternes totale udbredelse igennem en hel dag. \\
Aktiviteterne registreres af henholdsvis et accelerometer, et gyroskop og en pulssensor. Disse sensorer bidrager hver især til optællingen af den samlede udførte tid af hver aktivitet samt GUIs point system. %På baggrund af dette, problemanalysen, litteratur og et pilotforsøg blev der opstillet funktionelle og specifikke krav til det samlede system. Heraf undersøges det hvorvidt systemets funktionalitet overholder, og hermed af,- eller bekræfter disse krav. 

Systemet optræder  mobilt i en grad, hvor trådløs dataoverførsel mellem GAP peripheral og GAP central er mulig. Ydermere er størstedelen af kravene til systemets delelementer opfyldt. Dette bidrager til, at det samlede system kan adskille og detektere aktiviteterne gang, løb og cykling ved benyttelse af de valgte sensorer. \\
Det samlede system kan med en hvis afvigelse automatisk detektere og adskille de føromtalte aktiviteter med en tilnærmelsesvis lineær sammenhæng. Under udførelsen af gang ved 4,8 km/t var alle forsøgspersonernes resultater tilnærmelsesvis ens, hvilke ligeledes afspejledes under løb samt cykling. Systemets overholder dermed kravet vedrørende en afvigelse på maksimalt 10\% i forhold til detektion af aktiviteterne. Dog opfanger GUIen ikke en tredjedel af de samlede samples for samtlige forsøgspersoner og aktiviteter, hvorfor tidsenheden ikke angives korrekt. Dermed overholder systemet ikke det føromtalte krav med hensyn til afvigelse i varigheden. Dette medfører, at yderligere algoritme optimering er nødvendig, førend dette krav vil blive imødekommet til fulde. \\
Systemets funktionalitet blev testet igennem GUI, hvor algoritmernes resultater blev visualiseret i form af aktiviteternes pointfordeling og varighed. Pulssensoren skulle have bidraget som en justerbar variable til pointfordelingen, således et højt intensitetsniveau kunne belønnes. Dette fungerede efter hensigten, dog kunne pulssensoreren ikke leverer et output under udførelse af aktivitet, som kunne accepteres. Denne skal derfor optimeres, førend en implementering i et endeligt system kan foregå. \\
Det samlede system har et gennemsnitligt spændingsforbrug på 0,0956~V per time med antagelsen om, at spændingsforsyningen forbruges lineært i hele dets levetid. Resultatet af dette spændingsforbrug vurderes især at gå til MCUens databehandling, ICens dataopsamling samt transmission af data. Det samlede system vil med den benyttede spændingsforsyning antageligvis kunne operere i 15,48 timer, førend spændingsniveauet er under de tilladte 1,71~V. Det konkluderes hermed at det samlede system kan detektere og adskille aktiviteterne gang, løb og cykling igennem 15 timer. 

På baggrund af testene af systemets delelementer samt testen af det samlede system kan det konkluderes, at systemet skal optimeres for at imødekomme alle funktionelle krav og herefter operere fuldstændigt efter hensigten. Hvis systemet bliver optimeret i en grad, hvor disse krav imødekommes, vil den udviklede aktivitetsmåler have potentialet til at kunne medvirke til en reduktion af fysisk inaktive børn i aldersgruppen 9-12 år. En optimering af systemet er en nødvendighed, hvormed der bør tages forbehold for at det udviklede system er en prototype.% Resultatet heraf medfører at yderligere videreudvikling og undersøgelser vil være omdrejningspunkt for at systemet udvikles til produkt fra prototype.