\section{Diskussion}\label{sec:diskussion}
\textit{Afsnittet diskuterer og vurderer validiteten af rapportens opnåede resultater, for derefter at opstille eventuelle forbedringer med henblik på en systemoptimering. Afslutningsvis diskuteres prototypens egnethed i et samfundsfagligt perspektiv.}

Problemanalysen undersøger kvaliteten af en række aktivitetsmålere, men hensyn til udvalgte kriterier. Det fremgår her af, at ingen aktivitetsmålere opfylder alle opstillede kriterier, hvorfor et alternativt system vil være fordelagtigt at undersøge og udvikle. \\
Rapporten beskriver udarbejdelsen af en prototype, som potentielt kan opfylde de opstillede kriterier. 

\subsubsection{Pilotforsøget}
Pilotforsøget undersøgte blandt andet signalers udformning ved gang, løb og cykling. Der viste sig dog at være stor forskel på forsøgspersonernes amplituder ved de pågældende aktiviteter. Dette kan skyldtes, at forsøgspersonerne ikke benyttede samme type fodtøj. Derfor vil personer med en større stødabsorbering opnå lavere amplituder ved gang og løb, end personer i fodtøj med lav stødabsorbering. Det kan derfor antages, at en forsøgspopulation med sammenligneligt fodtøj vil medføre en mere valid dataopsamling.
Ydermere vil en større forsøgspopulation give et mere repræsentativt datasæt, hvilket ligeledes vil øge validiteten af dataopsamlingen. \\
Pilotforsøgets forsøgspersoner har en højere vægt end rapportens målgruppe. Det er derfor antageligt, at g påvirkningerne fra forsøgspersonerne under pilotforsøget var højere end målgruppens. Derfor vil det være essentielt, at udføre et pilotforsøg på en forsøgspopulation i aldersgruppen tilsvarende målgruppen. Datasættet vil dermed være mere repræsentativt for rapportens målgruppe.

Pilotforsøget undersøgte ikke gyroskopets output ved forsøgspersonernes maksimale antal omdrejninger ved cykling. Kravene til prototypens gyroskop er dermed opstillet på baggrund af en tilnærmelsesvis konstant omdrejningshastighed. Det er bestemt, at prototypens gyroskop blot skal overholde et minimum antal omdrejninger per sekund. Det vil derfor være fordelagtigt, at undersøge forsøgspopulationens maksimale hastighed og dermed maksimale antal omdrejninger ved cykling. Denne værdi kan benyttes til at opstille krav til arbejdsområdet for prototypens gyroskop. \\
Ydermere undersøgte pilotforsøget ikke hvilken betydning en acceleration under cykling har for signalets udformning. Det vil være særligt fordelagtigt, at undersøge betydningen af en acceleration for signalets frekvensdomæne. Dette er med henhold til detektering af cykling, hvoraf en frekvensdomæneanalyse benyttes til dette formål.

\subsubsection{Design, implementering og test af prototypens blokke}
Prototypens separate blokke er design, implementeret og testet enkeltvis for at opretholde variabelkontrol under processen. Prototypen er dog ikke et færdigudviklet produkt som tager højde for alle relevante system- og brugermæssige aspekter. \\
Prototypens spændingsforbrug er testet over en varighed på X timer. Det fremgik af testen, at GAP peripheral vil være funktionel i Y timer ved en spændingstilkobling svarende 3,17~V.  


\subsubsection{Samlet systemtest}



\subsubsection{Samfundsmæssige perspektiver}
Rapporten sætter særligt fokus på fysisk inaktivitet hos børn i aldersgruppen 9-12 år, og hvilke konsekvenser dette kan have for barnet i nutiden og fremtiden. Det er med udgangspunkt i disse fysiologiske og psykiske konsekvenser, at der er vurderet behov for en ny og forbedret aktivitetsmåler. \\
Væsentligt er især valget af målgruppe, hvor det er beskrevet, at implementeringen af nye vaner kan være hensigtsmæssigt i hos målgruppen. Det er dog vigtigt, at de sundhedsmæssige vaner som en aktivitetsmåler ligger op til, ikke bliver overdrevet hos børnene……
