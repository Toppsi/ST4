\section{Diskussion}\label{sec:diskussion}
%\textit{}

Indhold til diskussion i kort punktform:
\begin{itemize}
	\item Diskuter problemstillingen fra problemanalysen. Er det sundt og fornuftigt at gøre så unge mennesker bevidste om deres aktivitetsniveau og egen krop? Man kunne evt. kople lidt moderne livsstil ind over, hvor det er populært i vores moderne samfund at være fit og med på alle sociale medier (lidt langt ude...)
	\item Diskutter pilotforsøgets betydning for hele projektet. Hvorfor er der så stor forskel på de forskellige forsøgspersoners data? Hvilken betydning har denne spredning for hele designet af systemet? Var sensoren den mest optimale til forsøget? Hvis vi kunne lave noget om, hvad skulle det så være?
	\item Når der ikke tages højde for batterilevetid alligevel, havde det så ikke været en fordel at bruge gyroskop til detektering af alle aktiviteter, når dette signal var mere markant? 
	\item Havde almindelig bluetooth været bedre at benytte istedet for BLE? Hvilke andre muligheder havde det givet, både positiv (længere range) og negativ (kræver mere strøm)
	\item Diskutter at vores algoritme for cykling muligvis ikke tager højde for acceleration. Det er ikke noget vi kan sige med sikkerhed, da vi netop har tjekke for en fast hastighed. Vi mener, at under acceleration vil frekvenserne netop være fordelt mere end ved konstant hastighed - uanset hastighed - hvorved der muligvis kan forekomme fejl under acceleration.
	\item Tærskelværdien er ikke kalibreret. Dette Bliver over beskrevet i perspektiveringsafsnittet, hvordan dette kunne gøres - her kan man altså fokusere på, hvilken betydning har det for det samlede system?
	\item Hvis en person træder meget hårdt eller hopper meget under gang eller whatever og vedkommende får en værdi på eksempelvis 1500 under gang, så vil dette detekteres som løb. Hvad kan vi gøre for at løse dette?
	\item Diskussion om det samlede systemforsøg (der kommer sikkert til at være nogle diskussionselementer)
\end{itemize}
