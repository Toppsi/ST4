\section{Diskussion}\label{sec:diskussion}
\textit{Afsnittet diskuterer og vurderer validiteten af projektets indhold% opnåede resultater,
 for derefter at opstille eventuelle forbedringer med henblik på en systemoptimering. Afslutningsvis diskuteres det, hvorledes prototypen egner sig i et samfundsfagligt perspektiv.}

Problemanalysen undersøger kvaliteten af en række aktivitetsmålere med hensyn til udvalgte kriterier. Det fremgår heraf, at ingen aktivitetsmålere opfylder alle opstillede kriterier. Det er derfor fordelagtigt, at designe og udvikle en prototype som har potentiale til at opfylde de opstillede kriterier. Kriterierne for protypen beskrev, at prototypen skal:
\begin{itemize}
	\item Kunne detektere aktiviteterne gang, løb og cykling ved brug af gyroskop og accelerometer. Der accepteres ikke brug af andre sensorer.
	\item Kunne lave automatisk adskillelse af gang, løb og cykling ved hjælp af algoritmer. Der accepteres en afvigelse på 10\% i forhold til fejlvurdering af aktivitet og varighed.
	\item Kunne detektere puls ved brug af pulssensor og tilhørende algoritme samt derefter kategorisere intensiteten af en given aktivitet. Der accepteres en pulsafvigelse på 10\%.
	\item Videresende signaler til en ekstern enhed ved hjælp af BLE. Der accepteres ikke andre trådløse kommunikationsformer.
	\item Besidde batterilevetid for en hel dag svarende til 15 timer. Der accepteres ikke en batterilevetid på mindre end 15 timer.
	\item Repræsentere varigheden og pointfordelingen af en given aktivitet i GUI. Der accepteres ikke en anden form for visualisering. 
\end{itemize}


\subsubsection{Resultater opnået i pilotforsøget}
Pilotforsøgets opnåede resultater diskuteres, med henblik på at opnå mere normaliserede, og dermed valide, resultater til brug ved udvikling af en prototype.\\
Pilotforsøget havde til formål at undersøge signalernes udformning, frekvensområde og amplitude ved gang, løb og cykling. Ydermere blev signalerne undersøget i forhold til placeringen af sensoren på benet. 

Pilotforsøget undersøger blandt andet signalers udformning ved gang, løb og cykling. Det viser sig, at der er stor forskel på signalernes amplituder fra hver forsøgsperson ved gang og løb. Dette kan skyldtes, at forsøgspersonerne ikke benyttede samme type fodtøj. Derfor vil personer, der har sko med en større stødabsorbering, opnå lavere amplituder ved gang og løb, end personer i fodtøj med lavere stødabsorbering. Det kan derfor antages, at en forsøgspopulation med sammenligneligt fodtøj vil medføre en mere identisk dataopsamling. Det vil derfor være fordelagtigt, at udføre forsøget med samme type fodtøj for forsøgspersonerne. Ydermere vil en større forsøgspopulation give et mere repræsentativt datasæt, hvilket vil øge validiteten af dataopsamlingen. En større forsøgspopulation vil ligeledes bidrage til dannelsen af en normale, hvoraf eventuelle algoritmer med større sandsynlighed vil fungere i praksis.\\
Pilotforsøgets forsøgspersoner har en højere vægt end rapportens målgruppe. I 2014 befandt børn i alderen 9-12 år sig gennemsnitligt i vægtklassen 28-42 kg. \citep{Rigsholspitalet2014} Det er derfor antageligt, at g påvirkningerne fra forsøgspersonerne under pilotforsøget er højere, end hvis målgruppen udførte pilotforsøget. Derfor vil det være essentielt at udføre et pilotforsøg med en forsøgspopulation i aldersgruppen tilsvarende målgruppen. Datasættet og databehandlingen vil dermed være mere repræsentativt for rapportens målgruppe.\\
Pilotforsøget undersøger ikke gyroskopets output ved forsøgspersonernes maksimale antal omdrejninger ved cykling. Kravene til prototypens gyroskop er dermed opstillet på baggrund af en tilnærmelsesvis konstant omdrejningshastighed. Det er bestemt, at prototypens gyroskop blot skal overholde et minimum antal omdrejninger per sekund. Det vil derfor være fordelagtigt at undersøge forsøgspopulationens maksimale hastighed og dermed maksimale antal omdrejninger per sekund ved cykling. Denne værdi kan benyttes til at opstille krav til arbejdsområdet for prototypens gyroskop.\\

Pilotforsøget undersøgte signaludformningen for cykling ved et konstant antal omdrejninger. Dette forsøg tager derfor ikke højde for en acceleration eller skiftende antal omdrejningerne. Det vil derfor være fordelagtigt at undersøge betydningen af acceleration under cykling for signalets frekvensdomæne. Dette er med henhold til detektering af cykling, hvoraf en frekvensdomæneanalyse benyttes til dette formål. Da dette ikke er undersøgt, vides det ikke, hvorledes accelererende cykling vil blive opfattet af sensoren.

\subsubsection{Udformning af systemets blokke}
Prototypens separate blokke er designet, implementeret og testet enkeltvis for at opretholde variabelkontrol under processen. Prototypen er dog ikke et færdigudviklet produkt, som tager højde for alle relevante system- og brugermæssige aspekter for den enkelte blok. Det er eksempelvis ikke muligt at udforme en brugervenlig og komfortabel enhed som er let at påføre brugeren.

Prototypens spændingsforbrug er testet ved benyttelse af systemet i 2,5~time. Det fremgik af testen, at GAP peripheral vil være funktionel i \textbf{Y} timer ved en spændingstilkobling på 3,17~V fra ubrugte AAA batterier. Yderligere er denne antagelse men hensyn til et lineært spændingsforbrug for prototypen. Denne varighed overholder ikke det funktionelle krav for det samlede system, som fremsætter et krav på en batterilevetid på 15 timer. Det vil derfor være relevant at sænke systemets strømforbrug for at øge batteriernes levetid og dermed prototypens funktionelle levetid. Eksempelvis ville det være muligt, at slukke BLE modulerne hvis disse ikke er indenfor rækkevidde. Yderligere optimeringer til strømbesparelse, fremgår af \secref{sec:perspektivering}. 

Prototypen involverer yderligere en pulsdetektering som har til formål at kategorisere intensiteten af den pågældende aktivitet.
Testen af prototypen viser sig at have en afvigelse på 0\% ved pulsdetektering på et simuleret signal. Ydermere blev pulsdetekteringen testet på forsøgsperson, som ikke udførte fysisk aktivitet. Det fremgik heraf, at pulsdetekteringen har en afvigelse på 8,15\% fra den benyttede reference. Dette kan blandt andet være som følge af en dårlig kontakt mellem pulssensor og øreflip. Hvis dette er tilfældet, vil signalets amplitude ikke være tilstrækkeligt stort til at overskride tærskelværdien. Dette fremgik ved testens udførsel, hvor kontakten mellem sensor og hud blev bedre i takt som funktion af tiden. Som følge af en bedre kontakt, blev det observeret, at signalets amplitude blev markant forøget. \\
Yderligere bliver pulsdetektering testet ved gang på et løbebånd for at undersøge funktionaliteten heraf ved fysisk aktivitet. Testen påviser en afvigelse på -17,85\% og -46,4\% for henholdsvis gang og løb i forhold til den benyttede reference. Det blev observeret ved testens udførsel, at signalets udformning under gang og løb indeholder støj, hvoraf denne støj var markant større ved løb. Der var derfor ikke muligt at foretage en optimal pulsdetektering på dette signal. Testen blev ikke udført på cykling, idet denne aktivitet ikke indebærer betydelig bevægelse af overkroppen. Dermed antages det, at pulssensoren ikke belastes betydeligt mere i forhold til den stillesiddende position. \\
Som følge af de observerede signaler, samt de påviste afvigelser, er det fordelagtigt at udforme en anordning, som mindsker mængden af støj på signal. Dette kan eksempelvis opnås ved en anordning som sænker risikoen for bevægelse af sensorens ledninger ved selve sensoren. Yderligere vil det være fordelagtigt, at udforme en opsætning som sikrer en stabil og konstant kontakt mellem sensor og hud. 

Den benyttede IC i prototypen indeholder blandt andet et accelerometer og et gyroskop. Accelerometret blev testet og har en afvigelse på maksimalt 1,4\%. Dette output fra accelerometeret har en tilstrækkelig lav afvigelse, hvormed accelerometerets output kan accepteres. Ydermere er det ikke været muligt at bestemme det maksimale arbejdsområde for accelerometeret på en videnskabelig måde. Dette skyldtes, at der ikke er tilgængeligt udstyr, som vil kunne påvirke accelerometeret med op til 16~g. Det er derfor antaget, at accelerometerets arbejdsområde er tilsvarende de konfigurerede indstillinger, som fabrikanten foreskriver i databladet. Det har ikke været muligt at bestemme gyroskopets nøjagtighed eller arbejdsområde grundet mangel på udstyr til kontrol af konstante omdrejninger. Det er derfor antaget, at gyroskopets nøjagtighed og arbejdsområde er tilsvarende de konfigurerede indstillinger, som fabrikanten foreskriver i databladet. Med henhold til disse antagelser er det derfor ikke muligt at vurdere, hvorvidt outputdata fra gyroskopet er valide ved implementering i det endelig system. 

Prototypens rækkevidde for den trådløse kommunikation blev testet, hvor det fremgår, at ved en afstand på 4~meter mellem to MCUer mistes forbindelsen. Denne afstand blev testet, hvor begge PRoCs på MCUerne var placeret således, at de pegede mod hinanden og var koblet til spænding ved brug af USB porten. Det vil derfor være fordelagtigt yderligere at teste, hvilken betydning det har, når afstanden mellem enhederne er afskærmet af genstande eller personer. Dette kun eksempelvis testes ved, at en person har påmonteret MCUen på benet under et par bukser, for dermed at simulere en påmontering af prototypen. Rækkevidden for den trådløse kommunikation kan derfor antages at blive forkortet ved disse forhold. \\
Derudover bør der testet, hvorledes en ekstern spændingsforsyning har betydning for rækkevidden den trådløse kommunikation end spændingsforsyning fra USB porten. Dette er især essentielt, da GAP peripheral i det samlede system er tilkoblet en ekstern strømforsyning i form af to 1,5~V batterier.

\subsubsection{Samlet systemtest}
\textbf{\textit{Den samlede system test er ikke lavet endnu, hvoraf dette afsnit ikke er skrevet. Vi vil her diskutere resultaterne fra testen og beskrive hvorfor de er som de er, og hvad der kan være galt/godt med resultaterne.}} \\

\subsubsection{Resultater opnået ved test af prototype}
De enkelte blokke er testet og opfylder de opstillede krav, med undtagelse af pulssensoren. De funktionelle blokke er dermed samlet til prototypen. Denne er testet i forhold til de opstillede krav, som fremgår i \secref{sec:diskussion}.

Prototypen testes blandt andet for at undersøge, hvorvidt den udførte aktivitet detekteres og er svarende til varigheden som aktiviteten er udført. Testen påviser, at der er en maksimal forskel 5,13\% mellem den totale detektering af aktivitet og detekteringen af den pågældende aktivitet. Prototypen er dermed i stand til at detektere og adskille de gang, løb og cykling med en afvigelse på 5,13\%. \\
Det fremgår ydermere af testen, at de fysiske aktiviteter detekteres til en varighed på cirka 40~sekuder, på trods af at aktiviteten bliver udført i 60~sekunder. Dette svarer dermed til, at cirka hvert tredje sekund af den udførte aktivitet ikke detekteres som værende udført af prototypen. Denne kan være som følge af en række softwaremæssige aspekter. Eksempelvis er algoritmen, implementeret på MCUerne, udført med henblik på at fungere efter hensigten. Der er derfor ikke taget højde for varigheden af en eksekvering af algoritmen. Det vil derfor være fordelagtigt, at undersøge algoritmens varighed for dermed at nedsætte denne. Ved at forkorte varigheden for eksekvering af algoritmen, vil antageligt have en forbedret effekt på algoritmens detektering. Dermed vil der være en større varighed detekteret af den udførte aktivitet. 

Prototypen blev ydermere testet med hensyn til detektering af cykling. Det fremgår, at der er 0\% mellem totale mængde detekteret aktivitet og den udførte aktivitet. Denne test er dog udført for en konstant kadence. Det vil yderligere være fordelagtigt, at teste betydningen af acceleration under cykling. Ved acceleration antages det dog, at frekvensområdet er mere spredt over en række frekvenser. Derfor vil amplituderne omkring en given frekvens fortsat udgøre over 70\% af den samlede magnitude for frekvensdomænet. Denne antagelse vil dog påkræve en test, for at kunne optimere algoritmen hvis ikke acceleration af en kadence ikke detekteres som værende cykling. \\
Prototypen tester yderligere ikke hvilken betydning en langsommere eller hurtigere kadence har for detekteringen af cykling. Det fremgår af pilotforsøget, at signalets udformning for cykling er tilsvarende en sinusbølge. Derfor er det antageligt, at en langsommere eller kadence, blot vil flytte magnituderne i frekvensdomænet til henholdsvis en lavere eller højere frekvens. Denne antagelse skal derfor testes, for at undersøge hvorvidt ovenstående har indvirkning på detekteringen af cykling. 

Prototypen testes desuden ved en hastighed på 8~km/t med udførsel af aktiviteterne gang og løb. Der forekommer her en større procentvis afvigelse mellem den detekterede varighed for aktiviteten og den samlede varighede af detekteret aktivitet. Dette kan være som følge af den hastighed som både gang og løb udføres med. Denne hastighed er valgt med udgangspunkt i et studie, som beskriver 6,4~km/t som rask gang og 9,7~km/t som løb \citep{115}. Derfor er en hastighed på 8~km/t valgt, som er en overgang mellem gang og løb. Det blev desuden observeret i pilotforsøget, at ved en hastigheden mellem 8~km/t og 10~km/t skiftede forsøgspersonerne fra gang til løb. Det blev dog observeret under testen af prototypen, at forsøgspersonerne havde problemer med at opretholde gang ved 8~km/t. Særligt var de kvindelige forsøgspersoner udfordret, idet disse forsøgspersoner havde en gangcyklus som havde tendens til at ligne om en løbecyklus. Dette bekræfter prototypen, idet denne detekterede aktiviteten som løb, når de kvindelige forsøgspersoner udførte gang ved 8~km/t. Prototypen detekterede dog en større varighed af gang hos de mandlige forsøgspersoner. Denne forskel kan være som følge af højdeforskellen, og dermed benlængden, mellem de kvindelige og mandlige forsøgspersoner. De kvindelige forsøgspersoner havde dermed en unaturlig gangcyklus, hvorfor det antages at denne cyklus ikke vil forekomme under normale omstændigheder i dagligt brug.
