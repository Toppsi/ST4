\section{Diskussion}\label{sec:diskussion}
\textit{Afsnittet diskuterer og vurderer validiteten af projektets indhold% opnåede resultater,
 for derefter at opstille eventuelle forbedringer med henblik på en systemoptimering. Afslutningsvis diskuteres det, hvorledes prototypen egner sig i et samfundsfagligt perspektiv.}

Problemanalysen undersøger kvaliteten af en række aktivitetsmålere med hensyn til udvalgte kriterier. Det fremgår heraf, at ingen aktivitetsmålere opfylder alle opstillede kriterier, hvorfor et alternativt system vil være fordelagtigt at undersøge og udvikle. \\
Rapporten beskriver udarbejdelsen af en prototype, som potentielt kan opfylde de opstillede kriterier. 

\subsubsection{Pilotforsøget}
Pilotforsøget undersøger blandt andet signalers udformning ved gang, løb og cykling. Det viser sig, at der er stor forskel på signalernes amplituder fra hver forsøgsperson ved de pågældende aktiviteter. Dette kan skyldtes, at forsøgspersonerne ikke benyttede samme type fodtøj. Derfor vil personer, der har sko med en større stødabsorbering, opnå lavere amplituder ved gang og løb, end personer i fodtøj med lavere stødabsorbering. Det kan derfor antages, at en forsøgspopulation med sammenligneligt fodtøj vil medføre en mere enslignende dataopsamling. Ydermere vil en større forsøgspopulation give et mere repræsentativt datasæt, hvilket vil øge validiteten af dataopsamlingen. En større forsøgspopulation vil ligeledes bidrage til dannelsen af en normale, hvoraf eventuelle algoritmer med større sandsynlighed vil fungere i praksis.\\
Pilotforsøgets forsøgspersoner har en højere vægt end rapportens målgruppe. I 2014 befandt målgruppen sig gennemsnitligt i vægtklassen 28-42 kg. \citep{Rigsholspitalet2014} Det er derfor antageligt, at g påvirkningerne fra forsøgspersonerne under pilotforsøget er højere, end hvis målgruppen udførte pilotforsøget. Derfor vil det være essentielt at udføre et pilotforsøg med en forsøgspopulation i aldersgruppen tilsvarende målgruppen. Datasættet og databehandlingen vil dermed være mere repræsentativt for rapportens målgruppe.\\
Pilotforsøget undersøger ikke gyroskopets output ved forsøgspersonernes maksimale antal omdrejninger ved cykling. Kravene til prototypens gyroskop er dermed opstillet på baggrund af en tilnærmelsesvis konstant omdrejningshastighed. Det er bestemt, at prototypens gyroskop blot skal overholde et minimum antal omdrejninger per sekund. Det vil derfor være fordelagtigt at undersøge forsøgspopulationens maksimale hastighed og dermed maksimale antal omdrejninger per sekund ved cykling. Denne værdi kan benyttes til at opstille krav til arbejdsområdet for prototypens gyroskop.\\
Ydermere undersøger pilotforsøget ikke, hvilken betydning en acceleration under cykling har for signalets udformning. Det vil være særligt fordelagtigt at undersøge betydningen af acceleration under cykling for signalets frekvensdomæne. Dette er med henhold til detektering af cykling, hvoraf en frekvensdomæneanalyse benyttes til dette formål. Da dette ikke er undersøgt vides det ikke, hvorledes accelererende cykling vil blive opfattet af sensoren.

\subsubsection{Design, implementering og test af prototypens blokke}
Prototypens separate blokke er designet, implementeret og testet enkeltvis for at opretholde variabelkontrol under processen. Prototypen er dog ikke et færdigudviklet produkt, som tager højde for alle relevante system- og brugermæssige aspekter.\\
Prototypens spændingsforbrug er testet over en varighed på en time. Det fremgik af testen, at GAP peripheral vil være funktionel i \textbf{Y} timer ved en spændingstilkobling på 3,17~V fra ubrugte batterier. Denne test beskriver prototypens funktionelle varighed når alle sensorer og algoritmer er aktive. Det vil derfor være relevant at sænke systemets strømforbrug for at øge batteriernes levetid og dermed systemets levetid.\\
Prototypens pulsdetektering viser sig at have en afvigelse på 0\% ved et simuleret inputsignal. Ydermere blev pulsdetekteringen testet på forsøgsperson, som ikke udførte fysisk aktivitet. Det fremgik heraf, at pulsdetekteringen har en afvigelse på 8,15\% fra den benyttede reference. Dette kan blandt andet være som følge af en dårlig kontakt mellem pulssensor og øreflip. Hvis dette er tilfældet, vil signalets amplitude ikke være tilstrækkeligt stort til at overskride tærskelværdien. Yderligere bliver pulsdetektering testet ved gang på et løbebånd for at undersøge funktionaliteten heraf ved fysisk aktivitet. Testen påviser, at det ikke var muligt at detektere en puls uden at systemet giver ukorrekte resultater. Dette skyldes signalets udformning under gang indeholder støj, hvilket kan være opstået som følge af bevægelser af ledningerne. Ydermere kan kontakten mellem hud og sensor være blevet forringet, hvilket ligeledes kan have bidraget til et støjfuldt signal. Det kan derfor være fordelagtigt at udforme en anordning, der sænker risikoen for bevægelse af sensorens ledninger ved selve sensoren. Yderligere vil det være fordelagtigt, at udforme en opsætning som sikrer en stabil og konstant kontakt mellem sensor og hud.\\
Den benyttede IC i prototypen indeholder blandt andet et accelerometer og et gyroskop. Accelerometret blev testet og har en afvigelse på -0,007~g til 0,003~g. Dette output fra accelerometeret har en tilstrækkelig lav afvigelse, hvormed accelerometerets output kan accepteres. Ydermere er det ikke været muligt at bestemme det maksimale arbejdsområde for accelerometeret på en videnskabelig måde. Dette skyldtes, at der ikke er tilgængeligt udstyr, som vil kunne påvirke accelerometeret med op til 16~g. Det er derfor antaget, at accelerometerets arbejdsområde er tilsvarende de konfigurerede indstillinger, som fabrikanten foreskriver i databladet. Det har ikke været muligt at bestemme gyroskopets nøjagtighed eller arbejdsområde grundet mangel på udstyr til kontrol af konstante omdrejninger. Det er derfor antaget, at gyroskopets nøjagtighed og arbejdsområde er tilsvarende de konfigurerede indstillinger, som fabrikanten foreskriver i databladet. Med henhold til disse antagelser er det derfor ikke muligt at vurdere, hvorvidt outputdata fra gyroskopet er valide ved implementering i det endelig system. \\
Prototypens rækkevidde for den trådløse kommunikation er testet, hvor det fremgår, at ved en afstand på 4~meter mellem to MCUer mistes forbindelsen. Denne afstand blev testet, hvor begge PRoCs på MCUerne var placeret således, at de pegede mod hinanden og var koblet til strøm via USB porten. Det vil derfor være fordelagtigt yderligere at teste, hvilken betydning det har når afstanden mellem enhederne er afskærmet af genstande eller personer. Dette kun eksempelvis testes ved en person har påmonteret MCUen på benet under et par bukser for dermed at simulere en påmontering af prototypen. Rækkevidden for den trådløse kommunikation kan derfor antages kortere ved disse forhold end testen påviste. Derudover burde det testes, hvorledes en ekstern strømforsyning har betydning for BLE rækkevidden end strømforsyning fra USB porten. Dette er især essentielt, da GAP peripheral i det samlede system er tilkoblet en ekstern strømforsyning i form af to 1,5~V batterier.

\subsubsection{Samlet systemtest}
Vi gennemgår resultaterne fra testen og beskriver hvorfor de er som de er, og hvad der kan være galt/godt med resultaterne.

\subsubsection{Samfundsmæssige perspektiver}
Rapporten sætter fokus på fysisk inaktivitet hos børn i aldersgruppen 9-12 år, og hvilke konsekvenser dette kan have for barnet i nutiden og fremtiden. Det er med udgangspunkt i disse fysiologiske og psykologiske konsekvenser, at der er vurderet behov for en ny og forbedret aktivitetsmåler. Væsentligt er især valget af målgruppe, hvor det er beskrevet, at inkorporationen af nye vaner kan være hensigtsmæssigt hos målgruppen. Ydermere bør børn motiveres igennem spil og leg, hvoraf børnene i målgruppen bør blive påvirket herigennem. Børnene bør altså udsættes for en tilgang til sundhed igennem gode oplevelser, hvoraf en afhjælpning af inaktivitet og overvægt er muligt. Det er dog vigtigt, at de sundhedsmæssige vaner som en aktivitetsmåler ligger op til, ikke bliver overdrevet hos børnene. Det er ikke hensigten, at børnene skal have fokus på at være fysisk aktive igennem hele deres dag, men mere gøres til en god vane. Der eksisterer altså modsatrettede hensigter i forhold til inkorporering af betydningen af fysisk aktivitet for målgruppen. Det skal gøres til en god vane uden at blive en altafgørende faktor for hverdagen.