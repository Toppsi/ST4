\section{Grafisk bruger interface}\label{GUI_design}
\textit{Dette afsnit omhandler design, implementering og test af GUI til visualisering af de udførte aktiviteter. Først designes GUIen til det specifikke formål ud fra dets kravspecifikationer, hvorefter denne kan implementeres. Afslutningsvist bliver GUI testet i forhold til opstillede krav, som beskrevet i \secref{krav_GUI}.}

\subsection{Design}
GUI benyttes i dette projekt til at motivere børn til en mere aktiv hverdag. Dette gøres ud fra \secref{motivation_boern}, hvor det beskrives, at børn motiveres gennem succesoplevelser. GUI designes med henblik på at give børnene et overblik over, hvor lang tid de har udført en given aktivitet, og hvor mange point de har optjent som følge af dette. Pointene vægtes ud fra aktivitetstypen og varigheden heraf. 

Data fra gyroskopet sendes til MATLAB som tre bytes. Først sendes en identifikation (ID) på én byte, og derefter gyroskopdata i en pakke bestående af lowbyte og highbyte. Arrayet af data, som modtages fra gyroskopet, er dermed: [ID gyroskopdata(lowbyte) gyroskopdata(highbyte)]. Data fra accelerometeret sendes som i alt fem bytes. Først sendes et ID på én byte, hvorefter tiden mellem to peaks sendes som en pakke bestående af lowbyte og highbyte og til sidst peakværdien i en pakke, ligeledes bestående af lowbyte og highbyte. Arrayet af data fra accelerometeret er dermed: [ID tid(lowbyte) tid(highbyte) peak(lowbyte) tid(highbyte)]. Den første byte i begge arrays er altså et ID, som angiver hvorvidt dataet kommer fra accelerometeret eller gyroskopet, som det ses på \figref{fig:GUI}. IDet for accelerometeret er 2, mens IDet for gyroskopet er 3.\\
Hvis MATLAB registrerer accelerometerets ID, findes peakværdien til detektion af gang og løb, som en pakke bestående af tredje og fjerde byte i arrayet. Peakværdi gør, at gang og løb kan adskilles og informerer dermed om, hvilken aktivitet tidsvariablen skal lægges til. På fjerde og femte plads findes en pakke af tidsenheden, som er resultat af varighed siden sidst detekterede peak. Er peakværdien mellem 50 og 400 skal varigheden lægges i tidsvariablen for gang, og er den lig med eller over 400, skal den lægges over i tidsvariablen for løb.\\
Registreres gyroskopets ID, som den første byte, påbegyndes databehandlingen for pakken bestående af anden og tredje byte. Denne databehandling foregår i MATLAB, som beskrevet i \secref{sec:algocykel}. Giver resultatet af en værdi over 70\%, overføres fire sekunder til tidsvariablen for cykling til GUIen.\\ 

Hver aktivitet belønnes forskelligt, da cykling og løb har højere intensitetsniveau end gang. Herved stiger pulsen igennem disse aktiviteter, hvilket giver et andet fysiologisk udbytte end ved lav puls. Derfor omregnes tiden til point ved at multiplicere tiden for løb med 3, for cykling med 2 og for gang med 1.\\ 
Pointene visualiseres i GUI ud for den enkelte aktivitet, hvorved barnet kan se, hvor mange point vedkommende har opnået ved udførsel af hver aktivitet igennem en hel dag. For at give barnet bedre visualisering af udført aktivitet over en periode summeres pointene for hver dag og plottes i en graf. Pointene vises grafisk via tre søjler for hver aktivitet i individuelle farver, hvormed barnet yderligere kan se, hvor stor en del hver aktivitet har udgjort af dagens totale point.  
\begin{figure}[H]
	\centering
	\includegraphics[scale=0.4]{figures/cDesign/pseudo_GUI.png}
	\caption{På figuren ses et flowchart, der gennemgår hvorledes resultaterne fra de forskellige algoritmer behandles af GUI.}
	\label{fig:GUI}
\end{figure}

\subsection{Implementering}
GUI implementeres ved at skabe en figur i MATLAB, hvori brugerens tid og point printes i en tabel for hver aktivitet. Derudover fremkommer en tabel, som visualiserer dagens samlede aktiviteter. Hertil plottes et søjlediagram over brugerens point for dagen, som plottes i tre forskellige søjler og farver alt efter aktiviteten. %Dette plottes for tre dages aktivitet således, at brugeren kan følge sin progression. 
%GUI implementeres ved at anvende MATLABs funktion Graphical User Interface Design Environment (GUIDE). GUIDE er en funktion, der gør det muligt at lave en specifik brugerflade med indbyggede funktioner.

Programmet starter idet der trykkes 'run' i MATLAB, hvorved indhentning af data fra mikrokontrolleren begynder. Programmet benytter, ved hjælp af if løkker, IDet for data til at adskille accelerometerdata og gyroskopdata. Er den første byte 2 i det indhentede array, registreres mikrokontrollerens data som værende accelerometerdata. Den anden og tredje byte i arrayet er en repræsentation af en pakke for hvor mange samples der er mellem hvert peak. Denne skal omregnes til minutter og lægges over i tidsvariablen for den pågældende aktivitet. Omregning udføres som det ses i \eqref{eq:tidsvariabel}. Den fjerde og femte byte angiver peakværdien, hvilken indikerer om aktiviteten er gang eller løb. \\
Er den første byte i arrayet 3, registreres mikrokontrollerens data som værende gyroskopdata. Den anden og tredje byte repræsenterer en pakke i arrayet, som lægges over i et nyt array, der behandles som beskrevet i \secref{sec:algocykel}. En tidsvariabel på fire sekunder lægges derefter over i cykling, hvis databehandlingen viser at energien omkring den maksimale peak summeret fra $\pm$1~Hz er over 70\%.  
%Der indsættes en grøn toggle button med teksten START, og når der trykkes på knappen starter programmet. Herefter bliver knappen rød, og teksten ændres til STOP. Når programmet starter indhentes data fra mikrokontrolleren. \\
%Er den første værdi 2, registreres mikrokontrollerens array som værende data fra accelerometeret. Den anden værdi i arrayet er en repræsentation af, hvor mange samples der er mellem hvert peak. Denne skal omregnes til minutter og lægges over i tidsvariablen for den pågældende aktivitet. Omregning udføres som ses i \eqref{eq:tidsvariabel}. Den tredje værdi videreføres til løkker, hvori der tjekkes for, om peakværdien indikerer aktiviteten gang eller løb. \\
%Er den første værdi 3, registreres mikrokontrollerens array som værende data fra gyroskopet. Den anden værdi i arrayet lægges derefter over i et nyt array på 4*Fs og behandles som beskrevet i \secref{sec:algocykel}. En tidsvariabel på fire sekunder lægges derfor over i cykling, hvis databehandlingen viser at energien omkring maks peak summeret med værdien fra $\pm$1~Hz er over 70\%. 
\begin{equation}
Tidsvariabel = \frac{Samples}{Samplingsfrekvens} \cdot 0,016667
\label{eq:tidsvariabel}
\end{equation}
Tidsvariablen fra \eqref{eq:tidsvariabel} benyttes til at udregne point for aktiviteterne. Dette gøres ved at multiplicere med de tidligere nævnte værdier opnået som følge af aktivitetstypen. Dette ses i \eqref{eq:pointvariabel}. 
\begin{equation}
Pointvariabel = Tidsvariabel \cdot Aktivitetspoint
\label{eq:pointvariabel}
\end{equation}
Tidsvariablen og pointvariablen for de enkelte aktiviteter lægges over i forskellige static text felter. Disse tilhører henholdsvis point og tid for de tre forskellige aktivitetstyper. Ydermere er der to forskellige static text, hvor der i den ene samles tidsvariablerne for hele dagen, og i den anden samles point opnået gennem hele dagen. \\
Der implementeres et søjlediagram med faste axis værdier, hvor data plottes. I denne samles point fra gang, løb og cykling, som plottes ved siden af hinanden med forskellige farver. Dette gør det muligt at se, hvilke aktiviteter der er udført, og hvor mange point de summeret giver for en dag. I programmet er der aktiveret en timer, som gør det muligt at skifte til en ny dag efter 24 timer. Ved begyndelse på en ny dag nulstilles alle variabler, og der plottes i den næste dag.\\ 
Afslutningsvist kan programmet stoppes ved at der trykkes 'Q' på tastaturet. Herved stoppes dataindhentningen og optællingen af aktivitetstid samt tilhørende point, hvorfor figuren fryses og derefter lukkes. 

\subsection{Test}
Testen udføres på baggrund af de opstillede krav og tilhørende afvigelser opstillet i \secref{krav_GUI}, som beskriver, at GUIen skal:
\begin{itemize}
	\item Kunne visualisere tidsforbruget og point opnået ved henholdsvis gang, løb og cykling.  
\end{itemize}

Testen udføres ved at indsende kendte værdier, hvoraf hver aktivitet kan testes individuelt i forhold til multiplikation som følge af aktivitetstype. Igennem testen indsendes et datasæt med simuleret input fra algoritmen, som skal ramme tærskelværdien og derved optælles som de forskellige aktiviteter. For gang indsendes [2 473(lowbyte) 473(highbyte) 1050(lowbyte) 1050(highbyte)], for løb indsendes [2 473(lowbyte) 473(highbyte) 1150(lowbyte) 1150(highbyte)] og for cykling indsendes [3 (simuleret\_sinus)(lowbyte) simuleret\_sinus(highbyte) ]. Gang og løb simuleres samtidig med et sekunds delay gennem forskellige arrays fra PSoC. Den simulerede sinus laves i MATLAB, hvorfra det direkte benyttes i GUI. Alt det simulerede data indsendes i MATLAB, hvor det hver har en varighed af 30 sekunder.

Resultatet af GUIs design medfører, at forskellige aktivitetsformer bidrager til en forskellig mængde point grundet multiplikationsfaktoren, hvilket kan ses i \eqref{eq:pointvariabel}. Resultatet af at sende de simulerede data gennem GUI ses i \tabref{test:GUI} og på \figref{fig:GUI2}. Derudover lægges point for de forskellige aktiviteter i et søjlediagram, hvilket ligeledes kan ses på \figref{fig:GUI2}.
\begin{table}[H]
	\centering
	\begin{tabular}{ccc}
		\hline
		\rowcolor[HTML]{C0C0C0} 
		Aktivitet 	& Forventet antal point & Optalt antal point \\ \hline
		Gang 	&  30 & 30  \\ \hline
		Løb 	& 30 & 30 \\ \hline
		Cykling & 30 & 30 \\ \hline
	\end{tabular}
	\caption{I tabellen ses sammenhængen mellem forventede antal point og optalt antal point som resultat af den indsendte simulerede data.}
	\label{test:GUI}
\end{table}\vspace{-.2cm}

\begin{figure}[H]
	\centering
	\includegraphics[scale=0.35]{figures/cDesign/test_GUI.png}
	\caption{På figuren ses et udklip af GUI, hvor en simulering af aktiviteterne gang, løb og cykling visualiseres. Aktiviteternes samlede antal point og udført varighed ses på figuren.}
	\label{fig:GUI2}
\end{figure}\vspace{-.5cm}
Efterfulgt af design, implementering og test af GUI kan det konkluderes, at denne opfylder kravene heraf. GUI er i stand til at visualisere tidsforbruget samt antal opnåede point for alle aktiviteterne. GUI opdaterede kontinuert i testen hver gang et nyt input bliver indsendt, hvoraf kravet vedrørende opdatering af GUI mindst hvert femtende minut ligeledes er opfyldt.