\section{Mikrokontroller}
\textit{Dette afsnit beskriver design, implementering og test af GAP peripheral MCUen som spændingsforsyning til IC og pulssensor. MCUen designes på baggrund af de opstillede krav i \secref{krav_mikro_spaending} vedrørende spændingsforsyning til ICen og pulssensoren.}

%Design, implementering og test af mikrokontrolleren som spændingsforsyning, udføres med hensyn til de krav som er opstillet i \secref{krav_mikro_spaending}. Mikrokontrolleren skal derfor kunne levere en spænding til ICen og pulssensoren som overholder spændingsintervallerne for de pågældende komponeneter.

\subsection{Design}
MCUen, der fungerer som GAP peripheral, er tilkoblet en ekstern spændingsforsyning, da denne er mobil. Systemets pulssensor vil være tilkoblet denne MCU, hvorfor denne komponent vil benytte MCUen som spændingsforsyning. Den skal derfor kunne levere tilstrækkelig spænding til, at pulssensoren er funktionsdygtig. \\
MCUens targetboard har fire pins, hvor det er muligt at tilkoble spænding fra en ekstern spændingsforsyning eller udlede spændingsforsyning til komponenter. Yderligere har MCUen fire pins, hvor ground kan tilkobles. \citep{Semiconductor2016} \\
Systemets pulssensor kræver en spændingsforsyning i intervallet 3~V til 5~V \citep{Murphy2016}.

\subsection{Implementering}
Den eksterne spændingsforsyning tilkobles MCUen og leverer, som påvist i \secref{spaendingsforsyning}, et spændingsoutput på 3,14~V. Dermed bør spændingen til MCUen kunne aktivere enheden til funktionelt niveau. %spændingsoutputtet fra MCUen overholde det spændingsinterval som systemets IC og pulssensor påkræver.\\
Spændingsforsyningen til pulssensoren vil blive forbundet ved brug af række J2 på targetboardet. VDDD og GND fra J2 pinrækken benyttes til at forsyne pulssensoren.
%Ydermere konfigureres det pågældende pins i PSoC Creator, således den pågældende sensor vil blive forbundet med de rigitge pins i forhold til spændingstilkoblingen.

\subsection{Test}
Testen udføres med henhold til de opstillede krav og tilhørende tilladte afvigelser opstillet i \secref{krav_mikro_spaending}. Kravene beskriver, at MCUen skal:
\begin{itemize}
	\item Levere mindst 3~V til maksimalt 5~V til pulssensoren. Der accepteres ikke en spænding under minimumsgrænsen eller over maksimumsgrænsen.
\end{itemize}

Spændingsoutputtet fra MCUen testes ved tilkobling af en ekstern spændingsforsyning til targetboardet, og herefter måles outputspændingen fra henholdsvis pin VDDA og VDDD.  \newline
Testen viser, at spændingen på pin VDDD er 3,11~V. Dermed overholder MCUen spændingsintervallet for pulssensoren. Det kan yderligere antages, at spændingsoutput vil være konstant og overholde kravene, så længe spændingsforsyningen til MCUen overholder de opstillede krav i \secref{krav_spaendingsf}.

%Mikrokontrollerens outputspændinger skal testes, således output pins i J1 og J2 rækken henholdsvis leverer 3,3 V og mellem 3 - 5 V. Resultatet fra testen beskrives i \tabref{tab:design_mikrokonsp}:
%\begin{table}[H]
%	\centering
%	\begin{tabular}{cccc} \hline
%		\rowcolor[HTML]{C0C0C0} 
%		Outputpin & Spænding inden pedometer [V] &  Spænding efter pedometer [V] & \begin{tabular}[c]{@{}c@{}}Afvigelse fra\\ønsket værdi [\%] \end{tabular} [\%] \\ \hline
%		VDDIO & 4,663 & 3,341 & 1,24 \\ \hline
%		VDDD & 4,666 & & 0 \\ \hline
%	\end{tabular}%
%	\caption{I tabellen ses det, at multimetret nedsætter spændingen, som skal forsyne IC'en. Derudover ses det, at spændingsforsyningen fra MCU'en derved overholder kravene.}
%	\label{tab:design_mikrokonsp}
%\end{table}\vspace{-0.2cm}
%MCU'ens spændingsforsyning til IC'en er hermed justeret ved hjælp af et pedometer, således IC'en i teorien er funktionel. Derudover overholder MCU'ens spænding fra VDDD kravet til pulssensoren. MCU'en er derfor klar til at blive benyttet i det samlede system som spændingsforsyning til IC og pulssensor.