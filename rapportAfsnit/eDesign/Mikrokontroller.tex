\section{Mikrokontroller}
\textit{I dette afsnit beskrives designet, implementeringen og testen af GAP peripheral MCUen som spændingsforsyning til IC og pulssensor. MCUen designes på baggrund af de opstillede krav i \secref{krav_mikro_spaending} vedrørende spændingsintervaller til forsyning af ICen og pulssensoren.}

%Design, implementering og test af mikrokontrolleren som spændingsforsyning, udføres med hensyn til de krav som er opstillet i \secref{krav_mikro_spaending}. Mikrokontrolleren skal derfor kunne levere en spænding til ICen og pulssensoren som overholder spændingsintervallerne for de pågældende komponeneter.

\subsection{Design}
MCUen som fungerer som GAP peripheral er tilkoblet en ekstern spændingsforsyning, da denne er mobil. Systemets IC og pulssensor vil være tilkoblet denne MCU, hvorfor disse komponenter vil benytte MCUen som spændingsforsyning, den skal derfor kunne levere tilstrækkelig spænding til at disse er funktionsdygtige. \\
MCUens targetboard har fire pins hvor det er muligt at tilkoble spænding fra en ekstern spændingsforsyning eller udlede spændingsforsyning til komponenter. Yderligere har MCUen fire pins hvor ground kan tilkobles. \citep{Semiconductor2016}

Systemets IC indeholder et accelerometer og et gyroskop, hvoraf hele enheden påkræver en spændingsforsyning på 1,9~V til 3,6~V \citep{Jimb02016}.
Ydermere kræver pulssensoren, som også er tilkoblet MCUen, en spændingsforsyning i intervallet 3~V til 5~V \citep{Murphy2016}.

\subsection{Implementering}
Den eksterne spændingsforsyning som tilkobles MCUen leverer, som påvist i \secref{spaendingsforsyning}, et spændingsoutput på 3,14~V. Dermed bør spændingsoutputtet fra MCUen overholde det spændingsinterval som systemets IC og pulssensor påkræver.\\
Spændingsforsyningen til pulssensoren og ICen vil blive forbundet ved brug af række J2 på targetboardet. VDDA og GND benyttes til at forsyne ICen, mens VDDD og GND fra J2 pinrækken benyttes til pulssensoren.
%Ydermere konfigureres det pågældende pins i PSoC Creator, således den pågældende sensor vil blive forbundet med de rigitge pins i forhold til spændingstilkoblingen.

\subsection{Test}
Spændingsoutputtet fra MCUen testes ved tilkobling af en ekstern spændingsforsyning til targetboardet, på henholdsvis pin VDDA og VDDD.  \newline
Testen viser, at spændingen på pin VDDA er 3,12~V og på pin VDDD er 3,11~V. Dermed overholder MCUen spændingsintervallerne for de benyttede komponenter. Det kan yderligere antages, at spændingsoutput vil være konstant og overholde kravene sålænge spændingsforsyningen til MCUen overholder de opstillede krav i \secref{krav_spaendingsf}.

%Mikrokontrollerens outputspændinger skal testes, således output pins i J1 og J2 rækken henholdsvis leverer 3,3 V og mellem 3 - 5 V. Resultatet fra testen beskrives i \tabref{tab:design_mikrokonsp}:
%\begin{table}[H]
%	\centering
%	\begin{tabular}{cccc} \hline
%		\rowcolor[HTML]{C0C0C0} 
%		Outputpin & Spænding inden pedometer [V] &  Spænding efter pedometer [V] & \begin{tabular}[c]{@{}c@{}}Afvigelse fra\\ønsket værdi [\%] \end{tabular} [\%] \\ \hline
%		VDDIO & 4,663 & 3,341 & 1,24 \\ \hline
%		VDDD & 4,666 & & 0 \\ \hline
%	\end{tabular}%
%	\caption{I tabellen ses det, at multimetret nedsætter spændingen, som skal forsyne IC'en. Derudover ses det, at spændingsforsyningen fra MCU'en derved overholder kravene.}
%	\label{tab:design_mikrokonsp}
%\end{table}\vspace{-0.2cm}
%MCU'ens spændingsforsyning til IC'en er hermed justeret ved hjælp af et pedometer, således IC'en i teorien er funktionel. Derudover overholder MCU'ens spænding fra VDDD kravet til pulssensoren. MCU'en er derfor klar til at blive benyttet i det samlede system som spændingsforsyning til IC og pulssensor.