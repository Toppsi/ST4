\section{Analog til digital konvertering} \label{adc_design_impl}
\textit{Dette afsnit beskriver design, implementering og test af ADCen. ADCen designes og implementeres med henhold til teoretiske aspekter samt de opstillede krav. Afslutningsvis testes ADCens virken i forhold til kravene opstillet i \secref{krav_adc}.}

\subsection{Design}
Systemets IC har en ADC indbygget i sit kredsløb, hvorfor ADCen tilhørende denne blok er designet, implementeret og testet i \secref{sec_design_LSM9DS1}.\\
Pulssensoren kræver en AD konvertering for at kunne blive behandlet som digitalt data. Derfor benyttes MCUens 12 bits SAR ADC til at konvertere det analoge pulssignal til et digitalt signal. \\
Den benyttede ADC er designet i PSoC Creator således, at denne sender data ud fra ADCen så snart en sample er konverteret. Der benyttes derfor en kode til at undersøge tilstanden for ADCen og dermed om der er tilgængelige digitale samples klar. 

Det fremgår af \secref{krav_adc}, at ADCen skal have en samplingsfrekvens på mindst 35~Hz, for at kunne gengive et repræsentativt digitalt signal for pulssensoren. For at kunne konfigurere ADCen til at have den pågældende samplingsfrekvens, da skal varighed mellem hver samples, ved en frekvens på 35~Hz, bestemmes:
\begin{equation}
\frac{1~sekund}{35~Hz} = 0,0286~sekunder
\label{eq:adc_puls}
\end{equation}
Det fremgår af \eqref{eq:adc_puls}, at der er 0,0286~sekunder mellem hver sample ved en frekvens på 35~Hz. 

\subsection{Implementering}
Pulssensorens analoge signal, skal gennemgå en AD konvertering i MCUen. Topdesignet for ADCen indebærer derfor en analog komponent i form af en SAR ADC, som yderligere er placeret i PSoC 4200M \citep{Murphy2016}. AD konverteringen udføres ved at koble pulssensorens output til ADCens input. Topdesignet for ADCen i PSOC 4200M bliver derfor forbundet med en analog inputpin, som konfigureres til at være tilhørende outputsignalet fra pulssensoren. ADCen konverterer derfor hver sample fra den optagede kanal, og når ADCen har konverteret data klar, kan dette blive benyttet af pågældende algoritmer. \\
Yderligere konfigureres samplingsfrekensen for ADCen, med hensyn til den tid mellem hver sample som er bestemt i \eqref{eq:adc_puls}. Derfor skal konverteringstiden i Topdesign for ADC konfigureres til 0,0286~sekunder. Dette vil dermed konfigurere ADCen til at have en samplingsfrekvens på 35~Hz.

\subsection{Test}
ADCen testes i to omgange for at undersøge, om samplingsfrekvensen på inputkanalen sampler pulssensoren korrekt. Derudover testes ADCen for, om den konverterer signalet repræsentativt i forhold til LSB og afvigelser. \\
Testen udføres med henhold til de opstillede krav og tilhørende tilladte afvigelser opstillet i \secref{krav_adc}. Kravene beskriver, at ADCen skal:
\begin{enumerate}
	\item Sample pulssensorens output med mindst 35~Hz. Der accepteres ikke en samplingsfrekvens under 35~Hz. 
	\item Repræsentere det analoge signal med maksimalt 5\% afvigelse. 
\end{enumerate}

ADCen er konfigureret således, at denne sender samples afsted, så snart disse er klar. Det vil derfor være muligt at oprette en variabel, som står og tæller op, hver gang der er data klar fra ADCen. Ved at printe denne variabel vil det være muligt at se, hvor mange gange der har været data klar fra ADCen over et givent tidsinterval. Herved vil den respektive frekvens kunne bestemmes, med henhold til den konfigurerede frekvens på 35~Hz. \\
Den første test af ADCen undersøger derfor, hvilken frekvens som ADCen har data klar med. Denne værdi er svarende til samplingsfrekvensen for ADCen.\\
Der oprettes en variabel af størrelsen int16, som skal tælle op hver gang ADCen har data klar. Idet variablen har den valgte størrelse, vil denne kunne tælle op til $2^{16} = 65.536$. Samtidig er ADCen konfigureret til en samplingsfrekvens på 35~Hz, hvormed der er 35 samples per sekund. Hvis testen har en varighed af 1 minut, vil ADCen antageligt have konverteret følgende antal samples: $35~samples \cdot 60~sekunder = 2100~samples$. Dette antal samples vil ydermere ikke overskride variablens størrelse, idet denne er 16 bit.

Testen viser, at ved en varighed på 60 sekunder, da har ADCen konverteret 2134 samples. Dette betyder, at den egentlige samplingsfrekvens har en værdi af:
\begin{equation}
\frac{2134~samples}{60~sekunder} = 35,6~Hz 
\end{equation}
Testen påviser, at ADCen har en egentlig samplingsfrekvens på 35,6~Hz. Jævnfør kravene for samplingsfrekvensen for ADcen, da accepteres en samplingsfrekvens højere end 35~Hz. Derfor påviser testen, at den konfigurerede samplingsfrekvens for ADCen kan accepteres. 

Yderligere testes ADCen for korrekt konvertering. Dette gøres ved brug af en funktionsgenerator og et oscilloskop. MCUen indstilles til at sample 1 single ended kanal men en frekvens på 200~Hz med inputkanalen i P2[0]. Funktionsgeneratoren kobles til dette indput samt ground på MCUen og indstilles til at indsende et sinussignal med 10 Hz og 4 Vpp. Herefter påbegyndes dataopsamling ved hjælp af Matlab, som visualiserer inputsignalet. Udover denne datavisualisering, koples et oscilloskop til samme output fra funktionsgeneratoren. Oscilloskopets målte indput anses for værende det korrekte output fra funktionsgeneratoren. Resultatet fra denne test ses i \tabref{tab:ADC_test}.
\begin{table}[H]
	\centering
	\begin{tabular}{cccc} \hline
		\rowcolor[HTML]{C0C0C0} 
		Indstillet amplitude {[}Vpp{]}  & \begin{tabular}[c]{@{}c@{}} Målte amplitude\\fra oscilloskop {[}Vpp{]} \end{tabular} & Målte amplitude\\fra MCU ADC {[}Vpp{]} & \begin{tabular}[c]{@{}c@{}} Afvigelse \\ {[}\%{]} \end{tabular} \\ \hline
		4 & 4,001 & 3,993 & 0,20 \%\\ \hline
	\end{tabular}%
	\caption{I tabellen ses resultatet fra testen af MCUens AD konvertering.}
	\label{tab:ADC_test}
\end{table}\vspace{-0.2cm}
Der ses i \tabref{tab:ADC_test}, at MCUens ADC repræsenterer den analoge inputsignal med 0,20\% afvigelse. Dette overholder kravet på maks 5\%, hvorfor ADCen i MCUen accepteres til videre implementering.