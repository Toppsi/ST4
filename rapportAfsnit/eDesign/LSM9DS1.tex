\section{LSM9DS1}
Der benyttes en IC (LSM9DS1), som både indeholder magnometer, gyroskop og accelerometer. Det er muligt at indstille accelerometeret til $\pm$1, 4, 8 eller 16 g. Gyroskopet kan måle $\pm$245, 500 eller 2000 grader per sekund, og magnometeret kan måle $\pm$4, 8, 12 eller 16 G.\citep{Jimb02016} \newline
LSM9DS1 har ni frihedsgrader, hvormed det måler i x-, y- og z-aksen for både magnometeret, gyroskopet og accelerometeret, hvilket kan ses på \figref{vores_IC}. Akserne for gyroskopet og accelerometeret internt følger højrehåndsreglen, mens magnometerets x- og y-akse er flippet.\citep{Jimb02016}

\begin{figure}[H]
	\centering
	\includegraphics[scale=0.6]{figures/cDesign/LSM9DS1.png}
	\caption{På figuren ses akserne fra IC'en LSM9DS1, hvor magnometerets (til venstre) akse i realiteten er flippet i forhold til gyroskopet (i midten) og accelerometeret (til højre).\citep{Jimb02016}}
	\label{fig:sensor_placering}
\end{figure}

LSM9DS1 er valgt, da den både indeholder et gyroskop og et accelerometer, som der er mulige at benytte enkeltvis eller samlet. Da gyroskopet bruger 4mA og accelerometeret bruger 600\textmuA, er det væsentligt at kunne sætte gyroskopet i sleepmode, når dette ikke benyttes. 