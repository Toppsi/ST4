\section{Pulssensor}\label{sec_design_puls}
\textit{I dette afsnit beskrives designet, implementeringen og testen af den valgte pulssensor.}

\subsection{Design}
Pulssensoren SEN-11574 er valgt til dette projekt, da det er en optisk pulssensor og er derfor mere sikker for brugeren, som beskrevet i \secref{sec:pulssensor}. Derudover er en optisk pulssensor mere alsidig i forhold til placering, da den blot skal placeres over en arterie. En elektrisk pulssensor har risiko for at opfange muskelkontraktion frem for puls, hvis den ikke placeres korrekt. Pulssensoren SEN-11574 kræver 3 til 5 V for at være funktionel og forbruger 4 mA ved en forsyning på 5V. På sensorens board findes et aktivt filter\fxnote{Et aktivt filter er en type af analog elektronisk filter, der anvender aktive bestanddele, såsom en forstærker.} samt en forstærker, som tilsammen øger amplituden for pulsbølgen og normaliserer signalet omkring et referencepunkt, hvilket fjerne al DC spænding i signalet. Herved fremkommer en tydelig pulsbølge direkte fra sensoren, som umiddelbart ikke kræver behandling af signalet for at kunne detektere. \citep{Murphy2016,Murphy2016_sensor}

\subsection{Implementering}
Pulssensoren skal benyttes til at beskrive intensiteten af aktiviteten, som beskrevet i \secref{subsub:ak_int}. Herudfra kan effekten af aktiviteten bestemmes, hvilket kan have en motiverende faktor for brugeren. \\
Sensoren har 3 pins til henholdsvis spændingsforsyning, ground og outputsignal, som skal kobles med MCU'en. Outputsignalets pin skal designes i programmet PSoC Creater, således MCU'en kan finde ud af at modtage signalet fra denne pin. Dette gøres ved at indsætte henholdsvis en UART serie kommunikationsblok (SCB) og SAR ADC i topdesignet. UART'en bruges for at sensoren og MCU'en kan kommunikere og er tilpasset formålet fra standart. Puls data skal igennem en ADC, da det er et analog signal, der fortsætter så længe sensoren modtager en spænding. Dennes design skal derfor tilpasses pulssensoren, hvilket gøres ved at vælge antallet af kanaler og indstille sampleraten. I dette tilfælde skal der benyttes én kanal, som er single ended, og ADC'ens samplingsfrekvens sættes til 35 SPS. ADC'ens samplingsfrekvens vælges ud fra, at en persons makspuls kan beregnes med 220 - alder \citep{CooperBlair2005}. Ud fra målgruppen vurderes det, at makspulsen altså er cirka 210, som svarer til tre et halvt slag pr sekund. Det vurderes at ti samples per puls giver et præsentabelt resultat, hvilket derved giver 35 SPS.\\
Efter UART og ADC er konfigureret i topdesignet, skal de korrekte pins indstilles i pinopsætning. UART pins får typisk tildelte interne pins, hvorimod ADC'ens inputpin skal indstilles til den plads, som outputtet fra sensoren er. I dette tilfælde er det pin 2.0. Når overstående design er implementeret, indstilles MCU'en til at debugge via programmet PSoC Creater. \\

Igennem et matlab program er det muligt at visualisere pulsen ved hjælp af en GUI. I det samlede system skal denne GUI ikke være tilgængelig her i processen, da GAB peripheral modtager pulsdata og sender dette videre sammen med IC data til GAB central, som skal visualisere data i en endelig GUI.\\
I matlabscriptet vælges den pågældende port for MCU'en i computeren, således PC'en kan registrere hvor det visualiserede data skal komme fra. Herefter startes scriptet og en figur tvinges i front med pulsen, hvor et bestemt antal sekunders vindue visualiseres

\subsection{Test}
Pulssensoren skal testes for, om den kan opfange brugerens puls under fysisk aktivitet uden ukorrekt eller ikke tydelig puls. Dette gøres ved, at der optages to målinger henholdsvis med og uden fysisk aktivitet, hvor sensoren placeres på den distale phalanges på højre hånds pegefinger. Ud fra matlab scriptet og tilhørende optagede data vurderes det, om pulsen visualiseres korrekt, således den korrekte puls kan kalkuleres.\\
Den fysiske aktivitet bestod af gang på et løbebånd. Hver måling optages i 10 sekunder, hvilket ses på \figref{fig:puls_m_u}.\\
%\begin{figure}[H]
%	\centering
%	\includegraphics[scale=0.6]{figures/cDesign/ToPulser.png}
%	\caption{På figuren ses to plots for hver test af pulssensoren. På figur 1 til venstre står forsøgspersonen helt stille i 10 sekunder, hvorimod vedkommende går i en subjektiv vurdering af normal gang på figur 2 til højre. Der ses en tydelig puls på figur 1, hvorimod pulsen ikke fremgår på figur 2.}
%	\label{fig:puls_m_u}
%\end{figure}
Matlab scriptet udregnede

% Test af krav


% Grimme hjemmeside \citep{Murphy2016}
% \cite{Murphy2016_sensor}