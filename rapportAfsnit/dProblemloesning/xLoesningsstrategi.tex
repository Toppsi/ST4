\textit{I forbindelse med besvarelse af problemformuleringen udarbejdes en løsningsstrategi. Hertil opstilles en række funktionelle krav, som systemet skal overholde for at kunne besvare problemformuleringen. Der foretages en bevægelsesanalyse for gang, løb og cykling med henblik på at definere aktiviteternes karakteristika for at muliggøre senere algoritmedesign. Efterfølgende præcenteres teori for hardware og software, der sammen med bevægelsesanalysen og et pilotforsøg danner grundlag for udarbejdelse af specifikke krav.}

\section{Løsningsstrategi}
\textit{Dette afsnit beskriver en strategi for, hvordan projektet vil forsøge at besvare problemformuleringen.}

For at besvare det omtalte problem i \secref{Problemformulering} udarbejdes en prototype af en aktivitetsmåler, som kan detektere og adskille aktivitetsformerne gang, løb og cykling samt registrere intensiteten af den udførte aktivitet. Disse aktivitetsformer adskilles, idet barnets fysiologiske udbytte afhænger af aktiviteten samt intensiteten og varigheden heraf, som beskrevet i \secref{subsub:ak_int}. Barnet får feedback på det opsamlede data, således det er muligt at følge med i, hvilke aktiviteter der udføres samt varigheden heraf. Dette gøres for, at aktivitetsmåleren har potentialet til at motivere og dermed reducere antallet af fysisk inaktive børn i aldersgruppen 9-12 år.

Sensorerne skal i denne forbindelse undersøges med henblik på detektering og adskillelse af gang, løb og cykling. Hertil skal der udarbejdes en bevægelsesanalyse for aktiviteterne således, at forskellige bevægelsesmønstre kan beskrives for at kunne adskille disse gennem algoritmer i software.\\
Børnene skal motiveres til at være aktive med et højere intensitetsniveau for at opnå det største udbytte af deres præstation, som visualiseres gennem en brugerflade. 


