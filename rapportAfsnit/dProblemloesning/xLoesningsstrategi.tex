\textit{I forbindelse med løsning af problemformuleringen, udarbejdes en løsningsstrategi om, hvorledes problemet vil blive løst. Hertil opstilles en række funktionelle krav, som systemet skal overholde for at kunne løse problemet. Der foretages en bevægelsesanalyse med henblik på at definere karakteristika for de forskellige aktiviteter. Efterfølgende præcenteres teori for hardware og software, der sammen med bevægelsesanalysen og et pilotforsøg danner grundlag for valg af specifikke krav.}

\section{Løsningsstrategi}
\textit{Dette afsnit beskriver en strategi for, hvordan projektet vil forsøge at løse problemformuleringen.}

%Inaktivitet hos børn i aldersgruppen 9-12 år er et stigende problem, med helbredsmæssige og socioøkonomiske følger. Derfor ønskes en løsning i form af en aktivitetsmåler, som motiverer inaktive børn i denne aldersgruppe til en mere aktiv hverdag som nævnt i \secref{Problemformulering}. 
For at løse det omtalte problem i \secref{Problemformulering} udarbejdes en aktivitetsmåler, som kan detektere og adskille aktivitetsformerne gang, løb og cykling, samt registrere intensitet. Grunden til at disse aktivitetsformer adskilles, er grundet kroppens fysiologiske udbytte som er afhængig af aktiviteten, intensiteten samt varigheden, som beskrevet i \secref{subsub:ak_int}. Dataet gives som feedback tilbarnet, så det er muligt at følge med i hvilke aktiviteter der er blevet registreret og i hvor lang tid de er udført. Dette gøres således at aktivitetsmåleren har potentialet til at motivere børnene og dermed reducere antallet af fysisk inaktive børn i aldersgruppen 9-12 år.

Sensorerne, accelerometer og gyroskop, skal i denne forbindelse undersøges i forhold til hvilke, der er ideelle at benytte. Derudover skal der udarbejdes en bevægelsesanalyse for de aktuelle aktiviteter, således forskellige bevægelsesmønstre kan beskrives med henblik på at kunne adskille disse gennem algoritmer i software. Aktivitetsmåleren skal derudover kunne registrere intensiteten af den givne aktivitet, da det fysiologiske udbytte afhænger af intensiteten. Børnene skal derfor motiveres til at være aktive med et højere intensitetsniveau for at opnå det største udbytte af deres præstation, hvorfor præstationen skal visualiseres gennem en brugerflade, hvorved de kan følge deres progression.
%Ligeledes skal aktivitetsmåleren kunne monteres uden gene, for at undgå at dette ellers kan være en demotiverende faktor, som kan føre til at børnene fravælger at benytte den.\fxnote{Herved opfyldes de essentielle succeskriterier for aktivitetsmålere til børn i denne aldersgruppe som beskrevet i \secref{succeskrav}.} 





\textit{I forbindelse med løsning af problemformuleringen, udarbejdes en løsningsstrategi om, hvorledes problemet vil blive løst. Hertil opstilles en række funktionelle krav, som systemet skal overholde for at kunne løse problemet. Der foretages en bevægelsesanalyse med henblik på at definere karakteristika for de forskellige aktiviteter. Efterfølgende præcenteres teori for hardware og software, der sammen med bevægelsesanalysen og et pilotforsøg danner grundlag for valg af specifikke krav.}


