\section{Løsningsstrategi}

%Den overordnede metode til at svare på problemformuleringen → altså hvad er hele formålet med problemløsningen
%Inddrage de krav som opstilles til aktivitetsmålerne
%For at vi kan løse problemformuleringen som vi skriver ovenfor, da skal vi have opstillet nogle funktionelle krav for hele systemet (leder op til funktionelle krav)

%Problemformulering: Hvordan kan en aktivitetsmåler udvikles således at fysisk inaktive børn i aldersgruppen 9-12 år, motiveres til en mere aktiv livsstil?
%

%En aktivitetsmåler til børn i denne aldersgruppe skal essentielt kunne:
%\begin{itemize}
%	\item Detektere gang, løb og cykling.
%	\item Registrere aktivitetens intensitet.
%	\item Motivere inaktive børn.
%	\item Monteres uden gene.
%\end{itemize}


Da inaktivitet hos børn i aldersgruppen 9-12 år er et stigende problem, med helbredsmæssige og socioøkonomiske følger, ønskes en løsning i form af en aktivitetsmåler som motiverer inaktive børn i denne aldersgruppe til en mere aktiv hverdag som nævnt i \secref{problemformulering}. 
Måden hvorpå dette løses er igennem udarbejdelse af en aktivitetsmåler, som ved hjælp af sensorer, kan detektere og adskille aktiviteterne gang, løb og cykling. Hvorfor der vil blive undersøgt hvilke sensorer som er ideelle at benytte samt bevægelsesanalyse for de aktuelle aktiviteter, hvorledes forskellige bevægelsesmønstre beskrives med henblik på at kunne adskille disse gennem algoritmer. Herigennem skal aktivitetsmåleren også kunne registrere intensiteten af den givne aktivitet, da det fysiologiske udbytte afhænger af intensiteten, hvorledes børnene skal motiveres til at være aktive med et højere intensitetsniveau for at opnå det største udbytte af deres præstation. 
Ligeledes skal aktivitetsmåleren kunne monteres uden gene, for at undgå at dette kan være en demotiverende faktor, som kan føre til at børnene fravælger at benytte den.\fxnote{Herved opfyldes de essentielle succeskriterier for aktivitetsmålere til børn i denne aldersgruppe som beskrevet i \secref{succeskrav}.} 

For at sikre systemets funktionalitet i forhold til dette løsningsønske, opstilles en række funktionelle krav, og senere en kravspecifikation.  



   