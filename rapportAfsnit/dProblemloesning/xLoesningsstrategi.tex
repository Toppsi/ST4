\section{Løsningsstrategi}
\textit{Dette afsnit hvordan projektet vil forsøge at løse ovenstående problemet...}

%Inaktivitet hos børn i aldersgruppen 9-12 år er et stigende problem, med helbredsmæssige og socioøkonomiske følger. Derfor ønskes en løsning i form af en aktivitetsmåler, som motiverer inaktive børn i denne aldersgruppe til en mere aktiv hverdag som nævnt i \secref{Problemformulering}. 
For at løse det omtalte problem i \secref{Problemformulering} udarbejdes en aktivitetsmåler, som kan detektere og adskille aktivitetsformerne gang, løb og cykling, samt registrere intensitet. Grunden til at disse aktivitetsformer adskilles, er fordi kroppens fysiologiske udbytte af aktiviteten, afhænger af intensiteten, som beskrevet i \secref{subsub:ak_int}. Dette gøres således at aktivitetsmåleren har potentialet til at reducere antallet af fysisk inaktive børn i aldersgruppen 9-12 år.

Sensorerne, accelerometer og gyroskop, skal i denne forbindelse undersøges i forhold til hvilke, der er ideelle at benytte. Derudover skal der udarbejdes en bevægelsesanalyse for de aktuelle aktiviteter, således forskellige bevægelsesmønstre kan beskrives med henblik på at kunne adskille disse gennem algoritmer i software. Aktivitetsmåleren skal derudover kunne registrere intensiteten af den givne aktivitet, da det fysiologiske udbytte afhænger af intensiteten. Børnene skal derfor motiveres til at være aktive med et højere intensitetsniveau for at opnå det største udbytte af deres præstation, hvorfor præstationen skal visualiseres gennem en brugerflade, hvorved de kan følge deres progression.
%Ligeledes skal aktivitetsmåleren kunne monteres uden gene, for at undgå at dette ellers kan være en demotiverende faktor, som kan føre til at børnene fravælger at benytte den.\fxnote{Herved opfyldes de essentielle succeskriterier for aktivitetsmålere til børn i denne aldersgruppe som beskrevet i \secref{succeskrav}.} 

For at sikre systemets funktionalitet i forhold til disse løsningsønsker, opstilles en række funktionelle krav for hele systemet og senere en kravspecifikationer for hver blok, som vil ligge til grund for design, implementering og test. 
%Hver bloks krav inspireres fra litteratur, pilotforsøg eller komponenternes karakteristika. 
%Disse krav vil ligge til grund for design, implementering og test af hver blok, som til sidst kan samles og det samlede system kan derved testes.




