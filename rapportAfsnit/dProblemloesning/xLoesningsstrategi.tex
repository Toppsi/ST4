\section{Løsningsstrategi}

%Den overordnede metode til at svare på problemformuleringen → altså hvad er hele formålet med problemløsningen
%- Hvordan måles aktivitet - gennem sensorer
%- Hvordan motiveres de - gennem teknologi, uden gene, det skal være nemt --> den skal opfylde succeskriterierne 
%Inddrage de krav som opstilles til aktivitetsmålerne
%For at vi kan løse problemformuleringen som vi skriver ovenfor, da skal vi have opstillet nogle funktionelle krav for hele systemet (leder op til funktionelle krav)

%Problemformulering: Hvordan kan en aktivitetsmåler udvikles således at fysisk inaktive børn i aldersgruppen 9-12 år, motiveres til en mere aktiv livsstil?
%

%En aktivitetsmåler til børn i denne aldersgruppe skal essentielt kunne:
%\begin{itemize}
%	\item Detektere gang, løb og cykling.
%	\item Registrere aktivitetens intensitet.
%	\item Motivere inaktive børn.
%	\item Monteres uden gene.
%\end{itemize}


%Inaktivitet hos børn i aldersgruppen 9-12 år er et stigende problem, med helbredsmæssige og socioøkonomiske følger. Derfor ønskes en løsning i form af en aktivitetsmåler, som motiverer inaktive børn i denne aldersgruppe til en mere aktiv hverdag som nævnt i \secref{Problemformulering}. 
For at løse det omtalte problem i \secref{Problemformulering} udarbejdes en aktivitetsmåler, som kan detektere og adskille gang, løb og cykling. Denne skal motivere inaktive børn til mere fysisk hverdag, hvorfor systemet skal have en visuel brugerflade og sensorerne skal være præcise. Derfor skal forskellige sensorer undersøges i forhold til hvilke, som er ideelle at benytte. Derudover skal der udarbejdes en bevægelsesanalyse for de aktuelle aktiviteter, således forskellige bevægelsesmønstre kan beskrives med henblik på at kunne adskille disse gennem algoritmer i softwaren. Aktivitetsmåleren skal derudover også kunne registrere intensiteten af den givne aktivitet, da det fysiologiske udbytte afhænger af intensiteten, og børnene skal motiveres til at være aktive med et højere intensitetsniveau for at opnå det største udbytte af deres præstation. 
%Ligeledes skal aktivitetsmåleren kunne monteres uden gene, for at undgå at dette ellers kan være en demotiverende faktor, som kan føre til at børnene fravælger at benytte den.\fxnote{Herved opfyldes de essentielle succeskriterier for aktivitetsmålere til børn i denne aldersgruppe som beskrevet i \secref{succeskrav}.} 

For at sikre systemets funktionalitet i forhold til disse løsningsønsker, opstilles en række funktionelle krav for hele systemet og senere en kravspecifikationer for hver blok. Hver bloks krav inspireres fra litteratur, pilotforsøg eller komponenternes karakteristika. Disse krav vil ligge til grund for design, implementering og test af hver blok, som til sidst kan samles og det samlede system kan derved testes.