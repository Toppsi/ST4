\section{Løsningsstrategi}

%Den overordnede metode til at svare på problemformuleringen → altså hvad er hele formålet med problemløsningen
%Inddrage de krav som opstilles til aktivitetsmålerne
%For at vi kan løse problemformuleringen som vi skriver ovenfor, da skal vi have opstillet nogle funktionelle krav for hele systemet (leder op til funktionelle krav)

%Projektets definerede målgruppe er fysisk inaktive børn i aldersgruppen 9-12 år. Disse børn er særligt udsatte for fysisk inaktivitet, hvilket i Danmark er et stigende problem. Fysisk inaktivitet har en bred række helbredsmæssige konsekvenser, eksempelvis overvægt. Overvægt kombineret med fysisk inaktivitet forværrer barnets helbredsmæssige tilstand. Øget fysisk aktivitet afhjælper fysisk inaktivitet direkte, men har også andre åbenlyse fordele. Et øget aktivitetsniveau kan afhjælpe og forebygge overvægt, men også bidrage til en øget kognitiv aktivitet. Børn motiveres til handling forskelligt, og den valgte aldersgruppe motiveres særligt igennem spil og leg. Sideløbende med at disse børn motiveres af leg og spil, så er har deres teknologiske tilgang udviklet sig i en grad hvor benyttelsen teknologiske apparater er stødt stigende. Eksisterende teknologiske metoder benytter i dag disse motiverende faktorer, til at opnå et øget aktivitetsniveau, dog opfylder de ikke alle essentielle succeskritriterier, hvilket danner grundlag for forbedring. Det vil dermed være essentielt at undersøge:
%

%Problemformulering: Hvordan kan en aktivitetsmåler udvikles således at fysisk inaktive børn i aldersgruppen 9-12 år, motiveres til en mere aktiv livsstil?
%

Da inaktivitet hos børn i aldersgruppen 9-12 år er et stigende problem, med helbredsmæssige og socioøkonomiske følger, ønskes en løsning i form af en aktivitetsmåler som motiverer inaktive børn i denne aldersgruppe til en mere aktiv hverdag som nævnt i \secref{problemformulering}. 
Måden hvorpå dette løses er gennem udarbejdelsen af en aktivitetsmåler, som ved hjælp af sensorer, kan detektere og adskille aktiviteterne gang, løb og cykling.  
 


