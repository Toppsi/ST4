\section{Kravsspecifikationer}
Formålet med aktivitetsmåleren er, at kunne registrere og adskille aktivitetsformerne; gang, løb og cykling. Aktivitetsmåleren vil dermed indeholde hardware og software bestående af henholdsvis en dataopsamling og digital signalbehandling. Disse elementer skal samlet have et potentiale til at opfylde de opstillede succeskriterier for den optimale aktivitetesmåler, beskrevet i \secref{succeskrav}. \newline
Endvidere vil nedenstående kravsspecifikationer tage udgangspunkt i de opnåede resultater fra de udførte pilotforsøg beskrevet i \appref{PILOT} \fxnote{Sæt reference ind!}.

\subsection{Krav til hardware}
Aktivitetsmålerens hardware består af sensorer, spændingsforsyning og en analog-til-digital konvertering (ADC). Disse elementer benyttes til en signalopsamling, hvoraf signalet efterfølende bliver behandlet i forbindelse med aktivitetsmålerens software.

\subsubsection{Accelerometer}
Et accelerometer kræver en given spænding for at kunne optage data. Sensoren skal være i stand til at optage data ved tilførslen af en DC spænding fra et knapcelle batteri. Der vil blive benyttet et knapcelle batteri, med hensyn til brugervenlig og mobilitet for aktivitetsmåleren.\newline

Arbejdsområdet for et accelerometer er angivet i g, og er derfor påvirkelig overfor den accelerationen som sensoreren udsættes for. Den påvirkning som udøves på sensores er dermed afhængig af flere faktorer såsom vægt, bevægelsens hastighed og bevægelsens mønster. \newline
Pilotforsøget viste en maksimal acceleration på 25,7 g ($\pm$~12,85 g). Denne maksimale acceleration antages derfor som værende den største acceleration, som acceleromteret vil blive påvirket af. Dette skyldes, at pilotforsøget er udført på en forsøgspopulation (n=4) med voksne mennesker, hvilken gennemsnitligt har en højere vægt end målgruppens aldersgruppe. 


\textbf{Krav til accelerometer} \newline 
Accelerometeret skal:
\begin{itemize}
\item Være operativ ved en DC spænding fra knapcelle batteri.
\item Have et arbejdsområde på mindst $\pm$12,85 g.
\end{itemize}

\textbf{SKAL DER VÆRE NOGET OM AT PLACERINGEN SKAL VÆRE VED ANKLEN, JF. PILOTFORSØG}

\subsubsection{Gyroskop} 
Et accelerometer kræver en given spænding for at kunne optage data. Sensoren skal være i stand til at optage data ved tilførslen af en DC spænding fra et knapcelle batteri. Der vil blive benyttet et knapcelle batteri, med hensyn til brugervenlig og mobilitet for aktivitetsmåleren.\newline

Det maksimale arbejdsområde for gyroskopet blev undersøgt i pilotforsøget. Pilotforsøget viste at gyroskopet maksimalt blev udsat for $\pm$160 grader/sekund. Denne værdi blev bestemt for en given frekvens ved cykling, derfor bør gyroskopet have et større arbejdsområde for dermed at tage forbehold for en højere frekvens af omdrejninger på cyklen. 


\textbf{Krav til gyroskop} \newline
Gyroskopet skal:
\begin{itemize}
\item Være operativ ved en DC spænding fra knapcelle batteri.
\item Have et arbejdsområde på mindst $\pm$160 grader/sekund.
\end{itemize}

\textbf{SKAL DER VÆRE NOGET OM AT PLACERINGEN SKAL VÆRE VED ANKLEN, JF. PILOTFORSØG}

\subsubsection{Pulsmåler}
Pulsmåleren skal kunne optage data ved at være forsynet med en DC spænding fra et knapcelle batteri. \newline
Yderligere skal pulsmåleren kunne bestemme brugerens puls, med henblik på at bestemme intensiteten af den pågældende aktivitet.

\textbf{Krav til pulsmåler} \newline
Pulsmåleren skal:
\begin{itemize}
\item Være operativ ved en DC spænding fra knapcelle batteri.
\item Kunne bestemme brugerens puls.
\end{itemize}

\subsubsection{Spændingsforsyning}
%	o Maxwell CR2032 H, 3 V, DC (det der ligger i kassen)
%	o Tjekke op på hvor meget microcontrolleren skal have
%	o Skal kunne give 1,9-3,6 V til breakout boardet (-0,3 til 4,8 V)
%- fngerer ned til 1,9 V

\textbf{Krav til spændingsforsyning} \newline
Spændingsforsyningen skal:
\begin{itemize}
\item Bla bla
\item Bla bla
\end{itemize}



\subsubsection{ADC}
Pilotforsøget undersøgte frekvensområdet for de pågældende aktiviteter, i forhold til de sensorer som er påtænkt at detektere den givne aktivitet.\newline
Accelerometeret skal benyttes til at detektere gang og løb, hvorfor pilotforsøg blev undersøgt med henhold til frekvensområdet for gang og løb for accelerometeret. Pilotforsøget viste, at frekvensområdet for et accelerometer ved gang og løb maksimalt havde en frekvens af XXXX Hz. Ifølge Nyquist skal aktivitetsmålerens ADC derfor have en samlingshastighed som er dobbelt så stor som det maksimale frekvensområde, med henhold til samplingsfrekvensen for accelerometeret.\newline

Gyroskopet vil blive benyttet til at detektere aktiviteten, cykling. Derfor undersøgtes pilotforsøget ydermere med henblik på at bestemme frekvensområdet for gyroskopet ved cykling. Det fremgik heraf, at det maksimale frekvensområde var XXX Hz. Samplingsfrekvensen i ADC'en, for gyroskopet, skal derfor være dobbelt så stor som frekvensområdet med henhold til Nyquist. 


\textbf{Krav til ADC} \newline
ADC'en skal:
\begin{itemize}
\item Sample accelerometerets output med XXX Hz.
\item Sample gyroskopets output med XXX Hz. 
\end{itemize}



\subsection{Krav til software}


Aktivitetsmålerens software skal:
\begin{itemize}
\item Kunne arbejde med DC spænding fra knapcelle batteri.
\item Anvende digitale filtre til filtrering.
\item Detektere og adskille aktiviteterne; gang, løb og cykling. 
\item Trådløs overførsel.
\item Gemme en hel dags aktiviteter. 
\item 
\end{itemize}
 


