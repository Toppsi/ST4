	\section{Kravspecifikationer}\label{Sec:krav}
\textit{I det følgende afsnit opstilles krav samt tolerancer hertil for hver del i det samlede system. Det sikres herved, at hver enhed kan fungere efter hensigten.}

Formålet med aktivitetsmåleren er at kunne registrere og adskille aktivitetsformerne gang, løb og cykling. Aktivitetsmåleren vil dermed indeholde hardware og software, som tilsammen kan opsamle analoge signaler og udføre digital signalbehandling herpå. Det samlede system skal have et potentiale til at opfylde de funktionelle krav for systemet, beskrevet i \secref{funktionellekrav}. Endvidere vil nedenstående kravspecifikationer tage udgangspunkt i de opnåede resultater fra de udførte pilotforsøg, som er beskrevet i \appref{pilot} og kort opsummeret i \secref{opsamling_pilot}.
%
%\subsection{Krav til hardware}
%Aktivitetsmålerens hardware består af to sensorer, spændingsforsyning og en ADC. Disse elementer benyttes til en signalopsamling og -konvertering, hvoraf det digitale signal efterfølgende bliver behandlet af aktivitetsmålerens software.

\subsection{Spændingsforsyning} \label{krav_spaendingsf}
MCUen kræver en spændingsforsyning for at kunne fungere, hvilket enten kan ske igennem USB porten eller tilføres fra en mobil enhed. Spændingsforsyningen skal kunne forsyne MCUen i en hel dag samt være sikkert for brugeren. %Det samlede system skal benytte elektroniske komponenter, hvorfor en spændingsforsyning er nødvendig. Spændingsforsyningen skal tage hensyn til mobilitet samt brugersikkerhed.

\textbf{Krav til spændingsforsyning} \newline 
Spændingsforsyningen skal:
\begin{itemize}
	\item Levere mindst 1,71~V og maks 5,5~V til MCUen\fxnote{Alle mikroprocessorer kræver 1,71-5,5~V for at kunne fungere, selvom der står 3,3-5,5~V i databladet for mikroprocessoren.}. Der accepteres ikke en spænding under minimumsgrænsen eller over maksimumsgrænsen. %en tilstrækkelig spænding til alle systemets aktive komponenter, og må varierer med $\pm$5\%.
	\item Være i stand til at levere denne spænding i mindst 15 timer. Der accepteres ikke, at spændingsforsyningen leverer under 1,71 V eller over 5,5~V i mindre end 15 timer.
	%\item Muliggøre spændingsopsætning af systemet udenom elnettet og	være elektrisk sikkert.
	\item Være mobil og dermed besidde en opsætning udenom elnettet, hvilket gør systemet mere elektrisk sikkert. Der accepteres ikke, at systemet skal kobles til elnettet og derved ikke være mobilt.
\end{itemize}

\subsection{Mikrokontroller} \label{krav_mikro_spaending}
En specifik, udleveret MCU skal benyttes til projektet. Der kan derfor ikke stilles krav til selve hardwaren hertil. Dog skal MCUen fungere som spændingsforsyning til ICen samt pulssensor, hvorfor der skal stilles krav hertil. I \secref{sec_design_LSM9DS1} og \secref{sec_design_puls} beskrives de specifikke sensorer for dette projekt, hvorfor spændingsforsyningen hertil er bestemt.

\textbf{Krav til mikrokontrolleren} \newline 
Mikrokontrolleren skal:
\begin{itemize}
	\item Levere 1,9~V til 3,6~V til ICen. Der accepteres ikke, at pulssensoren modtager en spænding under minimumsgrænsen eller over maksimumsgrænsen.
	\item Levere mellem 3~V og 5~V til pulssensoren. Der accepteres ikke, at pulssensoren modtager en spænding under minimumsgrænsen eller over maksimumsgrænsen.
\end{itemize}

\subsection{Accelerometer}\label{krav:acc}
% Et accelerometer kræver en given spænding for at kunne optage data. Accelerometret i  LSM9DS1 kræver 3,3 V for at være operativt. %Sensoren skal være i stand til at optage data ved tilførslen af en DC spænding med baggrund i spændingsforsyningens krav. Arbejdsområdet for et accelerometer er angivet i g, og er derfor påvirkelig overfor den accelerationen som sensoreren udsættes for. Den påvirkning som udøves på sensoren er dermed afhængig af flere faktorer såsom vægt, bevægelsens hastighed og bevægelsens mønster. \newline
Pilotforsøget viste en maksimal acceleration på +16,95 g og -8,83 g. Den maksimale positive g værdi antages derfor som værende den største acceleration, som accelerometret vil blive påvirket af som prototype. Dog er pilotforsøget udført på en forsøgspopulation (n=4) med voksne mennesker. Det antages derfor, at den gennemsnitlige vægt er større end målgruppens, hvorfor et barn ikke vil kunne påvirke accelerometret med over 16 g. %Jævnfør pilotforsøget blev den optimale placering af sensorer med henblik på målgruppen bestemt. Placeringen af sensorer skal derfor være ud for den laterale malleolus.

\textbf{Krav til accelerometer} \newline 
Accelerometeret skal:
\begin{itemize}
%\item Være operativ ved 3,3 V fra MCUens VDD output spænding.\fxnote{Outputspændingen fra MCUen er omkring 4.8V, men det afhænger måske er, hvilken spænding den får tilført? Ellers skal det reguleres med potentiometer} Der accepteres en afvigelse på +5\%.
\item Have et arbejdsområde på $\pm$16 g. Der accepteres ikke, at accelerometret har et arbejdsområde på under $\pm$16 g.
\item Angive korrekt påvirkning i forhold til g påvirkning. Der accepteres en afvigelse på 5\%.
\end{itemize}

\subsection{Gyroskop} \label{krav:gyro}
%Et gyroskop kræver en given spænding for at kunne optage data. Gyroskopet i  LSM9DS1 kræver 3,3 V for at være operativt. %Sensoren skal være i stand til at optage data ved tilførslen af en DC spænding med baggrund i spændingsforsyningens krav.
Det maksimale arbejdsområde for gyroskopet blev undersøgt i pilotforsøget, der udledte et arbejdsområde på maks 334,69 dps for placering A. Dette blev bestemt for en given frekvens ved cykling, hvorfor gyroskopet bør have et større arbejdsområde for at tage forbehold for en højere frekvens. % af omdrejninger på cyklen . Jævnfør pilotforsøget blev den optimale placering af sensorer med henblik på målgruppen bestemt.

\textbf{Krav til gyroskop} \newline
Gyroskopet skal:
\begin{itemize}
%\item Være operativ ved 3,3 V fra MCUens VDD output spænding.\fxnote{Outputspændingen fra MCUen er omkring 4.8V, men det afhænger måske er, hvilken spænding den fårr tilført? Ellers skal det reguleres med potentiometer} Der accepteres en afvigelse på +5\%.
\item Have et arbejdsområde på mindst 334,69 dps. Der accepteres ikke et arbejdsområde herunder.
\end{itemize}

\subsection{Pulssensor} \label{puls_krav}
En pulssensor kræver en given spænding for at kunne optage data, som tilføres fra MCUen. Sensoren skal være i stand til at optage data ved tilførslen af en DC spænding. % med baggrund i spændingsforsyningens krav. 
Yderligere skal pulssensoren kunne opfange brugerens puls med henblik på at bestemme intensiteten af den pågældende aktivitet.

\textbf{Krav til pulssensor} \newline
Pulsmåleren skal:
\begin{itemize}
%\item Være operativ mellem 3 V og 5 V. Der accepteres ikke, at pulssensoren modtager en spænding under minimumsgrænsen eller over maksimumsgrænsen.\fxnote{MCUen leverer ca. 4,6 V fra sin VDD output}% eller ikke er funktionel i dette spændingsinterval.
\item Kunne opfange brugerens puls uden ukorrekt optagelse og udsving. Der accepteres en afvigelse af BPM på 10\% fra en reference måling.
\end{itemize}

\subsection{Analog til Digital konverter} \label{krav_adc}
%Pilotforsøget undersøgte frekvensområdet for de pågældende aktiviteter, i forhold til de sensorer som er påtænkt til at detektere den givne aktivitet.\newline
Accelerometret i LSM9DS1 skal benyttes til at opfange gang og løb, mens gyroskopet i LSM9DS1 skal benyttes til at opfange cykling. For at begge sensorer skal være i stand til dette, er det essentielt at vide det analoge signals frekvensområde. %Accelerometeret skal benyttes til at detektere gang og løb, hvorfor pilotforsøg blev undersøgt med henhold til frekvensområdet heraf. 
Pilotforsøget viste, at frekvensområdet for signalet ved gang og løb er 45 Hz, når det optages af et accelerometer. Ifølge Nyquist skal aktivitetsmålerens ADC derfor have en samlingshastighed, der er dobbelt så stor som det maksimale frekvensområde, altså 90 Hz \citep{Webster2011}. I praksis benyttes en samplingshastighed ofte, som er ti gange større end det maksimale frekvensområde. Derfor skal ADCen have en samplingshastighed på mindst 450 Hz for accelerometret. Det kan dog være fordelagtig at oversample. Dette giver mindre støj på signalet, da Nyquist frekvensen derved rykkes og fjerner aliasing. \citep{Webster2011} \newline
Frekvensområdet for gyroskopet viste sig at have et maksimalt frekvensområde på 6 Hz. Derfor skal ADCen sample gyroskopets data med mindst 60 Hz.

For at sikre at der registreres en valid puls for brugeren af systemet, skal den beregnede puls være <208 BPM. Dette er med udgangspunkt i beregningen, at en persons maksimale puls bestemmes ved: \citep{CooperBlair2005} 
\begin{equation}
220~[BPM] - alder~[$År$] = $Maksimale puls$~[BPM]
\end{equation}

Sampleraten bestemmes med antagelse om, at målgruppens maksimale puls er <208, hvilket svarer til cirka 3,5 hjerteslag i sekundet. Derfor skal samplingsfrekvensen konfigureres til 35 Hz, med henhold til en praktisk samplingsfrekvens som er 10 gange større end den maksimale frekvens for det optagede signal. \citep{Webster2011}

\textbf{Krav til ADC} \newline
ADCen skal:
\begin{itemize}
\item Sample accelerometerets output med mindst 450 Hz. Der accepteres ikke en samplingsfrekvens under 450~Hz.
\item Sample gyroskopets output med mindst 60 Hz. Der accepteres ikke en samplingsfrekvens under 60~Hz.
\item Sample pulssensorens output med mindst 35~Hz. Der accepteres ikke en samplingsfrekvens under 35~Hz. 
\item Repræsentere det analoge signal med maksimalt 5\% afvigelse. 
\end{itemize}
%
%\subsection{Krav til software}
%Aktivitetsmålerens software består af algoritmedesign til MCUens mikroprocessorer 4200M og EZ-BLE PRoC på henholdsvis GAP central og GAP peripheral, hvilket giver fire algoritmedesigns. Derudover skal disse to enheder kommunikere med hinanden, og en GUI designes for at give brugeren en visualisering af det behandlede data.

\subsection{Algoritmedesign til detektion af fysisk aktivitet} \label{krav_algoritme}
To MCUer skal agere som henholdsvis GAP central og GAP peripheral i forhold til deres BLE forbindelse. Dette opnås via et standart kodeeksempel fra Cypress, som skal debugges på hver af de to BLE PRoC på MCUerne. Enheden, som skal optage data fra brugeren, skal agere som GAP peripheral, mens MCUen tilkoblet en PC skal være GAP central. \\
Selve algoritmedesignet består altså af at lave en algoritme, som kan detektere henholdsvis gang, løb og cykling ved hjælp af data fra et accelerometer og et gyroskop. Der ses på data fra pilotforsøget i \appref{pilot}, at signalerne kræver databehandling førend en algoritme kan detektere forskellen imellem aktiviteterne. Derudover ses der, at peaket for hælnedslag har større amplitude under løb end ved gang, hvorfor en mulighed for detektion og adskillelse herimellem kunne være at indsætte en tærskelværdi.
%Designet af de fire algoritmer ligger til grund for hele funktionaliteten af systemet. \\
%Algoritmen til PSoC 4200M på den mobile MCU skal overordnet designes til at gøre hele MCUen til en GAP peripheral. Den skal indeholde elementer, som henvendes til ICen, da gyroskopet skal deaktiveres hvis det ikke er i brug. Hvert tiende sekund aktiveres gyroskopet, som skal tracke aktiviteten cykling. Hvis der ikke udføres aktiviteten cykling, slukkes gyroskopet, hvorefter det ti sekunder senere vågner igen. Algoritmen skal gøre mikroprocessoren i stand til at detektere, om den pågældende aktivitet er gang, løb eller cykling. Tiden, som brugeren bruger på den pågældende aktivitet, samt pulsmålingen skal gemmes. Hvert kvarter vækkes EZ-BLE PRoC fra stop power mode og den gemte data overføres hertil via UART kommunikation.\\
%Algoritmen til EZ-BLE PRoC på GAP peripheral sørger for, at enheden er i stop power mode med mindre, der modtages et interrupt fra PSoC 4200M på samme enhed. Når dette interrupt modtages, skal EZ-BLE PRoC modtage al data og overføre til en anden EZ-BLE PRoC på GAP central via BLE. \\
%Algoritmen på EZ-BLE PRoC på GAP central skal få enheden til at modtage data fra EZ-BLE PRoC på GAP peripheral. Denne data skal sendes til PSoC 4200M på GAP central via UART kommunikation. \\
%Algoritmen til PSoC 4200M på enheden tilkoblet en PC skal overordnet designes til at gøre hele MCUen til en GAP central. Desuden skal algoritmen gøre, at mikroprocessoren kan modtage data fra EZ-BLE PRoC på GAP central og videreføre dette til en GUI på en PC via USB porten.

\textbf{Krav til algoritmedesign} \newline 
Algoritmedesignet skal:
\begin{itemize}
	\item Indlede med databehandling således, hælnedslaget fra accelerometrets optagede data fremstår som en tydelig peak, og gyroskopets data for cykling fremstår som en tydelig sinus. Der accepteres ikke, hvis signalet ikke kommer til at fremstå som en tydelig peak eller sinus.
	\item Være i stand til at detektere henholdsvis gang, løb og cykling ved hjælp af eksempelvis tærskelværdier for peakværdien. Der accepteres ikke, at systemet ikke kan detektere og adskille de tre aktivitetsformer.
%	\item Gøre PSoC 4200M på GAP peripheral i stand til at sample signalet fra sensoren og pulssensoren samt filtrere og behandle dette. Der accepteres ikke, at mikroprocessoren ikke er i stand til dette grundet algoritmedesignet.
%	\item Aktivere gyroskopet hvert 10. sekund, som skal tracke om aktiviteten er cykling. Hvis dette er tilfældet, skal data fra gyroskopet samples istedet, og hvert 10. sekund skal algoritmen stadig tjekke for, om aktiviteten fortsat er cykling. Hvis aktiviteten ikke er cykling, skal algoritmen deaktivere gyroskopet igen for strømbesparelse.
%	\item Gøre PSoC 4200M på GAP peripheral i stand til at skelne imellem aktiviteterne gang, løb og cykling samt kunne tracke tiden for den pågældende aktivitet. Der accepteres, at enheden fejlbedømmer aktiviteten 5\% af tiden grundet tærskelværdien for skellet imellem.
%	\item Gøre, at PSoC 4200M på GAP peripheral kan gemme tidsenheden for hver aktivitet i et kvarter, hvorefter dette sendes til EZ-BLE PRoC på GAP peripheral. Der accepteres ikke, at algoritmen er skyld i, at dette ikke kan lade sig gøre.
%	\item Få EZ-BLE PRoC på GAP peripheral til at vågne ved et interrupt fra PSoC 4200M og derved kunne modtage data, som skal sendes videre via BLE. Der accepteres ikke, at algoritmen er skyld i, at dette ikke kan lade sig gøre.
%	\item Gøre, at EZ-BLE PRoC på GAP central kan modtage data fra EZ-BLE PRoC på GAP peripheral via BLE, hvorefter dette skal sendes til PSoC 4200M på GAP central. Der accepteres ikke, at algoritmen er skyld i, at dette ikke kan lade sig gøre.
%	\item PSoC 4200M på GAP central i stand til at modtage data fra EZ-BLE PRoC på GAP central og overføre dette til en PC via USB port. Der accepteres ikke, at algoritmen er skyld i, at dette ikke kan lade sig gøre.
\end{itemize}

\subsection{Trådløs kommunikation}\label{krav_BLE}
Den trådløse kommunikation imellem GAP Central og GAP Peripheral skal foregå over BLE. Dette kan give problematikker, da BLEs effektivitet afhænger af distancen imellem enhederne\fxnote{Bloetooth er maksimalt 100 meter, mens BLE er maksimalt 10 meter}. 

\textbf{Krav til den trådløse kommunikation} \newline 
Den trådløse kommunikation skal:
\begin{itemize}
	\item Foregå via BLE imellem de to MCU enheder. Der accepteres ikke andre former for trådløs kommunikation.
	\item Være i stand til at sende korrekt data i 3 meters afstand. Der accepteres ikke tab af data eller anden komplikation med dataoverførslen ved under 3 meters afstand imellem enhederne.
\end{itemize}

\subsection{Grafisk Bruger Interface}\label{krav_GUI}
GUIen skal være en motiverende faktor for brugeren, da denne netop skal motivere målgruppen til øget fysisk aktivitet. Den skal visualisere tidsforbruget på og intensiteten af henholdsvis gang, løb og cykling i løbet af en dag. %Hvert 15. minutter kan interfacet opdateres, da GAP peripheral afsender data til GAP central i dette tidsinterval.

\textbf{Krav til GUI} \newline 
GUIen skal:
\begin{itemize}
	\item Kunne visualisere tidsforbruget og point opnået ved henholdsvis gang, løb og cykling. 
%	\item Være i stand til at opdatere interfacet senest hvert 15. minut, hvis dette ønskes af brugeren. Der accepteres en forsinkelse på 5 minutter.
\end{itemize}

\subsection{Det samlede system} \label{krav_samlet_sys}
Når hver af de overstående dele fungerer, skulle det gerne være muligt at sammensætte systemet til en samlet enhed. Kravene hertil vil derfor tage udgangspunkt i \secref{funktionellekrav} med accepterede afvigelser. Men eftersom systemet anses som en prototype vurderes det, at nogle krav fra \secref{funktionellekrav} ikke kan testes og dermed muligvis ikke opfyldes. Dette blev ikke implementeret, da systemet er en prototype, hvoraf fokus er på dets funktionalitet. Eksempelvis er det inden for projektperioden ikke muligt at teste for, om systemet er komfortabelt og motiverende for børn i aldersgruppen 9-12 år.

\textbf{Krav til det samlede system} \newline
Det samlede system skal:
\begin{itemize}
	\item Kunne detektere aktiviteterne gang, løb og cykling gennem bestemte ved brug af gyroskop og accelerometer. Der accepteres ikke brug af andre sensorer.
	\item Kunne adskille gang, løb og cykling ved hjælp af algoritmer i softwaren. Der accepteres en afvigelse på 10\% ift. fejlvurdering af aktivitet.
	\item Kunne registrere puls igennem pulssensor og derefter omregne dette til intensitet af en given aktivitet. Der accepteres en pulsafvigelse på 10\%.
	\item Videresende signaler til en ekstern enhed ved hjælp af BLE. Der accepteres ikke andre trådløse kommunikationsformer.
	\item Besidde batteriledetid for en hel dag svarende til 14 timer. Der accepteres en batterilevetid på mindst 10 timer.
	\item Repræsentere varigheden og intensiteten af en given aktivitet i en GUI, hvor der gives point herudfra. Der accepteres ikke en anden form for visualisering. 
\end{itemize}

%Afspejling af en eventuel motiverende faktor, 
%	- Vise data fra en hel dag. 
%Notes: 
%- Trådløs overførsel af data
%	- BLE
%	- Mellem GAP Central og GAP Peripheral
%- Algoritmedesign 
%	- Aktivitet
%		- Detektere gang 
%			- (Detektere skridt)
%		- Detektere løb
%			- (Detektere skridt)
%		- Detektere cykling
%		- Detektere puls 
%		- Adskillese af ovenstående
%		- Digital filtrering 
%	- Strømbesparelse 
%		- Lowpower mode
%		- Gyroskop vs. accelerometer
%- MATLAB GUI
%	- Afspejling af en eventuel motiverende faktor, 
%	- Vise data fra en hel dag. 
%Aktivitetsmålerens software skal:
%\begin{itemize}
%\item Anvende digitale filtre til filtrering.
%\item Detektere og adskille aktiviteterne; gang, løb og cykling. 
%\item Trådløs overførsel.
%\item Gemme en hel dags aktiviteter. 
%\end{itemize}
