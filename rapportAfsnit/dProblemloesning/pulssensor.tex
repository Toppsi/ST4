\subsection{Pulssensorer}
Kroppens puls kan detekteres på en række forskellige måder, heriblandt elektrisk, optisk, akustisk og mekanisk eller magnetisk. 
\newline
Elektriske pulssensorer, måler pulsen ved elektrisk kontaktflade mellem sensor og person gennem elektroder. Pulsen detekteres ved de elektriske pulssensorer som forskellen i den elektriske ladning. Udfaldet af målingerne er subjektive, da de kan variere efter personens kropsvæsker såsom blod eller olier i huden. 

Optiske pulssensorer registrere puls gennem lys. En rød LED sender lys, som passerer huden og blodåren. Noget af lyset absorberes af hæmoglobinen i blodet, hvorefter en fotodiode opfanger mængden af det resterende lys. Jo mere blod der er i åren, jo mindre lys bliver tranmitteret. Outputtet fra pulssensoren er spejlvendt af det lys der opfanges, så signalet giver udslag der stemmer overens med mængden af blod. Denne type sensor bruges ofte i fingerspidsen eller på tåen. 

Akustisk opfangelse af puls sker gennem et stetoskop. Dette er en enhed hvor lægen opfanger lyden fra hjertet gennem patientens bryst, og derudfra bestemmer pulsen. 

Mekanisk kan pulsen opfanges ved brug af at piezo-elektrisk materiale, som presses mod huden, og opfanger presset fra hjerteslagene eller pulsen. 

En anden metode er at placere magneter på kroppen, hvorved blodmolekylerne polariseres. Derefter kan potentialeforskellen registreres gennem elektroder tæt ved det pålagte magnetfelt. 