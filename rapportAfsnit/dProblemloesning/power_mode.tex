\subsection{Power mode for PSoC 4 moduler}
Alle PSoC 4 moduler besidder fem forskellige power modes: active, sleep, deep-sleep, hibernate og stop. Strømforbruget samt tiden det tager at vågne op fra denne tilstand ses i \tabref{tab:Powermode}. \citep{Semiconductor2016PowerMode}
\begin{table}[H]
	\centering
	\begin{tabular}{ccc} \hline
		\rowcolor[HTML]{C0C0C0} 
		Power Mode & \begin{tabular}[c]{@{}c@{}}Strømområde\\(Vdd = 3,3V til 5,0V) \end{tabular} &  \begin{tabular}[c]{@{}c@{}}Opvågningstid\\ for PSoC 4200M \end{tabular} \\ \hline
		Active & 1,3mA til 14 mA & - \\ \hline
		Sleep & 1,0mA til 3mA & 0 \\ \hline
		Deep-sleep & 1,3$\mu$A til 15$\mu$A & 25$\mu$s \\ \hline
		Hibernate & 150 nA til 1$\mu$A & 0,7 ms \\ \hline
		Stop & 20 nA til 80 nA & 2 ms \\ \hline
	\end{tabular}%
	\caption{I tabellen ses strømforbruget samt opvågningstiden for PSoC 4200M, som er den valgte mikrokontrollers target processor. (Modificeret) \citep{Semiconductor2016PowerMode}}
	\label{tab:Powermode}
\end{table}\vspace{-0.2cm}
Det kan ses på \tabref{tab:Powermode}, at mikroprocessorens strømforbrug afhænger af, hvor meget af indholdet i mikroprocessoren der gøres utilgængeligt. Det kan derfor være fordelagtigt at køre i en lav strømforbrugende tilstand, hvis der benyttes et batteri til at forsyne MCU'en. %Power moden har indflydelse på mikroprocessorens funktioner, hvorfor det ikke altid er fordelagtigt at fokusere for meget på lavt strømforbrug.
En kort oversigt over aktiv og inaktiv indhold i mikroprocessoren ses i \tabref{tab:aktiv_inaktiv_CPU}. \citep{Semiconductor2016PowerMode}  % For eksempel er mikroprocessorens CPU kun aktiv, når power moden er indstillet til aktiv. Derudover er eksempelvis ADCen kun aktiv under de to power modes aktiv og sleep.\fxnote{Der findes en hel tabel inde på kilden over ting, som er aktive eller slukkede under de forskellige modes. Er det fint bare med disse to eksempler, som står i teksten, eller er der behov for at tabellen skal indsættes?} 
\begin{table}[H]
	\centering
	\begin{tabular}{cccccc} \hline
		\rowcolor[HTML]{C0C0C0} 
		Subsystem & Active & Sleep & Deep-sleep & Hibernate & Stop \\ \hline
		CPU & Aktiv & Tilbageholdt & Tilbageholdt & Deaktiv & Deaktiv \\ \hline
		RAM &  Aktiv &  Aktiv & Tilbageholdt & Tilbageholdt & Deaktiv \\ \hline
		I$^{2}$C slave & Aktiv &  Aktiv &  Aktiv & Deaktiv & Deaktiv \\ \hline
		Høj-hastigheds clock & Aktiv & Aktiv & Deaktiv & Deaktiv & Deaktiv \\ \hline
		ADC &  Aktiv &  Aktiv & Deaktiv & Deaktiv & Deaktiv \\ \hline
		\begin{tabular}[c]{@{}c@{}}Low-power\\ komparatorer \end{tabular}&  Aktiv &  Aktiv &  Aktiv &  Aktiv & Deaktiv \\ \hline
		\begin{tabular}[c]{@{}c@{}}	Generel-purpose \\ input/output (GPIO)\end{tabular} &  Aktiv &  Aktiv &  Aktiv &  Aktiv & Frossen \\ \hline
	\end{tabular}%
	\caption{I tabellen ses hvilke subsystemer, der er henholdsvis tændt, tilbageholdt slukket eller frossen under de fem forskellige power modes. Hvis en funktion tilbageholdes, fungerer dens centrale konfiguration ikke, men alle eksterne funktioner er aktive. Når GPIO er frossen, er alle pins låst og kan derfor ikke modtage input- eller sende outputinformation. (Modificeret) \citep{Semiconductor2016PowerMode}}
	\label{tab:aktiv_inaktiv_CPU}
\end{table}\vspace{-0.2cm}
I sleep mode tilbageholdes CPU'en, hvilket betyder at den venter på ethvert interrupt eller manuel genstart for at vågne op. RAM'en bevares, men CPU'en skriver eller læser ikke fra den, hvorfor den altså ikke kører sine instruktioner.\fxnote{Den sover ikke helt eller er ikke helt inaktiv - man kan sige, at den sover meget let, for den er til dels inaktiv, men den vækkes meget let.} Denne tilstand er fordelagtig til at reducere strømforbruget mellem hændelser som for eksempel ADC konvertering. \citep{Semiconductor2016PowerMode} \\
Under deep-sleep er højfrekvense clocks deaktive, og da ADC'en kræver sådanne clocks, er denne ligeledes inaktiv under denne tilstand. Operationssystemet I$^{2}$C er aktiv, hvorfor mikroprocessoren fortsat kan fungere som slave. Denne tilstand kræver specifikke interrupts for at vågne, som for eksempel et I$^{2}$C eller GPIO interrupt. Deep-sleep mode kan benyttes, hvis høj performance analoge eller digitale enheder ikke kræves for funktionen. \citep{Semiconductor2016PowerMode} \\
Når mikroprocessoren er i hibernate mode, er alle clocks og eksterne enheder deaktiveret. Ram'en gemmes men når et specifikt interrupt vækker mikroprocessoren, nulstilles og genstarter hele enheden. Denne tilstand kan vælges, hvis der kun er behov for periodiske opvågninger, og enheden skal bruge under 1$\mu$A strøm. \citep{Semiconductor2016PowerMode} \\
Stop mode er den laveste strømforbrugende tilstand. Her modtager mikroprocessoren strøm, som bevarer logiske tilstande, men alt andet er deaktiveret. Kun én dedikeret pin kan vække enheden fra denne stop mode, hvorefter den nulstilles og genstarter. Stop mode kan benyttes, når enheden ikke skal slukkes helt og skal være i stand til at tænde igennem for eksempel en trykknap med input i den dedikerede pin. \citep{Semiconductor2016PowerMode}
%Noget med funktionaliteten også afhænger af power moden, fordi nogle ting bliver slukket. Tabel 2 i URL link. Skriv det ind, så det hænger sammen med de andre steder i teksten, hvor der er nævnt noget med dens tilstand.
% https://cdn.sparkfun.com/assets/learn_tutorials/3/7/3/LSM9DS1_Datasheet.pdf