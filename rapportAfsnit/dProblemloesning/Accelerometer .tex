\subsection 

Et accelerometer er et elektromekanisk apparat som anvendes til at m�le accelerationskr�fter. 
Det kan blandt andet registrere om et objekt bev�ger sig opad, nedad og line�r acceleration\citep{Goodrich2013}. Enheden er m�lt i meter pr sekund i anden(m/s^2) eller i G kr�fter (g). En enkelt G kraft p� jorden er tilsvarende til 9,8 m/s^2 , men det kan variere med elevation. \citep{Sparkfun}
Hvis outputtet af en sensor er vendt opad vil kraften v�re +1g, hvis outputtet af en sensor er horisontal er det tilsvarende 0g og hvis outputtet er vendt nedad vil det v�re tilsvarende -1g. 

Accelerometre kan opdeles i flere typer, hvor piezoelektriske accelerometer er den mest anvendte af typen. Den anvender mikroskopiske krystalstrukturer, der bliver aktiveret ved accelerationskr�fter. Disse krystaliske strukturer danner en sp�nding og accelerometeret fortolker denne sp�nding fart og orientering. \citep{Goodrich2013}

Et accelerometer m�ler to former for acceleration, henholdsvis statisk og dynamisk, hvor de statiske kr�fter inkluderer tyngdekraften og hvilken vinkelretningen enheden bliver tiltet hen. De dynamiske kr�fter inkluderer hvilken retning enheden bev�ger sig imod og dens vibrationer. \citep{Sparkfun,Engineering, Goodrich2013}

Man kan m�le accelerationer i flere retninger ved af bruge flere end en accelerometre. \citep{Sparkfun}. Man kan m�le acceleration af en akse, to akser(x,y) og tre akser (x,y,z) akser, hvor den sidste er den hyppigst anvendte. For eksempel har en bil to akser i en og de fleste smartphones har 3 akser. \citep{Sparkfun}

Man kan anvende AC og DC forsyninger til et accerlerometer, hvilket bestemmer typen af hardware og s�tter en gr�nseflade for accelerometeren. \citep{Engineering}

I et AC-koblet accelerometer er outputtet AC, hvilket vil betyder at enheden ikke kan m�le statisk acceleration s�som tyngdekraften, men kun dynamisk acceleration. 
En DC-koblet accelerometer kan m�le nul hertz, og kan derfor anvendes til at m�le b�de statisk og dynamisk acceleration. 



http://www.livescience.com/40102-accelerometers.html


https://learn.sparkfun.com/tutorials/accelerometer-basics