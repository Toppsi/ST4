\section{Funktionelle krav}\label{funktionellekrav}
\textit{For at sikre systemets funktionalitet i forhold til ovenstående løsningsstrategi opstilles en række overordnede funktionelle krav til det samlede system. Disse krav danner grundlag for efterfølgende indhold i kapitlet. Der opstilles ydermere et blokdiagram for at give et overblik af kravene til systemet og heraf systemets funktionalitet.}

Formålet med systemet er at udvikle en aktivitetsmåler, som har potentialet til at reducere antallet af fysisk inaktive børn. Der ønskes et system, som detekterer og adskiller aktiviteterne gang, løb og cykling, idet disse vurderes at være gængse aktiviteter i børns hverdag. Detekteringen af disse kan ske gennem et accelerometer og et gyroskop, hvorefter systemet teoretisk kan adskille gang, løb og cykling. Intensiteten af aktiviteterne registreres igennem puls, da dette giver en indikation af barnets fysiologiske udbytte af aktiviteten. Det vil være væsentligt at sammenholde puls og tid, da det anbefales, at børn skal være aktive 30~minutter med høj intensitet mindst tre gange om ugen. Derudover er aktivitetens varighed en væsentlig parameter, idet børns kognitive funktion er øget i op til 50~minutter efter 11-20~minutters fysisk aktivitet.

For at systemet har en motiverende effekt på børn, skal der være en brugerflade, som børn finder motiverende. Denne skal visuelt give feedback på dagens samlede præstationer samt progressionen i aktivitetsniveauet. Børnene bliver gennem brugerfladen belønnet med point afhængigt af, hvilken aktivitet der udføres og intensiteten heraf.  

Systemet skal være i stand til at detektere børns aktivitet igennem en hel dag uden at være til gene. Det skal derfor være et batteridrevet trådløst system, som kan sende data til en ekstern enhed med faste intervaller. Derudover skal det være elektrisk sikkert, således barnet ikke udsættes for fare. 

På baggrund af ovenstående udformes de funktionelle krav således, at systemet skal: 
\begin{itemize}
	\item Kunne detektere aktiviteterne gang, løb og cykling gennem bestemte sensorer.
	\item Kunne adskille gang, løb og cykling ved hjælp af algoritmer i softwaren.
	\item Kunne registrere intensiteten af de givne aktiviteter gennem pulsdetektering.
	\item Være komfortabelt for brugeren.
	\item Trådløst sende data til en ekstern enhed.
	\item Være batteridrevet gennem en hel dag.
	\item Være elektrisk sikkert for brugeren.
	\item Behandle og repræsentere signalerne visuelt i forhold til tid.
	\item Motivere børn i aldersgruppen 9-12~år. 
\end{itemize}

\subsection{Det overordnede system}
Med udgangspunkt i de funktionelle krav til det overordnede system udformes et blokdiagram, som illustreres på \figref{fig:blokdiagram}. På denne fremgår rækkefølgen af blokkene, samt hvor i systemet funktionerne finder sted.
\begin{figure}[H]
 	\centering
 	\includegraphics[scale=0.6]{figures/bProblemloesning/blokdiagram_funktionelle_krav.png}
 	\caption{På figuren ses blokdiagrammet for det samlede system. Overordnet deles systemet op i tre, som markeres med forskelligt farvede kasser. Sensorerne opsamler data, som den perifære enhed signalbehandler, hvorefter den centrale enhed visualiserer dette.} 	
 	\label{fig:blokdiagram}
 \end{figure}\vspace{-0.25cm}
De analoge sensorer, som er omkredset af en rød firkant på \figref{fig:blokdiagram}, står for opsamling af data. Disse skal måle på forsøgspersonen og være tilkoblet den perifære enhed. Signalerne herfra konverteres fra analoge til digitale signaler gennem en analog-til-digital konverter (ADC). Herefter behandles signalerne således, at der kan adskilles mellem de specifikke aktiviteter gang, løb og cykling. Gennem trådløs kommunikation mellem enhederne overføres data til den centrale enhed, der viderefører data til PCen. Heri visualiseres dataet i en grafisk brugerflade (GUI), så børnene kan se varigheden og intensiteten af en given aktivitet. 