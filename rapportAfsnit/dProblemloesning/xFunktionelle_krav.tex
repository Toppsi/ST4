\section{Funktionelle krav}\label{funktionellekrav}
\textit{For at sikre systemets funktionalitet i forhold til ovenstående løsningsstrategi opstilles en række funktionelle krav for det samlede system. Disse krav danner grundlag for efterfølgende indhold i kapitlet. Der opstilles ydermere et blokdiagram for at give et overblik kravene til systemet.}

Formålet med systemet er at udvikle en aktivitetsmåler, som har potentialet til at reducere antallet af inaktive børn. Dette gøres med henblik på at ændre den teknologiske udviklings påvirkning på børns aktivitetsvaner fra inaktivitet til aktivitet. Der ønskes et system, som detekterer aktiviteterne gang, løb og cykling, da disse motionsformer vurderes at være gængse aktiviteter i et barns hverdag. Detekteringen af disse kan ske gennem et accelerometer og og et gyroskop, hvorefter systemet teoretisk kan adskille gang, løb og cykling. Intensiteten af aktiviteterne registreres igennem puls, da dette giver en indikation af det fysiologiske udbytte, som barnet får ud af en given aktivitet. Det vil være væsentligt at sammenholde puls og tid, da det anbefales, at børn skal være aktive 30 minutter med høj intensitet mindst tre gange om ugen. Derudover er et barns kognitive funktion øget i op til 50 minutter efter 11-20 minutters fysisk aktivitet.

For at systemet har en motiverende effekt på børn, skal der være en brugerflade, som børnene finder interessant. Denne skal visuelt give feedback på dagens samlede præstationer samt progressionen i aktivitetsniveauet. Dette gøres, da intensiteten af en aktivitet er essentiel for udbyttet, som beskrevet i \secref{subsub:ak_int}. % Børnene udfordres dermed på intensiteten, hvilken kan variere for det enkelte barn ved den samme aktivitet.
Børnene bliver belønnet med point afhængigt af, hvilken aktivitet der udføres og intensiteten heraf.  

Systemet skal være i stand til at detektere børns aktivitet igennem en hel dag uden at være til gene. %, hvormed det skal fungere uafhængigt af andre systemer. 
Det skal dermed være et batteridrevet trådløst system, som kan sende data til en ekstern enhed med faste intervaller. Derudover skal det være elektrisk sikkert, således barnet ikke bliver skadet som følge af aktivitetsmålerens design. 

På baggrund af ovenstående udformes de funktionelle krav således, at systemet skal: 
\begin{itemize}
	\item Kunne detektere aktiviteterne gang, løb og cykling gennem bestemte sensorer.
	\item Kunne adskille gang, løb og cykling ved hjælp af algoritmer i softwaren 
	\item Kunne registrere intensiteten af de givne aktiviteter igennem pulssensor.
	\item Være komfortabelt for brugeren.
	\item Trådløst videresende signaler til en ekstern enhed og være batteridrevet en hel dag.
	\item Være elektrisk sikkert for brugeren.
	\item Behandle og repræsentere signalerne visuelt som intensiteten af en aktivitet i forhold til tid.
	\item Motivere børn i aldersgruppen 9-12 år. 
\end{itemize}

\subsection{Overordnede system}
Ud fra kravene til det overordnede system udformes et blokdiagram, som illustreres på \figref{fig:blokdiagram}. På denne fremgår rækkefølgen af blokkene, samt hvorhenne i systemet funktionerne finder sted.\\
De analoge sensorer, som er omkredset af en rød firkant på \figref{fig:blokdiagram}, står for opsamling af data. Denne skal måle på forsøgspersonen og være tilkoblet den perifære enhed. Signalerne herfra konverteres fra analoge til digitale signaler gennem en analog-til-digital konverter (ADC). Herefter behandles signalerne således, at der kan adskilles mellem de specifikke aktiviteter gang løb og cykling. Gennem trådløs kommunikation mellem de enheder overføres data til den centrale enhed, der viderefører data til PCen. Heri visualiseres dataet på en GUI, så børnene kan se perioden og intensiteten af en given aktivitet. 
\begin{figure}[H]
 	\centering
 	\includegraphics[scale=0.6]{figures/bProblemloesning/blokdiagram_funktionelle_krav.png}
 	\caption{På figuren ses blokdiagrammet for det samlede system. Dette vil indebære en opsamling af data fra sensorer på brugeren efterfulgt af den perifære enhed, som blandt andet adskiller gang, løb og cykling. Afslutningsvis er der en central enhed, som visualiserer det fysiske aktivitetsniveau.}
 	\label{fig:blokdiagram}
 \end{figure}