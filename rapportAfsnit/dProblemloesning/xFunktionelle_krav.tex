\section{Funktionelle krav}
%Problemformulering: Hvordan kan en aktivitetsmåler udvikles således at fysisk inaktive børn i aldersgruppen 9-12 år, motiveres til en mere aktiv livsstil?

%Hvordan kan en aktivitetsmåler udvikles således, at den har potentialet til at reducere antallet af inaktive børn i aldersgruppen 9-12 år?


Formålet med systemet er, at udvikle en aktivitetsmåler som har potentialet til at reducere antallet af inaktive børn i aldersgruppen 9-12 år. Dette i forsøg på at ændre den teknologiske udviklings påvirkning af børns aktivitetsvaner fra inaktivitet til aktivitet.
Der ønskes derfor et analogt system som detekterer aktiviteterne gang, løb og cykling, da disse er gængse aktiviteter i et barns hverdag. Måden hvorpå et analogt system kan detektere disse aktiviteter kan ske gennem forskellige sensorer.\fxnote{skriver vi noget om dette i aktivitetsmåler afsnittet, kan nogle pointer evt. benyttes her}
Hertil skal intensiteten af aktiviteterne registreres igennem puls, da dette giver en indikation af det fysiologiske udbytte, barnet får ud af en given aktivitet. Herved vil det ligeledes kunne registreres om barnet er aktiv med høj intensitet i 30 minutter tre gang om ugen, som det anbefales\secref{subsec:fysio_aktivitet}.

Systemet skal kunne detektere børns aktivitet igennem en hel dag, uden at være til gene, hvorfor det skal kunne fungere uafhængigt af andre systemer. Det skal derfor være et trådløst system, som er batteridrevet og kan sende data til en ekstern enhed. Derudover skal det være elektrisk sikker, således at barnet ikke kan komme til skade som følge af aktivitetsmålerens design. 

Den eksterne enhed, hvortil dataet bliver sendt, skal efterfølgende kunne behandle og visualisere dataet, hvorigennem brugeren får feedback på sin præstation. De analoge signaler skal altså konverteres til digitale signer hvorefter systemet, gennem algoritmer, skal kunne adskille gang, løb og cykling. For at dataet tydeligt kan repræsenteres visuelt, skal det, igennem den digitale databehandling, filtreres for at fjerne uønskede signaler. 

For at systemet har en motiverende effekt på børn, skal de have feedback på deres præstationer ved at kunne se dagens totale aktivitet samt progressionen i deres aktivitetsniveau. Dette skal gøres gennem en motiverende brugerflade som børnene finder interessant. 

De funktionelle krav til systemet, lyder derfor således: 
\begin{itemize}
	\item Systemet skal kunne detektere og adskille gang, løb og cykling.
	\item Systemet skal kunne registrere intensiteten af de givne aktiviteter.
	\item Systemet skal være komfortabelt, hvorfor det trådløst skal kunne videresende signaler til en ekstern enhed og være batteridrevet.
	\item Signalerne som er sendt til en ekstern enhed, skal behandles og repræsenteres visuelt.
	\item Systemet skal være elektrisk sikkert for brugeren.
	\item Systemet skal motivere børn i aldersgruppen 9-12 år. 
\end{itemize}










