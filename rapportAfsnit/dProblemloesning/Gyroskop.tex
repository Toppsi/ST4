\subsection{Gyroskop}

Gyroskoper er et sensorapparat som anvendes til at registrere et objekts omdrejningsvinkel eller dets vinkelhastighed omkring en bestemt akse, som er dens mest anvendte metode. De bruges 
blandt andet til stabilitet, autopilotfeedback, flyvejsensor eller stabilisering af platforme samt til navigation. 
I nogle moderne gyroskoper er det muligt at bruge alle disse funktioner i et sensor, men oftest er de opdelt i flere grupper. 
Et gyroskop fungere ved at anvende inerti egenskaberne der opstår når et hjul spindes med en høj hastighed. Ved at hjulet fastholder den samme retning omkring aksen, kan impulsmomentmomentet, dets inertiprodukt samt hastighed være med til at definere en referenceretning. 

%De fundementale principper bag virkningen af et gyroskop er blandt andet det gyroskopiske inerti, som er når hjulet drejer om sin egen akse og står vinkelret på aksen. impulsmomentet som er fordelingen af en masse på et rotor, hvor vinkelhastigheden også har en betydning, og præcession som er rotationen omkring egen akse. 

De signaler som opfanges af et accelerometrer, inkluderer ikke signaler fra den roterende akse og derfor kan en præcis orientering ikke opfanges. For at forbedre nøjagtigheden, kan man anvende gyroskoper som et supplement til accelerometre .
Et gyroskop måler vinkelhastighed, hvor ændringen i orientering kan måles ved at integrere vinkelhastigheden på baggrund af en algoritme. \citep{LuingeVeltink2005}