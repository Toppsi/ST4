\section{Opsamling af pilotforsøg}
Igennem pilotforsøget, som ses i \appref{pilot}, blev aktiviteterne gang, løb og cykling undersøgt og behandlet. Databehandlingen heraf tog udgangspunkt i pilotforsøgets formål, hvormed fortolkningen af dette er pilotforsøgets essens. \\
Tre mulige placeringer af en aktivitetsmåler blev undersøgt for alle aktiviteter. Resultatet af dette medførte at aktiviteternes signalamplituder for henholdsvis accelerometer og gyroskop, blev undersøgt. Placering A blev valgt, hvorigennem accelerometeret bør have et arbejdsområde på $\pm$16 g, og gyroskopet på minimum 320,5 dps. \\
Aktiviteterne blev undersøgt med henblik på hvorvidt en mulig adskillelse af disse var mulige. Resultatet af dette medførte, at fælles for gang og løb forekom signal events på accelerometerets y-akse, hvorigennem aktiviteterne antageligvis kan adskilles. Karakteristika vedrørende cykling blev undersøgt med henhold til gyroskopets z-akse. Resultatet heraf medførte et tydeligt sinus lignende signal. Det antages at et sådan signal skaber mulighed for detektering. Det blev ydermere sikret at signaler fra gang og løb ikke havde en sinus lignende tendens vedrørende gyroskopets z-akse, hvoraf adskillelse ligeledes antages at være mulig. \\
Aktiviteternes frekvensindhold blev ligeledes undersøgt med henblik på at kunne fastsætte systemets samplingsfrekvens, samt knækfrekvens for eventuelle filtre. Resultatet heraf er at det største frekvensspektrum for gang og løb befinder sig op til 45 Hz, og cykling op til 6 Hz. 

