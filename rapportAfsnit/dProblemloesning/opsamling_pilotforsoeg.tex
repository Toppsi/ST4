\section{Opsamling af pilotforsøg}\label{opsamling_pilot}
\textit{Dette afsnit er en opsamling på projektgruppens pilotforsøg, som blev udført med henblik på optimal valg af sensor. Derudover blev signalernes udformning undersøgt for senere at kunne udvikle algoritmer.}

Igennem pilotforsøget, som ses i \appref{pilot}, blev aktiviteterne gang, løb og cykling undersøgt med henhold til biomekaniske egenskaber. \\
Tre mulige placeringer af en aktivitetsmåler blev undersøgt for alle aktiviteter, hvormed aktiviteternes signalamplituder for henholdsvis accelerometer og gyroskop blev undersøgt. Placering A blev valgt, hvorved accelerometeret bør have et arbejdsområde på $\pm$16~g og gyroskopet på minimum 320~dps. \\
Aktiviteterne blev undersøgt med henblik på, hvorvidt en adskillelse af disse var mulige. Signalamplituden for gang og løb har markant forskel i accelerometerets y-akse, hvorved de to aktivitetsformer antageligt kan adskilles. Karakteristika vedrørende cykling blev undersøgt i gyroskopets z-akse, hvoraf et tydeligt sinus lignende signal fremkom. Det antages, at et sådan signal skaber mulighed for detektering af cykling. Det blev ydermere sikret, at signaler fra gang og løb ikke havde en sinus lignende tendens i gyroskopets z-akse, hvoraf adskillelse ligeledes antages at være mulig. \\
Aktiviteternes frekvensindhold blev ligeledes undersøgt med henblik på at kunne fastsætte systemets samplingsfrekvens samt knækfrekvens for eventuelle filtre. Resultatet heraf er, at frekvensspektrum for gang og løb er på 0-45~Hz og for cykling på 0-6~Hz. 

