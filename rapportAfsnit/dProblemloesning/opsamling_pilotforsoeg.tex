\section{Resume af pilotforsøg}\label{opsamling_pilot}
\textit{Dette afsnit er et resume af pilotforsøget. Afsnittet opsummerer væsentlige resultater, som benyttes til udvikling af prototypen.% som blev udført med henblik på optimal valg af sensor. Derudover blev signalernes udformning undersøgt for senere at kunne udvikle algoritmer.
	}

Pilotforsøget, som ses i \appref{pilot}, undersøger gang, løb og cykling i forhold til biomekaniske egenskaber. \\
Tre mulige placeringer af en aktivitetsmåler bliver undersøgt for alle aktiviteter, hvormed aktiviteternes signalamplituder for henholdsvis accelerometer og gyroskop bliver undersøgt. Placering A vælges, hvorved accelerometeret bør have et arbejdsområde på $\pm$16~g og gyroskopet bør have et arbejdsområde på minimum 320~dps. \\
Aktiviteterne undersøges for, hvorvidt en adskillelse af disse er mulige. Signalamplituden for gang og løb har markant forskel i accelerometerets y-akse, hvorved de to aktivitetsformer antageligt kan adskilles. Karakteristika vedrørende cykling undersøges i gyroskopets z-akse, hvoraf et lignende sinussignal fremkommer. Det antages, at dette signal skaber mulighed for detektering af cykling. Det sikres ydermere, at signaler for gang og løb ikke har en sinus lignende tendens i gyroskopets z-akse, hvoraf adskillelse ligeledes antages at være mulig. \\
Aktiviteternes frekvensindhold undersøges ligeledes med henblik på at kunne fastsætte systemets samplingsfrekvens samt knækfrekvens for eventuelle filtre. Resultatet heraf er, at frekvensspektrum for gang og løb er på 0-45~Hz og for cykling på 0-6~Hz.

