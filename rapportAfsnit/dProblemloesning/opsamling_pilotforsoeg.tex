\section{Opsamling af pilotforsøg}\label{opsamling_pilot}
\textit{Dette afsnit er en opsamling på projektgruppens pilotforsøg, som blev udført med henblik på optimal valg af sensor. Derudover blev signalernes udformning undersøgt for senere at kunne udvikle algoritmer.}

Igennem pilotforsøget, som ses i \appref{pilot}, bliver aktiviteterne gang, løb og cykling undersøgt med henhold til biomekaniske egenskaber. \\
Tre mulige placeringer af en aktivitetsmåler bliver undersøgt for alle aktiviteter, hvormed aktiviteternes signalamplituder for henholdsvis accelerometer og gyroskop bliver undersøgt. Placering A vælges, hvorved accelerometeret bør have et arbejdsområde på $\pm$16~g og gyroskopet på minimum 320~dps. \\
Aktiviteterne bliver undersøgt med henblik på, hvorvidt en adskillelse af disse er mulige. Signalamplituden for gang og løb har markant forskel i accelerometerets y-akse, hvorved de to aktivitetsformer antageligt kan adskilles. Karakteristika vedrørende cykling bliver undersøgt i gyroskopets z-akse, hvoraf et tydeligt sinus lignende signal fremkommer. Det antages, at et sådanne signal skaber mulighed for detektering af cykling. Der bliver ydermere sikret, at signaler fra gang og løb ikke har en sinus lignende tendens i gyroskopets z-akse, hvoraf adskillelse ligeledes antages at være mulig. \\
Aktiviteternes frekvensindhold bliver ligeledes undersøgt med henblik på at kunne fastsætte systemets samplingsfrekvens samt knækfrekvens for eventuelle filtre. Resultatet heraf er, at frekvensspektrum for gang og løb er på 0-45~Hz og for cykling på 0-6~Hz.

