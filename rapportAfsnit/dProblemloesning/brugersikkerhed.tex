\section{Brugersikkerhed}
\textit{Nedenstående afsnit beskriver kort hvilke ricisi der kan forekomme når en bruger til tilkoblet elektronisk udstyr. Metoder hvorpå de omtalte risici kan forebygges, beskrives også. Afsnittet underbygges af det funktionelle krav, hermed at systemet skal været sikkert for brugeren at anvende.}

Medikoteknisk udstyr er oftest tilsluttet elektricitet, og  kan dermed være en potentiel fare for brugeren. En elektrisk fare kan opstå hvis brugeren er direkte tilsluttet det medikotekniske udstyr, og kan dermed risikere at blive udsat for makro- og mikroshock. Makroshock er defineret som elektrisk strøm, som løber igennem hele kroppen. Denne strøm løber oven på huden, og er overfladisk. Mikroshock er defineret som elektrisk strøm, som direkte løber igennem vævet. Den elektriske størm der påvirkes med under mikroshock løber igennem hjertet, hvilket oftest bevirker en større potentiel fare end makroshock. \citep{Webster2011} \newline
I og med medikoteknisk udstyr potentielt kan udsætte brugerene for fare, så er der nogle væsentlige sikkerhedsprocedure der skal inkorporeres i udstyrets design. Disse sikkerhedsmæssige procedurer omhandler blandt andet isolation samt jording. For at sikre lækstrømme ikke opnår en størrelse hvormed makro- og mikroshock kan være alvorligt skadelige, kan isolation benyttes. Ved isolation sikre man at det medikotekniske udstyr ikke er i direkte forbindelse med en betydelig spændingskilde. I og med at udstyret er forsynet med en lav spændingskilde, begrænses størrelsen af de lækstrømme som kan forekomme. Den anden sikkerhedsforanstaltning som kan implementeres for at gøre udstyret sikkert for brugeren er jording. Jording sikre at alle aktive komponenter føres til jord, altså et fælles nulpunkt. Ved at de aktive komponenter er ført til et fælles nulpunkt vil eventuelle lækstrømme løbe til jord fremfor brugeren. \citep{Webster2011} \newline 
Eftersom systemet skal forsynes med at lav spænding samtidig med at være mobilt, så skal der benyttes batterier. Batterier kan være forbundet med nogle sikkerhedsmæssige farer. Farerne opstår hvis batterierne bliver misbrugt, og dermed ikke håndteret efter hensigten. Dette kan resultere i at batterierne ødelægges hvilket kan give første, anden og tredje grads forbrændinger. Et ødelagt batteri kan også medfører åndedrætsbesvær. Disse farer kan undgås hvis man følger batteriets sikkerhedsanvisninger. \citep{NREL2011}
