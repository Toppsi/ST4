\section{Brugersikkerhed}
\textit{Nedenstående afsnit beskriver, hvilke risici der kan forekomme, når en bruger tilkobles elektronisk udstyr. Dette gøres med henblik på at kunne opbygge et system med sikkerhedsforanstaltninger.}

Medikoteknisk udstyr er tilsluttet en spændingsforsyning i form af eksempelvis strømnettet eller et batteri. Der indgår derfor en spænding i det elektroniske kredsløb og dermed også en elektrisk strøm, som kan være farlig for brugeren, når vedkommende er tilkoblet det medikotekniske udstyr. Der kan derved være risiko for lækstrømme, som fører til makro- og mikroshock. Makroshock er defineret som en elektrisk strøm oven på huden, der løber igennem den tilsluttede person.\fxnote{Skader heraf er sjældent alvorlige, da det oftest leder til muskelkontraktioner} Mikroshock er defineret som elektrisk strøm, der løber igennem en persons væv, heriblandt hjertet. Skader fra mikroshok kan være store vævsskader eller dødelige elektriske påvirkninger af personen. Mikroshok medfører derfor oftest en større potentiel fare end makroshock.~\citep{Webster2011} \newline
Det er essentielt, at det elektroniske udstyr involverer sikkerhedsmæssige elementer, således risikoen for lækstrømme sænkes. Isolation benyttes til at isolere brugeren fra elektriske spændingskilder i det medikotekniske udstyr. Ydermere benyttes jording som en sikkerhedsforanstaltning, hvor alle aktive komponenter føres til samme jord. Ved at forbinde alle komponenter til samme nulpunkt vil eventuelle lækstrømme løbe denne vej og dermed væk fra brugeren.~\citep{Webster2011} \newline 
Systemet skal være mobilt, som det fremgår af \secref{funktionellekrav}, hvorfor systemet skal forsynes med spænding fra et batteri. Ved at bruge et batteri tilføres en lav spænding, hvorfor risici ved benyttelse af systemet begrænses. Brug af batterier medfører dog andre sikkerhedsmæssige farer, hvis det ikke bruges efter forskrevne regler. Ved fejlbrug kan brugeren risikere at batteriet ødelægges, hvilket blandt andet kan medføre forbrændinger eller andre skader på huden, da indholdet af batteriet kan være giftigt.~\citep{NREL2011}