\section{Brugersikkerhed}
\textit{Nedenstående afsnit beskriver hvilke ricisi der kan forekomme når en bruger tilkobles elektronisk udstyr. Metoder hvorpå de omtalte risici kan forebygges, beskrives også. Afsnittet underbygges af det funktionelle krav, hermed at systemet skal været sikkert for brugeren at anvende.}

Medikoteknisk udstyr er tilsluttet en spændingsforsyning i form af eksempelvis strømnettet eller et batteri. Der indgår derfor en spænding og dermed en elektrisk strøm i det elektroniske kredsløb. En elektrisk fare kan opstå når brugeren er tilkoblet det medikotekniske udstyr, og kan dermed risikere at blive udsat for makro- og mikroshock fra hele det elektriske kredsløb. Makroshock er defineret som en elektrisk strøm, som løber igennem kroppen på den tilsluttede person. Denne strøm løber oven på huden, og er overfladisk. Mikroshock er defineret som elektrisk strøm, som løber igennem en persons væv deriblandt hjertet. Den elektriske strøm som personen påvirkes med under mikroshock, medfører oftest en større potentiel fare end makroshock. Eksempelvis kan makroshock forårsage mindre muskelkontraktioner og er ofte ikke-dødelige skader. Derimod kan mirkoshock være store vævsskader samt dødelige elektriske påvirkninger af personen. \citep{Webster2011} \newline
Medikoteknisk udstyr har dermed en risiko for at påføre brugeren en strøm som potentielt kan være farlig. Det er derfor væsentligt, at det elektroniske udstyr involverer sikkerhedsmæssige elementer således risikoen for lækstrømme sænkes. Eksempelvis benyttes isolation og jordning som sikkerhedsmæssige procedurer, for at nedbringe risikoen for at tilføre brugeren lækstrømme i form af henholdsvis makroshock eller mikroshock. Isolation benyttes til at isolere brugeren fra elektriske spændingskilder i det medikotekniske udstyr. Ydermere benyttes jording som en sikkerhedsforanstaltning, idet alle aktive komponenter føres til jord, altså et fælles nulpunkt. De aktive komponenter er forbundet til jord, hvormed eventuelle lækstrømme vil løbe denne vej og dermed væk fra brugeren. \citep{Webster2011} \newline 
Systemet skal være mobilt som det fremgår af \secref{succeskrav}. Systemet vil dermed have en spændingsforsyning i form af et knapcelle batteri, hvilket vil tilføre en lav spænding. Benyttelsen af batterier kan dog være forbundet med enkelte, mindre sikkerhedsmæssige farer. Farerne kan opstå hvis batterierne ikke bliver brugt efter de foreskrevne regler for det pågældende batteri. Dette kan risikere at ødelægge batteriet, hvormed brugeren vil kunne blive udsat for forbrændinger som følge af fejlbrug af batteriet. Et ødelagt batteri kan ydermere risikere at medføre åndedrætsbesvær for brugeren. Disse farer kan undgås hvis man følger batteriets sikkerhedsanvisninger. \citep{NREL2011}


%  For at sikre lækstrømme ikke opnår en størrelse hvormed makro- og mikroshock kan være alvorligt skadelige, kan isolation benyttes. Ved isolation sikre man at det medikotekniske udstyr ikke er i direkte forbindelse med en betydelig spændingskilde. I og med at udstyret er forsynet med en lav spændingskilde, begrænses størrelsen af de lækstrømme som kan forekomme. Den anden sikkerhedsforanstaltning som kan implementeres for at gøre udstyret sikkert for brugeren er jording. Jording sikre at alle