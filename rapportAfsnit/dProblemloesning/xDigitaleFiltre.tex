\subsection{Digitale filtre}
Digitale filtre benyttes til adskillelse eller genskabelse af signaler. Signaladskillelse benyttes ofte i forbindelse med at filtrere støj fra det ønskede signal.\fxnote{eksempelvis ved ekg eller hjertelyd, hvor vejrtrækning og andre kropssignaler/lyde skal fjernes} Signalgenskabelse benyttes, hvis signalet er blevet beskadiget eller forvrænget.\fxnote{eksempelvis ved rystelser hvis måleudstyret er dårligt eller hvis ikke vi får sat den ordentlig fast på forsøgspersonen.} \citep{Smith1997}

Ethvert lineært filter har en impulsrespons, steprespons og frekvensrespons. Disse responser indeholder information om filteret på forskellig vis og giver tilsammen information om, hvordan filteret vil agere i en given situation.\fxnote{hvis en af disse er oplyst eller defineret for filteret, kan de to andre udregnes matematisk.} \citep{Smith1997} Et filter designes ud fra dets responser. Den mest benyttede metode kaldes filterkernen eller Finite Impulse Response (FIR) filter, hvor signalets input summeres med filterets impulsrespons. En anden metode kaldes et rekursivt filter eller Infinite Impulse Response (IIR) filter, hvor tidligere outputværdier benyttes sammen med inputtet. For at finde impulsresponsen for et IIR filter, indsendes en impuls som input i filtret, og impulsresponsen er outputtet. Denne impulsrespons består af en sum af sinussignaler, som eksponentielt falder i amplitude. Dette resulterer i, at impulsresponsen bliver uendelig lang. \citep{Smith1997,Blandford2013} \newline
Et filter kan derudover designes ved, at filtrets steprespons eller frekvensrespons sammenholdes med impulsresponsen. Stepresponsen er integralet af impulsresponsen, som kan findes ved at indsende en stepbølge i filteret, hvoraf stepresponsen er outputtet. Frekvensresponsen kan findes ved at finde den diskrete Fourier transformationen (DFT) eller Fast Fourier transformationen (FFT) af impulsresponsen. \citep{Smith1997} 

\subsubsection{Finite Impulse Response filter}
FIR filtre er defineret som digitale filtre med et endeligt antal impulsresponser. Det vil sige, at filteret har en impulsrespons med et endeligt antal ikke-nulværdier\fxnote{nonzeros}, hvorfor filtret kan designes stabilt og med en lineær fase. \citep{Blandford2013} Den generelle formel for FIR filtre ses i \eqref{eq:fir}.
\space
\begin{flalign}
	Y[n] = \sum_{m=0}^{m} b_m X[n-m]
	\label{eq:fir}
\end{flalign}
\space
Der ses i \eqref{eq:fir}, at filteret benytter tidligere og nutidige inputs. Dette er den afgørende faktor for, at responsen har et endeligt antal impulsresponser. Hvorledes filtret designes bestemmes ud fra 'b' filterkoefficienten.  \newline
FIR filtre optræder som forskellige konfigurationer, heriblandt Parks-McClellan algoritmen, frekvens sampling, window type og moving average.~\citep{Blandford2013}

\subsubsection{Infinite Impulse Response filter}
Et IIR filter er defineret som et digitalt filter med et uendeligt antal impulsresponser. Derfor har dette filter en impulsrespons med uendeligt mange nulværdier\fxnote{zeros}, hvilket gør at filterets impulsresponsen falder eksponentielt i amplitude og resulterer i en uendelig respons. Filteret kan derfor kun tilnærmelsesvis designes med en lineær fase, hvorfor det kan risikere at være ustabilt. \citep{Blandford2013} Af den generelle formel for IIR filtre i \eqref{eq:iir} fremgår det, at filteret benytter tidligere og nutidige inputs men også tidligere outputs, hvormed det får uendeligt mange impulsresponser. 
\space
\begin{flalign}
	Y[n] = \sum_{k=1}^{k} a_k Y[n-k] + \sum_{m=0}^{m} b_m X[n-m]
	\label{eq:iir}
\end{flalign}
\space 
IIR filtre optræder som forskellige filterkonfigurationer, heriblandt Butterworth, Chebyshev og elliptisk. Den afgørende faktor for, hvilket filter der benyttes er 'a' og 'b' filterkoefficienterne. Disse kan alle benyttes til lavpas-, højpas-, båndpas- eller båndstopfilter og designes ud fra krav om ripples, linearitet, dæmpningsgrad og faseforskydelse. \citep{Blandford2013}