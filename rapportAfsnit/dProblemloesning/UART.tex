\subsection{Universal Asynchronous Receiver Transmitter}
En UART er en enhed, der implementerer seriel data og er dermed et led mellem et parallelt og et serielt interface, der både modtager data (RX) og sender data (TX). Dataen sendes som bits og kan både sendes serielt eller parallelt. \citep{Jimb02016a,Chun-zhiYin-shuiLun-yao2011} Parallelt bliver flere bits overført på samme tid, hvormed et 8-bit system er nødsaget til at have otte ledninger, hvor der i den ene ende er en RX kobling og i den anden er TX kobling. \citep{Jimb02016a} Den serielle kommunikation foregår asynkront og kan foregå med én ledning, da kun en bit overføres af gangen. Den serielle kommunikation benyttes oftest da der ikke kræves så mange ledninger. Det er i den serielle kommunikation nødvendigt at bestemme en baudrate\fxnote{baudrate = Hvor hurtigt der sendes data (bits per sekund[bps])}, da TX sender og RX sampler i forhold til denne. Det modtagne data gemmes ofte i en buffer, som efterfølgende videregives i form af firs-in-first-out (FIFO). \citep{Jimb02016a}\newline 
Overførsel af data sker asynkront, som det ses på \figref{fig:asynkron}. Ved denne kommunikationsform opererer RX og TX ved to forskellige clocks. For at kunne lave dataen synkront sættes et startbit og et stopbit, og alt udenfor disse overføres ikke, hvilket er illustreret på \figref{fig:asynkron}. UARTens opgave er at læse data fra FIFO parallel data, som laves om til seriel data. Dette data kan så sendes til andre enheder. Når RX i en anden enhed modtager et startbit, laves data om fra seriel til parallel data, hvorefter data kan skrives til modtager-FIFO. \citep{Jimb02016a,Chun-zhiYin-shuiLun-yao2011}
\begin{figure}[H]
	\centering
	\includegraphics[scale=0.6]{figures/bProblemloesning/asynkron.png}
	\caption{På figuren ses asynkron dataoverførsel. \citep{Jimb02016}}
	\label{fig:asynkron}
\end{figure}
Nogle enheder indeholder mere end en seriel-linje, og disse enheder fungerer enten som fuld-duplex eller halv-duplex. Fuld-duplex enheder kan både sende og modtage data på samme tid. Halv-duplex enheder sender og modtager data på skift. \citep{Jimb02016a,Chun-zhiYin-shuiLun-yao2011}