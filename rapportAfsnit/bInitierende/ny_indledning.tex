\textit{Dette kapitel belyser de samfundsmæssige problemstillinger, som forekommer i forbindelse med inaktive børn. De opstillede problemstillinger vil danne grundlag for et initierende problem, som yderligere undersøges i problemanalysen.}

\section{Indledning}
Fysisk inaktivitet er et stigende problem i det danske samfund, hvor 45~\% af danske børn i alderen 11-15 år er fysisk inaktive \citep{Sundhedsstyrelsen2006}. Desuden påpeger studier, at menneskers fysiske aktivitetsniveau er faldende med alderen. Som følge af et lavt fysisk aktivitetsniveau kan dette medføre en række helbredsmæssige konsekvenser \citep{Sundhedsstyrelsen2006}. Dette har resulteret i, at fysisk inaktivitet er relateret til 4.500 dødsfald årligt. Endvidere er det påvist, at fysisk inaktive personer ofte lever 5-6 år mindre end fysisk aktive personer. \citep{JuelSoerensenBroennum-Hansen2006} Det anses derfor som væsentligt at give børn fysisk aktive vaner i barndommen, for dermed at sikre et højt helbredsniveau \citep{L.MeyerP.Gullotta2012}. \newline
Fysisk inaktivitet kan være medvirkende til en række helbredsmæssige konsekvenser, heriblandt overvægt. Overvægtige børn har en stor risiko for at udvikle
livsstilssygdomme, såsom type-2-diabetes og hjertekarsygdomme. Ydermere har undersøgelser vist, at overvægtige børn har 70~\% risiko for at forblive overvægtige som voksne. \citep{Reilly2006}. Overvægt, og særligt fysisk inaktivitet, har desuden en stor betydning for barnets psykiske velvære. Danske børn har det seneste årti haft en faldende vurdering af deres livstilfredshed, som følge af deres fysiske fremtonen og formåen \citep{Universitet2014,StatensInstitutforFolkesundhed2007}. \newline
Fysisk inaktivitet har helbredsmæssige konsekvenser for den pågældende person, hvilket ydermere kan medføre konsekvenser for samfundet. Studier har påvist, at fysisk inaktivitet er relateret til et årligt medforbrug på 3,1 milliarder kroner for det danske sundhedsvæsen. \citep{JuelSoerensenBroennum-Hansen2006}

I sammenhæng med moderne teknologi og udviklingen af elektroniske spil samt sociale medier, foretrækker mange børn stillesiddende aktiviteter fremfor fysiske aktiviteter \citep{Universitet2014}. Dette har medført konsensus om, at teknologiens udvikling er en af hovedårsagerne til at fysisk inaktivitet er en stigende tendens, særligt hos børn \citep{Kiens2007}. \newline
Særligt børn i den tidlige pubertet har fået et øget tidsforbrug i forbindelse med stillesiddende aktiviteter. En undersøgelse har vist, at 15~\% af 11-årige børn i år 2000 brugte mere end 4 timer på elektroniske spil. I år 2014 var der sket en fordobling af dette tal, hvormed 30~\% af 11-årige brugte mere end 4 timer på elektroniske spil. \citep{Universitet2014} \newline
Der fremkommer en tydelig sammenhæng mellem fysisk inaktivitet og teknologiens udvikling. Dette kan være som følge af børns psykiske tilstand, idet særligt børn i den tidligere pubertetsalder finder spil og leg interessant \citep{Wied2011}. Spil og leg i forbindelse med teknologi er dermed motiverende elementer for børn som skal udføre en aktivitet. En sammenkobling af disse motiverende elementer og fysisk aktivitet har firmaet PlayWare implementeret på en række legepladser. PlayWare indeholder intelligent teknologi som motiverer børn til at få et øget fysisk aktivitetsniveau. Denne sammenkobling af teknologi, leg og fysisk aktivitet som PlayWare benytter, har resulteret i et øget fysisk aktivitetsniveau idet teknologien initierede en række fysiske aktiviteter hos børnene \citep{Rishoej2010}. 

\section{Initierende problemstilling}
Der er et stigende antal børn, som i dag er inaktive og overvægtige. Inaktive børn, der lever en stillesiddende livsstil, udsættes med forøget risiko for en lang række følgesygdomme. For at kunne motivere børn til en mere fysisk aktiv hverdag ønskes der en teknologisk tilgang til problemet:

\begin{center}
\textit{Hvilke teknologiske muligheder findes der for at motivere inaktive og overvægtige børn til et øget aktivitetsniveau?}
\end{center}