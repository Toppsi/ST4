\textit{Dette kapitel belyser de samfundsmæssige problemstillinger, som forekommer i forbindelse med fysisk inaktive børn. De opstillede problemstillinger vil danne grundlag for et initierende problem, som yderligere undersøges i problemanalysen.}

\section{Indledning}
Fysisk inaktivitet er et problem i det danske samfund, da 45~\% af danske børn i alderen 11-15 år er fysisk inaktive. Desuden påpeger studier, at menneskets fysiske aktivitetsniveau er faldende med alderen. Der kan opstå en række helbredsmæssige konsekvenser som følge af et lavt fysisk aktivitetsniveau. \citep{Sundhedsstyrelsen2006} Dette har resulteret i, at fysisk inaktivitet er relateret til 4.500 dødsfald årligt i Danmark. Endvidere er det påvist, at fysisk inaktive danskere ofte lever 5-6 år mindre end fysisk aktive personer. \citep{JuelSoerensenBroennum-Hansen2006} Dermed bør fysisk aktive vaner inkorporeres i barndommen for at afhjælpe problemet tidligst muligt. %Det antages, at hvis fysisk aktive vaner inkorporeres i barndommen, vil dette sikre et bedre helbred. \citep{L.MeyerP.Gullotta2012}. \newline
Overvægt kan være en af de helbredsmæssige konsekvenser som resultat af fysisk inaktivitet. Overvægtige børn har i højere grad end normalvægtige børn risiko for at udvikle
livsstilssygdomme, såsom type-2-diabetes og hjertekarsygdomme. Ydermere har undersøgelser vist, at overvægtige børn har 70~\% risiko for at forblive overvægtige som voksne, hvormed risikoen for livsstilsygdomme forstørres. \citep{Reilly2006} Overvægt og særligt fysisk inaktivitet har desuden en stor betydning for barnets psykiske velvære. Danske børn har det seneste årti haft en faldende vurdering af deres livstilfredshed, hvilket blandt andet kommer til udtryk på baggrund af deres vurdering af fysiske fremtonen og formåen \citep{Universitet2014,StatensInstitutforFolkesundhed2007}. \newline
%Fysisk inaktivitet har helbredsmæssige konsekvenser for den pågældende person, 
Fysisk inaktivitet kan medføre konsekvenser for samfundet. Dette er et resultat af, at flere børn bliver inaktive, hvormed en stigning i antallet af overvægtige børn kan forekomme. I takt med at størstedelen af de overvægtige børn forbliver overvægtige som voksne, antages det, at tilfælde af livsstilssygdomme i relation med inaktivitet og overvægt vil stige. En stigning af livsstilssygdomme vil medføre et merforbrug på 3,1 milliarder kroner, hvorfor inaktive børn er et problem for det danske sundhedsvæsen. \citep{JuelSoerensenBroennum-Hansen2006}

I sammenhæng med udviklingen af moderne teknologi og af elektroniske spil foretrækker mange børn stillesiddende aktiviteter fremfor fysiske aktiviteter \citep{Universitet2014}. Dette har medført konsensus om, at teknologiens udvikling er en af hovedårsagerne til, at fysisk inaktivitet er en stigende tendens hos børn \citep{Kiens2007}.
Særligt børn i den tidlige pubertet har fået et øget tidsforbrug i forbindelse med stillesiddende aktiviteter. En undersøgelse har vist, at 15\% af danske 11-årige i år 2000 brugte mere end fire timer dagligt på elektroniske spil. I år 2014 var der sket en fordobling af dette tal, hvor 30\% af danske 11-årige brugte mere end fire timer dagligt på elektroniske spil. \citep{Universitet2014} \newline
Der forekommer en tydelig sammenhæng mellem fysisk inaktivitet og teknologiens udvikling. Dette kan være som følge af børns psykiske tilstand, idet særligt børn i den tidlige pubertetsalder finder spil og leg interessant \citep{Wied2011}. Spil og leg kan dermed i forbindelse med teknologi være motiverende for børn, som skal udføre en aktivitet. En sammenkobling af disse motiverende elementer og fysisk aktivitet har eksempelvis firmaet PlayWare implementeret på en række legepladser. PlayWare indeholder intelligent teknologi, som motiverer børn til at få et øget fysisk aktivitetsniveau. Denne sammenkobling af teknologi, leg og fysisk aktivitet, som PlayWare benytter, har resulteret i et øget fysisk aktivitetsniveau, idet teknologien initierede en række fysiske aktiviteter hos børnene \citep{Rishoej2010}. \\
En teknologi, der benytter teknologiske elementer som motiverende faktor, kan derfor have potentialet til at øge det fysiske aktivitetsniveau hos børn.

\section{Initierende problemstilling}
Fysisk inaktivitet blandt danske børn er et stort problem, hvilket blandt andet kommer til udtryk ved følgesygdommene heraf. Disse indbefatter fysiske såvel som psykiske konsekvenser for den pågældende person. Ydermere medfører disse helbredsmæssige konsekvenser et årligt merforbrug på 3,1 milliarder kroner for det danske sundhedsvæsen. Der er dermed et behov for at sænke antallet af fysisk inaktive børn med henhold til helbredsmæssige og økonomiske parametre. Studier har vist, at børn kan få et øget aktivitetsniveau ved en kombination af teknologi og fysisk aktivitet. Det er derfor væsentligt at undersøge: 
%\textbf{Gamle forslag}
%Der er et stigende antal børn, som i dag er inaktive og overvægtige. Inaktive børn, der lever en stillesiddende livsstil, udsættes med forøget risiko for en lang række følgesygdomme. For at kunne motivere børn til en mere fysisk aktiv hverdag ønskes der en teknologisk tilgang til problemet:

\begin{center}
\textit{Hvilke teknologiske muligheder findes der for at motivere fysisk inaktive børn til et øget fysisk aktivitetsniveau?}
\end{center}