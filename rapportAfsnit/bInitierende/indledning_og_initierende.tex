\chapter{Introduktion}
\textit{Dette kapitel belyser de samfundsmæssige problemstillinger som forekommer i forbindelse med inaktive børn.
De opstillede problemstillinger vil danne grundlag for et initierende problem, som yderligere undersøges i en problemanalyse.}

\section{Indledning}
Inaktivitet er et stigende problem hos danske børn. To tredjedele af danske børn i alderen, 11 år til 15 år, er inaktive\citep{SundhedsstyrrelsenFaktaark}. Disse børn har derfor ikke et tilstrækkeligt aktivitetsniveau, hvilken kan  afspejles i antallet af overvægtige børn. Hvert femte danske barn er overvægtigt, og hvert tiende barn er svært overvægtigt\citep{Universitet2014}. Antallet af overvægtige børn er dermed tredoblet de seneste 30 år\citep{Vindum2012}. \newline
Inaktivitet hos børn kan være som følge af en række faktorer. Særligt stillesiddende aktiviteter er væsentlig faktor til inaktivitet hos børn. En dansk undersøgelse har vist, at over halvdelen af alle børn er inaktive mindst fire timer hver dag, i forbindelse med videospil og TV\citep{Universitet2014}.

Inaktivitet og manglende motion kan medføre helbredsmæssige konsekvenser for den pågældende person. Inaktivitet har vist sig at være årsag til udviklingen af en række kroniske sygdomme, herunder overvægt. \newline 
Overvægt hos børn kan få en række helbredsmæssige følger. Overvægtige børn har en stor risiko for at udvikle livsstilssygdomme, heriblandt type-2-diabetes og hjertekar-sygdomme. Undersøgelser har vist, at overvægtige børn har 70\% risiko for at blive overvægtige som voksen.\citep{Reilly2006}
Der forefindes dokumenterede fordele ved at fokusere på børn aktivitet, med henblik på at undgå livsstilssygdomme i barndommen og særligt i voksenlivet. En målrettet indsats som har potentiale til at øge aktivitetsniveauet, og muligvis nedbringe antallet af overvægtige børn, vil derfor være en fordelagtig økonomisk investering\citep{COWI2015}.

Inaktivitet og overvægt hos børn kan have en række psykosociale følger. Danske børn har over en lang årrække haft en faldende vurdering af deres livstilfredshed. Dette har resulteret i, at i 2014 havde 80\% af børnene en lav livstilfredshed\citep{Universitet2014}. Denne selvvurdering kan have betydning for barnets sociale relationer. Barnets sociale relationer kan både lide følger i barndommen og voksenlivet som følge af overvægt\citep{StatensInstitutforFolkesundhed2007}. \newline
Børns fysiologiske udvikling og dermed fysisk fremtoning, kan derfor have stor betydning for barnets senere voksenliv. Det er derfor væsentligt, at justere børns kost- og motionsvaner i en barndommen/ungdommen, således dårlige vaner ikke overføres til voksenlivet. 

\section{Initierende problemstilling}
Lille indledende tekst

\textit{Hvilke teknologiske muligheder findes der for at motivere inaktive og overvægtige børn, i folkeskolens mellemtrin, til et øget aktivitetsniveau/til en aktiv livsstil?}