\textit{Dette kapitel belyser de samfundsmæssige problemstillinger, som forekommer i forbindelse med inaktive børn. De opstillede problemstillinger vil danne grundlag for et initierende problem, som yderligere undersøges i problemanalysen.}

\section{Indledning}
Inaktivitet er et stigende problem hos danske børn, da $\sfrac{2}{3}$ af drengene og $\sfrac{4}{5}$ af pigerne i alderen 11 år til 15 år er inaktive \citep{SundhedsstyrrelsenFaktaark}. %Antallet af overvægtige børn er tredoblet de seneste 30 år, hvor hvert femte danske barn i dag er overvægtigt mens hvert tiende er svært overvægtigt. \citep{Universitet2014,Vindum2012} Inaktivitet hos børn kan være som følge af flere faktorer, hvor det særligt er stillesiddende aktiviteter der er en væsentlig faktor for inaktivitet hos børn. 
En dansk undersøgelse har vist, at der siden 2006 er sket en stigning i antallet af inaktive børn i forbindelse med elektroniske spil. I 2014 spillede mellem hvert fjerde og hvert femte barn i alderen 11-13 år elektroniske spil mindst fire timer i hverdagene. Endnu større var andelen i weekenderne, hvor halvdelen af børn i alderen 11-13 år spillede elektroniske spil i mindst fire timer dagligt. \citep{Universitet2014}  I sammenhæng med moderne teknologi  og udviklingen af elektroniske spil og sociale medier, foretrækker mange børn stillesiddende aktiviteter fremfor fysiske aktiviteter.\citep{Universitet2014} Ifølge Sundhedsstyrelsen, er man inaktiv, hvis vedkommende udøver mindre end 2,5 timers fysisk aktivitet om ugen\citep{Kiens2007}, hvilket er tilfældet for mange af disse børn. 

Denne manglende motion kan medføre helbredsmæssige konsekvenser for den pågældende person, blandt andet overvægt. Overvægtige børn har en stor risiko for at udvikle livsstilssygdomme, heriblandt type-2-diabetes og hjertekarsygdomme. Undersøgelser har vist, at overvægtige børn har 70~\% risiko for at blive overvægtige som voksne. \citep{Reilly2006} 
Overvægten som voksen kan lede til omkostninger for samfundet og individet, da det har virkninger på arbejdsmarkedsdeltagelsen gennem de nævnte livsstilsygdomme. Virkningerne er blandt andet reduktioner af arbejdsevnen, sygefravær og reduceret arbejdsfunktion ved tilstedeværelse på arbejdspladsen. Omkostningerne for samfundet ligger på ca 3 milliarder årligt, \citep{BarudThomsen2007, Sundhedsministeriet2007} og derfor er det essentielt at undgå inaktivitet i en tidlig alder hos børn.

%Samtidig findes der dokumenterede fordele ved at fokusere på børns inaktivitet, med henblik på at undgå livsstilssygdomme i barndommen og særligt i voksenlivet. En målrettet indsats, som har potentiale til at øge aktivitetsniveauet, kan muligvis nedbringe antallet af overvægtige børn og vil derfor være en fordelagtig økonomisk og sundhedsmæssig investering. \citep{COWI2015}

Psykosociale følger kan opstå hos børn der er inaktivite og eventuel overvægtige. Danske børn har over en lang årrække haft en faldende vurdering af deres livstilfredshed, hvilket kan have betydning for barnets sociale relationer. Disse kan få følger i barndommen såvel som voksenlivet som følge af overvægt. Det er derfor væsentligt at justere børns kost- og motionsvaner i barndommen. \citep{Universitet2014,StatensInstitutforFolkesundhed2007} Hvis børn skal motiveres til en mere aktiv hverdag tyder et studie på, at dette skal ske igennem leg. Arrangementer, hvor der bliver fokuseret på at informere børn om fordelene ved at være aktive, viste sig ikke at være lige så motiverende som sjove lege for børnene, hvorigennem de yder fysisk aktivitet. %Råd til indflydelsesrige voksne kan også være en motiverende faktor, fordi barnet kan søge til et forbillede for opmuntringer. 
\citep{J.Sebire2013}

I Danmark har der været flere kampagner, som fokuserer på at få bestemte målgrupper til at yde mere fysisk aktivitet eller indformere om fordelene heraf. I 2013 var der blandt andet kampagnen "Get moving" af Sundhedsstyrelsen, hvor målgruppen var børn og unge men appellerede i høj grad til forældres hjælp. Her kunne forældre via hjemmesiden hente et inspirationsdokument med 84 forskellige lege, og de kunne dele erfaringer omkring, hvordan de støtter deres børn til mere fysisk aktivitet.  \citep{Sundhedsstyrelsen2013} Der er dog stadig behov for flere motivationsmåder, hvorpå man kan gøre børn mere fysisk aktive. Dette kunne være en teknologi, der kan måle bestemte faktorer, såsom skridt, løb og cykling.  

\section{Initierende problemstilling}
%Forslag 1: 
%Børn, som er inaktive og lever en stillesiddende livsstil, udsættes med forøget risiko, for en lang række følgesygdomme. En teknologisk tilgang til problemet, som motiverer en større gruppe børn, vil for disse børn være fordelagtigt. Børnene vil med en mulig løsning, tilvendes en mere aktiv livsstil: 
%
%Forslag 2: 
%Der er et stigende antal børn, som i dag er inaktive og overvægtige. Børn, som er inaktive og lever en stillesiddende livsstil, udsættes med forøget risiko, for en lang række følgesygdomme. Det er derfor væsentligt at børnene i en tidlig alder tilvendes en mere aktiv livsstil. Eftersom mange børn i dag bruger mange timer på elektroniske spil, kan der formegentligt appalleres til børnene gennem en teknologisk tilgang:
Der er et stigende antal børn, som i dag er inaktive og overvægtige. Inaktive børn, der lever en stillesiddende livsstil, udsættes med forøget risiko for en lang række følgesygdomme. For at kunne motivere børn til en mere fysisk aktiv hverdag ønskes der en teknologisk tilgang til problemet:

\begin{center}
\textit{Hvilke teknologiske muligheder findes der for at motivere inaktive og overvægtige børn til et øget aktivitetsniveau?}
\end{center}