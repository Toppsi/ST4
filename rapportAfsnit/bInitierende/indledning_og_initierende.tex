\textit{Dette kapitel belyser de samfundsmæssige problemstillinger som forekommer i forbindelse med inaktive børn. De opstillede problemstillinger vil danne grundlag for et initierende problem, som yderligere undersøges i en problemanalyse.}

\section{Indledning}
Inaktivitet er et stigende problem hos danske børn. To tredjedele af danske børn i alderen, 11 år til 15 år, er inaktive\citep{SundhedsstyrrelsenFaktaark}. Disse børn har derfor ikke et tilstrækkeligt aktivitetsniveau, hvilket kan  afspejles i antallet af overvægtige børn. Hvert femte danske barn er overvægtigt, og hvert tiende er svært overvægtigt.\citep{Universitet2014} Antallet af overvægtige børn er tredoblet de seneste 30 år\citep{Vindum2012}. Inaktivitet hos børn kan være som følge af flere faktorer, hvor det særligt er stillesiddende aktiviteter der er en væsentlig faktor for inaktivitet hos børn\fxnote{Kilde}. En dansk undersøgelse har vist, at over halvdelen af alle børn er inaktive mindst fire timer hver dag, i forbindelse med videospil og TV\citep{Universitet2014}. 

Inaktivitet og manglende motion kan medføre helbredsmæssige konsekvenser for den pågældende person. Inaktivitet har vist sig at være årsag til udviklingen af overvægt samt en række kroniske sygdomme. Overvægtige børn har en stor risiko for at udvikle livsstilssygdomme, heriblandt type-2-diabetes og hjertekar-sygdomme. Undersøgelser har vist, at overvægtige børn har 70\% risiko for at blive overvægtige som voksen.\citep{Reilly2006} Der er dokumenterede fordele ved at fokusere på børns aktivitet, med henblik på at undgå livsstilssygdomme i barndommen og særligt i voksenlivet. En målrettet indsats som har potentiale til at øge aktivitetsniveauet, og muligvis nedbringe antallet af overvægtige børn, vil derfor være en fordelagtig økonomisk investering\citep{COWI2015}.

Inaktivitet og overvægt hos børn kan have en række psykosociale følger. Danske børn har over en lang årrække haft en faldende vurdering af deres livstilfredshed, og i 2014 havde 80\% af børnene en lav livstilfredshed\citep{Universitet2014}. Denne selvvurdering kan have betydning for barnets sociale relationer. Barnets sociale relationer kan få følger i barndommen såvel som voksenlivet, som følge af overvægt \citep{StatensInstitutforFolkesundhed2007}. \newline
Børns fysiologiske udvikling og dermed fysiske fremtoning, kan derfor have stor betydning for barnets voksenliv. Det er derfor væsentligt, at justere børns kost- og motionsvaner i barndommen. \newpage

\section{Initierende problemstilling}
Børn, som er inaktive og lever en stillesiddende livsstil, udsættes med forøget risiko, for en lang række følgesygdomme. En teknologisk tilgang til problemet, som motiverer en større gruppe børn, vil for disse børn være fordelagtigt. Børnene vil med en mulig løsning, tilvendes en mere aktiv livsstil: 

\begin{center}
\textit{Hvilke teknologiske muligheder findes der for at motivere inaktive og overvægtige børn, til et øget aktivitetsniveau?}
\end{center}