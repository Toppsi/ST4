% !TeX spellcheck = da_DK
\section*{Forord}
Denne rapport er udarbejdet som et 4. semesters projekt på bacheloruddannelsen i Sundhedsteknologi på Aalborg Universitet. Projektetperioden forløb fra 1. februar 2016 til 27. maj 2016. \\
Projektet tager udgangspunkt i studieordningen for bacheloruddannelsen i Sundhedsteknologi. Semesterets fokusområde er 'Behandling af fysiologiske signaler', hvor dette projekt tager udgangspunkt i projektforslaget 'Udvikling af aktivitetsmåler'. Formålet er blandt andet design, implementering og test af en prototype, der kan detektere fysisk aktivitet. Protoypen udvikles med henblik på at bestemme det fysiske aktivitetsniveau for børn i aldersgruppen 9-12 år. %Prototypen vil derfor involvere en dataopsamling fra analoge komponenter i form af et et accelerometer, et gyroskop og en pulssensor. Ydermere vil prototypen indeholde signal- og databehandling som muligøre digital visualisering på et grafisk brugerinterface.

%Projektet henvender sig til studerende på samme niveau eller andre interessere med et kendskab til basal analog og digital databehandling. \\
Der rettes en tak til vejleder Sabata Gervasio for et godt og lærerigt samarbejde under udarbejdelsen af denne rapport. Yderligere rettes der en tak til semesterkoordinater, John Hansen, for råd og vejledning til forståelse af semesterets nye mikrokontroller. 

\section*{Læsevejledning}
Projektet er opbygget af fem kapitler, en litteraturoversigt samt et bilag. Hvert kapitel og hovedafsnit indledes med et kursiv afsnit, som har til formål at vejlede læseren i henholdsvis kapitlets og hovedafsnittets indhold og sammenhæng i rapportens helhed.\\
Første kapitel består af en indledning og initierende problemstilling. Herefter er problemanalysen, der bearbejder den initierende problemstilling, hvilket leder frem til en problemformulering. Det tredje kapitel er problemløsning, hvori løsningsstrategi og essentielle teoretiske elementer beskrives. Yderligere indeholder kapitlet krav til prototypen og dets delelementer. Det efterfølgende kapitel består af design, implementering og test af systemets delementer samt en test af det samlede system. Afslutningsvis findes syntesen, indeholdende diskussion, konklusion og perspektivering.

Rapporten benytter Vancouver metoden til kildehenvisning. Alle benyttede kilder er at finde på side~\pageref{litteraturliste}, hvor de er listet i numerisk rækkefølge. I tilfælde, hvor kilden befinder sig inden for punktum, tilhører denne kildehenvisning indholdet i den pågældende sætning. Er kildehenvisningen placeret efter punktummet i sætningen, tilhører kilden indholdet i det foregående afsnit. \\
Tabeller og figurer er nummereret efter deres respektive afsnit, hvorfor eksempelvis figur 1.1 er den første figur i kapitel 1.

Rapporten benytter forkortelser, hvor ordet skrives ud første gang det præsenteres med tilhørende forkortelse i parentes efter ordet. Efterfølgende vil denne forkortelse blive benyttet i resten af rapporten med undtagelse af overskrifter. 