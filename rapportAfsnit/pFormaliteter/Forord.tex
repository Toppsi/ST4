% !TeX spellcheck = da_DK
\section*{Forord}
Denne rapport er udarbejdet af fem 4. semesters sundhedsteknologistuderende på Aalborg universitet. Projektet forløb fra d. 1. februar til d. 27. maj 2016 under temaet \textit{behandling af fysiologiske signaler}. Formålet med dette semester er at arbejde videre med opnået viden fra 3. semester men med mere fokus på signalbehandling og datakommunikation. Dette projekt tager udgangspunkt i projektforslaget \textit{udvikling af aktivitetsmåler}, hvor målet blandt andet var design, implementering og test af et prototypesystem, der kunne registrere skridt og cykelaktivitet. Dette mål fastholdes til dels i dette projekt, da der ønskes at aktivitetsmåleren i dette projekt kan registrere gang, løb og cykling samt intensiteten heraf.

Projektet henvender sig til studerende på samme niveau eller andre interessere med et kendskab til basal analog og digital databehandling. \\
projektaktørerne vil gerne takke vejleder Sabata Gervasio for et godt samarbejde samt John Hansen for råd og vejledning til forståelse af semestrets nye mikrokontroller

\section*{Læsevejledning}
Projektet er opbygget af 5 kapitler, litteraturoversigt samt bilag. Det første kapitel består af en indledning og initierende problemstilling. Herefter findes problemanalysen, der bearbejder den initierende problemstilling og leder ud i en problemformulering. 3. kapitel er problemløsning, hvori blandt andet løsningsstrategi, nødvendig teori hertil samt krav til systemet og dets underdele beskrives. Det efterfølgende kapitel består af design, implementering og test af systemet mindre dele samt det samlede system. Afslutningsvis findes syntesen indeholdende diskussion, konklusion samt perspektivering i det 5. kapitel.

Igennem denne rapport henvises der til kilder ved hjælp af vancouver-metoden, hvor den første kilde kaldes [1], den næste [2] og så videre. Samtlige kilder findes efter kapitel 5, hvor de står i numerisk rækkefølge. I tilfælde, hvor kilden befinder sig inden for punktum, tilhører denne reference udelukkende indholdet i denne sætning. Er referencen skrevet uden for sætningen, tilhører kilden det forrige afsnit indtil sidste kilde. Derudover er tabeller og figurer nummereret efter deres respektive afsnit, hvorfor eksempelvis figur 3.1 er den første figur i kapitel 3.

Når et ord kan angives med en forkortelse, vil dette fremgå i en parentes efter det pågældende ord. Efterfølgende vil denne forkortelse benyttes med undtagelse af overskrifter. 