% !TeX spellcheck = da_DK
\chapter*{Forord}
Denne rapport er udarbejdet som et 4. semesters projekt på bacheloruddannelsen Sundhedsteknologi, Aalborg Universitet. Projektetperioden forløb fra februar 2016 til maj 2016. \\
Projektet tager udgangspunkt i studieordningen for bacheloruddannelsen i Sundhedsteknologi. Det fremgår, at semesterets fokusområde er 'Behandling af fysiologiske signaler'. Projektgruppen har derfor valgt at 
Dette projekt tager udgangspunkt i projektforslaget 'Udvikling af aktivitetsmåler', hvor formålet blandt andet er design, implementering og test af en prototype, der kan detektere fysisk aktivitet. Protoypen udvikles med henblik på at bestemme det fysisk aktivitetsniveau for børn i aldersgruppe 9-12 år. Prototypen vil derfor involvere en dataopsamling i form af analoge komponenter fra et accelerometer, et gyroskop og en pulssensor. Ydermere vil prototypen indeholde signal- og databehandling som muligøre digital visualisering på et grafisk brugerinterface.

%Projektet henvender sig til studerende på samme niveau eller andre interessere med et kendskab til basal analog og digital databehandling. \\
Der rettes en tak til vejleder Sabata Gervasio for et godt og lærerigt samarbejde under udarbejdelsen af denne rapport. Yderligere rettes der en tak til semesterkoordinater, John Hansen, for råd og vejledning til forståelse af semesterets nye mikrokontroller. 

\section*{Læsevejledning}
Projektet er opbygget af 5 kapitler, litteraturoversigt samt bilag. Før hvert kapitel er der et indledende kursiv afsnit, som har til formål at vejlede læseren i form af det følgende kapitels indhold og sammenhæng i rapportens helhed.\\
Første kapitel består af en indledning og initierende problemstilling. Herefter er problemanalysen, der bearbejder den initierende problemstilling og leder ud i en problemformulering. Det tredje kapitel er problemløsning, hvori blandt andet løsningsstrategi og essentielle teoretiske elementer hertil beskrives. Yderligere indeholder kapitlet krav til protoypen og dets delelementer. Det efterfølgende kapitel består af design, implementering og test af prototpyens delementer samt en test af den samlede prototype. Afslutningsvis er syntesen, indeholdende diskussion, konklusion og perspektivering.

Rapporten benytter Vancouver metoden til kildehenvisning, hvor den første kilde i litteraturoversigten er [1], den næste [2] og så fremdeles. Alle benyttede kilder er at finde i kapitel 5, hvor de er listet i numerisk rækkefølge. I tilfælde, hvor kilden befinder sig inden for punktum, tilhører denne kildehenvisning indholdet i den pågældende sætning sætning. Er kildehenvisningen placeret efter punktummet i sætningen, da tilhører kilden indholdet i det ovenstående afsnit indtil forrige kildehenvisning. \\
Tabeller og figurer er nummereret efter deres respektive afsnit, hvorfor eksempelvis figur 1.1 er den første figur i kapitel 1.

Rapporten benytter forkortelser, hvor ordets fulde længde skrives første gang ordet præsenteres, med tilhørende forkortelse i parantes efter ordet. Efterfølgende vil denne forkortelse benyttes i resten af rapporten, med undtagelse af overskrifter. 