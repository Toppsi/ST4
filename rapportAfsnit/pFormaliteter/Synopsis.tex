Fysisk inaktivitet er en stigende tendens i det danske samfund, som kan føre til en række helbredsmæssige problemer som overvægt, osteoporose og hjertekarsygdomme. I Danmark er det vurderet, at 4.500 dødsfald årligt er relateret til fysisk inaktivitet. Konsekvenserne af fysisk inaktivitet kan være reversibel, hvorfor forebyggelse og behandling heraf især i en ung alder er at foretrække.\\
Derfor designes, implementeres og testes et system, som kan detektere hverdagsaktiviteterne gang, løb og cykling. Dette system giver undervejs point ud fra detekterede aktiviteter gennem en brugerflade. Derved skal dette fungere som en motiverende faktor til et øget fysisk aktivitetsniveau for målgruppen 9-12-årige børn. Der opstilles krav til hver del af det samlede system, hvilket testes undervejs. Derved burde det samlede system fungere efter hensigten.\\
Systemet er designet, implementeret og afslutningsvist testet på fem forsøgspersoner. Resultatet heraf er, at systemet til dels er funktionelt men ikke opfylder alle krav, da aktiviteterne detekteres men ikke optages med korrekt varighed. Der vurderes derfor at være optimeringsmuligheder, som afslutningsvist præsenteres og diskuteres.