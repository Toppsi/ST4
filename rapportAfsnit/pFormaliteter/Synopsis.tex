Fysisk inaktivitet er en stigende tendens i det danske samfund, som ofte leder til en række helbredsmæssige problemer som overvægt, osteoporose og hjertekarsygdomme. I Danmark er det vurderet, at 4.500 dødsfald årligt er relateret til fysisk inaktivitet. Konsekvenserne af fysisk inaktivitet er dog reversibel, hvorfor forebyggelse og behandling heraf især i en ung alder er at foretrække.\\
Derfor designes, implementeres og testes et system, som kan detektere hverdagsaktiviteterne gang, løb og cykling. Dette system giver undervejs point ud fra målte aktivitet, hvorved det skal fungere som en motiverende faktor til et øget fysisk aktivitetsniveau for målgruppen, 9-12 årige inaktive børn. Der opstilles krav til hver del af det samlede system, hvilket testet undervejs. Derved burde det samlede system fungere efter hensigten.\\
Systemet blev designet, implementeret og afslutningsvist testet på fem forsøgspersoner. Resultatet heraf var, at systemet til dels er funktionelt men ikke opfylder alle sine krav, da aktiviteterne detekteres men ikke optages med korrekt varighed. Der vurderes derfor at være optimeringsmuligheder, som afslutningsvist præsenteres og diskuteres.