\newpage \section{Problemformulering}\label{Problemformulering}
Projektets definerede målgruppe er fysisk inaktive børn i aldersgruppen 9-12 år. Disse børn er udsatte for fysisk inaktivitet, hvilket i Danmark er et stigende problem. Fysisk inaktivitet har en bred række helbredsmæssige konsekvenser. Eksempelvis overvægt, som kombineret med fysisk inaktivitet, forværrer barnets helbredsmæssige tilstand. Øget fysisk aktivitet afhjælper fysisk inaktivitet direkte men har også andre åbenlyse fordele. Et øget aktivitetsniveau kan afhjælpe og forebygge overvægt og kan derudover bidrage til en øget kognitiv aktivitet. Børn motiveres til handling forskelligt, og den valgte aldersgruppe motiveres særligt igennem spil og leg. Denne aldersgruppe benytter sig desuden af teknologiske apparater i høj grad til dagligt. %Sideløbende med at disse børn motiveres af leg og spil, har deres teknologiske tilgang udviklet sig i en grad, hvor benyttelsen teknologiske apparater er stødt stigende. 
Eksisterende teknologiske apparater benytter i dag disse motiverende faktorer til at opnå et øget aktivitetsniveau. Disse eksisterende aktivitetsmålere opfylder dog ikke alle essentielle succeskriterier, hvilket danner grundlag for forbedring. Det vil dermed være essentielt at undersøge:

%\begin{center}
%\textit{Hvordan kan en aktivitetsmåler udvikles således at fysisk inaktive børn i aldersgruppen 9-12~år motiveres til en mere aktiv livsstil?}
%\end{center}

\begin{center}
\textit{Hvordan kan en aktivitetsmåler udvikles således, at den har potentialet til at reducere antallet af fysisk inaktive børn i aldersgruppen 9-12 år?}
\end{center}

%Problemer ved Inaktivitet	
%	-Stigende problem            			    √
%		-inddrag teknoglogien    			    √ (er gjort senere)
%	-Helbred 									√
%		-Overvægt								√
%	-Socioøkonimisk				 			   -/-
%
%Gevinster ved aktivitet             			√
%	-Afhjælper inaktivitet og har mange fordele  √
%		-Nævn særligt kognitive 					√
%	-Udbyttet afhænger af intensiteten (inddrag forskellige aktiviteter - gang, løb og cykling)										   -/-
%	
%Motivation										√
%	-Hvordan motiveres børn i denne aldersgruppe - dette skal danne baggrund for hvordan vi udarbejder en løsning.						√
%	-Teknologi → motiverende faktor → aktivitet → mulig løsning. √
%	-Eksisterende teknologier skal forbedres.. Eksisterende teknologi benytter i dag disse motiverende faktorer, dog opfylder de ikke alle essentielle succeskritriterier, hvilket danner grundlag for forbedring.   		√

