\section{Problemformulering}\label{Problemformulering}
Projektets definerede målgruppe er 9-12-årige børn, idet et stort antal i denne aldersgruppe er fysisk inaktive, hvilket i Danmark er et stigende problem. Fysisk inaktivitet har en række helbredsmæssige konsekvenser. Eksempelvis overvægt, som kombineret med fysisk inaktivitet forværrer barnets helbredsmæssige tilstand. Øget fysisk aktivitet afhjælper fysisk inaktivitet direkte men har også andre åbenlyse fordele. Et øget fysisk aktivitetsniveau kan afhjælpe og forebygge overvægt samt bidrage til en øget kognitiv respons. Børn motiveres forskelligt, og den valgte aldersgruppe motiveres særligt igennem spil og leg. Denne aldersgruppe benytter desuden teknologiske apparater i høj grad. Eksisterende aktivitetsmålere benytter disse motiverende faktorer til at opnå et øget aktivitetsniveau. Disse opfylder ikke alle essentielle succeskriterier, hvilket danner grundlag for forbedring. Det vil dermed være essentielt at undersøge:

\begin{center}
\textit{Hvordan kan en aktivitetsmåler udvikles således, at den har potentialet til at forebygge og reducere antallet af fysisk inaktive børn i aldersgruppen 9-12 år?}
\end{center}

