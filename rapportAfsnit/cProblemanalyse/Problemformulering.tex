\section{Problemformulering}
Projektets definerede målgruppe er fysisk inaktive børn i aldersgruppen 9-12 år. (\textbf{Mangler teknologi.}) Disse børn er udsat for fysisk inaktivitet, hvilket i Danmark er et stigende problem. Fysisk inaktivitet har en bred række helbredsmæssige konsekvenser, eksempelvis overvægt. Overvægt kombineret med fysisk inaktivitet er en tilstand for forværrer barnets helbredsmæssige tilstand. (\textbf{EVT socioøkonomisk})


%Problemer ved Inaktivitet
%	-Stigende problem 
%		-inddrag teknoglogien
%	-Helbred 
%		-Overvægt	
%	-Socioøkonimisk
%
%Gevinster ved aktivitet
%	-Afhjælper inaktivitet og har mange fordele
%		-Nævn særligt kognitive 	
%	-Udbyttet afhænger af intensiteten (inddrag forskellige aktiviteter - gang, løb og cykling)
%	
%Motivation
%	-Hvordan motiveres børn i denne aldersgruppe - dette skal danne baggrund for hvordan vi udarbejder en løsning.
%	-Teknologi → motiverende faktor → aktivitet → mulig løsning.
%	-Eksisterende teknologier skal forbedres.. Eksisterende teknologi benytter i dag disse motiverende faktorer, dog opfylder de ikke alle essentielle succeskritriterier, hvilket danner grundlag for forbedring. 


\begin{center}
\textit{Hvordan kan en aktivitetsmåler udvikles således, at fysisk inaktive børn i aldersgruppen 9-12 år, motiveres til en mere aktiv livsstil?}
\end{center}