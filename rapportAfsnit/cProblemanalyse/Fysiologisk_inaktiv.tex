\textit{For at kunne løse den initierende problemstilling, analyseres en række aspekter af problemet. Dette gøres med henblik på at belyse problemet fra flere vinkler, hvorefter en problemformulering kan opstilles. \newline 
I dette kapitel beskrives fysiologisk inaktivitet og aktivitet samt dets indvirkning på kroppen. Derudover defineres en målgruppe for projektet, hvilket gør, at denne målgruppes motivationsfaktorer kan forklares. Dette giver nogle succeskriterier til aktivitetsmålere, der benyttes til at udvælge og analysere eksisterende aktivitetsmålere.}

\section{Effekt af fysisk aktivitet for børn}\label{sec:fysio}
\textit{I afsnittet beskrives, hvilke fysiologiske konsekvenser der er forbundet med inaktivitet og aktivitet. Det belyses hvilke forskelle der er på inaktivitet og overvægt, og hvilken der på længere sigt kan have de største fysiologiske konsekvenser. Derudover beskrives intensiteten af en fysisk aktivitet, og i hvilken grad den kognetive respons afhænger af fysisk aktivitet.}


\subsection{Fysiologisk risici ved inaktivitet}\label{subsec:inover}
Hvis et individ udfører mindre end 2,5 times fysisk aktivitet om ugen med moderat intensitet, defineres vedkommende som værende fysisk inaktiv. Moderat intensitet defineres som aktivitet hvor personen skal opnå 64-74\% af maxpuls\fxnote{Moderat intensitet svarer til 40-59\% af den maksimale iltoptagelse, eller 40-59\% af pulsreserven (maxpuls – hvilepuls), eller 64-74\% af maxpuls eller 12-13 RPE (rate of percieved excertion, Borgskala) og er yderligere defineret som fysisk aktivitet, hvor man bliver lettere forpustet men hvor samtale er mulig}. \citep{Kiens2007} Overvægt og inaktivitet hænger ofte sammen, idet inaktivitet har en stor sammenhæng med overvægt. Grundlæggende opstår overvægt som resultat af et større kalorieindtag i forhold til ligevægtsindtag. \citep{Nestle2014} Definitionen for overvægt er blandt andet defineret igennem body mass index (BMI), hvilket er forholdet mellem en persons vægt og højde \citep{Academic2016}. Der findes en specifik BMI oversigt for henholdsvis piger og drenge i aldersgruppen 2-20 år, hvor grænseområder er fast defineret for begge køn. Der er ikke signifikant forskel på denne BMI oversigt imellem kønnene, men derimod afhænger grænseområderne for BMI oversigten af alderen. \citep{DiseaseControl2015}\newline
Fysisk inaktivitet og overvægt er ikke det samme, hvoraf de helbredsmæssige konsekvenser tilsvarende ikke er ens. Det er derfor muligt at være overvægtig men samtidig have en aktiv livstil. \citep{Kiens2007} Undersøgelser viser, at en overvægtig men aktiv person kan have samme metabolske sundhed som en normalvægtig. En overvægtig person kan igennem en aktiv livsstil nedsætte insulinresistens, højt kolesterol og højt bloktryk, selvom vedkommende forbliver overvægtig. \citep{Lunau2012,Marcelino2012}

%Det er veldokumenteret, at der sker et fald i fysisk aktivitet med alderen samtidig med, at der sker en stigning i vægt \citep{Kaprio2008}. Undersøgelser tyder på, at hvis kroppens cellulære vedligeholdelse styrkes med fysisk aktivitet kan aldringsprocessen nedsættes \citep{Knight2012}. 
%Fysisk inaktivitet forstærker altså den generelle aldring og anses som værende mindst lige så farligt som overvægt. De to fænomener forekommer dog ofte samtidig, da inaktivitet kan forsage overvægt, men fysisk inaktivitet har en selvstændig helbredsmæssig betydning ligesom overvægt. Det er muligt at være overvægtig men samtidig have en aktiv livsstil. \citep{Kiens2007,Kaprio2008,Hjort1997} 

Fysisk inaktivitet kan lede til flere af de store folkesygdomme som hjertekarsygdomme, diabetes, osteoporose og psykiske lidelser. Menneskekroppen er ikke skabt til at være inaktiv, og derfor vil kroppen reagere kraftigt på det. Eksempelvis kan kroppen begynde at nedbryde knoglerne indefra, således det fysiske aktivitetsniveau får betydningen for knoglernes samlede vægt, da der ikke er behov for store og stærke knogler, hvis de ikke benyttes tilstrækkeligt. \citep{Kiens2007,Reshma2002,Martini2012} \\
Ifølge et longitudinelt studie fra Holland, hvor børn og unge blev fulgt over en 15-årig periode, har inaktivitet hos børn før puberteten alvorlige konsekvenser. Studiet konkluderede, at inaktivitet før puberteten medfører stor risiko for knoglefrakturer og mulig immobilitet herfra. Dette er et resultat af, at fysisk aktivitet i barndom og ungdom er stærkt relateret til knoglemineraltætheden i ryggen og hoften. \citep{Kemper2000} I et andet studie med 2.429 børn i alderen 5-14 år blev det konkluderet, at fysisk inaktive børn havde mere end dobbelt så stor risiko for høfeber end aktive børn \citep{Kohlhammer2006}. Inaktivitet i barndommen kan altså være særligt skadeligt, da det medfører kroniske konsekvenser.

Fysisk inaktivitet kan føre til overvægt, hvormed overvægt ligeledes kan medføre en række helbredsmæssige konsekvenser for den pågældende person. Overvægt øger risikoen for forhøjet kolesteroltal, forhøjet blodtryk og diabetes og følgesygdomme heraf som slagtilfælde og nyresygdomme. Det er dokumenteret, at der er større risiko for tidlig død, jo tidligere den pågældende person pådrager sig overvægt. Det er derfor essentielt at øge børns aktivitetsniveau og dermed mindske risikoen for inaktivitet i kombination med overvægt. \citep{Nestle2014} Derudover ses der, at overvægtige børn ofte lider af psykologiske og sociale problemer, hvilket kombineret med overvægten kan have en negativ indvirkning på barnets fremtid i forhold til uddannelse og socioøkonomiske status \citep{Academic2016}.

%Der er udarbejdet en specifik oversigt for børn i denne aldersgruppe, da et BMI på for eksempel 20 for en femårig ikke er det samme som for en tolvårig. En femårig med dette BMI vil være defineret som kraftig overvægtig, mens en tolvårig vil være inden for den normale zone. Der er ikke signifikant forskel imellem kønnene, men BMI for denne aldersgruppe afhænger meget af alderen. \citep{DiseaseControl2015}\\
%Overvægt opstår grundlæggende fordi der indtages mere energi end der forbruges. Nogle mennesker kan lagre fedt bedre end andre, hvorfor overvægt også kan være genetisk betinget. \citep{Nestle2014}\newline

%Definitionen for overvægt er globalt sat ud fra et body mass index (BMI), hvilket er forholdet mellem en persons vægt og højde\citep{Academic2016}. Der findes en BMI oversigt for henholdsvis piger og drenge i aldersgruppen 2-20 år, hvorefter grænseområderne for, hvornår en person er undervægtig, normal, overvægtig eller kraftig overvægtig er fast defineret for begge køn. Der er udarbejdet en specifik oversigt for børn i denne aldersgruppe, da et BMI på for eksempel 20 for en femårig ikke er det samme som for en tolvårig. En femårig med dette BMI vil være defineret som kraftig overvægtig, mens en tolvårig vil være inden for den normale zone. Der er ikke signifikant forskel imellem kønnene, men BMI for denne aldersgruppe afhænger meget af alderen. \citep{DiseaseControl2015}\\
%Overvægt opstår grundlæggende fordi der indtages mere energi end der forbruges. Nogle mennesker kan lagre fedt bedre end andre, hvorfor overvægt også kan være genetisk betinget. \citep{Nestle2014}\\
%Overvægt øger risikoen for højt kolesteroltal, forhøjet blodtryk og diabetes samt følgesygdomme heraf som slagtilfælde og nyresygdomme. Det er dokumenteret, at der er størst risiko for tidlig død jo yngre mennesker opnår overvægt. Det er derfor essentielt at forbedre børns aktivitet og dermed mindske risikoen for overvægt. \citep{Nestle2014} Derudover ses der, at overvægtige børn ofte lider af psykologiske og sociale problemer, hvilket kombineret med overvægten kan have en negativ indvirkning på barnets fremtid i forhold til uddannelse og socioøkonomiske status \citep{Academic2016}.

Det tyder på, at inaktivitet er mere skadeligt end overvægt, hvis de sammenlignes som inaktiv normalvægtig mod aktiv overvægtig.
Inaktivitet kombineret med overvægt øger risikoen for diverse sygdomme, men en normalvægtig inaktiv person er i større risiko for tidlig dødsfald end en overvægt aktiv person. I et 12-års studie lavet over 334.161 europæiske deltagere blev fysisk aktivitet, BMI og taljemål holdt op mod dødeligheden iblandt deltagerne. Igennem studiet konkluders det, at dobbelt så mange vil dø af inaktivitet i forhold til overvægt. Det antydes igennem dette, at inaktivitet er en større risikofaktor i sammenhæng med dødelighed. \citep{Ekelund2015} 