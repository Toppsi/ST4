\textit{For at kunne løse den initierende problemstilling analyseres en række aspekter af problemet. Dette gøres med henblik på at belyse problemet fra flere vinkler. I dette kapitel beskrives fysiologisk inaktivitet og aktivitet samt dets indvirkning på kroppen. Derudover defineres en målgruppe for projektet, hvilket medfører at denne målgruppes motivationsfaktorer kan forklares. Heraf opstilles succeskriterier til aktivitetsmålere, der benyttes til at udvælge og analysere eksisterende aktivitetsmålere. Afslutningsvist præsenteres en problemformulering.}

\section{Fysiologisk effekt af fysisk aktivitet og inaktivitet for børn}\label{sec:fysio}
\textit{I afsnittet beskrives, hvilke fysiologiske konsekvenser der er forbundet med fysisk inaktivitet og aktivitet. Det belyses hvilke forskelle der er ved fysisk inaktivitet og overvægt, og hvilken der på længere sigt kan have de største fysiologiske konsekvenser. Derudover beskrives intensiteten af en fysisk aktivitet, og i hvilken grad den kognetive respons afhænger af fysisk aktivitet.}

\subsection{Fysiologiske risici ved fysisk inaktivitet}\label{subsec:inover}
Et individ, som udfører mindre end 2,5~timers fysisk aktivitet om ugen med moderat intensitet, defineres som værende fysisk inaktiv. Moderat intensitet defineres som aktivitet, hvor personen skal opnå 64-74\% af den maksimale puls\fxnote{Moderat intensitet svarer til 40-59\% af den maksimale iltoptagelse, eller 40-59\% af pulsreserven (maxpuls – hvilepuls), eller 64-74\% af maxpuls eller 12-13 RPE (rate of percieved excertion, Borgskala) og er yderligere defineret som fysisk aktivitet, hvor man bliver lettere forpustet men hvor samtale er mulig}.~\citep{Kiens2007} \\
Fysisk inaktivitet kan medføre følgesygdomme såsom hjertekarsygdomme, diabetes, osteoporose og psykiske lidelser\fxnote{OPGAVE: skriv en fxnote om hvilke psykiske lidelser, der er tale om}. Menneskekroppen er ikke skabt til at være inaktiv, og derfor vil kroppen reagere kraftigt heraf. Eksempelvis kan kroppen begynde at nedbryde knoglerne indefra som resultat af fysisk inaktivitet. Det fysiske aktivitetsniveau får således betydningen for knoglernes samlede vægt, da der ikke er behov for store og stærke knogler, hvis de ikke benyttes tilstrækkeligt.~\citep{Kiens2007,Reshma2002,Martini2012} \\
Et longitudinelt studie fra Holland fulgte børn og unge over en 15-årig periode. Studiet påviste, at inaktivitet hos børn før puberteten har alvorlige konsekvenser. Studiet konkluderede, at inaktivitet før puberteten medfører forøget risiko for knoglefrakturer og mulig immobilitet. Dette er et resultat af, at fysisk aktivitet i barndommen og ungdommen er stærkt relateret til knoglemineraltætheden i ryggen og hoften.~\citep{Kemper2000} I et andet studie med 2.429~børn i alderen 5-14~år blev det konkluderet, at fysisk inaktive børn havde mere end dobbelt så stor risiko for at udvikle høfeber end aktive børn~\citep{Kohlhammer2006}. Inaktivitet i barndommen kan dermed være skadeligt, da det kan medføre kroniske følger.

Fysisk inaktivitet kan føre til overvægt, som ydermere kan medføre en række helbredsmæssige konsekvenser for den pågældende person. Overvægt øger risikoen for forhøjet kolesteroltal, forhøjet blodtryk og diabetes samt følgesygdomme heraf, såsom slagtilfælde og nyresygdomme. Det er dokumenteret, at der er større risiko for tidlig død, jo tidligere den pågældende person pådrager sig overvægt. Det er derfor essentielt at øge børns fysiske aktivitetsniveau og dermed mindske risikoen for fysisk inaktivitet i kombination med overvægt.~\citep{Nestle2014} Derudover ses det, at overvægtige børn ofte lider af psykologiske og sociale problemer. Kombineret med overvægten kan dette have en negativ indvirkning på barnets fremtid i forhold til uddannelse og socioøkonomiske status~\citep{Academic2016}. \\
Overvægt kan opstå som følge af et større kalorieindtag i forhold til individets ligevægtsindtag.~\citep{Nestle2014} Definitionen for overvægt er blandt andet defineret af body mass index (BMI), hvilket er forholdet mellem en persons vægt og højde~\citep{Academic2016}. Der findes en specifik BMI oversigt for henholdsvis piger og drenge i aldersgruppen 2-20 år, hvor grænseområder er fast defineret for begge køn. Der er ikke betydelig forskel på BMI oversigten mellem kønnene, men derimod afhænger grænseområderne for BMI oversigten af alderen.~\citep{DiseaseControl2015}

Fysisk inaktivitet og overvægt er ikke det samme, hvoraf de helbredsmæssige konsekvenser tilsvarende ikke er ens. Det er derfor muligt at være overvægtig men samtidig have en aktiv livstil.~\citep{Kiens2007} Undersøgelser viser, at en overvægtig men aktiv person kan have samme metabolske sundhed som en normalvægtig. En overvægtig person kan igennem en aktiv livsstil nedsætte insulinresistens, højt kolesterol og højt bloktryk, selvom vedkommende forbliver overvægtig.~\citep{Lunau2012,Marcelino2012} \\
Det tyder på, at fysisk inaktivitet kan være mere skadeligt end overvægt, hvis de sammenlignes som inaktiv normalvægtig mod aktiv overvægtig. Fysisk inaktivitet kombineret med overvægt øger risikoen for en række sygdomme. Derimod er en normalvægtig fysisk inaktiv person i større risiko for tidlig dødsfald end en overvægtig fysisk aktiv person. I et 12-års studie lavet over 334.161 europæiske deltagere blev fysisk aktivitet, BMI og taljemål holdt op mod dødeligheden blandt deltagerne. Studiet konkluderer, at dobbelt så mange vil dø af fysisk inaktivitet i forhold til overvægt. Det antydes igennem dette, at fysisk inaktivitet er en større risikofaktor i sammenhæng med dødelighed.~\citep{Ekelund2015} 