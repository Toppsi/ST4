\subsubsection{Metabolisme}\label{subsub:metab}
Ved fysisk aktivitet kræver musklerne energi, hvilket dannes under anaerob og aerob processer, hvor der i begge tilfælde dannes adenosintrifosfat (ATP). ATP er musklers drivkræft, hvilket gør kontraktion og derved bevægelse muligt. Fysisk aktivitet bevirker blandt andet kroppens stofskifte og kredsløbssystem. Selv hverdagsmotion med lav intensitet som gang og cykling til og fra skole kan forbedre kroppens stofskifte, hvilket betyder lavere koncentration af kolesterol i blodet samt bedre kontrol af blodtrykket og blodsukkeret. Høj intenst aktivitet forbedrer kroppens kredsløbssystem. Den perifere modstand i kredsløbet mindskes samt blodvolumen forøges, hvorfor hjertets arbejde reduceres. \citep{Martini2012,Kiens2007,Sundhedsstyrelsen2001} \\
Aerob træning medfører fysiologiske forandringer, specielt for børn. Ifølge fransk undersøgelse kan aerob træning nedsætter børns risiko for det metabolske syndrom, som blandt andet dækker over hypertension, hyperglykæmi og abdominal fedme. På celleplan forøges kapillariseringen af fysisk aktivitet, og tætheden af kapillærerne er tæt relateret til den aerobe kapacitet. \citep{Sundhedsstyrelsen2001,Guinhouya2009} Igennem aerob træning vil der dannes flere mitokondrier, hvilket giver flere oxidative enzymer og dermed bedre forbrænding af fedt \citep{Sundhedsstyrelsen2001}.\fxnote{Den areobe proces foregår i mitokondrierne og kan danne cirka 12 gange mere ATP per gram kulhydrat i forhold til den anaerobe proces \citep{Engelbreth2010,Stanfield2013}.}