\subsection{UNICEF kid power band}
UNICEF Kid power band er en aktivitetsmåler, som appellerer til børn ved at de kan hjælpe andre børn i de ressourcefattige lande, og fører sloganet "vær aktiv og red liv". Børnene optjener point ved at være  aktive mens de har aktivitetsmåleren på. Aktivitetsmåleren indeholder både et pedometer og et 3-akse accelerometer, hvormed det både kan registrere skridt, og andre aktiviteter. Aktivitetsmåleren monteres på armen, og skridtene opfanges derfor ud fra børnenes bevægelse med armen. Børnene samler derved flere point, jo mere energiske de er gennem øvelserne. Hver dag nulstilles aktivitetsmåleren, så børnene hver dag kan følge med i hvor aktive de har været den pågældende dag. Derudover gemmer det data 30 dage tilbage, så det er muligt at sammenligne med tidligere dage. På aktivitetsmåleren er der en skærm, hvor det er muligt at følge med i klokken, antal skridt, KidPower points, fremskridt på missioner og navnet på brugeren.\citep{PowerAbout2015,PowerManual2015} \newline 
professionelle atleter står i spidsen for forskellige missioner, som børnene kan vælge at deltage i. Ved at deltage i disse missoner, er børnene ikke bare aktive, men lærer også om forskellige kulturer. Et eksempel er en mission, som basketballspilleren Tyson Chandler står i spidsen for, hvor børnene lærer om hvordan børn i ressourcefattige lande, hjælper familien med at gro deres eget mad.\citep{PowerMission2015} \newline
Mængden af aktiviteten børnene udfører, omregnes til point, og når nogle mål er nået sendes en sum penge, sponsoreret af fans, firmaer og forældre, til at hjælpe i de ressourcefattige lande. \newline
Alle resultater samles i en app, hvor børnene både har mulighed for at følge med i progressionen for dem selv og deres venner, men også for de missioner de deltager i.\citep{PowerAbout2015}


\subsubsection{Vurdering af succeskrav}
Unicef Kid power band opfylder en række af de opstillede krav i \secref{tracker_intro}. Aktivitetsmåleren giver mulighed for at tælle skridt, som både registreres under løb og gang, samt ved andre aktiviteter. Der skelnes dog ikke mellem løb og gang, og da armene ikke bevæges ved cykling er dette ikke muligt at registrere. Derudover måles der ikke puls, hvormed intensiteten af det udførte arbejde udelukkende måles ud fra hvor energisk armene bevæges under en givne øvelse. Aktivitetsmåleren er designet, så det nemt kan sættes på barnet, da det kommer i en størrelse med justerbart bånd.\citep{PowerManual2015} \newline
Børnene aktiveres socialt, da alle aktiviteter udføres med henblik på at de sammen med jævnaldrende skal hjælpe børn i de ressourcefattige lande. Derudover bliver børnene gennem appen opdateret på venners progression, samt progressionen af den mission de deltager i. Flere skoler i USA har også benyttet aktivitetsmåleren, som en del af klasseprojekter, for at få børnene til at blive mere aktive.\citep{PowerAbout2015}

Dermed opfylder Unicef Kid power band 2 ud af 6 succeskrav, mens det delvist opfylder 2 succeskrav.

\subsection{The Sqord Booster}
Sqord Booster er en aktivitetsmåler, som appellerer til børn gennem konkurrence og fællesskab. Der er en tilhørende hjemmeside, hvor al aktivitet gemmes i en avatar, som børnene selv kan designe. Aktiviteterne skal børnene selv uploade til hjemmesiden. Forældrene kan oprette et forældrelogin til siden, så de ligeledes kan følge med i deres børns aktivitet. Børnene tjener point ved at deltage i forskellige konkurrencer, hvor deres aktivitet måles gennem et 3-akse accelerometer, som måler aktivitetens intensitet og varighed. Aktivitetsmåleren placeres oftest om håndleddet, men kan også placeres i en lomme eller bundet til skoen. \newline
Aktivitetsmåleren er designet til at blive brugt i grupper, hvor børnene ikke skal at være fysisk sammen for at være aktive. De kan enten konkurere mod hinanden, eller arbejde sammen som et hold. Det er dog også muligt at benytte aktivitetsmåleren, selvom børnene ikke er i en gruppe. \noindent
Hjemmesiden hvor børnene kan følge med i deres avatar, fungerer som et forum, hvor de har mulighed for at give hinanden highfives for gode præstationer, chatte indbyrdes, eller lave talebobler, hvor alle kan se hvad de skriver. 
Sqord tilgodeser alle præstationer, da alle får en medalje ved blot at have deltaet i en given aktivitet. Vinderen får imidlertid flere point end de andre deltagere. Spillet er lavet, så alle har mulighed for at vinde, da der i det enkelte spil, ses på børnenes individuelle form, ved at se på tidligere præstationer.

\subsubsection{Vurdering af succeskrav}
Sqord Booster opfylder en række af de opstillede krav i \secref{tracker_intro}. Aktivitetsmåleren registrerer både børnenes aktivitet ved gang og løb, men kan ikke skelne mellem de to forskellige former for aktivitet, og der registreres ikke cykling. der måles derudover ikke puls, hvormed intensiteten af det udførte arbejde findes ud fra accelerometerets fart. 
Børnene bliver aktiveret socialt, da hjemmesiden er en blanding mellem et chatforum og en oversigt over præstationer. Derudover har børnene mulighed for at konkurrere med og mod hinanden. Sqord har derudover sørget for at fange både de børn der er i god form, og de der ikke er, da alle har mulighed for at vinde baseret på tidligere præstationer. Aktivitetsmåleren er mulig at placere flere steder, hvormed børnene har mulighed for at vælge en placering, hvor det er til mindst gene. Derudover er det designet efter målgruppen, hvormed aktivitetsmåleren både kan modstå stød og tåle at komme i vand.  

Dermed opfylder Sqord Booster 2 ud af 6 succeskrav, mens det delvist opfylder 2 succeskrav.