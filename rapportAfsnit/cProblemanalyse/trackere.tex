\subsection{UNICEF kid power band}
UNICEF Kid Power Band er en aktivitetsmåler, som appellerer til børn ved at hjælpe andre børn i ressourcefattige lande, hvilket fører til sloganet "vær aktiv og red liv". Børnene optjener point ved at være  aktive mens de har aktivitetsmåleren på. Aktivitetsmåleren indeholder både et pedometer og et tre-akse accelerometer, hvormed det både kan registrere skridt, og andre aktiviteter. Aktivitetsmåleren monteres på armen, og skridtene opfanges derfor ud fra børnenes bevægelse med armen. Børnene samler flere point, jo mere energiske de er gennem øvelserne. Hver dag nulstilles aktivitetsmåleren, så børnene hver dag kan følge med i hvor aktive de har været den pågældende dag. Derudover gemmes der data 30 dage tilbage, så det er muligt at sammenligne med tidligere dage. På aktivitetsmåleren er der en skærm, hvor det er muligt at følge med i klokken, antal skridt, KidPower points, fremskridt på missioner og navnet på brugeren.\citep{PowerAbout2015,PowerManual2015} \newline 
professionelle atleter står i spidsen for forskellige missioner, som børnene kan vælge at deltage i. Ved at deltage i disse missoner, er børnene ikke blot aktive, men lærer også om forskellige kulturer. Et eksempel er en mission, som basketballspilleren Tyson Chandler står i spidsen for, hvor børnene lærer om hvordan børn i ressourcefattige lande, hjælper familien med at gro deres eget mad.\citep{PowerMission2015} \newline
Mængden af aktiviteten børnene udfører, omregnes til point, og når nogle mål er nået sendes en sum penge, sponsoreret af fans, firmaer og forældre, til at hjælpe i de ressourcefattige lande. \newline
Alle resultater samles i en app, hvor børnene både har mulighed for at følge med i progressionen for dem selv og deres venner, samt for de missioner de deltager i.\citep{PowerAbout2015}


\subsubsection{Vurdering af succeskrav}
UNICEF Kid power band opfylder en række af de opstillede krav i \secref{tracker_intro}. Aktivitetsmåleren giver mulighed for at tælle skridt, som både registreres under løb og gang, samt ved andre aktiviteter. Der skelnes dog ikke mellem løb og gang, og da armene ikke bevæges ved cykling er dette ikke muligt at registrere. Derudover måles der ikke puls, hvormed intensiteten af det udførte arbejde udelukkende måles ud fra hvor energisk armene bevæges under en givne øvelse. Aktivitetsmåleren er designet, så den nemt kan sættes på barnet, da det kommer i en størrelse med justerbar rem.\citep{PowerManual2015} \newline
Børnene aktiveres socialt, da alle aktiviteter udføres med henblik på at de sammen med jævnaldrende skal hjælpe børn i de ressourcefattige lande. Derudover bliver børnene gennem appen opdateret på venners progression, samt progressionen af den mission de deltager i, hvorved det ikke kun er den individuelle præstation der er i fokus. Flere skoler i USA har i fjerde klasse også benyttet aktivitetsmåleren, som en del af klasseprojekter, for at få børnene til at blive mere aktive.\citep{PowerAbout2015}

Dermed opfylder UNICEF Kid Power Band 2 ud af 6 succeskrav, mens det delvist opfylder 2 succeskrav.

\subsection{The Sqord Booster}
Sqord Booster er en aktivitetsmåler, som appellerer til børn i alderen 8-14 år gennem konkurrence og fællesskab. Der er en tilhørende hjemmeside, hvor al aktivitet uploades og gemmes i en avatar, som børnene selv designer. Forældrene kan oprette et forældrelogin til siden, så de ligeledes kan følge med i deres børns aktivitet. Børnene tjener point ved at deltage i forskellige konkurrencer, hvor deres aktivitet måles gennem et tre-akse accelerometer, som måler aktivitetens intensitet og varighed. Aktivitetsmåleren placeres oftest om håndleddet, men kan også placeres i en lomme eller bundet til skoen. \newline
Aktivitetsmåleren er designet til at blive brugt i grupper, hvor børnene ikke fysisk skal at være sammen for at være aktive. De kan enten konkurere mod hinanden, eller arbejde sammen som et hold. Det er dog også muligt at benytte aktivitetsmåleren, selvom børnene ikke er i en gruppe. \newline
Hjemmesiden hvor børnene kan følge med i deres avatar, fungerer som et forum, hvor de har mulighed for at give hinanden highfives for gode præstationer, chatte indbyrdes, eller lave talebobler, hvor alle kan se hvad de skriver. \newline
Sqord tilgodeser alle præstationer, da alle får en medalje ved blot at have deltaet i en given aktivitet. Vinderen får imidlertid flere point end de andre deltagere. Spillet er lavet, så alle har mulighed for at vinde, da der i det enkelte spil, vurderes ud fra børnenes individuelle form, ved at se på tidligere præstationer.

\subsubsection{Vurdering af succeskrav}
Sqord Booster opfylder en række af de opstillede krav i \secref{tracker_intro}. Aktivitetsmåleren registrerer både børnenes aktivitet ved gang og løb, men kan ikke skelne mellem de to forskellige former for aktivitet, og der registreres ikke cykling. Der måles derudover ikke puls, hvormed intensiteten af det udførte arbejde findes ud fra accelerometerets fart. \newline
Børnene bliver aktiveret socialt, da hjemmesiden er en blanding mellem et chatforum og en oversigt over præstationer. Derudover har børnene mulighed for at konkurrere med og mod hinanden. Sqord har derudover sørget for at fange både de børn der er i god form, og de der ikke er, da alle har mulighed for at vinde baseret på tidligere præstationer. Aktivitetsmåleren er mulig at placere flere steder, hvormed børnene har mulighed for at vælge en placering, hvor det er til mindst gene. Derudover er det designet efter målgruppen, hvormed aktivitetsmåleren både kan modstå stød og tåle at komme i vand.  

Dermed opfylder Sqord Booster 2 ud af 6 succeskrav, mens det delvist opfylder 2 succeskrav.

\subsection{Nabi Compete}
Nabi Cpmpete er en aktivitetsmåler, som appellerer til børn over seks år gennem deres madvaner og samvær med andre. Der er muligt for børnene at konkurrerer individuelt, men hovedformålet er at konkurrere mod andre, eller med andre som et hold. Konkurrencerne kan bestå i at løbe en bestemt rute, som de selv kan tegne ind, men kan også bestå i at forbrænde nok kalorier til at have forbrændt forskelligt fastfood. Det er muligt at opnå mål sammen med andre, eller dyste i hvem der når forskellige mål først. derudover lærer børnene om kalorier og distance ved at bruge appen, hvor det er muligt at følge med i progressionen. Gennem konkurrencerne optjenes der point, som kan bruges til at købe et virtuelt dyr, som også gennem point er muligt at få til og vokse. 
Aktiviteten måles gennem at tre-akse accelerometer, som sidder i et armbånd. Dataet synkroniseres til en smartphone eller tablet via bluetooth, hvor der kan gemmes data 90 dage tilbage, så barnet og forældrene har mulighed for at følge med i barnets progression. 

\subsubsection{Vurdering af succeskrav}
Nabi Compete opfylder en række af de opstillede krav i \secref{tracker_intro}. Aktivitetsmåleren registrer både gang og løb, men det er ikke muligt at skelne mellem de to former for aktivitet. Der registreres desuden ikke cykling og puls med aktivitetsmåleren. 
Børnene aktiveres socialt, da appen er designet med mulighed for at konkurrere mod hinanden eller arbejde sammen som et hold. Derudover har børnene mulighed for at have et kæledyr på appen, hvorved de, udover konkurrence mod andre, har et formål ved at forbrænde en mængde kalorier. Aktivitetsmåleren monteres uden gene, da den er placeret i en justerbar rem, som let kan monteres om barnets håndled. Derudover er det designet så det kan tåle sved og regn, hvilket gør at børnene kan bruge det i al slags vejr. 

Dermed opfylder Nabi Compete 2 ud af 6 succeskrav, mens det delvist opfylder 2 succeskrav.

\subsection{Ibitz}
Ibitz er en aktivitetsmåler, som apellerer til børn over fem år gennem udfordringer i samarbejde med forældrene. Ibitz har egne udfordringer, men der lægges særligt op til at forældrene sætter nogle mål for børnene gennem deres dag. Dette kan være for hvornår der er legetid, hvornår de må sidde foran skærmen eller hvornår de skallave aktiviteter med forældrene. Ved at gennemføre målene forældrene har sat, eller ibitz egne udfordringer, kan børnene tjene point, som kan bruges på to forskellige spil - minecraft og Club Penguin. Aktivitetsmåleren består af et pedometer, som måler skridt, der trådløst synkroniseres med en app på en smartphone eller tablet via bluetooth. Aktivitetsmåleren monteres ved en klemme, som kan sættes på bukserne eller på skoen. Appen gemmer aktiviteter 30 dage tilbage, hvorved barnet og forældrene har mulighed for at følge med i progressionen. 

\subsubsection{Vurdering af succeskrav}
Ibitz opfylder en række af de opstillede krav i \secref{tracker_intro}. Aktivitetsmåleren registrer både gang og løb, dog er et ikke muligt at skelne mellem de to former for aktivitet, samt at registrere puls og cykling. Børnene bliver delvist aktiveret socialt, hvor det primært er sammen med familien. Derudover aktiveres børnene ved at tjene point til forskellige spil, som oftest spilles sammen med andre børn. Aktivitetsmåleren monteres uden gene, da børnene selv kan vælge mellem at montere den på buksen eller skoen. Derudover kan den tåle vand, hvorved børn også kan bruge den i regnvejr.  

Dermed opfylder Ibitz 1 ud af 6 succeskrav, mens det delvist opfylder 3 succeskrav.

\subsection{Opsummering af de udvalgte aktivitetsmålere}

\begin{table}[H]
	\centering
	\label{tab:sammenhold_tracker}
	\resizebox{\textwidth}{!}{%
		\begin{tabular}{|l|c|c|c|c|}
			\hline
			Krav                                            & \multicolumn{1}{l|}{Unicef Kid Power Band} & \multicolumn{1}{l|}{Sqord Booster} & \multicolumn{1}{l|}{Nabi Compete} & \multicolumn{1}{l|}{Ibitz} \\ \hline
			Registrere gang                                 & (x)                                        & (x)                                & (x)                               & (x)                        \\ \hline
			Registrere løb                                  & (x)                                        & (x)                                & (x)                               & (x)                        \\ \hline
			Registrere cykling                              &                                            &                                    &                                   &                            \\ \hline
			Registrere intensitet gennem puls               &                                            &                                    &                                   &                            \\ \hline
			Motivere inaktive såvel som aktive børn socialt & x                                          & x                                  & x                                 & (x)                        \\ \hline
			Monteres uden gene                              & x                                          & x                                  & x                                 & x                          \\ \hline
		\end{tabular}
	}
	\caption{Tabellen viser en oversigt over de fire aktivitetsmålere og hvorvidt de lever op til kravene. (x) betyder at de delvist lever op til kravene. x betyder at de lever op til kravene}
\end{table}

Ud fra analysen ses det at de aktivitetsmålere der i dag benyttes til børn i projektets aldersgruppe ikke lever op til samtlige af de krav der er stillet. De kan alle registrere løb og gang, men har ikke mulighed for at skelne mellem de to aktivitetsformer. Ingen af aktivitetsmålerne registrere cykling eller intensitet gennem puls. Alle aktivitetsmålerne appellerer til både inaktive og aktive børn, men kun tre af dem er rettet social aktivitet med jævnaldrende børn. Alle aktivitetsmålere er beregnet til at have rundt om armen, hvor den spændes på med en justerbar rem. Derudover er alle aktivitetsmålere designet efter at børnene både skal kunne bruge dem i såvel regnvejr som solskin.

For at optimere de aktivitetsmålere der benyttes i dag, skal de kunne skelne mellem løb gang og cykling, så barnet ikke kun kan måle hvor mange skridt de har gået, og hvor langt de er nået, men også kan måle  hvilken aktivitet der er udført. Derudover skal intensiteten af øvelsen kunne registreres med puls, da det har en afgørende betydning for det fysiologiske udbytte af den givne aktivitet, hvilket kan ses på \figref{fig:PA_Procentpuls} i \secref{sec:fysio}.\newline
Aktivitetsmåleren skal, som de der findes i dag, aktivere børnene socialt sammen med jævnaldrene. Derudover skal aktiviteterne foregå i leg eller spil, som både skal være baseret på konkurrence mod og sammenspil med andre. 
