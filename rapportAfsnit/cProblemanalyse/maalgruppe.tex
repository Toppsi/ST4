\section {Udsat målgruppe for inaktivitet}
\textit{Dette afsnit omhandler hvilken aldersgruppe af inaktive børn som vil kunne påvirkes til en mere aktiv livsstil som resultat af en teknologisk mulighed. Afsnittet beskriver hvilken aldersgruppe af børn i folkeskolen som har den største tendens til at være inaktive og hvilken aldersgruppe der vil tilknytte sig bedre aktivitetsvaner. Yderligere beskrives det hvilken aldersgruppe der vil finde leg og teknologi motiverende.}
				
Det er påvist at hvert femte barn er overvægtig, hvilket formentlig er et resultat af at danske børn er blevet mindre fysisk aktive\citep{Universitet2014}. Denne tendens sænkes ikke når ovenstående aldersgruppe overskrides, da både danske og internationale studier hævder at to tredjedele af børn i aldersgruppen 11-15 år er fysisk inaktive. De er altså fysisk aktive i mindre end 2,5 time ugentligt. 
Der er flere bud på verdensplan om, hvad definitionen for fysisk inaktivitet er. Sundhedsstyrelsen har derfor udarbejdet en generel definition ud fra de flere forskellige som lyder, at et individ er fysisk inaktiv, hvis vedkommende udfører mindre end 2,5 times fysisk aktivitet om ugen med moderat intensitet\fxnote{Moderat intensitet svarer til 40-59\% af den maksimale iltoptagelse, eller 40-59\% af pulsreserven (maxpuls – hvilepuls), eller 64-74\% af maxpuls eller 12-13 RPE (rate of percieved excertion, Borgskala) og er yderligere defineret som fysisk aktivitet hvor man bliver lettere forpustet men hvor samtale er mulig.}.\citep{Kiens2007}
Denne målgruppe er under en udvikling, hvor tendensen tyder på, at desto ældre disse børn bliver, jo mindre fysisk aktive er de. Halvdelen af børnene i denne aldersgrupper ønsker at leve en mere aktiv livsstil, med den rette mængde fysisk aktivitet. Med stor overvægt er det de inaktive børn, som har dette ønske, hvilket oftest ikke opnås. Det kan antages at de mangler den primære motivation for at opfylde denne lyst. Samtidig med, at en stor del af denne aldersgruppe er inaktive, så er antallet af 10-13 årige børn der cykler i skole, de seneste 15 år, faldet med 30\%. \citep{Sundhedsstyrelsen2006}

Børn inkorporerer vaner til forskellige tiltag, i forskellige aldre af deres barndom. Når børn nærmer sig teenageårene bliver de generelt bevidste om en bred række vaner, blandt andet aktivitetsvaner. I og med at børnene bliver bevidste om flere vaner, så til- og fravælges vaner ligeledes. Dette medfører en ydre påvirkning med henblik på øget aktivitet, ville være optagelig for børn i aldersgruppen op mod teenageårene, og dermed fordelagtigt for at modarbejde statistikken for inaktivitet. \citep{Laub2011} Når børnene når teenagealderen bliver fest og alkohol også et interessant bekendtskab. Et bekendtskab 20 \%, af 13 årige børn kender til, i en grad hvor de har været fulde \citep{Sundhedsstyrelsen2016a}. Ikke langt fra denne alder, kan det antages at barnlig leg ikke længere fanger interessen, som tidligere. Dette er et resultat af at børnene når puberteten og mentalt udvikles, og påbegynder en tilnærmelse af modning. \citep{Skovby2014}

Tilstedeværelsen af teknologi for børn er et tilfælde som ikke længere er en sjældenhed. Antallet af teknologiske gadgets som var til stede i folskeskolerne ved børn i alderen 5-8 år er steget drastisk inden for de seneste par år. I 2013 havde 3 \% i denne aldersgruppe teknologiske gadgets med i skole hverdag, og i 2014 var dette steget til 33\%. Denne tendens hvor teknologiske gadgets medbringes dagligt stiger meget i aldersgruppen de efterfølgende år. Dette har medført, at i 2014 havde 87 \% af børn i aldersgruppen 9-12 år dagligt medbragt teknologiske gadgets. \citep{GjensidigeForsikring2014}

Det ønskes at reducere mængden af inaktive børn i folkeskolerne, og for at inddrage den målgruppe hvor en teknologisk metode vil have størst påvirkning, så skal ovenstående problemstillinger sammenkobles. Børn i alderen fra 10 år og op, er i den kategori hvor størstedelen er inaktive. Ligeledes er denne aldersgruppe også problematisk da 30\% færre cykler til skole. De befinder sig i en alder hvor vaner tages til efterretning og de er vant til at omgås teknologiske gadgets. Dette betyder at en teknologisk mulighed for at motivere inaktive børn til et øget aktivitetsniveau ville have størst påvirkning på børn i alderen fra 10 år og op. Dog når børnene kommer ind i puberteten fjernes fokus oftest fra barnlig leg og andre interesser, dermed skal disse kommende teenagere ikke inkluderes som en del af målgruppen. 

Dermed er målgruppen for dette projekt defineret som: 

\begin{itemize}
\item Børn der befinder sig i folkeskolens mellemtrin, som har alderen 9-13 år.
\end{itemize}