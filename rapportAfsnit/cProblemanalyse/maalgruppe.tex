\section {Udsat aldersgruppe for inaktivitet}
\textit{Afsnittet præciserer en målgruppe for dette projektet i forhold til, hvilken aldersgruppe der er hensigtsmæssig at vælge, hvis inaktivitet skal mindskes i fremtiden. Derudover fokuseres der på hvordan børnene kan aktiveres.}
%\textit{Dette afsnit omhandler, hvilken aldersgruppe af inaktive børn, som vil kunne påvirkes til en mere aktiv livsstil som resultat af en teknologisk mulighed. Afsnittet beskriver, hvilken aldersgruppe af børn i grundskolen, der har den største tendens til at være inaktive, og hvilken aldersgruppe der vil tilknytte sig bedre aktivitetsvaner. Yderligere beskrives der, hvilken aldersgruppe, som vil finde leg og teknologi motiverende.}

Den teknologiske udvikling har stor betydning for den stigende andel af inaktive danskere %(den gamle sætning) Endvidere menes der, at den teknologiske udvikling kan have stor betydning for inaktivitet 
\citep{Kiens2007}. Ifølge Sundhedsstyrelsen var 45\% af danske 11–15 årige fysisk inaktive i 2006 \citep{Sundhedsstyrelsen2006}. Derudover mener de, at børn og unge bliver mindre aktive med alderen, hvilket kan have en sammenhæng med, at tilstedeværelsen af teknologi for børn stiger med alderen. %Denne målgruppe er under en udvikling, hvor tendensen tyder på, at desto ældre børnene bliver, jo mindre fysisk aktive er de. Halvdelen af børnene i denne aldersgruppe ønsker at leve en mere aktiv livsstil, med den rette mængde fysisk aktivitet. Med stor overvægt er det de inaktive børn, som har dette ønske, hvilket oftest ikke opnås. Det kan antages at de mangler den primære motivation for at opfylde denne lyst. Samtidig med, at en stor del af denne aldersgruppe er inaktive, så er antallet af 10-13 årige børn der cykler i skole, de seneste 15 år, faldet med 30\%. \citep{Sundhedsstyrelsen2006} 
I 2013 havde 3\% af børn i alderen 5-8 år teknologiske apparater med i skole hverdag, og i 2014 var dette steget til 33\% for samme aldersgruppe. Denne tendens, hvor teknologiske apparater medbringes dagligt, stiger med alderen, da 87\% af børn i aldersgruppen 9-12 år dagligt medbragt teknologiske apparater i 2014. \citep{Sundhedsstyrelsen2006,GjensidigeForsikring2014}

Børns vaner i forhold til deres fysiske aktivitetsniveau dannes i barndommen og den tidlige pubertetsalder \citep{F.SallisG.Simons-MortonJ.Stone1992}. For denne alder har autoritære roller, såsom forældre og lærere, fortsat en stærk påvirkning med henhold til at inkorporere vaner hos børnene \citep{L.MeyerP.Gullotta2012}. \newline
Det anses som nødvendigt, at børn vænnes til at være fysisk aktive i en tidlig alder, da vaner bringes med videre til voksenlivet. Hvis ikke børnene får en fysisk livsstil, vil vænnes de til en stillesiddende adfærd \citep{Nabe-NielsenSundhedsministerietetal.2005,P.J.KremersBrug2008,L.MeyerP.Gullotta2012}. Endvidere påpeger studier, at det er fordelagtigt at give børn gode vaner før puberteten. Dette skyldtes en række fysiske og psykiske faktorer, som børnene undergår i puberteten. Gode vaner med en fysisk aktiv livsstil skal dermed videreføres til børnene forinden folkeskolens udskoling. 

Der ønskes at reducere antallet af inaktive, hvormed der med fordel kan appelleres til børn inden pubertetsaleden. Når børnene aktiveres i denne aldersgruppe, er chancen større for videreførelse af vaner. For at aktivere børnene kan det med fordel gøres gennem teknologi, da børnene i stigende grad benytter dette, hvilket er en af de store grunde til inaktivitet. Der ønskes dermed at optimere aktivitetsniveauet for børn i aldren 9-12 år, da det er denne aldersgruppe der især bruger teknologien i for høj en grad. 

%Det ønskes at reducere antallet af inaktive børn ved at appellere til en målgruppe, som er bekendte og fortrolige med teknologiske apparater. Ydermere skal der være mulighed for at påvirke målgruppens vaner i forhold til aktivitetsniveau. 
%I den forbindelse anses børn i den tidlige pubertet som en essentiel målgruppe for videreførelsen af en aktiv livsstil. Dette gøres på baggrund af at børn i denne alder stadig har mulighed for at tilegne sig nye vaner, samtidig med at de har et stort kendskab til teknologier. 


%Det antages, at hvis ikke skolerne engagerer børnene til fysisk aktivitet, vil helbredsniveauet blive dårligere end tidligere. Skolerne skal derved være forløber for at give børnene gode vaner hvad angår deres fysiske aktivitetsniveau. \citep{L.MeyerP.Gullotta2012} \newline
%% Eksempel på afrunding %% 
%Det ønskes at reducere mængden af inaktive børn i grundskolen, og for at inddrage den målgruppe hvor en teknologisk metode vil have størst påvirkning, så skal ovenstående problemstillinger sammenkobles. Børn i alderen fra 11-15 år, er overvejende inaktive, og ligeledes er denne aldersgruppe også problematisk da 30~\% færre, fra 10 år, cykler til skole. De er i en alder hvor vaner tages til efterretning og de er vant til at omgås teknologiske apparater. Dette betyder at en teknologisk mulighed for at motivere inaktive børn til et øget aktivitetsniveau vil have størst påvirkning på børn i alderen fra 10 år og op. Når børnene kommer ind i puberteten fjernes fokus dog oftest fra barnlig leg og andre interesser, hvormed kommende teenagere ikke skal inkluderes som en del af målgruppen. 

Dermed er målgruppen for dette projekt defineret som børn i aldersgruppen 9-12 år.