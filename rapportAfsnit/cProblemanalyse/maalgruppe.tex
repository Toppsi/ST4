\section{Udsat aldersgruppe for inaktivitet} \label{sec:maalgruppe}
\textit{Dette afsnit præciserer en målgruppe ud fra forbrugsudviklingen af teknologiske apparater. Derudover undersøges hvordan børns vaner udvikles, hvormed en aldersgruppe der er modtagelig over for nye vaner kan fastlægges.}

Den teknologiske udvikling har stor betydning for den stigende andel af inaktive danske børn~\citep{Kiens2007}. Ifølge Sundhedsstyrelsen var 45\% af danske unge i alderen 11–15 år fysisk inaktive i 2006~\citep{Sundhedsstyrelsen2006}. Derudover mener Sundhedsstyrelsen, at børn og unge bliver mindre aktive med alderen. Dette kan have en sammenhæng med, at tilstedeværelsen af teknologi for børn ligeledes stiger med alderen. 
I 2013 havde 3\% af børn i alderen 5-8 år teknologiske apparater med i skole hver dag. Dette tal var i 2014 steget til 33\% for samme aldersgruppe. Denne tendens stiger med alderen, da 87\% af børn i aldersgruppen 9-12 år dagligt medbragte teknologiske apparater i 2014.~\citep{Sundhedsstyrelsen2006,GjensidigeForsikring2014} 

Børns vaner i forhold til fysisk aktivitetsniveau dannes i barndommen og den tidlige pubertetsalder, hvilken er defineret som 8-12 år afhængigt af køn \citep{Wied2011}. I denne aldersgruppe har autoritære roller, såsom forældre og lærere, fortsat en påvirkning med hensyn til at inkorporere vaner hos børnene. \citep{Wied2011,F.SallisG.Simons-MortonJ.Stone1992,L.MeyerP.Gullotta2012} \newline
Det anses som nødvendigt, at børn vænnes til at være fysisk aktive i en tidlig alder, idet vaner bringes med videre til voksenlivet. Hvis ikke børnene får tilegnet sig en fysisk aktiv livsstil, vil børnene vænnes til en stillesiddende adfærd \citep{Nabe-NielsenSundhedsministerietetal.2005}. Endvidere påpeger studier, at det kan være fordelagtigt at give børn gode vaner før puberteten. Dette skyldtes en række fysiske og psykiske faktorer, som børnene undergår i puberteten. Gode vaner, som en fysisk aktiv livsstil, skal dermed videreføres til børnene forinden 13 års alderen. \citep{F.SallisG.Simons-MortonJ.Stone1992,L.MeyerP.Gullotta2012,P.J.KremersBrug2008}

Der ønskes at reducere antallet af inaktive børn, hvormed der med fordel kan appelleres til børn inden pubertetsaleden. Når børnene aktiveres i denne aldersgruppe, er chancen større for en fremadrettet videreførsel af de tilegnede vaner. For at aktivere børnene kan det med fordel gøres gennem teknologi, da børnene i stigende grad benytter det, hvilket kan have en betydning for den stigende andel af inaktive børn. Det ønskes dermed at benytte teknologien til at forebygge fysisk inaktivitet for børn i alderen 9-12 år\fxnote{Vi har valgt 9-12 år istedet for 8-12 år, fordi vi ønsker "overlappet" imellem den tidligere pubertetsalder og aldersgruppen for dem, som bruger teknologi mest.}. Denne aldersgruppe bruger teknologien i høj grad og har ligeledes tendens til at være fysisk inaktive. 

Dermed er målgruppen for dette projekt defineret som børn i aldersgruppen 9-12 år.