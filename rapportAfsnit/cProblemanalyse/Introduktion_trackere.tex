\section{Aktivitetsmåler som motivationsfaktor for børn}

\textit{INDLEDENDE AFSNIT.} \label{tracker_intro}
Inaktivitet og overvægt er en tendens som er stigende for midaldrende børn i folkeskolerne. Børnene får ikke dyrket den mængden motion som flytter dem udenfor risikogruppen for diverse følgesygdomme. For at bevæge sig udenfor denne risikogruppe kræves det ifølge sundhedsstyrelsen 2.5 timers motion ugentligt. Manglen på denne motion kan være resultatet af den teknologiske udvikling, som medfører en mere stillesiddende livsstil. [1]  \newline 
Den teknologiske udvikling som umiddelbart medvirker til inaktivitet, og den stillesiddende livsstil, er forsøgt udnyttet som modarbejdende faktor. Flere producenter har benyttet teknologi som et led i at motivere børn til at leve et liv med mere motion. Fælles for disse producenter er at de motiverer børn til at motionere gennem spil og leg. Producenterne benytter aktivitetsmålere der registrerer aktivitet og sideløbende med aktivitet, optjenes der point hvormed børnene bliver belønnet. Børnene har i mange tilfælde mulighed for at spille alene, men også i hold. Dette medfører en mulig implementering af motions motiverende teknologier i et skoleregi. \newline
Potentialet af en teknologi som motiverer børn til en aktiv livsstil har flere fordele. 
