\section{Aktivitetsmåler for børn} \label{tracker_intro}
\textit{INDLEDENDE AFSNIT.}

Inaktivitet og overvægt er en tendens som er stigende for midaldrende børn i folkeskolerne. Børnene får ikke dyrket den mængden motion som flytter dem udenfor risikogruppen for diverse følgesygdomme. For at bevæge sig udenfor denne risikogruppe kræves det ifølge sundhedsstyrelsen 2.5 timers motion ugentligt. Manglen på denne motion kan være resultatet af den teknologiske udvikling, som medfører en mere stillesiddende livsstil. \citep{ObesityActionCoalition}  \newline 
Den teknologiske udvikling som umiddelbart medvirker til inaktivitet, og den stillesiddende livsstil, er forsøgt udnyttet som modarbejdende faktor. Flere producenter har benyttet teknologi som et led i at motivere børn til at leve et liv med mere motion. Fælles for disse producenter er at de motiverer børn til at motionere gennem spil og leg. Producenterne benytter aktivitetsmålere der registrerer aktivitet og sideløbende med aktivitet, optjenes der point hvormed børnene bliver belønnet. Børnene har i mange tilfælde mulighed for at spille alene, men også i hold. Dette medfører en mulig implementering af motions motiverende teknologier i et skoleregi. \newline
Potentialet af en teknologi som motiverer børn til en aktiv livsstil har flere fordele. Det menes blandt andet at hvis gode vaner inkorporeres i et barns hverdag i en tidlig alder, så er chancen for at disse hænger ved større. Dermed vil man kunne forebygge flere former for følgesygdomme ved at aktiverer børn i større grad end hidtil. (Se \secref{motivationsafsnit})

\subsection{Succeskrav til optimering af nuværende aktivitetsmålere til børn}
Hvis et eventuelt system skal udvikles med henblik på optimering af nuværende systemer, så bør der tages højde for essentielle kriterier. Essentielle kriterier indebærer at al aktivitet gennem et barns hverdag skal opfanges, og dermed indgå i den daglige totale af aktivitet. I takt med at al gængs aktivitet skal opfanges, så bør en aktivitetsmåler kunne adskille gang fra løb, gang fra cykling og løb fra cykling. Igennem aktivitetsformer som indebærer, gang, løb og cykling, vil det kunne udføres med forskellig intensitet. Kondition forøges mest effektivt gennem motion af høj intensitet, hvilket skal belønnes og dermed kunne registreres \citep{Hjerteforeningen}. Idet, det er børn som aktivitetsmåleren skal benyttes af skal det indebærer en måde hvorved de bliver motiveret til at motionere. Igennem \secref{motivationsafsnit} tyder det på at børn i den målrettede målgruppe motiveres til aktivitet gennem leg og spil. Det er dermed et essentielt krav kunne motivere er bred målgruppe, spredt over alder og køn. En aktivitetsmåler som skulle benyttes af børn skal ikke være til gene. En eventuelt gene i form af placeringen af en aktivitetsmåler ville kunne medføre at motion blev fravalgt, dermed er der ydermere et krav vedrørende komfort. Komforten skal altså medføre at børnene med en aktivitetsmåler påsat, er lige så frie som foruden.

Den optimerede aktivitetsmåler skal kunne: 
\begin{enumerate}
\item Registrere gang
\item Registrere løb
\item Registrere cykling
\item Registrere intensitet igennem puls
\item Motivere inaktive såvel som aktive børn
\item Monteres uden gene
\end{enumerate}
