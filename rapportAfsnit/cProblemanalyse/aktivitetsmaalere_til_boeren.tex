\section{Aktivitetsmålere til børn} \label{tracker_intro}
%%Den gamle intro
%\textit{Dette afsnit omhandler, en beskrivelse af funktionaliteten for en række udvalgte aktivitetsmålere. Disse aktivitetsmålere bliver vurderet og analyseret på baggrund af opstillede succeskrav. Afslutningsvis præsenteres den samlede vurdering af aktivitetsmålerne, og hvordan disse opfylder opstillede kriterier.}
\textit{Dette afsnit omhandler optimale egenskaber for en aktivitetsmåler samt funktionaliteten af nuværende aktivitetsmålere til børn. Hertil vil en række udvalgte aktivitetsmålere blive vurderet og analyseret på baggrund af opstillede succeskrav. Afslutningsvis præsenteres den samlede vurdering af aktivitetsmålerne, og i hvilken grad disse opfylder de opstillede kriterier.}
%%Gammel indledning 
%
%Sundhedsstyrelsen anbefaler børn at motionere 2,5 ugentligt, hvis dette ikke opfyldes karakteriseres barnet som inaktivt. Manglende motion kan være som resultatet af den teknologiske udvikling, som medfører en mere stillesiddende livsstil. \citep{ObesityActionCoalition} Den teknologiske udvikling som medvirker til inaktivitet og stillesiddende livsstil, er forsøgt udnyttet som modarbejdende faktor. Flere producenter har benyttet teknologi som et led i at motivere børn til et mere aktivt liv \citep{Fuhu2015,PowerAbout2015}. Fælles for disse producenter er, at de motiverer børn til at fysisk aktivitet gennem spil og leg. Producenterne benytter aktivitetsmålere til at registrere aktivitet. Sideløbende belønnes børnene med et antal point, afhængigt af aktivitetsniveauet. Børnene har i mange tilfælde mulighed for at spille alene, men også i hold. Dette medfører en mulig implementering af motions motiverende teknologier i et skoleregi. \newline
%Potentialet af en teknologi som motiverer børn til en aktiv livsstil kan have flere fordele. Den primære fordel ved en aktiv livsstil er forebyggelsen af følgesygdomme. Dette har vist sig at være en fordelagtig økonomisk og sundhedsmæssig investering.
%% Ny indledning

Den teknologiske udvikling medvirker til inaktivitet og en stillesiddende livsstil. Derfor ønskes det at få inkorporeret fysisk aktivitet i den teknologiske udvikling, hvorved nogle af de fysiologiske konsekvenser ved denne udvikling kan ændres til positive effekter. \citep{ObesityActionCoalition} Flere producenter benytter teknologi, i form af aktivitetsmålere, som et led i at motivere børn til et mere aktivt liv gennem spil og leg. Børnene har i mange tilfælde mulighed for at spille alene men også i hold, hvorfor det er muligt at implementere aktivitetsmotiverende teknologier i et skoleregi. \citep{Fuhu2015,PowerAbout2015}
Potentialet af en teknologi, som motiverer børn til en aktiv livsstil, har flere samfundsøkonomiske og sundhedsmæssige fordele, idet en aktiv livsstil blandt andet er forebyggende for diverse af følgesygdomme, som beskrevet i \secref{subsec:inover}.


\subsection{Succeskrav til aktivitetsmålere til børn} \label{succeskrav}

Aktivitetsmålere til børn bør tage højde for en række essentielle kriterier, som indebærer at al daglig aktivitet registreres, og dermed indgår i den daglige totale aktivitet. Nævnt i \secref{subsec:fysio_aktivitet} anbefales det, at børn dagligt udfører 60 minutters aktivitet med moderart til høj intensitet. Idet aktivitetsmåleren skal anvendes igennem en skoledag, så skal aktivitetsmåleren også kunne registrere aktivitetsformer, der er tilgængelige i skolen. Sundhedsstyrelsen har opstillet en række aktivitetsformer, hvor det ønskede intensitetsniveau opnås. Aktivitetsformer, som er tilgængelige for børn igennem en skoledag, er eksempelvis lege, der indebærer løb, leg i skolegården, cykling, fodbold og basketbold. Fælles for disse aktivitetsformer er, at de kan registreres som gang, løb og cykling. \citep{Sundhedsstyrrelsen2003}
I takt med at den daglige aktivitet opfanges bør en aktivitetsmåler kunne registrere, og dermed også adskille, gang, løb og cykling, hvilket gøres gennem forskellige sensorer. 
%For at en aktivitetsmåler kan registrere aktivitet, kræves det at aktivitetsmåleren indeholder sensorer. Med den rette algoritme kan sensorer automatisk skelne mellem de nævnte former for aktivitet. 
Idet de fysiologiske effekter i forbindelse med aktivitet er forskellige alt efter intensitetsniveauet, skal aktivitetsmåleren kunne registrere intensiteten af aktiviteten og belønne brugeren gennem brugerfladen. Intensiteten kan ifølge \secref{subsec:inover} og \secref{subsec:fysio_aktivitet} bestemmes ud fra puls. Derudover kan intensiteten også bestemmes ud fra maksimal iltoptagelse eller Borg skalaen, som vurderer mængden af anstrengelse \citep{Kiens2007}. 

Da det er børn, aktivitetsmåleren skal benyttes af, skal den indeholde en måde, hvorpå de bliver motiveret til fysisk aktivitet. Ifølge \secref{motivation_boern} tyder det på, at børn i den udvalgte målgruppe motiveres til aktivitet gennem leg og spil, hvorfor et essentielt krav er, at kunne motivere denne målgruppe, ubegrænset alder og køn, hvilket kan gøres gennem. \newline
En aktivitetsmåler skal ikke være til gene, da en eventuelt gene i forbindelse med placeringen af en aktivitetsmåler muligvis vil medføre, at motion bliver fravalgt, hvorfor der er et yderligere krav vedrørende komfort. Komforten skal altså medføre, at børnene med en aktivitetsmåler påsat er lige så frie som foruden.

Den optimale aktivitetsmåler skal kunne: 
\begin{itemize}
\item Registrere gang
\item Registrere løb
\item Registrere cykling
\item Registrere intensitet %igennem puls
\item Motivere inaktive såvel som aktive børn %socialt
\item Monteres uden gene
\end{itemize}

\subsubsection{Baggrund for analyse og vurdering af aktivitetsmålere}
Der er udvalgt fire aktivitetsmålere, som er udviklet med henblik på at hjælpe projektets målgruppe. Derudover er de fysiske aktivitetsmålere, som trådløst virker sammen med en app eller hjemmeside. 
De udvalgte aktivitetsmålere vil blive analyseret og vurderet på baggrund af de opstillede succeskrav i \secref{succeskrav}.


\subsection{UNICEF kid power band}
UNICEF Kid Power Band er en aktivitetsmåler, som appellerer til børn ved at hjælpe andre børn i ressourcefattige lande, hvilket fører til sloganet "vær aktiv og red liv". Børnene optjener point ved at være aktive, mens de har aktivitetsmåleren på, hvilken er monteret på armen. Børnene samler flere point, jo mere energiske de er gennem øvelserne. Aktivitetsmåleren opfanger, ud fra børnenes bevægelse med armen, skridt og andre bevægelser gennem et pedometer og et tre-akse accelerometer. \citep{PowerAbout2015,PowerManual2015} \newline 
For at få point, skal børnene gennemføre forskellige missioner, som professionelle atleter står i spidsen for, hvorigennem børnene ikke blot er aktive men også lærer om forskellige kulturer.\fxnote{Et eksempel er en mission, som basketballspilleren Tyson Chandler står i spidsen for, hvor børnene lærer om hvordan børn i ressourcefattige lande, hjælper familien med at gro deres eget mad.} \citep{PowerMission2015} 
Børnene kan selv følge med i, hvor langt de er i den pågældende mission på aktivitetsmåleren eller gennem en applikation (app). Når børnene har gennemført en mission, omregnes deres point til en sum penge, sponsoreret af fans, firmaer og forældre, som sendes til det pågældende ressourcefattige land, som missionen støtter. \newline

%Hver dag nulstilles aktivitetsmåleren, så børnene hver dag kan følge med i hvor aktive de har været den pågældende dag. Derudover gemmes der data 30 dage tilbage, så det er muligt at sammenligne med tidligere dage. 
%På aktivitetsmåleren er der en skærm, hvor det er muligt at følge med i klokken, antal skridt, KidPower points, fremskridt på missioner og navnet på brugeren. 
%Alle resultater samles i en applikation (app), hvor børnene både har mulighed for at følge med i progressionen for dem selv og deres venner, samt for de missioner de deltager i. \citep{PowerAbout2015}

\subsubsection{Vurdering af succeskrav}
Aktivitetsmåleren giver mulighed for at tælle skridt, som både registreres under løb, gang og andre aktiviteter, dog skelnes der ikke mellem aktiviteterne. Da armene ikke bevæges ved cykling, er dette ikke muligt for aktivitetsmåleren at registrere. Derudover måles intensitet af det udførte arbejde ikke, idet der udelukkende måles hvor energisk armene bevæges under en givne øvelse, og ikke puls, iltoptagelse eller anstrengelse. Aktivitetsmåleren er designet som et armbånd, som nemt kan sættes på barnet, da den har en justerbar rem. \citep{PowerManual2015} \newline
Børnene aktiveres socialt, da alle aktiviteter udføres med henblik på at de sammen med jævnaldrende, skal hjælpe børn i ressourcefattige lande. Derudover bliver børnene gennem appen opdateret på progression i de missioner de deltager i, samt venners progression, hvorved det ikke kun er den individuelle præstation der er i fokus. %Flere skoler i USA har i fjerde klasse også benyttet aktivitetsmåleren, som en del af klasseprojekter, for at få børnene til at blive mere aktive. 
\citep{PowerAbout2015} 

Dermed opfylder UNICEF Kid Power Band 2 ud af 6 succeskrav, mens det delvist opfylder 2 succeskrav.

\subsection{The Sqord Booster}
The Sqord Booster er en aktivitetsmåler, som appellerer til børn i alderen 8-14 år gennem konkurrence og fællesskab. Måden hvorpå aktivitetsmåler motiverer børnene er gennem spil, hvori al aktivitet de udfører gemmes i en avatar. Denne avatar designer børnene selv på en hjemmeside, hvor de også kan kommunikere med deres venner. Forældrene har mulighed for at oprette et forældrelogin til siden, så de ligeledes kan følge med i deres børns aktivitet. Aktivitetsmåleren er designet til at blive brugt i grupper, dette er dog uafhængigt af om børnene fysisk eller online er sammen. \citep{Sqord_family2015}

Børnene optjener point ved at deltage i forskellige konkurrencer, hvor deres aktivitet måles gennem et tre-akse accelerometer, som måler hastigheden af aktiviteterne. Aktivitetsmåleren placeres oftest om håndleddet men kan også placeres i en lomme eller bundet til skoen. \citep{Sqord_family2015} \newline Børnene kan enten konkurrere mod hinanden, eller arbejde sammen som et hold. Det er dog også muligt at benytte aktivitetsmåleren individuelt. \citep{Sqord_family2015,Sqord_group2015}
%Hjemmesiden hvor børnene kan følge med i deres avatar, fungerer som et forum, hvor de har mulighed for at give hinanden highfives for gode præstationer, chatte indbyrdes, eller lave talebobler, hvor alle kan se hvad de skriver. \citep{Sqord_family2015} \newline

The Sqord Booster tilgodeser alle præstationer, da alle får en medalje ved blot at have deltaget i en given aktivitet. Vinderen får imidlertid flere point end de andre deltagere. Spillet er lavet, så alle har mulighed for at vinde, da der i det enkelte spil, vurderes ud fra børnenes individuelle form, ved at se på tidligere præstationer. \citep{Sqord_family2015}

\subsubsection{Vurdering af succeskrav}
Aktivitetsmåleren registrerer både børnenes aktivitet ved gang og løb men kan ikke skelne mellem de to forskellige former for aktivitet, og der registreres ikke cykling. Der måles derudover ikke intensitet af det udførte arbejde, idet kun accelerometerets fart vurderes. \newline
Børnene bliver aktiveret socialt, da hjemmesiden er en blanding mellem et chatforum og en oversigt over præstationer. Derudover har børnene mulighed for at konkurrere med og mod hinanden. The Sqord Booster har derudover sørget for at fange både de børn der er i god form, og dem som ikke er, da alle har mulighed for at vinde baseret på tidligere præstationer. Aktivitetsmåleren er mulig at placere flere steder, hvormed børnene har mulighed for at vælge en placering, hvor det er til mindst gene.\fxnote{Derudover er det designet efter målgruppen, hvormed aktivitetsmåleren både kan modstå stød og tåle at komme i vand.}

Dermed opfylder The Sqord Booster 2 ud af 6 succeskrav, mens det delvist opfylder 2 succeskrav.

\subsection{Nabi Compete}
Nabi Compete er en aktivitetsmåler, som appellerer til børn over seks år gennem deres madvaner og samvær med andre. Der er muligt for børnene at konkurrerer individuelt, men hovedformålet er at konkurrere mod eller med andre som et hold. Konkurrencerne kan bestå i at løbe en bestemt rute, som de selv kan tegne ind, men kan også bestå i at forbrænde nok kalorier til at have forbrændt forskelligt fastfood. 
%Det er muligt at opnå mål sammen med andre, eller dyste i hvem der når forskellige mål først. 
\fxnote{Derudover lærer børnene om kalorier og distance ved at bruge appen, hvor det er muligt at følge med i progressionen.}
Gennem konkurrencerne optjenes der point, som kan bruges til at købe et virtuelt dyr, som ved hjælp af point kan vokse. 
Aktiviteten måles gennem et tre-akse accelerometer, som sidder i et armbånd. Dataet synkroniseres til en app via bluetooth, hvor der kan gemmes data 90 dage tilbage, så barnet og forældrene har mulighed for at følge med i barnets progression. \citep{Fuhu_tech2015,Fuhu2015} 

\subsubsection{Vurdering af succeskrav}
Aktivitetsmåleren registrer både gang og løb, men det er ikke muligt at skelne mellem de to former for aktivitet, der registreres heriblandt ikke cykling eller intensitet med aktivitetsmåleren. 
Børnene aktiveres socialt, da appen er designet med mulighed for at konkurrere mod hinanden eller arbejde sammen i hold. Derudover har børnene mulighed for, 
%at have et kæledyr på appen, hvorved de, 
udover at konkurrere mod andre, kan se hvor mange kalorier de har forbrændt. Aktivitetsmåleren monteres uden gene, da den er placeret i en justerbar rem, som let kan monteres om barnets håndled. Derudover er den designet således at den kan tåle sved og regn, hvilket gør at børnene kan bruge det i al slags vejr. 

Dermed opfylder Nabi Compete 2 ud af 6 succeskrav, mens det delvist opfylder 2 succeskrav.

\subsection{Ibitz}
Ibitz er en aktivitetsmåler, som apellerer til børn over fem år gennem udfordringer i samarbejde med forældrene. Ibitz har generelle udfordringer, men der lægges særligt op til at forældrene sætter nogle mål for børnene gennem deres dag og derved bestemmer udfordringerne. Disse udfordringer kan indebære hvor meget tid børnene skal bruge på aktivitet og hvor land tid de må bruge på elektroniske spil. Ved at gennemføre udfordringerne forældrene eller Ibitz har sat, kan børnene tjene point, som kan bruges på to forskellige spil. \newline
%Dette kan være for hvornår der er legetid, hvornår de må sidde foran skærmen eller hvornår de skal lave aktiviteter med forældrene. 
Aktivitetsmåleren består af et pedometer, som måler skridt, der trådløst synkroniseres med en app via bluetooth. Aktivitetsmåleren monteres ved en klemme, som kan sættes på bukserne eller på skoen. Appen gemmer aktiviteter 30 dage tilbage, hvorved barnet og forældrene har mulighed for at følge med i progressionen. \citep{Ibitz_features2016}

\subsubsection{Vurdering af succeskrav}
Aktivitetsmåleren registrer både gang og løb, dog er det ikke muligt at skelne mellem de to former for aktivitet, samt at registrere puls og cykling. Børnene bliver delvist aktiveret socialt, hvor det primært er sammen med familien. Derudover aktiveres børnene ved at tjene point til forskellige spil, som oftest spilles sammen med andre børn. Aktivitetsmåleren monteres uden gene, da børnene selv kan vælge mellem at montere den på buksen eller skoen. Derudover kan den tåle vand, hvorved børn også kan bruge den i regnvejr.  

Dermed opfylder Ibitz 2 ud af 6 succeskrav, mens det delvist opfylder 2 succeskrav.

\subsection{Opsummering af de udvalgte aktivitetsmålere}
Ud fra vurderingen ses det, at de aktivitetsmålere, der i dag benyttes til børn i projektets aldersgruppe, ikke lever op til samtlige af de succeskrav, som er stillet. De kan alle registrere løb og gang men har ikke mulighed for at skelne mellem de to aktivitetsformer. Ingen af aktivitetsmålerne registrerer cykling eller intensitet. Alle aktivitetsmålerne appellerer til både inaktive og aktive børn. Alle aktivitetsmålere er beregnet til at have rundt om armen, hvor den spændes på med en justerbar rem. Derudover er alle aktivitetsmålere designet efter, at børnene både skal kunne bruge dem i såvel regnvejr som solskin.

\begin{table}[H]
	\centering
	\resizebox{\textwidth}{!}{%
		\begin{tabular}{|l|c|c|c|c|}
			\hline
			Krav                                            & \multicolumn{1}{l|}{Unicef Kid Power Band} & \multicolumn{1}{l|}{Sqord Booster} & \multicolumn{1}{l|}{Nabi Compete} & \multicolumn{1}{l|}{Ibitz} \\ \hline
			Registrere gang                                 & (x)                                        & (x)                                & (x)                               & (x)                        \\ \hline
			Registrere løb                                  & (x)                                        & (x)                                & (x)                               & (x)                        \\ \hline
			Registrere cykling                              &                                            &                                    &                                   &                            \\ \hline
			Registrere intensitet gennem puls               &                                            &                                    &                                   &                            \\ \hline
			Motivere inaktive såvel som aktive børn         & x                                          & x                                  & x                                 & x                          \\ \hline
			Monteres uden gene                              & x                                          & x                                  & x                                 & x                          \\ \hline
		\end{tabular}
	}
	\caption{Tabellen viser en oversigt over de fire aktivitetsmålere og hvorvidt de lever op til kravene. (x) betyder at de delvist lever op til kravene. x betyder at de lever op til kravene}
	\label{tab:sammenhold_tracker}
\end{table}

For at optimere de aktivitetsmålere, der benyttes i dag, skal de kunne skelne mellem løb, gang og cykling, så barnet ikke kun kan måle, hvor mange skridt vedkommende har gået, eller hvor langt de er nået, men også kan måle hvilken aktivitet, som er udført. Derudover skal intensiteten af øvelsen kunne registreres ved hjælp af puls, da det har en afgørende betydning for det fysiologiske udbytte af den givne aktivitet, hvilket kan ses på \figref{fig:PA_Procentpuls} i \secref{sec:fysio}.\newline
Aktivitetsmåleren skal, som de der findes i dag, aktivere børnene socialt sammen med jævnaldrende børn. Derudover skal aktiviteterne foregå igennem leg eller spil, som både skal være baseret på konkurrence mod andre eller sammenspil i hold. 


