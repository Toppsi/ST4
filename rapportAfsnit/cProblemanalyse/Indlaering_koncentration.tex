% !TeX spellcheck = da_DK
\subsubsection{Aktivitet og kognitiv respons}
Fysisk aktivitet har ikke blot positive effekter for kroppens fysiske helbred men også for hjernens kognitive funktioner heriblandt indlæring, hukommelse samt kontrolprocesser, som multitasking, planlægning og koncentration. Derudover medvirker længerevarende træningsperioder til en positiv virkning på den numeriske intelligens\fxnote{Matematiske færdigheder} \citep{Bugge2015,Berchtold2010,Schmidt2015}.\\
Måden hvorpå fysisk aktivitet gavner hjernes kognitive funktioner er, at øget fysisk aktivitet resulterer i øget aktivitet i hippocampus \fxnote{lokaliseret i det limbiske system i hjernen}, som er det område i hjernen, der processerer hukommelse og navigation, hvorved øget fysisk aktivitet forbedre evnen til læring og hukommelse. Ved en længerevarende træningsperiode vil der ske en ændring i hjernens plasticitet, hvorved hjernen adaptere sig til det ændrede aktivitetsniveau\fxnote{Den tilpasser sig til at dyrke mere motion, hvorved området for indlæring og hukommelse vokser - ligesom en muskel man bruger mere}. Blodkarrene i hjernen\fxnote{hippocampus, cortex og cerebellum} udvides, som følge af det øgede aktivitetsniveau, på samme vis som i resten af kroppen\fxnote{reference til fysiologiafsnit}, hvilket medfører at der kan tilføres flere næringsstoffer og mere energi. \citep{Cotman2007}\\
Den fysiske aktivitets effekter på hjernens kognitive funktioner er dog ikke permanente, og aftager langsomt efter aktiviteten er opholdt. Efter fysisk aktivitet i 11-20 minutter, vil de øgede kognitive funktioner for børn vare i op til 50 minutter, mens de for voksne vil vare i 25-45 minutter. \citep{Cotman2007}



