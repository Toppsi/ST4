% !TeX spellcheck = da_DK
\subsubsection{Fysisk aktivitet og kognitiv respons}
Fysisk aktivitet bidrager med et positivt udbytte vedrørende encephalons\fxnote{den del af centralnervesystemet, som er indholdt i kraniekassen - aka altså hjernen.} kognitive funktioner. Eksempelvis øges de kognitive funktioner som indlæring, hukommelse og koncentration.~\citep{Berchtold2010,Bugge2015,Schmidt2015} 
Fysisk aktivitet gavner encephalons kognitive funktioner ved at øge aktiviteten i hippocampus, som er lokaliseret i det limbiske system i encephalon. Dette område i encephalon processerer hukommelse, indlæring og navigation, hvilket resulterer i at øget fysisk aktivitet forbedrer evnen heraf. Ved en længerevarende træningsperiode vil der ske en ændring i encephalons plasticitet, hvorved encephalon adapteres til det ændrede aktivitetsniveau. Dermed vokser områder i hippocampus, som processerer indlæring og hukommelse, som resultat af øget fysisk aktivitet. Blodkarrene i encephalon\fxnote{hippocampus, cortex og cerebellum påvirkes mest - altså mere end de andre} udvides som følge af det øgede aktivitetsniveau på samme vis som i resten af kroppen. Dette medfører, at der kan tilføres flere næringsstoffer og mere energi, hvilket er medvirkende til kortvarig øget kognitiv funktion. 
Efter fysisk aktivitet i 11-20 minutter vil de øgede kognitive funktioner for børn vare op til 50 minutter, mens de for voksne vil vare 25 til 45 minutter. Den fysiske aktivitets effekt på encephalons kognitive funktioner er ikke permanente og aftager langsomt efter aktiviteten er opholdt.\fxnote{Der findes ikke noget grundlag for, hvorfor voksne har kortere kognitiv effekt af fysisk aktivitet end voksne. Det vurderes ud fra litteraturen, at den "voksne" aldersgruppe dækker over en bred alder - altså også ældre. Det kan tænkes, at disse ældre ikke har lige så stor effekt af fysisk aktivitet som "unge voksne", hvorfor gennemsnittet sættes ned.}~\citep{Cotman2007,Schmidt2015} Ydermere tyder studier på, at fysisk aktivitet kan have en længerevarende positiv effekt på børns kognition. Dette kommer eksempelvis til udtryk ved, at længerevarende træningsperioder kan bidrage til en positiv virkning på matematiske færdigheder.~\citep{SibleyEtnier2003,Bugge2015}
