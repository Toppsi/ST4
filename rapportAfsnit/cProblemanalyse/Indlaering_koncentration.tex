% !TeX spellcheck = da_DK
\subsection{Indlæring og koncentration}

% Opbygning af afsnit %%%%%%%%%%%%%%%%%%%%%%%%%%%%%%%%%%%%%%%%%%%%%%
%-- Inaktivt og/eller overvægtig 
% Kort introduktion, beskrivelse af hh. inaktivitet og overvægt.
% Følgesygdomme fra begge
% Sammenligning (Hvad er farligst, hvad er forskellen)
%-- Aktivt
% Kort introduktion, hvad betyder det at være aktiv(definition)
% Hvilken betydning har puls for forbrænding?
% Helbred 
%-- Kongnitiv funkktion (indlæring/Koncentration)
% Inaktivitet, fedme og aktivitets påvirkning af kongitiv funktion (I hvor lang tid kan aktivitet hjælpe på indlæring?)
% Hvornår skal man være aktiv i forhold til undervisning?

Fysisk aktivitet har ikke blot positive effekter på kroppens fysiske helbred, men også hjernens kognitive funktioner, heriblandt indlæring, hukommelse og kontrolprocesser som multitasking, planlægning og koncentration\cite{Berchtold2010}.



http://www.accessscience.com.zorac.aub.aau.dk/content/exercise-and-cognitive-functioning/YB100072
participation in exercise also benefits higher cognitive function, particularly aspects of higher cognitive function that decline with aging, such as learning and memory, and “executive control processes” (involved in multitasking, decision making, planning, attention, and dealing with distraction).

http://static.sdu.dk/mediafiles//C/E/E/%7BCEE2E548-DBAB-42EC-A284-7753E1C6EFD0%7DRapport_Forsøg_Læring_i_Bevægelse_2015.pdf
Kognition. Der kan ikke ud fra de forskellige delprojekter i ”Forsøg med Læring i Bevægelse” udledes entydige konklusioner om fysisk aktivitets betydning for kognition. Projektet har beskæftiget sig med fysisk aktivitets betydning for eksekutiv funktion, akademisk kunnen og intelligens. 

Laboratorieforsøgene viste, at perioder med fysisk aktivitet havde en umiddelbar positiv effekt på eksekutiv funktion og længerevarende træningsforsøg havde positiv virkning på numerisk intelligens. Et kontrolleret skoleforsøg med øget fysisk aktivitet i matematiktimerne viste positive resultater på matematikfærdigheder i 1. klasse, hvorimod en bredere fysisk aktivitet indsats ingen effekt viste i 6. og 7. klasser. Modelforsøg og kvalitative studier viste, at øget fysisk aktivitet i undervisningen kan gennemføres i alle de deltagende institutions- og skoleformer og have positiv betydning for læring, under de rette betingelser.

http://onlinelibrary.wiley.com/doi/10.1002/14651858.CD007651.pub2/epdf
