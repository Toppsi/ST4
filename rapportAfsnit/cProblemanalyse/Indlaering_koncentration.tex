% !TeX spellcheck = da_DK
\subsection{Kognitive ændringer ved fysisk aktivitet}

% Opbygning af afsnit %%%%%%%%%%%%%%%%%%%%%%%%%%%%%%%%%%%%%%%%%%%%%%
%-- Inaktivt og/eller overvægtig 
% Kort introduktion, beskrivelse af hh. inaktivitet og overvægt.
% Følgesygdomme fra begge
% Sammenligning (Hvad er farligst, hvad er forskellen)
%-- Aktivt
% Kort introduktion, hvad betyder det at være aktiv(definition)
% Hvilken betydning har puls for forbrænding?
% Helbred 
%-- Kongnitiv funktion (indlæring/Koncentration)
% Inaktivitet, fedme og aktivitets påvirkning af kongitiv funktion (I hvor lang tid kan aktivitet hjælpe på indlæring?)
% Hvornår skal man være aktiv i forhold til undervisning?

Fysisk aktivitet har ikke blot positive effekter for kroppens fysiske helbred, men også hjernens kognitive funktioner, heriblandt indlæring, hukommelse og kontrolprocesser som multitasking, planlægning og koncentration\citep{Berchtold2010,Schmidt2015}. Ydermere har studier vist, at fysisk aktivitet også har positive effekter på kontrolprocesser og reaktionstid, mens længerevarende træningsperioder har positiv virkning på numerisk intelligens\citep{Bugge2015,Berchtold2010,Schmidt2015}.\\
Måden hvorpå fysisk aktivitet gavner hjernes kognitive funktioner er, at øget fysisk aktivitet resulterer i øget aktivitet i hippocampus\fxnote{lokaliseret i det limbiske system i hjernen}, som er det område i hjernen, som processerer hukommelse og navigation, hvorved øget fysisk aktivitet forbedre evnen til læring og hukommelse. Ved en længerevarende træningsperiode, vil der ske en ændring i hjernens plasticitet, hvorved hjernen adaptere dig til det ændrede aktivitetsniveau\fxnote{Den tilpasser sig til at dyrke mere motion, hvorved området for indlæring og hukommelse vokser - ligesom en muskel man bruger mere}\fxnote{Blodårene i hippocampus, cortex og cerebellum udvides, hvorved der kan komme flere næringsstoffer og mere energi hertil - sker som følge af fysisk aktivitet}.\citep{Cotman2007}\\

Den fysiske aktivitets effekter på hjernens kognitive funktioner er dog ikke permanente, og aftager langsomt efter aktiviteten er opholdt. Efter fysisk aktivitet i 11-20 minutter, vil de øgede kognitive funktioner, for børn, vare i op til 50 minutter, mens de for voksne vare i 25-45 minutter.\citep{Cotman2007}



