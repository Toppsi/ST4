% !TeX spellcheck = da_DK
\subsubsection{Aktivitet og kognitiv respons}
Fysisk aktivitet bidrager med et positivt udbytte vedrørende encephalons kognitive funktioner. Eksempelvis øges de kognitive funktioner som indlæring, hukommelse og koncentration. %kontrolprocesser, som multitasking, planlægning og koncentration. 
\citep{Berchtold2010,Bugge2015,Schmidt2015}. 
Måden hvorpå fysisk aktivitet gavner encephalons kognitive funktioner er øget aktivitet i hippocampus, som er lokaliseret i det limbiske system i encephalon. Dette område i encephalon processerer hukommelse, indlæring og navigation, hvilket resulterer i, at øget fysisk aktivitet forbedrer evnen heraf. Ved en længerevarende træningsperiode vil der ske en ændring i encephalons plasticitet, hvorved encephalon adapterer sig til det ændrede aktivitetsniveau. Den tilpasser sig til at dyrke mere motion, hvorved områder for eksempelvis indlæring og hukommelse vokser ligesom en muskel, der bruges mere. Blodkarrene i encephalon\fxnote{hippocampus, cortex og cerebellum påvirkes mest - altså mere end de andre} udvides som følge af det øgede aktivitetsniveau på samme vis som i resten af kroppen, hvilket også er nævnt i \secref{subsec:fysio_aktivitet}. Dette medfører, at der kan tilføres flere næringsstoffer og mere energi. \citep{Cotman2007}\\
Den fysiske aktivitets effekt på encephalons kognitive funktioner er dog ikke permanente og aftager langsomt efter aktiviteten er opholdt. Efter fysisk aktivitet i 11-20 minutter vil de øgede kognitive funktioner for børn vare op til 50 minutter, mens de hos voksne vil vare 25 til 45 minutter. \citep{Cotman2007,Schmidt2015} Ydermere tyder studier på, at fysisk aktivitet kan have en længerevarende positiv effekt på børns kognition. Dette kommer eksempelvis til udtryk ved, at længerevarende træningsperioder kan bidrage til en positiv virkning på matematiske færdigheder \citep{Bugge2015,SibleyEtnier2003}.
