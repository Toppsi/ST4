\subsection{Vaner}
Børns vaner, angående deres fysiske aktivitetsniveau, dannes i barndommen og den tidlige pubertetsalder \citep{F.SallisG.Simons-MortonJ.Stone1992}. For disse alderstrin har autoritære roller, såsom forældre og lærere, fortsat en stærk påvirkning med henhold til at inkorporere vaner hos børnene \citep{L.MeyerP.Gullotta2012}. \newline
Voksne menneskers vaner hvad angår fysisk aktivitet og stillesiddende opførsel, er ofte tilsvarende vanerne de havde som børn \citep{P.J.KremersBrug2008}. Det anses dermed som nødvendigt at børnene vænner sig til at være fysisk aktive i en tidlig alder. Hvis ikke denne fysiske livsstil forekommer, da vil børnenes kroppe vænne sig til en stillesiddende adfærd \citep{Nabe-NielsenSundhedsministerietetal.2005}. \newline
Børn bør derfor have en sund og aktiv livsstil, for hermed at have en større chance for at kunne videreføre disse gode vaner til voksenlivet \citep{L.MeyerP.Gullotta2012}. Endvidere påpeger studier, at det er fordelagtigt at give børn gode vaner før puberteten. Dette skyldtes en række fysiske og psykiske faktorer, som børnene undgår i puberteten. Gode vaner med en fysisk aktiv livsstil skal dermed videreføres til børnene forinden folkeskolens mellemtrin. I den forbindelse anses folkeskolen som en essentiel faktor for videreførelsen af en aktiv livsstil. Det antages, at hvis ikke skolerne engagerer sig i at gøre børnene mere fysisk aktive, da vil helbredsniveauet blive dårligere end tidligere. Skolerne skal derved være forløber for at give børnene gode vaner hvad angår deres fysiske aktivitetsniveau. \citep{L.MeyerP.Gullotta2012}

