\subsection{Vaner}
 {\color{red} \textbf{Kildehenvisningerne til dette afsnit fungerer ikke helt endnu.
Afsnittet har fokus på overvægt på nuværende tidspunkt. Dette vil blive ændret, således afsnittet får meget mere fokus på inaktivitet.
Desuden vil sørge for, at der ikke står særlig meget om forældrenes rolle i forhold til børns vaner.}}
 
Dårlige vaner og overvægt går ofte tillige. Undersøgelser har vist at i de fleste tilfælde har overvægt hos et barn en sammenhæng mellem barnets forældre og deres kost- og motionsvaner. Undersøgelser viser at ved 70\% af tilfældene, så kan forældrene til overvægtige børn mellem 6-12 år ikke se overvægtigen. Nogle forældre forbinder det i stedet som at barnet har en høj grad af trivsel.\citep{videnskab} 
Forbedring af børns og unges kost- og motionsvaner har derfor stor betydning i håndtering af overvægt og udvikling af kroniske sygdomme. Dette er relevant, da vaner etableret i de tidlige år, har tendens til at fortsætte ind i voksenlivet.\citep{trends}  Det er særligt i teenageårene, hvor børn bliver bevidste om deres vaner i hverdagen.\citep{dansker} 
Disse sociale uligheder kan beskrives samfundsøkonomisk, hvor børn/unge som bor med familier med høj socioøkonomisk status, har sundere kostvaner sammenlignet med familier med lavere socioøkonomisk status.\citep{trends} 

Familier med/fra lavere socioøkonomisk status, heriblandt familier hvor forældrene har en kortere skolegang har en større forekomst af overvægt. 
Vanerne i disse familier kan bestå af usund mad og mangel på fysisk aktivitet.\fxnote{KILDE}

Det har vist sig at en pædagogisk tilgang, hvor motiverende mål, kan hjælpe børn, med at få indsigt i effekten af deres vaner. De fleste børn ved godt, igennem undervisningen, hvilke vaner er sunde herunder mad, motion, søvn osv. De er samtidig også bevidste om, hvilke vaner som er usunde og på trods af denne viden, selv har vaner i større eller mindre omfang som kan anses for at være usunde. 
Motivation og ændring i vaner hænger sammen børn perception af de har haft i livet. Ændringen af vaner skal opnås igennem erfaringer og rollemodeller som børnene kan iagttage eller imitere. I denne sammenhæng imiterer børn ofte personer som de kan identificerer sig med og føle sig trygge med.\citep{opsporing}
