\section{Motivationsfaktor til øget fysisk aktivitet}\label{motivation_boern}
\textit{Dette afsnit beskriver, hvad der kan motivere den valgte aldersgruppe til øget fysisk aktivitet. Dette gøres med henblik på at have et optimalt grundlag til at designle et motiverende apparat til denne aldersgruppe.}

Motivation er menneskets drivkraft i forhold til opførsel og udførslen af handlinger \citep{V.Brown2007}. Fysisk aktivitet bliver udført på baggrund af den enkelte persons motivation til en aktivitet. Motivationen til en given aktivitet kan deles op i to overordnede typer af motivation: Intrinsisk og ekstrinsisk. Den intrinsiske motivation omhandler individets egen drivkraft til at udføre en opgave. Denne type motivation fokuserer på individets holdning til aktiviteten, og hvordan aktiviteten kan opfylde de personlige behov. Den intrinsiske motivation er derfor karakteriseret af interessen og glæden ved en aktivitet. Den ekstrinsisk motivation omhandler en ekstern påvirkning af et individ. Denne type motivation kan eksempelvis være forældres forventninger til et barns skolekarakterer eller sportsaktiviteter. Barnet udfører aktiviteten på baggrund af en ekstern motivation, som kan risikere at blive udført med frygten for at fejle. Ekstrinsisk motivation fokuserer derfor på effekten af en aktivitet udført med en ekstern motivation. \citep{J.Sebire2013} 

Motiverende faktorer kan være aldersmæssigt betinget, hvorfor børn og voksne motiveres forskelligt. Dette kommer blandt andet som følge af det psykologiske stadie, som børn befinder sig i \citep{V.Brown2007}. Børn handler instinktivt og impulsivt, hvormed de kan have svært ved at fastholde deres koncentration på en given aktivitet. Derfor er det essentielt, at børnene har en motivationsfaktor, som giver dem glæde og lysten til at udføre en aktivitet. \citep{V.Brown2007} \newline
For børn er det væsentligt, at en aktivitet opleves sjovt, anerkendende og har sociale dimensioner. Der kan midlertid opstå problemer ved fysiske gruppeaktiviteter, da børnene kan være forhindret i at møde til de givne tidspunkter. Besværligheden ved tidsplanlægning kan gøre, at flere børn fravælger gruppeaktiviteter. %Træningen må gerne være fysisk krævende, men kritikken, der kan kommer på indsatsen, skal gives som positiv konstruktiv kritik, der giver børnene en naturlig tilfredsstillelse.  Derudover er det essentielt, at der er sociale dimensioner ved aktiviteten, da de fleste børn forbinder aktivitetsformer med et socialt fællesskab. \textbf{Der kan opstå problemer med gruppeaktiviteter, da børnene kan være forhindret i at møde til de givne tidspunkter.} %Dette kan få barnet til at tvivle om de sociale tilhørsforhold og egen udvikling af færdigheder. Derfor fravælger flere børn muligvis aktiviteten, da det bliver for besværligt. 
Det kan dermed være fordelagtigt, hvis en fysisk gruppeaktivitet ikke involverer et fysisk fremmøde eller skal foregå på et bestemt tidspunkt. %Dermed kan barnet uafhængigt af tidspunktet Dermed vil den fysiske aktivitet kunne blive udført af det enkelte barn, uden at skulle nødsaget til at møde et bestemt sted på et bestemt tidspunkt.
\citep{Wied2011,Romani2013} %Sociale sammenhænge og legen ved en fysisk aktivitet er de væsentligste  ekstrinsiske motivationsfaktorer for børn, som skal øge aktivitetsniveauet \citep{McWhorter2003,J.Sebire2013}. Det fremhæves, at en fysisk aktivitet, som giver børn naturlig tilfredsstillelse og glæde, vil medføre et fremtidigt øget aktivitetsniveau for barnet \citep{Romani2013}. Derudover giver autonom fysisk aktivitet med intrinsisk motivation det bedste udbytte for børn \citep{J.Sebire2013}.

Måden, hvorpå børn motiveres til og gennem fysisk aktivitet, er forskellig, alt efter hvilken aldersgruppe de befinder sig i. Børn i den valgte målgruppe, altså i alderen 9-12 år, motiveres særligt gennem leg, hvor det er essentielt, at alle deltagere oplever succes gennem aktiviteten. Børn i denne alder motiveres endvidere intrinsisk gennem en positiv tilgang, hvor der særligt fokuseres på de ting, som lykkedes. \citep{Wied2011} Dermed giver frivillig fysisk aktivitet med intrinsisk motivation det bedste udbytte for børn \citep{J.Sebire2013}. Konkurrencer vil ofte være en del af sociale fysiske aktiviteter, idet børnene sammenligner sig med andre. Disse konkurrencer kan medføre nederlag og dårlige oplevelser for det enkelte barn. Det er dog essentielt at bibeholde barnets gode oplevelse ved den fysisk aktivitet. Konkurrencer skal derfor holdes på et plan, hvor det ikke er en begrænsende faktor for barnet. Overordnet skal der appelleres til børnene i denne aldersgruppe gennem fairplay og positiv syn på aktiviteterne. \citep{Wied2011}\\
%Øvelserne, der skal udføres, skal være lette og korte, og der skal sættes mål, så barnet har mulighed for at arbejde konkret med én øvelse. Det er essentielt, at børnene instrueres nøje i udførelsen af den givne aktivitet, således barnet opnår de ønskede og mest optimale resultater. \citep{Wied2011} 
Sociale sammenhænge, forældrenes støtte og leg igennem fysiske aktiviteter er de væsentligste ekstrinsiske motivationsfaktorer for børn, som skal øge aktivitetsniveauet. Generelt virker intrinsisk motivation bedre end ekstrinsisk motivation. Hvis barnet ikke selv har lysten og interessen i fysisk aktivitet, vil eksempelvis forældres opfordring ikke gøre en forskel. \citep{J.Sebire2013,McWhorter2003} En fysisk aktivitet, som giver børn naturlig tilfredsstillelse og glæde, kan medføre et fremtidigt øget aktivitetsniveau for barnet \citep{Romani2013}.

%Denne betragtning understøttes endvidere af undersøgelser som påpeger, at børn oftest nævner, at en fysisk aktivitet skal være sjov at udføre. Den fysiske aktivitet skal derfor give børnene glæde og en naturlig tilfredsstillelse \citep{Romani2013}. \newline
%Det fremgår tydeligt, at fysiske aktiviteter skal have et socialt og sjovt perspektiv, for at give det bedste udbytte af aktiviteten. Dette understøttes desuden af studiet \cite{McWhorter2003}. Studiet undersøger overvægtige børns fysiske aktivitetsniveau i forhold til typen af motivation. Det fremgår heraf, at den fysiske aktivitet skal udføres med fokus på at forbedre de fysiologiske egenskaber for den overvægtige. Denne aktivitet skal desuden udføres med en motiverende baggrund, som involverer leg og socialt samvær, for at være mest effektiv. \citep{McWhorter2003} \newline