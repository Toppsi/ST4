\section{Motivationsfaktor til øget fysisk aktivitet}\label{motivation_boern}
Motivation er menneskets drivkraft i forhold til opførsel og udførslen af handlinger \citep{V.Brown2007}. Fysisk aktivitet bliver derfor udført på baggrund af den enkelte persons motivation til en aktivitet. Motivationen til en given aktivitet kan deles op i to overordnede typer af motivation: Intrinsisk og ekstrinsisk. Den intrinsiske motivation omhandler individets egen drivkraft til at udføre en opgave. Denne type motivation fokuserer dermed på individets holdning til aktiviteten, og hvordan aktiviteten kan opfylde de personlige behov. Den intrinsiske motivation er derfor karakteriseret af interessen og glæden ved en aktivitet. Omvendt for den ekstrinsisk motivation er dette en ekstern påvirkning af et individ. Denne type motivation kan eksempelvis være forældres forventninger til et barns skolekarakterer eller sportsaktiviteter. Barnet udfører dermed aktiviteten på baggrund af en ekstern motivation, som kan risikere at blive udført med frygten for at fejle. Ekstrinsisk motivation fokuserer derfor på effekten af en aktivitet udført med en ekstern motivation. \citep{J.Sebire2013} 

Motiverende faktorer kan være aldersmæssigt betinget. Der er dermed forskellige måder, hvorpå børn og voksne motiveres mest optimalt. Dette kommer som følge af det psykologiske stadie som børn befinder sig i \citep{V.Brown2007}. Børn handler instinktivt og impulsivt, hvormed de kan have svært ved at fastholde deres koncentration på en given aktivitet. Derfor er det vigtigt, at børnene har en motivationsfaktor som giver dem glæde og lysten til at udføre en aktivitet. \citep{V.Brown2007} \newline
For børn er det væsentligt, at et træningsmiljø opleves sjovt og anerkendende. Træningen må gerne være fysisk hård, men kritikken, der kan kommer på indsatsen, skal gives som positiv konstruktiv kritik, der giver børnene en naturlig tilfredsstillelse. Derudover er det vigtigt, at der er sociale dimensioner ved træningen, da de fleste børn forbinder træningsaktiviteter med et socialt fællesskab. Der opstår imidlertid problemer med gruppetræningsaktiviteter, da børnene kan være forhindret i at møde til de givne træningstidspunkter. Her fravælger flere børn aktiviteten, da de dermed ikke kan opnå samme niveau som andre. \citep{Wied2011,Romani2013}\newline
Sociale sammenhænge og legen ved en fysisk aktivitet er de væsentligste motivationsfaktorer for børn, som skal øge aktivitetsniveauet \citep{McWhorter2003,J.Sebire2013}. Det fremhæves, at en fysisk aktivitet, som giver børn naturlig tilfredsstillelse og glæde, vil medføre et fremtidigt øget aktivitetsniveau for barnet \citep{Romani2013}. Derudover giver autonom fysisk aktivitet med intrinsisk motivation det bedste udbytte for børn \citep{J.Sebire2013}.

Måden hvorpå børn motiveres til og gennem træning er forskellig, alt efter hvilken aldersgruppe de befinder sig i. Børn i den valgte målgruppe, altså i alderen 8-12 år, motiveres særligt gennem leg, hvor det er vigtigt, at alle deltagere oplever succes gennem aktiviteten. Børnene i denne alder motiveres endvidere gennem en positiv tilgang, hvor der særligt fokuseres på de ting, som lykkedes. Konkurrencer er ofte en del af aktiviteten, da børnene sammenligner sig med andre, men konkurrencedelen skal ikke fylde meget. Overordnet skal der appelleres til børnene i denne aldersgruppe gennem fairplay og positiv syn på præstationerne. Øvelserne, der skal udføres, skal være lette og korte, og der skal sættes mål, så barnet har mulighed for at arbejde konkret med én øvelse. Det er vigtigt, at børnene instrueres nøje, da de ikke får meget ud af egentræning. Børnene kan ydermere aktiveres til at tænke taktisk gennem træningen, men da denne evne ikke er færdigudviklet i denne alder, skal dette foregå på et lavt niveau. \citep{Wied2011} 

%Denne betragtning understøttes endvidere af undersøgelser som påpeger, at børn oftest nævner, at en fysisk aktivitet skal være sjov at udføre. Den fysiske aktivitet skal derfor give børnene glæde og en naturlig tilfredsstillelse \citep{Romani2013}. \newline
%Det fremgår tydeligt, at fysiske aktiviteter skal have et socialt og sjovt perspektiv, for at give det bedste udbytte af aktiviteten. Dette understøttes desuden af studiet \cite{McWhorter2003}. Studiet undersøger overvægtige børns fysiske aktivitetsniveau i forhold til typen af motivation. Det fremgår heraf, at den fysiske aktivitet skal udføres med fokus på at forbedre de fysiologiske egenskaber for den overvægtige. Denne aktivitet skal desuden udføres med en motiverende baggrund, som involverer leg og socialt samvær, for at være mest effektiv. \citep{McWhorter2003} \newline

