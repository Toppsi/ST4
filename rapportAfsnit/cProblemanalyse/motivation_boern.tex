\section{Motivationsfaktor til øget fysisk aktivitet}\label{motivation_boern}
\textit{Dette afsnit beskriver, hvad der kan motivere den valgte målgruppe til øget fysisk aktivitetsniveau. Dette gøres med henblik på at have et grundlag til at designe et motiverende teknologisk apparat til målgruppen.}

Motivation er menneskets drivkraft i forhold til opførsel og udførslen af handlinger~\citep{V.Brown2007}. Fysisk aktivitet bliver udført på baggrund af den enkelte persons motivation til en aktivitet. Motivationen til en given aktivitet kan deles op i to overordnede typer: Intrinsisk og ekstrinsisk. Den intrinsiske motivation omhandler individets drivkraft til at udføre en opgave. Denne type motivation fokuserer på individets holdning til aktiviteten, og hvordan aktiviteten kan opfylde personlige behov. Den intrinsiske motivation er derfor karakteriseret af interessen og glæden ved en aktivitet. Den ekstrinsisk motivation omhandler en ekstern påvirkning af et individ. Denne type motivation kan eksempelvis være forældres forventninger til et barns skolekarakterer eller sportsaktiviteter. Barnet udfører aktiviteten på baggrund af en ekstern motivation, som kan risikere at blive udført med frygten for at fejle. Ekstrinsisk motivation fokuserer derfor på effekten af en aktivitet udført med en ekstern motivation.~\citep{J.Sebire2013} 

Motiverende faktorer kan være aldersmæssigt betinget, hvorfor børn og voksne motiveres forskelligt. Dette kommer blandt andet som følge af det psykologiske stadie, som børn befinder sig i. Børn handler instinktivt og impulsivt, hvormed de kan have svært ved at fastholde koncentrationen på en given aktivitet. Derfor er det essentielt, at børn har en motivationsfaktor, som giver glæde og lysten til at udføre en aktivitet.~\citep{V.Brown2007} For børn er det væsentligt, at en aktivitet opleves sjovt, anerkendende og har sociale dimensioner. Der kan imidlertid opstå problemer ved fysiske gruppeaktiviteter, da børnene eksempelvis kan være forhindret i at møde til de givne tidspunkter. Det kan dermed være fordelagtigt, hvis en fysisk gruppeaktivitet ikke udelukkende afhænger af et fysisk fremmøde.~\citep{Wied2011,Romani2013}

Børn i målgruppen motiveres særligt gennem leg, hvor det er essentielt, at alle deltagere oplever succes ved aktiviteten. Børn i denne alder motiveres endvidere intrinsisk gennem en positiv tilgang, hvor der særligt fokuseres på de ting, som lykkes. Dermed bidrager frivillig fysisk aktivitet med intrinsisk motivation til det bedste udbytte for børn~\citep{J.Sebire2013}. Konkurrencer vil ofte være en del af sociale fysiske aktiviteter, idet børnene sammenligner sig med andre. Disse konkurrencer kan medføre nederlag og dårlige oplevelser for det enkelte barn. Det er dog essentielt at bibeholde barnets gode oplevelse ved den fysisk aktivitet. Konkurrencer skal derfor holdes på et plan, hvor det ikke er en begrænsende faktor for barnet. Overordnet skal der appelleres til børnene i denne aldersgruppe gennem fairplay og et positivt syn på aktiviteterne.~\citep{Wied2011}\\
Sociale sammenhænge, forældrenes støtte og leg igennem fysiske aktiviteter er de væsentligste ekstrinsiske motivationsfaktorer for børn, som skal øge det fysiske aktivitetsniveau. Generelt virker intrinsisk motivation bedre end ekstrinsisk motivation. Hvis barnet ikke selv har lysten og interessen i en given fysisk aktivitet, vil eksempelvis forældres opfordring ikke gøre en væsentlig forskel.~\citep{J.Sebire2013,McWhorter2003} En fysisk aktivitet, som giver børn naturlig tilfredsstillelse og glæde, kan medføre et fremtidigt øget fysisk aktivitetsniveau for barnet~\citep{Romani2013}.