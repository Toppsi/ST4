\subsection{Motivation til øget fysisk aktivitet hos børn} \label{motivation_boern}
 {\color{red} \textbf{Der står endnu ikke særlig meget om motivationen for netop vores målgruppe, da det ikke har været til at finde litteratur på indtil videre. Der vil derfor blive skrevet noget om hvilke motivationsfaktorer som er gældende for vores målgruppe. Samtidig vil der komme en kort delkonklusion for afsnittet.}}


Motivation er menneskets drivkraft i forhold til opførsel og udførslen af handlinger \citep{V.Brown2007}. Fysisk aktivitet bliver derfor udført med baggrund i den enkelte persons motivation til en aktivitet. Motivationen til en given aktivitet kan deles op i to overordnede typer af motivation: Intrinsisk og ekstrinsisk. \newline
Den intrinsiske motivation omhandler individets egen drivkraft til at udføre en opgave. Denne type motivation fokuserer dermed på individets holdning til aktiviteten, og hvordan aktiviteten kan opfylde de personlige behov. Den intrinsiske motivation, er derfor karakteriseret af interessen og glæden ved en aktivitet. \newline
Ekstrinsisk motivation er en ekstern påvirkning af et individ. Denne type motivation kan eksempelvis være forældres forventninger til et barns skolekarakterer eller sportsaktiviteter. Barnet udfører dermed aktiviteten på baggrund af en ekstern motivation, som kan risikere at blive udført med frygten for at fejle. Ekstrinsisk motivation fokuserer derfor på effekten af en aktivitet udført med en ekstern motivation. \citep{J.SebireJagoR.FoxEtAl2013} 

Motiverende faktorer kan være aldersmæssigt betinget. Der er dermed forskellige måder hvorpå børn og voksne motiveres mest optimalt. Dette kommer som følge af det psykologiske stadie som børn befinder sig på. \newline
Børn handler instinktivt og impulsivt, hvormed de kan have svært ved at fastholde deres koncentration på en given aktivitet. Derfor er det vigtigt, at børnene har en motivationsfaktor som giver dem glæde og lysten til at udføre en aktivitet. \citep{V.Brown2007} \newline
Studiet af \cite{J.SebireJagoR.FoxEtAl2013} hævder, at en autonom fysisk aktivitet med intrinsisk motivation giver det bedste udbytte for børn. Denne betragtning understøttes endvidere af undersøgelser som påpeger, at børn oftest nævner, at en fysisk aktivitet skal være sjov at udføre. Den fysiske aktivitet skal derfor give børnene glæde og en naturlig tilfredsstillelse \citep{Q.Romani2013}. \newline
Det fremgår tydeligt, at fysiske aktiviteter skal have et socialt og sjovt perspektiv, for at give det bedste udbytte af aktiviteten. Dette understøttes desuden af studiet \cite{McWhorter2003}. Studiet undersøger overvægtige børns fysiske aktivitetsniveau i forhold til typen af motivation. Det fremgår heraf, at den fysiske aktivitet skal udføres med fokus på at forbedre de fysiologiske egenskaber for den overvægtige. Denne aktivitet skal desuden udføres med en motiverende baggrund, som involverer leg og socialt samvær, for at være mest effektiv. \citep{McWhorter2003} \newline

Sociale sammenhænge og legen ved en fysisk aktivitet er de mest væsentlige motivationsfaktorer for børn, for hvilke at det fysiske aktivitetsniveau skal øges \citep{McWhorter2003,J.SebireJagoR.FoxEtAl2013}. Det fremhæves, at en fysisk aktivitet som giver børn naturlig tilfredsstillelse og glæde, vil medføre et fremtidigt øget aktivitetsniveau for barnet.\fxnote{kilde?}

