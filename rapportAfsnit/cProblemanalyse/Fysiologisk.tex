\section{Fysiologiske konsekvenser}\label{sec:fysio}
%I forhistorien, da mennesket var jægere, var der en naturlig favorisering af de mennesker, som kunne lagre fedt bedre end andre, da der kunne gå lang tid imellem måltiderne, hvilket ikke er nødvendigt med den moderne livsstil, hvor teknologi og højere velstand har medført et mere fysisk inaktivt liv samtidig med der er let adgang til føde. Idet evolutionen ikke har tilpasset sig denne moderne livsstil, søger kroppen stadig at lagre fedt, hvorved personer med et lavere aktivitetsniveau end den energi de indtager, langsomt vil ophobe fedtdepoter, hvilket kan resultere i overvægt.\citep{Ahmad2014,Kiens2007}

\subsection{Fysiske konsekvenser ved inaktivitet og overvægt}
 {\color{red} \textbf{Vi vil sørge for, at dette afsnit fokuserer lidt mere på at man kan både være aktiv men også overvægtig. Desuden vil vi gå lidt mere i dybden med de fysiologiske konsekvenser af at være inaktiv og overvægtig.}}


Det er veldokumenteret, at der sker et fald i fysisk aktivitet med alderen samtidig med der sker en stigning i vægt\citep{Kaprio2008}. Undersøgelser tyder på, at hvis kroppens cellulære vedligeholdelse styrkes med fysisk aktivitet, så kan aldringsprocessen nedsættes\citep{Knight2012}. Fysisk inaktivitet forstærker altså den generelle aldring og anses som værende mindst lige så farligt som overvægt. De to fænomener forekommer dog ofte samtidig, da inaktivitet kan forsage overvægt, men fysisk inaktivitet har en selvstændig helbredsmæssig betydning ligesom overvægt har. Det er muligt at være overvægtig men samtidig have en aktiv livsstil.\citep{Kaprio2008,Kiens2007,Hjort1997}

Fysisk inaktivitet kan lede til flere af de store folkesygdomme som hjertekarsygdomme, diabetes, osteoporose og psykiske lidelser. Menneskekroppen er ikke skabt til at være inaktiv, og derfor vil kroppen reagere kraftigt på det. For eksempel kan kroppen påbegynde nedbrydelse af knoglerne indefra, så de ikke vejer ret meget. 60 til 85\%\fxnote{Find nogle danske tal istedet} af verdensbefolkningen lever en stillesiddende livsstil, hvilket forstærker forekomsten af disse folkesygdomme.\citep{Kiens2007,Reshma2002,Martini2012} Derudover kan inaktivitet lede til disuse syndromet, som blandt andet indebærer svækket hud integritet, ændret respiratorisk funktion og nedsætning af sanserne\citep{Knight2012,Mosby2009}. \\
Definitionen for overvægt er globalt sat ud fra et body mass index (BMI), hvilket er forholdet mellem en persons vægt og højde. Et BMI på 25 eller derover er defineret som værende overvægt.\citep{Academic2016} \fxnote{Men er BMI egentlig den bedste metode? Tager udgangspunkt i færdigudviklede højde, så er måske ikke bedst for man} Overvægt opstår grundlæggende fordi der indtages mere energi end der forbruges. Nogle mennesker kan lagre fedt bedre end andre, hvorfor overvægt også kan være genetisk betinget.\citep{Nestle2014}\\
Overvægt øger risikoen for højt kolesteroltal, forhøjet blodtryk og diabetes samt følgesygdomme heraf som slagtilfælde og nyresygdomme. Det er dokumenteret, at der er størst risiko for tidlig død jo yngre mennesker opnår overvægt. Det er derfor essentielt at forbedre børns aktivitet og dermed mindske risikoen for overvægt.\citep{Nestle2014} Derudover ses der, at overvægtige børn ofte lider af psykologiske og sociale problemer, hvilket kombineret med overvægten kan have en negativ indvirkning på barnets fremtid i forhold til uddannelse og socioøkonomiske status\citep{Academic2016}.

Inaktivitet kombineret med overvægt øger risikoen for diverse sygdomme, men en normalvægtig inaktiv person er i større risiko for tidlig dødsfald end en overvægt aktiv person. Ifølge et 12-års studie lavet over 334.161 europæiske deltagere så tyder det på, at dobbelt så mange vil dø af inaktivitet end overvægt.\citep{Ekelund2015} En aktiv overvægt person har derudover ikke større chance for at udvikle hjertesygdomme end normalvægtige, så længe de er trænede og dyrker motion\citep{Nichols2014}. Det tyder altså på, at inaktivitet er mere skadeligt end overvægt, hvis de sammenlignes som normalvægtig inaktiv mod overvægtig aktiv.

