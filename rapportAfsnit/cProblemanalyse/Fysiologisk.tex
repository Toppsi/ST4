% !TeX spellcheck = da_DK
\section{Fysiologiske konsekvenser}


\subsection{fysisk inaktivitet og overvægt}
Den moderne teknologi samt høje velstand har medført et mere fysisk inaktivt liv samtidig med at man har let adgang til føde. \cite{Kiens2007} Det er veldokumenteret, at der sker et fald i fysisk aktivitet med alderen samtidig med der sker en stigning i vægt \cite{Kaprio2008}. Fysisk inaktivitet forstærker den generelle aldring og anses som værende mindst lige så farligt som overvægt. De to fænomener forekommer dog ofte samtidig, men fysisk inaktivitet har en selvstændig helbredsmæssig betydning ligesom overvægt \cite{Kaprio2008,Hjort1997,Kiens2007}.\\
Der er flere bud på verdensplan om, hvad definitionen er for fysisk inaktivitet. Sundhedsstyrelsen har derfor udarbejdet en generel definition ud fra de flere forskellige som lyder, at et individ er fysisk inaktiv, hvis vedkommende udfører mindre end 2,5 timers fysisk aktivitet om ugen med moderat intensitet\fxnote{Moderat intensitet svarer til 40-59\% af den maksimale iltoptagelse, eller 40-59\% af pulsreserven (maxpuls – hvilepuls), eller 64-74\% af maxpuls eller 12-13 RPE (rate of percieved excertion, Borgskala) og er yderligere defi neret som fysisk aktivitet hvor man bliver lettere forpustet men hvor samtale er mulig.}. \cite{kiens2007}

\subsection{Fysiologisk aktiv}

\subsection{Indlæring og koncentration}

%Hvad sker der fysiologisk i kroppen (2 pers.)
%Inaktivt og/eller overvægtig 
%--Følgesygdomme 
%--Hvad er farligst 
%--Hvad er forskellen 
%Aktivt
%--Hvilken betydning har puls for forbrænding?
%--Helbred 
%Indlæring/Koncentration 
%--I hvor lang tid kan aktivitet hjælpe på indlæring? 
%--Hvornår skal man være aktiv i forhold til undervisning?
