% !TeX spellcheck = da_DK
\section{Fysiologiske konsekvenser}


\subsection{fysisk inaktivitet og overvægt}
Den moderne teknologi samt høje velstand har medført et mere fysisk inaktivt liv samtidig med at man har let adgang til føde \cite{Kiens2007}. Det er veldokumenteret, at der sker et fald i fysisk aktivitet med alderen samtidig med der sker en stigning i vægt \cite{Kaprio2008}. Undersøgelser tyder på, at hvis kroppens cellulære vedligeholdelse styrkes med fysisk aktivitet, så kan aldringsprocessen nedsættes \cite{Knight2012}. Fysisk inaktivitet forstærker altså den generelle aldring og anses som værende mindst lige så farligt som overvægt. De to fænomener forekommer dog ofte samtidig, da inaktivitet kan forsage fedme, men fysisk inaktivitet har en selvstændig helbredsmæssig betydning ligesom overvægt har \cite{Kiens2007,Kaprio2008,Hjort1997}.\\
Der er flere bud på verdensplan om, hvad definitionen for fysisk inaktivitet er. Sundhedsstyrelsen har derfor udarbejdet en generel definition ud fra de flere forskellige som lyder, at et individ er fysisk inaktiv, hvis vedkommende udfører mindre end 2,5 timers fysisk aktivitet om ugen med moderat intensitet\fxnote{Moderat intensitet svarer til 40-59\% af den maksimale iltoptagelse, eller 40-59\% af pulsreserven (maxpuls – hvilepuls), eller 64-74\% af maxpuls eller 12-13 RPE (rate of percieved excertion, Borgskala) og er yderligere defineret som fysisk aktivitet hvor man bliver lettere forpustet men hvor samtale er mulig.}. \cite{kiens2007} Fysisk inaktivitet kan lede til flere af de store folkesygdomme som hjerte-kar-sygdomme, diabetes, osteoporose og psykiske lidelser. 60 til 85\% af verdensbefolkningen lever en stillesiddende livsstil, hvilket forstærker forekomsten af disse folkesygdomme. \cite{kiens2007,Reshma2002april} Derudover kan inaktivitet lede disuse syndromet, som blandt andet indebærer svækket hud integritet, ændret respiratorisk funktion og nedsætning af sanserne \cite{Knight2012,Mosby2009}.\\


\cite{Knight2012}


\subsection{Fysiologisk aktiv}

\subsection{Indlæring og koncentration}

% Opbygning af afsnit %%%%%%%%%%%%%%%%%%%%%%%%%%%%%%%%%%%%%%%%%%%%%%
%-- Inaktivt og/eller overvægtig 
% Kort introduktion, beskrivelse af hh. inaktivitet og overvægt.
% Følgesygdomme fra begge
% Sammenligning (Hvad er farligst, hvad er forskellen)
%-- Aktivt
% Kort introduktion, hvad betyder det at være aktiv(definition)
% Hvilken betydning har puls for forbrænding?
% Helbred 
%-- Kongnitiv funkktion (indlæring/Koncentration)
% Inaktivitet, fedme og aktivitets påvirkning af kongitiv funktion (I hvor lang tid kan aktivitet hjælpe på indlæring?)
% Hvornår skal man være aktiv i forhold til undervisning?
