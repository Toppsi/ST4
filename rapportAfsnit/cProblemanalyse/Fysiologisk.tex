\section{Fysiologiske konsekvenser}


\subsection{fysisk inaktivitet og overvægt}
Den moderne teknologi samt høje velstand har medført et mere fysisk inaktivt liv samtidig med at man har let adgang til føde \citep{Kiens2007}. Det er veldokumenteret, at der sker et fald i fysisk aktivitet med alderen samtidig med der sker en stigning i vægt \citep{Kaprio2008}. Undersøgelser tyder på, at hvis kroppens cellulære vedligeholdelse styrkes med fysisk aktivitet, så kan aldringsprocessen nedsættes \citep{Knight2012}. Fysisk inaktivitet forstærker altså den generelle aldring og anses som værende mindst lige så farligt som overvægt. De to fænomener forekommer dog ofte samtidig, da inaktivitet kan forsage fedme, men fysisk inaktivitet har en selvstændig helbredsmæssig betydning ligesom overvægt har. Det er muligt at være overvægtig men samtidig have en aktiv livsstil. \citep{Kiens2007,Kaprio2008,Hjort1997}\\
Der er flere bud på verdensplan om, hvad definitionen for fysisk inaktivitet er. Sundhedsstyrelsen har derfor udarbejdet en generel definition ud fra de flere forskellige som lyder, at et individ er fysisk inaktiv, hvis vedkommende udfører mindre end 2,5 timers fysisk aktivitet om ugen med moderat intensitet\fxnote{Moderat intensitet svarer til 40-59\% af den maksimale iltoptagelse, eller 40-59\% af pulsreserven (maxpuls – hvilepuls), eller 64-74\% af maxpuls eller 12-13 RPE (rate of percieved excertion, Borgskala) og er yderligere defineret som fysisk aktivitet hvor man bliver lettere forpustet men hvor samtale er mulig.}. \citep{Kiens2007} Fysisk inaktivitet kan lede til flere af de store folkesygdomme som hjerte-kar-sygdomme, diabetes, osteoporose og psykiske lidelser. Menneskekroppen er ikke skabt til at være inaktiv, og derfor vil kroppen reagere kraftigt på det. For eksempel kan kroppen påbegynde nedbrydelse af knoglerne indefra, så de ikke vejer ret meget. 60 til 85\% af verdensbefolkningen lever en stillesiddende livsstil, hvilket forstærker forekomsten af disse folkesygdomme. \citep{Kiens2007,Reshma2002,Martini2012} Derudover kan inaktivitet lede til disuse syndromet, som blandt andet indebærer svækket hud integritet, ændret respiratorisk funktion og nedsætning af sanserne \citep{Knight2012,Mosby2009}. \\
Definitionen for overvægt er globalt sat ud fra et body mass index (BMI), hvilket er forholdet mellem en persons vægt i kg og højde i m$^2$. Et BMI på 25 eller derover er defineret som værende overvægt. \citep{Academic2016} Overvægt opstår grundlæggende fordi der indtages mere endegi end der forbruges. Nogle mennesker kan lagre fedt bedre end andre, hvorfor fedme også kan være genetisk betinget. \citep{Nestle2014} I forhistorien, da mennesket var jægere, var der en naturlig favorisering af de mennesker, som kunne lagre fedt bedre end andre, da der kunne gå lang tid imellem måltiderne. Evolutionen har endnu ikke tilpasset sig til den moderne livsstil, hvor der er let adgang til føde. \citep{Ahmad2014} \\
Fedme øger risikoen for højt kolesteroltal, forhøjet blodtryk og diabetes samt følgesygdomme heraf som slagtilfælde og nyresygdomme. Det er dokumenteret, at der er størst risiko for tidlig død jo yngre mennesker opnår overvægt. Det er derfor essentielt at forbedre børns aktivitet og dermed mindske risikoen for overvægt. \citep{Nestle2014} 

Forskelle..


\subsection{Fysiologisk aktiv}

% Opbygning af afsnit %%%%%%%%%%%%%%%%%%%%%%%%%%%%%%%%%%%%%%%%%%%%%%
%-- Inaktivt og/eller overvægtig 
% Kort introduktion, beskrivelse af hh. inaktivitet og overvægt.
% Følgesygdomme fra begge
% Sammenligning (Hvad er farligst, hvad er forskellen)
%-- Aktivt
% Kort introduktion, hvad betyder det at være aktiv(definition)
% Hvilken betydning har puls for forbrænding?
% Helbred 
%-- Kongnitiv funkktion (indlæring/Koncentration)
% Inaktivitet, fedme og aktivitets påvirkning af kongitiv funktion (I hvor lang tid kan aktivitet hjælpe på indlæring?)
% Hvornår skal man være aktiv i forhold til undervisning?