\section{Fysiologiske konsekvenser}
\textit{Følgende afsnit beskriver hvordan......... }

\subsection{Fysisk inaktivitet og overvægt}
Den moderne teknologi samt høje velstand har medført et mere fysisk inaktivt liv samtidig med at man har let adgang til føde\citep{Kiens2007}. Det er veldokumenteret, at der sker et fald i fysisk aktivitet med alderen samtidig med der sker en stigning i vægt\citep{Kaprio2008}. Undersøgelser tyder på, at hvis kroppens cellulære vedligeholdelse styrkes med fysisk aktivitet, så kan aldringsprocessen nedsættes\citep{Knight2012}. Fysisk inaktivitet forstærker altså den generelle aldring og anses som værende mindst lige så farligt som overvægt. De to fænomener forekommer dog ofte samtidig, da inaktivitet kan forsage overvægt, men fysisk inaktivitet har en selvstændig helbredsmæssig betydning ligesom overvægt har. Det er muligt at være overvægtig men samtidig have en aktiv livsstil.\citep{Kiens2007,Kaprio2008,Hjort1997}\\
Der er flere bud på verdensplan om, hvad definitionen for fysisk inaktivitet er. Sundhedsstyrelsen har derfor udarbejdet en generel definition ud fra de flere forskellige som lyder, at et individ er fysisk inaktiv, hvis vedkommende udfører mindre end 2,5 timers fysisk aktivitet om ugen med moderat intensitet\fxnote{Moderat intensitet svarer til 40-59\% af den maksimale iltoptagelse, eller 40-59\% af pulsreserven (maxpuls – hvilepuls), eller 64-74\% af maxpuls eller 12-13 RPE (rate of percieved excertion, Borgskala) og er yderligere defineret som fysisk aktivitet hvor man bliver lettere forpustet men hvor samtale er mulig.}.\citep{Kiens2007} Fysisk inaktivitet kan lede til flere af de store folkesygdomme som hjerte-kar-sygdomme, diabetes, osteoporose og psykiske lidelser. Menneskekroppen er ikke skabt til at være inaktiv, og derfor vil kroppen reagere kraftigt på det. For eksempel kan kroppen påbegynde nedbrydelse af knoglerne indefra, så de ikke vejer ret meget. 60 til 85\%\fxnote{Find nogle danske tal istedet} af verdensbefolkningen lever en stillesiddende livsstil, hvilket forstærker forekomsten af disse folkesygdomme.\citep{Kiens2007,Reshma2002,Martini2012} Derudover kan inaktivitet lede til disuse syndromet, som blandt andet indebærer svækket hud integritet, ændret respiratorisk funktion og nedsætning af sanserne\citep{Knight2012,Mosby2009}. \\
Definitionen for overvægt er globalt sat ud fra et body mass index (BMI), hvilket er forholdet mellem en persons vægt i kg og højde i m$^2$. Et BMI på 25 eller derover er defineret som værende overvægt.\citep{Academic2016} \fxnote{Men er BMI egentlig den bedste metode? Tager udgangspunkt i færdigudviklede højde, så er måske ikke bedst for man} Overvægt opstår grundlæggende fordi der indtages mere endegi end der forbruges. Nogle mennesker kan lagre fedt bedre end andre, hvorfor overvægt også kan være genetisk betinget.\citep{Nestle2014}\fxnote{I forhistorien, da mennesket var jægere, var der en naturlig favorisering af de mennesker, som kunne lagre fedt bedre end andre, da der kunne gå lang tid imellem måltiderne. Evolutionen har endnu ikke tilpasset sig til den moderne livsstil, hvor der er let adgang til føde. \citep{Ahmad2014}}\\
Fedme øger risikoen for højt kolesteroltal, forhøjet blodtryk og diabetes samt følgesygdomme heraf som slagtilfælde og nyresygdomme. Det er dokumenteret, at der er størst risiko for tidlig død jo yngre mennesker opnår overvægt. Det er derfor essentielt at forbedre børns aktivitet og dermed mindske risikoen for overvægt.\citep{Nestle2014} Derudover ses der, at overvægtige børn ofte lider af psykologiske og sociale problemer, hvilket kombineret med overvægten kan have en negativ indvirkning på barnets fremtid i forhold til uddannelse og socioøkonomiske status\citep{Academic2016}.

Inaktivitet kombineret med overvægt øger risikoen for diverse sygdomme, men en normalvægtig inaktiv person er i større risiko for tidlig dødsfald end en overvægt aktiv person. Ifølge et 12-års studie lavet over 334.161 europæiske deltagere så tyder det på, at dobbelt så mange vil dø af inaktivitet end overvægt.\citep{Ekelund2015} En aktiv overvægt person har derudover ikke større chance for at udvikle hjertesygdomme end normalvægtige, så længe de er trænede og dyrker motion\citep{Nichols2014}. Det tyder altså på, at inaktivitet er mere skadeligt end overvægt, hvis de sammenlignes som normalvægtig inaktiv mod overvægtig aktiv.

\subsection{Fysisk aktiv}
Fysisk aktivitet er defineret som enhver bevægelse, hvor skeletmuskler skal kontrahere og derved forbrænde energi. Der er forskellige former for fysisk aktivitet, som har forskellige intensitetsniveauer.\citep{Academic2016a} Ifølge Sundhedsstyrelsen skal et barn i alderen 5-17 år være fysisk aktiv i mindst 60 minutter om dagen med moderat til høj intensitet. Derudover anbefales det, at der tre gange om ugen skal indgå en aktivitet på 30 minutter med høj intensitet.\fxnote{Sundhedsstyrelsen2016} Hvis kroppen holdes fysisk aktiv, kan dette mindske risikoen for flere kroniske sygdomme som diabetes og hjertesygdomme. Derudover har fysisk aktivitet flere positive effekter på for eksempel knoglers metabolisme og menneskers psyke. \citep{Smith1991,Academic2016a} \\
Kroppen har mange reaktioner på fysisk aktivitet, hvilket blandt andet afhænger af aktivitetens krav til kroppen\fxnote{Skal muskelgrupper fremskynde en position som ved svømning og derved være udholdende eller skal muskelgrupper løfte en vægt som ved vægtløftning og derfor være eksplosiv men knap så udholdende?} og intensiteten heraf. Ved anstrengende fysisk aktivitet overtager sympatikus størstedelen af det autonome nervesystem og sætter for eksempel fordøjelsen på pause, da fordøjelse ikke længere er førsteprioritet og al kroppens energi kan bruges til aktivering af de pågældende skelletmuskelgrupper. Hjertet slår hurtigere, hvilket gør at pulsen stiger og ilt og næringsstoffer hurtigere sendes rundt i kroppen\citep{Hjerteforeningen}. Blodkar ved hudoverfladen vil spile ud, så kroppen kan slippe af med varmen fra bevægende muskler. \citep{Martini2012,Stanfield2013,Berchtold2010}
	

	% Sundhedsstyrrelsens anbefalinger, sygdomme det hjælper på, fortsat af notater.
% Opbygning af afsnit %%%%%%%%%%%%%%%%%%%%%%%%%%%%%%%%%%%%%%%%%%%%%%
%-- Inaktivt og/eller overvægtig
% Kort introduktion, beskrivelse af hh. inaktivitet og overvægt.
% Følgesygdomme fra begge
% Sammenligning (Hvad er farligst, hvad er forskellen)
%-- Aktivt
% Kort introduktion, hvad betyder det at være aktiv(definition)
% Hvilken betydning har puls for forbrænding?
% Helbred 
%-- Kongnitiv funkktion (indlæring/Koncentration)
% Inaktivitet, overvægt og aktivitets påvirkning af kongitiv funktion (I hvor lang tid kan aktivitet hjælpe på indlæring?)
% Hvornår skal man være aktiv i forhold til undervisning?