\subsection{Fysiologisk udbytte ved aktivitet}\label{subsec:fysio_aktivitet}
Fysisk aktivitet er defineret som enhver bevægelse, hvor skeletmuskler skal kontrahere og derved forbrænde energi. Der er forskellige former for fysisk aktivitet, som har forskellige intensitetsniveauer. \citep{Academic2016a} Ifølge Sundhedsstyrelsen skal et barn i alderen 5-17 år være fysisk aktiv i mindst 60 minutter om dagen med moderat til høj intensitet. Derudover anbefales det, at børn i denne alder skal indgå i en aktivitet i 30 minutter med høj intensitet tre gange om ugen. Det vil dermed være fordelagtigt for barnets helbredsniveau at følge disse anbefalinger. \citep{Sundhedsstyrelsen2016}\newline
Fysisk aktivitet kan mindske risikoen for flere kroniske sygdomme såsom overvægt, diabetes og hjertesygdomme. Eksempelvis kan overvægt både forbygges og afhjælpes af fysisk aktivitet, idet overvægt kan opstå ved et højere energiindtag end energiforbrug. Ydermere er fysisk aktivitet et forebyggende samt udviklende element for børns led, knogler og muskler. Eksempelvis dannes der mere synovialvæske ved fysisk aktiviteter, hvorved bevægelse af led faciliteres. Knogler vedligeholdes af desuden fysisk aktivitet, hvorved der undgås, at knoglens densitet mindskes som beskrevet i \secref{subsec:inover}. Ydermere udvikles og vedligeholdes muskler ligeledes af fysisk aktivitet, som følge af den belastning en fysisk aktivitet påfører muskelfibrene.  % Aktivitet medfører forbrænding af denne indtagede energi, hvorved ligevægtsindtaget muligvis ikke overskrides.
\citep{Academic2016a,Smith1991,Academic2016b,Cotman2007,CenterforDiseaseControlandPrevention2015}

Kroppen har mange reaktioner på fysisk aktivitet, hvilket blandt andet afhænger af aktivitetens krav til kroppen\fxnote{Skal muskelgrupper fremskynde en position som ved svømning og derved være udholdende eller skal muskelgrupper løfte en vægt som ved vægtløftning og derfor være eksplosiv men knap så udholdende} og intensiteten heraf. %Ved anstrengende fysisk aktivitet overtager sympatikus størstedelen af det autonome nervesystem og sætter for eksempel fordøjelsen på pause, da fordøjelse ikke længere er førsteprioritet og al kroppens energi kan bruges til aktivering af de pågældende skeletmuskler. 
Eksempelvis typer studier på, at fysisk aktivitet har en positiv indvirkning på børns kognition \citep{SibleyEtnier2003}. Ydermere vil en anstrengende fysisk aktivitet få hjertet til at slå hurtigt, hvilket medfører en øget puls, hvormed ilt og næringsstoffer hurtigere sendes rundt i kroppen \citep{Hjerteforeningen}. Blodkar vil desuden blive udspilet, således blodet i større grad kan komme til hudoverfladen og afgive den varme, som blodet fører væk fra de aktive muskler. Der sker altså en stigning i pulsen og blodtrykket, og denne stigning afhænger af den pågældende aktivitets påvirkning på kroppen. \citep{Martini2012,Stanfield2013,Berchtold2010}


%  OBS: Generelt - vi skal have flyttet den defination vi snakkede om.
%  OBS: Vi skal måske lige snakke om relevansen for dette afsnit
%		Det eneste formål vi ønsker med dette afsnit er 'At sikre hvorfor vi ønsker 
%       de krav som vi gør. --> Burde det så ikke bare være et kort afsnit inde i 
%       opstillingen af succeskrav
%
%	Hvis afsnittet skal være her, er dette en mulig opstilling med indhold.
%	
%	Afsnittets opbygning: 
%	- Nævn betegnelsen inaktivitet og anbefalet aktivitet --> Skal lede op til: Hvad skal man egentlig lave så?
%	- Opremsning af aktiviteter som hører under moderart til høj intensitet -> aktivitet som ville være fordelagtige at udeføre for at opnå anbefalingerne.
%		- Tabel fra sundhedsstyrrelsen nævner mange til som børn bør lave som ville 
%		  aktivere dem nok til at få nok ud af motionen (Foldbold, cykling, leg i skolegård)
%		- Disse aktiviteter skal blandt andet ligge til grundlag for senere valg af krav
%		- Disse aktiviteter skal definere 'gængs aktivitet blandt skolebørn' (Uden at gøre det for groft, = Gang, løb og cykling bør dække nævnte aktiviteter)
%		- https://sundhedsstyrelsen.dk/da/sundhed-og-livsstil/fysisk-aktivitet/anbefalinger/~/media/31FF5ED226F643D0A6948B52948E5DB3.ashx
% 	- Argument for cykling er/bør være en gængs aktivitet
% 		- Giver øget koncentrationsevne+indlæringsevne 
%		- Børn der cykler er sundere
%		- https://www.cyklistforbundet.dk/Om-os/Vi-mener/Cyklistforbundet-mener/Boerns-cykling-til-skole-og-i-fritiden
%	- Opsummering af hvorfor dette er/bør være gængs aktivitet for et barns hverdag
%	  og hvordan det påvirker børn rent fysiologisk.
		

%
%
	% Sundhedsstyrrelsens anbefalinger, sygdomme det hjælper på, fortsat af notater.
% Opbygning af afsnit %%%%%%%%%%%%%%%%%%%%%%%%%%%%%%%%%%%%%%%%%%%%%%
%-- Inaktivt og/eller overvægtig
% Kort introduktion, beskrivelse af hh. inaktivitet og overvægt.
% Følgesygdomme fra begge
% Sammenligning (Hvad er farligst, hvad er forskellen)
%-- Aktivt
% Kort introduktion, hvad betyder det at være aktiv(definition)
% Hvilken betydning har puls for forbrænding?
% Helbred 
%-- Kongnitiv funkktion (indlæring/Koncentration)
% Inaktivitet, overvægt og aktivitets påvirkning af kongitiv funktion (I hvor lang tid kan aktivitet hjælpe på indlæring?)
% Hvornår skal man være aktiv i forhold til undervisning?