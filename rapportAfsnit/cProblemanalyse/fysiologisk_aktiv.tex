\subsection{Fysiologisk udbytte ved fysisk aktivitet}\label{subsec:fysio_aktivitet}
%Fysisk aktivitet er defineret som enhver bevægelse, hvor skeletmuskler skal kontrahere og derved forbrænder energi. 
Der er forskellige former for fysisk aktivitet, som har forskellige intensitetsniveauer~\citep{Academic2016a}. Ifølge Sundhedsstyrelsen skal et barn i alderen 5-17~år være fysisk aktiv i mindst 60 minutter om dagen med moderat til høj intensitet. Derudover anbefales det, at børn i denne alder skal udføre fysisk aktivitet i 30 minutter med et højt intensitetsniveau tre gange om ugen.~\citep{Sundhedsstyrelsen2016}\newline
Fysisk aktivitet kan mindske risikoen for flere sygdomme såsom overvægt, diabetes og hjertekarsygdomme. Eksempelvis kan overvægt både forbygges og afhjælpes af fysisk aktivitet. Ydermere er fysisk aktivitet et forebyggende samt udviklende element for børns led, knogler og muskler. Eksempelvis dannes der mere synovialvæske ved fysisk aktiviteter, hvorved bevægelse af led faciliteres. Knogler vedligeholdes desuden af fysisk aktivitet, hvorved det kan undgås, at knoglens densitet mindskes. Ydermere udvikles og vedligeholdes muskler af fysisk aktivitet, som følge af den belastning en fysisk aktivitet påfører muskelfibrene.~\citep{Smith1991,Academic2016b,CenterforDiseaseControlandPrevention2015}

Kroppens udbytte af fysisk aktivitet afhænger blandt andet af aktivitetstypen\fxnote{Skal muskelgrupper fremskynde en position som ved svømning og derved være udholdende eller skal muskelgrupper løfte en vægt som ved vægtløftning og derfor være eksplosiv men knap så udholdende} og intensiteten heraf. Eksempelvis ter et studie på, at fysisk aktivitet har en positiv indvirkning på børns kognition. \citep{SibleyEtnier2003} Ydermere vil en anstrengende fysisk aktivitet få hjertet til at slå hurtigt, hvormed ilt og næringsstoffer hurtigere sendes rundt i kroppen \citep{Hjerteforeningen}. Blodkar vil desuden blive udspilet, således blodet i større grad kan komme til hudoverfladen og afgive den varme, som blodet fører væk fra de aktive muskler. Der sker altså en stigning i pulsen og blodtrykket, og denne stigning afhænger af den pågældende aktivitets påvirkning på kroppen. \citep{Martini2012,Stanfield2013,Berchtold2010}