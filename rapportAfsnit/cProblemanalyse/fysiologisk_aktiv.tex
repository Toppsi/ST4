\subsection{Fysiologisk udbytte ved aktivitet}\label{subsec:fysio_aktivitet}
Fysisk aktivitet er defineret som enhver bevægelse, hvor skeletmuskler skal kontrahere og derved forbrænde energi. Der er forskellige former for fysisk aktivitet, som har forskellige intensitetsniveauer. \citep{Academic2016a} Ifølge Sundhedsstyrelsen skal et barn i alderen 5-17~år være fysisk aktiv i mindst 60 minutter om dagen med moderat til høj intensitet. Derudover anbefales det, at børn i denne alder skal indgå i en aktivitet i 30 minutter med høj intensitet tre gange om ugen. Det vil dermed være fordelagtigt for barnets helbredsniveau at følge disse anbefalinger. \citep{Sundhedsstyrelsen2016}\newline
Fysisk aktivitet kan mindske risikoen for flere kroniske sygdomme såsom overvægt, diabetes og hjertekarsygdomme. Eksempelvis kan overvægt både forbygges og afhjælpes af fysisk aktivitet. Ydermere er fysisk aktivitet et forebyggende samt udviklende element for børns led, knogler og muskler. Eksempelvis dannes der mere synovialvæske ved fysisk aktiviteter, hvorved bevægelse af led faciliteres. Knogler vedligeholdes desuden af fysisk aktivitet, hvorved det kan undgås, at knoglens densitet mindskes som beskrevet i \secref{subsec:inover}. Ydermere udvikles og vedligeholdes muskler ligeledes af fysisk aktivitet, som følge af den belastning en fysisk aktivitet påfører muskelfibrene.  % Aktivitet medfører forbrænding af denne indtagede energi, hvorved ligevægtsindtaget muligvis ikke overskrides.
\citep{Academic2016a,Smith1991,Academic2016b,Cotman2007,CenterforDiseaseControlandPrevention2015}

Kroppens reaktion på fysisk aktivitet afhænger blandt andet af aktivitetens krav til kroppen\fxnote{Skal muskelgrupper fremskynde en position som ved svømning og derved være udholdende eller skal muskelgrupper løfte en vægt som ved vægtløftning og derfor være eksplosiv men knap så udholdende} og intensiteten heraf. %Ved anstrengende fysisk aktivitet overtager sympatikus størstedelen af det autonome nervesystem og sætter for eksempel fordøjelsen på pause, da fordøjelse ikke længere er førsteprioritet og al kroppens energi kan bruges til aktivering af de pågældende skeletmuskler. 
Eksempelvis tyder studier på, at fysisk aktivitet har en positiv indvirkning på børns kognition. \citep{SibleyEtnier2003} Ydermere vil en anstrengende fysisk aktivitet få hjertet til at slå hurtigt, hvilket medfører en øget puls, hvormed ilt og næringsstoffer hurtigere sendes rundt i kroppen \citep{Hjerteforeningen}. Blodkar vil desuden blive udspilet, således blodet i større grad kan komme til hudoverfladen og afgive den varme, som blodet fører væk fra de aktive muskler. Der sker altså en stigning i pulsen og blodtrykket, og denne stigning afhænger af den pågældende aktivitets påvirkning på kroppen. \citep{Martini2012,Stanfield2013,Berchtold2010}