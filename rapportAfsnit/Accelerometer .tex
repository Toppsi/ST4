\subsection 

Et accelerometer er et elektromekanisk apparat som anvendes til at m�le accelerationskr�fter. 
Det kan blandt andet bestemme om et objekt bev�ger sig opad, nedad og i hvilken vinkel. Enheden ber m�lt i meter pr sekund i anden(m/s^2) eller i G kr�fter (g) 
Accelerometre kan opdeles i flere typer hvor piezoelektriske accelerometer er den mest anvendte at typen. Den anvender mikroskopiske krystalstrukturer der bliver aktiveret ved accelerationskr�fter. Disse krystaliske strukturer danner en sp�nding og accelerometeret fortolker denne sp�nding til at bestemme fart og orientering. 

Et accelerometer m�ler to former for acceleration henholdsvis statisk og dynamisk, hvor de statiske kr�fter inkluderer tyngdekraften og dynamisk kr�fter inkluderer vibrationer og bev�gelse. 

Accerometre m�ler acceleration af en, to(x,y) eller tre(x,y,z) akser. For eksempel er der en to akser i en bil, og de fleste smartphones har 3 akser. 
