\chapter{Pilotforsøg}\vspace{-.75cm}\label{pilot}
\section{Formål}
Pilotforsøget udføres med henblik på at kunne designe algoritmer, som adskiller gang, løb og cykling. Det undersøges derudover, hvilke af accelerometerets og gyroskopets akser der er essentielle at lave algoritmer ud fra. Ydermere undersøges signalernes frekvens for at undgå aliasing i det endelige system. Sidst undersøges hvilken indflydelse placering af sensoren har på signalets udformning. Dette gøres, så det endelige systems signal ikke går i mætning på grund af for stor kraftpåvirkning, og for at undersøge om placering har indflydelse på signalernes udformning.

Til opsamling af data anvendes en Shimmer3. Dette er en enhed, som indeholder en række sensorer\fxnote{accelerometer, gyroskop, tryksensor, magnometer, højdemåler}, hvor der til forsøget benyttes et accelerometer og et gyroskop. 

Formålet med pilotforsøget er dermed:
\begin{itemize}
	\item At undersøge hvordan signalerne for gang, løb og cykling kan adskilles. 
	\item At undersøge hvilken betydning placering af sensorene har for signalets udformning ved gang, løb og cykling. 
	\item At bestemme frekvensområdet for signalerne.
	\item At bestemme amplituderne for signalerne.
\end{itemize}

\section{Metode}
Til forsøget medtages forsøgspersoner, som ikke lider af gener, der forhindrer dem i at udføre aktiviteterne gang, løb og cykling. Er en person skadet eller syg, eksluderes vedkommende dermed fra forsøget. Der udføres kun forsøg på gruppemedlemmer, og det er derfor ikke muligt at udføre forsøget på børn fra målgruppen, som er på 9-12 år. Resultaterne kan dermed eriere i forhold til målgruppen, da deres vægt og højde antages ikke at være tilserende forsøgspersonernes~\citep{Rigsholspitalet2014}. 

Forsøget tager udgangspunkt i tre forudbestemte placeringer på underbenet af enheden Shimmer3, hvilke kan ses på \figref{fig:sensor_placering}. Disse placeringer er udvalgt på baggrund af \secref{bevaegelse}, hvor det ses, at de største bevægelser optræder her i forbindelse med gang, løb og cykling. Accelerometeret registrerer position og acceleration, og det forventes derfor, at den største forskel vil kunne ses ved disse placeringer. I databehandlingen behandles kun data fra accelerometerets y-akse, da denne bør have den største kraftpåvirkning på baggrund af \secref{bevaegelse}.

\subsection{Materialer}
\begin{itemize}
	\item Løbebånd med justerbar hastighed og sikkerhedsbæresele.
	\item Motionscykel.
	\item Shimmer3 sensor med tilhørende holder og strap.
	\item Sportstape.
	\item Computer med følgende software:
	\begin{itemize}\vspace{-.15cm}
		\item Labview.
		\item Shimmer Sensing.
	\end{itemize}
\end{itemize}

\subsection{Fremgangsmåde}
Forsøgets fremgangsmåde er opdelt i to. Første del indeholder en opsætning af Shimmer3, mens den anden del er fremgangsmåden for optagelse af data fra forsøget.

\subsubsection{Opsætning af Shimmer3}
Før forsøgene kan udføres skal Shimmer3 forbindes korrekt med computeren og indstilles til at bruge de sensorer, der ønskes i pilotforsøget. \vspace{-3mm}
\begin{itemize}
	\item Shimmer3 forbindes til programmet Labview gennem bluetooth.
	\item Shimmer3 indeholder en række sensorer, hvoraf følgende skal aktiveres: 
	\begin{itemize}\vspace{-.15cm}
		\item Widerange Accelerometer.
		\item Gyroscope.
	\end{itemize}
	\item De maksimale arbejdsområder på $\pm$16~g og $\pm$2000~dps vælges, idet signalets amplitude endnu er ukendt.
	\item Samplingsfrekvensen indstilles på 512~Hz, da signalets frekvens er ukendt, og denne samplingsfrekvens er den maksimale der kan vælges, når både gyroskopet og accelerometeret er i brug.  
	\item Det er nu muligt at starte stream og derefter følge optagelserne i realtid.
\end{itemize}

\subsubsection{Udførsel af forsøget}
Forsøget udføres på fire forsøgspersoner, som alle skal udføre aktiviteterne gang, løb, hastighedsstigning og cykling. Den nedenstående beskrivelse af forsøgets fremgangsmåde er gældende for én af de forudbestemte placeringer af Shimmer3 på forsøgspersonens højre ben. Alle fire aktiviteter udføres før placeringen ændres, dog benyttes den samme fremgangsmåde til de resterende to placeringer. De tre placeringer kan ses på \figref{fig:sensor_placering}.
\begin{figure}[H]
	\centering
	\includegraphics[scale=0.55]{figures/qBilag/Sensor_placering2.png}
	\caption{På figuren ses hvor sensoren skal placeres under pilotforsøget. Placering A: proximalt for den laterale malleolus. Placering B: medialt på den laterale side af tibia. Placering C: distalt for patella på den laterale side. \citep{Perna2016,Shimmer2016} (Modificeret)}
	\label{fig:sensor_placering}
\end{figure}
Inden forsøget skal forsøgspersonen fastspændes i en sikkerhedssele, således der ikke opstår skader, hvis personen snubler på løbebåndet. Derudover skal forsøgspersonen inden hver måling fortælle, hvor på Borgskalaen vedkommende befinder sig, som kan ses på \figref{fig:borgskala}. Er dette under 11, kan målingen påbegyndes. Denne værdi er valgt for, at forsøgspersonen ikke foretager en aktivitet anderledes grundet muskeltræthed. Det sikres dermed, at alle forsøgspersoner har samme startbetingelser for alle forsøg. 
\begin{figure}[H]
	\centering
	\includegraphics[scale=0.35]{figures/qBilag/Borg-skala.jpg}
	\caption{På figuren ses Borgskalaen, som benyttes inden forsøgets start.~\citep{Patientinformationen2013} (Modificeret)}
	\label{fig:borgskala}
\end{figure}\vspace{-.25cm}

Første måling er gang, hvor hastigheden er 4,8~km/t og betegnes som værende en moderat hastighed~\citep{Miles2007}. %\vspace{-3mm}
\begin{itemize}
	\item Der foretages en baseline på 10~sekunder, hvor forsøgspersonen skal stå oprejst med ret ryg og fødderne placeret parallelt og kigge lige frem.
	\item Løbebåndet indstilles til 4,8~km/t, hvor forsøgspersonen går på løbebåndet indtil en konstant hastighed opnås. 
	%		\item Forsøgspersonen indikerer når denne føler en homogen bevægelses-cyklus.
	\item Målingen på 45~sekunder igangsættes.
\end{itemize}

Anden måling er løb, hvor hastigheden 11,3~km/t og betegnes som værende en energisk hastighed~\citep{Miles2007}. Fremgangsmåden for denne test er tilserende ovenstående fremgangsmåde for gang.

Tredje måling foretages på løbebåndet, hvor forsøgspersonen gradvist skal stige i tempo under hele forsøget. Der noteres under forsøget, hvornår forsøgspersonen skifter fra gang til løb.  %\vspace{-3mm}
\begin{itemize}
	\item Der foretages en baseline på 10~sekunder, hvor forsøgspersonen skal stå oprejst med ret ryg og fødderne placeret parallelt og kigge lige frem.
	\item Målingen igangsættes.
	\item Løbebåndet indstilles til 2~km/t, hvor forsøgspersonen skal gå i 20~sekunder.  
	\item Hastigheden stiger herefter med 2~km/t hvert 20. sekund, indtil forsøgspersonen har opnået sin vurderede maksimale hastighed, eller løbebåndets maksimale hastighed på 18~km/t opnås. 
	\item Målingen stoppes. 
\end{itemize}

Sidste måling fortages under cykling med en hastighed på 20,9~km/t, hvilket betegnes som værende et højt cykeltempo \citep{Miles2007}. Tempoet er dog underordnet, da der kun ønskes at se på forskellen i selve bevægelsen fra de andre aktivitetsformer, men der er valgt et fast tempo for at få et ensformigt signal. %\vspace{-3mm}
\begin{itemize}
	\item Der foretages en baseline på 10~sekunder, hvor forsøgspersonen skal sidde i en naturlig cykelposition på motionscyklen med begge fødder på pedalerne, hvoraf den højre pedal skal være helt i bund. Denne position er valgt, da den er mulig at lave tilnærmelsesvis ens for alle forsøgspersoner, hvormed de får den samme baseline.
	\item Forsøgspersonen træder i pedalerne, indtil vedkommende opnår en konstant hastighed på 20,9~km/t ved en belastning på 35~W. Dermed sikres det, at alle forsøgspersoner bruger den samme belastning gennem forsøget.  
	\item Målingen på 45~sekunder igangsættes. 
\end{itemize}
Efter de tre placeringer skulle forsøgspersonerne vurdere hvilken placering der er mest behagelig.

\section{Databehandling}
\subsection{Kalibrering af Shimmer3}
Forud for pilotforsøgets målinger blev Shimmer3 kalibreret og testet. For at undersøge hvorvidt kalibreringen af Shimmer3 fungerede optimalt, blev der opsamlet data til at be- eller afkræfte dette. Data fra de tre akser, x, y og z blev behandlet. \\
Når Shimmer3 er placeret i en kalibreringsboks på et bord med henblik på en respektiv akse, bør accelerometeret blive påvirket med $\pm$1~g, mens de resterende akser ikke bør påvirkes.
\begin{figure}[H]
	\centering
	\includegraphics[width=1\textwidth]{figures/qBilag/kalibreringsdata}
	\caption{På figuren ses kalibreringsdataene tilhørende accelerometerets x-, y- og z-akse.}
	\label{fig:Ap_Kalibrering}
\end{figure}
For hver akse udregnes den gennemsnitlige værdi for henholdsvis den positive- og negative akse og sammenholdes med $\pm 1$g. Dermed bestemmes blev den procentmæssige afvigelse fra tyngdeaccelerationen. Resultatet heraf ses i \tabref{fig:akser_pilot}
\begin{table}[H]
	\centering
	\begin{tabular}{ccc}		\hline
		\rowcolor[HTML]{C0C0C0} Akse & Positiv retning {[}\%{]} & Negativ retning {[}\%{]} \\ \hline
		x & 3,5 & -2,2 \\ \hline
		y & -0,6 & -2,6 \\ \hline
		z & 8,0 & 8,8 \\ \hline
	\end{tabular}
	\caption{I tabellen ses afvigelsen for hver af accelerometerets akser ved kalibrering.}
	\label{fig:akser_pilot}
\end{table}\vspace{-.25cm}
Kalibreringen foretages for at sikre, at et offset ikke er til stede. 

\subsection{Baseline for gang, løb og cykling}
Forud for hver enkelt måling foretages en baselinemåling som indikation for hvorvidt Shimmer3 fungerer forud for aktiviteten. Derudover benyttes baseline til at teste, hvorvidt Shimmer3 er i samme position for alle forsøgspersoner ved de forskellige målingers start. Dataene skal afspejle en tilnærmelsesvis fuldstændig tyngdekraftpåvirkning på accelerometerets y-akse, som resultat af Shimmer3s placering på benet. Baseline foretages for at sikre, at Shimmer3 tilnærmelsesvis bliver placeret ens på alle forsøgspersoner, hvormed data kan sammenholdes. 
\begin{table}[H]
	\centering
	\begin{tabular}{cccc}
		\hline
		\cellcolor[HTML]{C0C0C0} Forsøgsperson & \cellcolor[HTML]{C0C0C0}\begin{tabular}[c]{@{}c@{}}Placering A, \\ y-akse {[}g{]}\end{tabular} & \cellcolor[HTML]{C0C0C0} \begin{tabular}[c]{@{}c@{}}Placering B,\\ y-akse {[}g{]}\end{tabular} & \cellcolor[HTML]{C0C0C0} \begin{tabular}[c]{@{}c@{}}Placering C,\\ y-akse {[}g{]}\end{tabular} \\ \hline
		F1 & 0,98 & 0,99 & 0,97 \\ \hline
		F2 & 1 & 0,99 & 0,96 \\ \hline
		F3 & 0,98 & 0,98 & 0,98 \\ \hline
		F4 & 0,97 & 0,99 & 0,95 \\ \hline
	\end{tabular}
	\caption{I tabellen ses de gennemsnitlige baselineresultater fra accelerometerets y-akse forud for gang.}
	\label{fig:Ap_baselinegang}
\end{table}\vspace{-0.5cm}
\begin{table}[H]
	\centering
	\begin{tabular}{cccc}
		\hline
		\cellcolor[HTML]{C0C0C0} Forsøgsperson & \cellcolor[HTML]{C0C0C0} \begin{tabular}[c]{@{}c@{}}Placering A, \\ y-akse {[}g{]}\end{tabular} & \cellcolor[HTML]{C0C0C0} \begin{tabular}[c]{@{}c@{}}Placering B,\\ y-akse {[}g{]}\end{tabular} & \cellcolor[HTML]{C0C0C0} \begin{tabular}[c]{@{}c@{}}Placering C,\\ y-akse {[}g{]}\end{tabular} \\ \hline
		F1 & 0,99 & 0,99 & 0,97 \\ \hline
		F2 & 0,99 & 0,99 & 0,96 \\ \hline
		F3 & 0,97 & 0,98 & 0,98 \\ \hline
		F4 & 0,97 & 0,99 & 0,95 \\ \hline
	\end{tabular}
	\caption{I tabellen ses de gennemsnitlige baselineresultater fra accelerometerets y-akse forud for løb.}
	\label{fig:Ap_baselineloeb}
\end{table}\vspace{-0.5cm}

Ved cykling benyttes gyroskopets data, da cykling detekteres som en roterende bevægelse omkring z-aksen. Enheden af dataet heraf er dps, og dermed bør baselineresultaterne ligge omkring nul.
\begin{table}[H]
	\centering
	\begin{tabular}{cccc}
		\hline
		\cellcolor[HTML]{C0C0C0} Forsøgsperson & \cellcolor[HTML]{C0C0C0} \begin{tabular}[c]{@{}c@{}}Placering A, \\ z-akse {[}dps{]}\end{tabular} & \cellcolor[HTML]{C0C0C0} \begin{tabular}[c]{@{}c@{}}Placering B,\\ z-akse {[}dps{]}\end{tabular} & \cellcolor[HTML]{C0C0C0} \begin{tabular}[c]{@{}c@{}}Placering C,\\ z-akse {[}dps{]}\end{tabular} \\ \hline
		F1 & -0,98 & -0,83 & -0,87 \\ \hline
		F2 & -0,90 & -0,79 & -0,77 \\ \hline
		F3 & -0,68 & -0,58 & -0,99 \\ \hline
		F4 & -0,89 & -0,92 & -0,85 \\ \hline
	\end{tabular}
	\caption{I tabellen ses de gennemsnitlige baselineresultater fra gyroskopets z-akse forud for cykling.}
	\label{fig:Ap_baselinecykling}
\end{table}\vspace{-0.5cm}

\subsection{Minimum og maksimum g-påvirkning under gang, løb og hastighed}
Dataene fra aktiviteterne gang, løb og hastighedsstigning behandles med henblik på bestemmelse af den maksimale g-påvirkning heraf. Dette bliver bestemt af den maksimale påvirkning i henholdsvis accelerometerets positive og negative y-akse samt placeringer. Før forsøgene bestemmes baseline forinden hvert forsøg. 

Den største afvigelse fra tyngdeaccelerationen for gang er på 0,9969\%, hvormed det vurderes, at alle baselines er uden betydeligt offset.\Tabref{fig:Ap_maxggang} viser resultaterne fra gang med et hastighed på 4,8~km/t.
\begin{table}[H]
	\centering
	\begin{tabular}{cccc}
		\hline
		\rowcolor[HTML]{C0C0C0} 
		Forsøgsperson & \begin{tabular}[c]{@{}c@{}} Placering A {[}g{]}\end{tabular} & \begin{tabular}[c]{@{}c@{}} Placering B {[}g{]}\end{tabular} & \begin{tabular}[c]{@{}c@{}} Placering C {[}g{]}\end{tabular} \\ \hline
		F1 &  0,09 ; 2,51  & 0,00 ; 2,32  & -2,51 ; 3,33 \\ \hline
		F2 &  -0,19 ; 3,19 & -0,43 ; 3,04 & -0,97 ; 2,84 \\ \hline
		F3 &  -0,24 ; 3,52 & -0,39 ; 3,38 & -0,20 ; 2,51 \\ \hline
		F4 &  -0,04 ; 2,84 & -0,29 ; 3,62 & -0,50 ; 3,52 \\ \hline
	\end{tabular}%
	\caption{I tabellen ses de maksimale positive og negative værdier fra accelerometerets y-akse som resultat af gang med en hastighed på 4,8~km/t. Værdierne er fundet for både placering A, B og C.}
	\label{fig:Ap_maxggang}
\end{table}\vspace{-.25cm}
Det ses i \tabref{fig:Ap_maxggang}, at den maksimale g-påvirkning under gang for placering A er i intervallet -0,24~g til 3,52~g. Den maksimale g-påvirkning for placering B er i intervallet -0,43~g til 3,62~g mens den for C er -2,51~g til 3,52~g.

På samme måde findes baseline for løb ved en hastighed på 11,3~km/t, som maksimalt afviger med 0,9930\%. Det vurderes derfor, at baseline for alle forsøgspersoner inden løb er uden betydeligt offset. Herefter blev der fundet de maksimale positive og negative værdier for løb, som kan ses i \tabref{fig:Ap_maxgloeb}.  
\begin{table}[H]
	\centering
	\begin{tabular}{cccc}
		\hline
		\rowcolor[HTML]{C0C0C0} 
		Forsøgsperson & \begin{tabular}[c]{@{}c@{}} Placering A {[}g{]}\end{tabular} & \begin{tabular}[c]{@{}c@{}} Placering B {[}g{]}\end{tabular} & \begin{tabular}[c]{@{}c@{}} Placering C {[}g{]}\end{tabular} \\ \hline
		F1 &  -2,03 ; 8,59 & -2,80 ; 5,07 & -4,10 ; 3,33 \\ \hline
		F2 &  -0,97 ; 5,35 & -2,51 ; 6,13 & -4,44 ; 6,52 \\ \hline
		F3 &  -2,12 ; 5,55 & -1,83 ; 5,60 & -2,46 ; 5,60 \\ \hline
		F4 &  -3,48 ; 6,42 & -4,63 ; 6,76 & -3,52 ; 8,30 \\ \hline
	\end{tabular}%
	\caption{I tabellen ses de maksimale positive og negative værdier fra accelerometerets y-akse som resultat af løb med en hastighed på 11,3~km/t. Værdierne er fundet for både placering A, B og C.}
	\label{fig:Ap_maxgloeb}
\end{table}\vspace{-.25cm}
Det ses i \tabref{fig:Ap_maxgloeb}, at den maksimale g-påvirkning under gang for placering A er i intervallet -3,48~g til 4,59~g. Den maksimale g-påvirkning for placering B er i intervallet -4,63~g til 6,76~g mens den for C er -4,44~g til 8,30~g.

Afslutningsvis bliver accelerometerets y-akse undersøgt ved forsøget, hvor forsøgspersonerne gradvist stiger i tempo. Baseline for disse målinger afviger med 0,9954\%, hvormed det vurderes, at baseline for alle målinger er neutrale. Den maksimale påvirkning i henholdsvis positiv og negativ retning, som detekteres under hastighedsstigningen, kan ses i \tabref{fig:Ap_maxghastighed1} 
\begin{table}[H]
	\centering
	\begin{tabular}{cccc}
		\hline
		\rowcolor[HTML]{C0C0C0} 
		Forsøgsperson & \begin{tabular}[c]{@{}c@{}} Placering A {[}g{]}\end{tabular} & \begin{tabular}[c]{@{}c@{}} Placering B {[}g{]}\end{tabular} & \begin{tabular}[c]{@{}c@{}} Placering C {[}g{]}\end{tabular} \\ \hline
		F1 &  -3,04 ; 8,20  & -4.59 ; 6,28  & -6,66 ; 7,10 \\ \hline
		F2 &  -3,19 ; 10,96 & -4,49 ; 10,48 & -7,58  ; 9,61 \\ \hline
		F3 &  -4,92 ; 10,48 & -4,59 ; 13,13 & -4,63 ; 9,70 \\ \hline
		F4 &  -8,83 ; 16,95 & -7,48 ; 16,32 & -8,01 ; 15,35 \\ \hline
	\end{tabular}%
	\caption{I tabellen ses de maksimale positive og negative værdier fra accelerometerets y-akse som resultat af hastigheds stigning. Værdierne er fundet for både placering A, B og C.}
	\label{fig:Ap_maxghastighed1}
\end{table}\vspace{-.25cm}
Det ses i \tabref{fig:Ap_maxghastighed1}, at den maksimale g-påvirkning under gang for placering A er i intervallet -8,83~g til 16,95~g. Den maksimale g-påvirkning for placering B er i intervallet -7,48~g til 16,32~g mens den for C er -8,01~g til 15,35~g.

\subsection{Maksimal omdrejninger per sekund under cykling}
Dataene fra aktiviteten, cykling, behandles med henblik på bestemmelse af den maksimale amplitude fra gyroskopet. Dette bestemmes ved at beregne den maksimale peak-to-peak under udførelsen af cykling. Dataene behandles med henblik på gyroskopets z-akse som resultat af \secref{bevaegelse}. Inden dataopsamling for cykling måles en baseline. Den maksimale afvigelse fra en værdi på 0 er -0,9979\%, hvormed det vurderes, at alle målinger har en neutral baseline. Dataene fra forsøget kan ses i \tabref{fig:Ap_maxghastighed}.
\begin{table}[H]
	\centering
		\begin{tabular}{cccc}
			\hline
			\rowcolor[HTML]{C0C0C0} 
			Forsøgsperson & \begin{tabular}[c]{@{}c@{}} Placering A {[}dps{]}\end{tabular} & \begin{tabular}[c]{@{}c@{}} Placering B {[}dps{]}\end{tabular} & \begin{tabular}[c]{@{}c@{}} Placering C {[}dps{]}\end{tabular} \\ \hline
			F1 & -148,23 ; 108,29   & -209,82 ; 118,60   & -188,66 ; 98,29  \\ \hline
			F2 & -108,42 ; 108,11   & -133,11 ; 114,94	 & -150,43 ; 120,61 \\ \hline
			F3 & -208,29 ; 136,28  	& -196,95 ; 140,18	 & -195,43 ; 151,10 \\ \hline
			F4 & -182,56 ; 152,13  	& -159,82 ; 138,35	 & -152,62 ; 136,83 \\ \hline
		\end{tabular}%
	\caption{I tabellen ses de maksimale positive og negative værdier fra gyroskopets z-akse som resultat af cykling med en hastighed på 20,9 km/t. Værdierne er fundet for både placering A, B og C.}
	\label{fig:Ap_maxghastighed}
\end{table}\vspace{-.25cm}
Det ses i \tabref{fig:Ap_maxghastighed}, at den maksimale dps under gang for placering A er i intervallet -208,29~dps til 152,13~dps. Den maksimale g-påvirkning for placering B er i intervallet -209,82~dps til 140,18~dps mens den for C er -195,43~dps til 151,10~dps. 

\subsection{Afgrænsning af placering}
Databehandling tager udgangspunkt i de maksimale værdier fra placering A. Dette gøres på baggrund af, at denne er den mest optimale placering i forhold til komfort for brugeren, da tre ud af fire forsøgspersoner foretrak denne placering. Den maksimale værdi for placering A overskrider den maksimale accelerationskraftpåvirkning med 0,95 g. Det vurderes dog at placering A er optimal at bruge, da de 16,95~g repræsenteres i form af hælnedslag. Det vil stadig være muligt at adskille hælnedslag fra det resterende signal, selvom det vil klippes ved 16~g. \newline
Gyroskopets data viser ligeledes, at det er muligt at benytte placering A, da denne viser at cykling ikke resulterer i en høj dps. På baggrund af dette vil der i det resterende databehandling tages udgangspunkt i placering A, som kan ses i fem sekunders interval for hver af de fire forsøgspersoner på \figref{raa_data}.
\begin{figure}[H]
	\centering
	\includegraphics[width=1\textwidth]{figures/qBilag/raa_data}
	\caption{På figuren ses det ubehandlede data fra de tre aktivitetstyper gang, løb og cykling, hvoraf data for gang og løb er fra accelerometret mens data for cykling er fra gyroskopet ved placering A.}
	\label{raa_data}
\end{figure}\vspace{-.25cm}

\subsection{Frekvensindhold af gang, løb og cykling}
Dataene fra aktiviteterne gang og løb behandles for at bestemme signalernes frekvensindhold. Resultatet af dette muliggør bestemmelsen af samplingsfrekvensen vedrørende accelerometeret og gyroskopet. Der foretages en frekvensdomæne analyse, hvilket muliggør visualisering af signalets magnitude ved forskellige frekvenser, hvoraf energien af signalet kommer til udtryk.
\begin{figure}[H]
	\centering
	\includegraphics[width=1\textwidth]{figures/qBilag/fft_f1_loeb}
	\caption{På figuren ses frekvensdomænet af aktiviteten løb for forsøgsperson 1. Den fuldstændige magnitude for de lave frekvenser vises ikke til fulde. Hvis dette skulle være tilfældet ville de mindste magnituder på figuren blive udskalleret.}
	\label{fig:Ap_FFt}
\end{figure}\vspace{-.25cm}
Frekvensdomæneanalysen vises kun for løb af F1, da frekvensspektrummet er størst heraf. Derudover vises den ikke for gang, da denne ydermere er lavere end for løb, og da begge aktiviteter skal detekteres med et accelerometer, skal de have samme samplingsfrekvens. Dermed vises kun frekvensspektrummet for løb, da systemets samplingsfrekvens bestemmes i forhold til den højest målte frekvens.

Data fra aktiviteten, cykling behandles for at bestemme signalernes frekvensindhold, med henblik på bestemmelsen af samplingsfrekvensen vedrørende gyroskopet.
\begin{figure}[H]
	\centering
	\includegraphics[width=1\textwidth]{figures/qBilag/cykling_frekvens}
	\caption{På figuren ses frekvensdomænet af aktiviteten cykling for alle forsøgspersoner. Den fuldstændige magnitude for de lave frekvenser vises ikke til fulde. Hvis dette skulle være tilfældet ville de mindste magnituder på figuren blive udskalleret.}
	\label{fig:Ap_cyklingfrekvens}
\end{figure}\vspace{-.25cm}

\subsection{Accelerometer karakteristika vedrørende gang og løb}
Dataene fra aktiviteten gang og løb behandles med henblik på bestemmelse af signalets karakteristika, således en sammenligning og senere algoritmedesign muliggøres. Dataene fra accelerometerets y-akse blev for alle forsøgspersoner lavpasfiltreret ved 45~Hz grundet frekvensspektret på \figref{fig:Ap_FFt}. Derudover blev signalet differentieret, hvorved områderne med størst hældningskoefficient kommer til udtryk. Herigennem fremhæves hælnedslag og tåafsæt, idet disse events har en stor hældning.
\begin{figure}[H]
	\centering
	\includegraphics[width=1\textwidth]{figures/qBilag/gang_diff}
	\caption{På figuren ses det filtrerede og differentierede data fra aktiviteten gang for alle forsøgspersoner.}
	\label{fig:Ap_gangdiff}
\end{figure}\vspace{-.25cm}
Det ses på \figref{fig:Ap_gangdiff}, at hælnedslag og tåafsæt fremgår tydligere end på \figref{raa_data} for både gang og løb. Ydermere ses det på \figref{fig:Ap_loebdiff}, at amplituderne for hælnedslag og tåafsæt øges ved løb.
\begin{figure}[H]
	\centering
	\includegraphics[width=1\textwidth]{figures/qBilag/loeb_diff}
	\caption{På figuren ses det filtrerede differentierede data fra aktiviteten løb for alle forsøgspersoner.}
	\label{fig:Ap_loebdiff}
\end{figure}\vspace{-.25cm}

\subsection{Gyroskop karakteristika vedrørende gang, løb og cykling}
Dataene fra aktiviteterne gang, løb og cykling behandles med henblik på bestemmelse af signalets karakteristika. Dette udføres ved at sammensætte forsøgspersonernes data, således en sammenligning muliggøres. Aktiviteterne gang og løb behandles for at sikre dette ikke har tilsvarende karakteristika som cykling, med henblik på algoritmedesign. Dataene behandles med henblik på gyroskopets z-akse, som resultat af \secref{bevaegelse}. Dette kan ses på \figref{fig:Ap_cykling1}, \figref{fig:Ap_cykling2} og \figref{fig:Ap_cykling3}.
\begin{figure}[H]
	\centering
	\includegraphics[width=1\textwidth]{figures/qBilag/gang_gyro}
	\caption{På figuren ses dataene fra gang ved 4,8 km/t fra de fire forsøgspersoner. Dataene er fra gyroskopets z-akse.}
	\label{fig:Ap_cykling1}
\end{figure}\vspace{-.25cm}

\begin{figure}[H]
	\centering
	\includegraphics[width=1\textwidth]{figures/qBilag/loeb_gyro}
	\caption{På figuren ses dataene fra løb ved 11,3 km/t fra de fire forsøgspersoner. Dataene er fra gyroskopets z-akse.}
	\label{fig:Ap_cykling2}
\end{figure}\vspace{-.25cm}

\begin{figure}[H]
	\centering
	\includegraphics[width=1\textwidth]{figures/qBilag/cykling_gyro}
	\caption{På figuren ses dataene fra cykling ved 20,9 km/t fra de fire forsøgspersoner. Dataene er fra gyroskopets z-akse.}
	\label{fig:Ap_cykling3}
\end{figure}\vspace{-.25cm}

\section{Diskussion}
\subsection{Kalibrering af Shimmer3}
Resultatet af databehandlingen bevirker, at kalibreringen af Shimmer3 antages at være tilstrækkelig. Dette antages at være tilstrækkeligt, da y-aksen afviger med henholdsvis -2,6\% i den negative akse og -0,6\% i den positive akse fra den teoretiske værdi. En eventuel fejlkilde til at denne fejlmargin kan være, at bordet hvorpå Shimmer3 er placeret, ikke er helt i vatter.

\subsection{Baseline for gang, løb og cykling}
Baselinemålingerne for henholdsvis gang og løb resulterer i en sammenligning af påvirkningen. g-påvirkningen af Shimmer3 er ikke præcis 1~g under kalibrering, hvilket kan være et resultat af, at Shimmer3 ikke er placeret ortogonalt på y-aksen på benet. Idet Shimmer3 ikke er placeret ortogonalt på benet, kan der være opstået en mindre hældning, hvorfor y-aksen ikke påvirkes med præcist 1~g. Resultaterne fra disse målinger indikerer, at Shimmer3 har optaget data som stemmer overens med antagelsen om den tilnærmelsesvise påvirkning på 1~g. 

Resultaterne fra baselinemålingerne vedrørende cykling ligger som forventet omkring 0, hvilket er et resultat af, at Shimmer3 ikke er blevet påvirket i z-aksen i nogen væsentlig grad, da benet ikke bevæges. Resultaterne af disse målinger indikerer, at Shimmer3 har optaget data, som stemmer overens med antagelsen om den tilnærmelsesvise påvirkning på 0~dps\fxnote{maksimal afvigelse på -0,9979}. 

\subsection{Maksimal g-påvirkning under gang, løb og hastighed} \label{app:maxg}
Resultatet af databehandlingen vedrørende de tre aktiviteter med henblik på bestemmelsen af den maksimale g-påvirkning medfører, at aktiviteten med hastighedsstigning har den største påvirkning. Resultaterne fra placering A, B eller C fra F1, F2 og F3 overskrider ikke $\pm16$~g. Resultaterne fra F4 overskrider 16~g med 0,95~g. Dette vurderes dog til ikke at have en væsentligt betydning, hvoraf den mest fordelagtige placering vælges. Med baggrund i \secref{succeskrav} og \secref{funktionellekrav} skal placeringen ikke være til gene for barnet og skal nemt af- og påmonteres, hvoraf placering A er valgt, da denne vurderes som værende mest komfortabel blandt forsøgspersonerne. Dette medfører, at den videre resultatbehandling udelukkende tager udgangspunkt i placering A. 

\subsection{Maksimal omdrejninger per sekund under cykling}
Resultatet af databehandlingen vedrørende maksimal omdrejninger ved cykling resulterer i et spænd mellem 216,5~dps og 320,4~dps. Dette kan være et resultat af, at forsøgspersonerne ikke har holdt samme hastighed. En pludselig acceleration kan derfor muligvis give en ændring, som ikke er relateret til cykling ved 20,9~km/t. I takt med at der maksimalt bliver registreret 320,4~dps, er dette medbestemmende vedrørende valg af et endeligt gyroskop. Et gyroskop til det endelige system skal heraf have et arbejdsområde som er større end 320,4 dps, men det præcise arbejdsområde vides ikke, da en hastighedsstigning ikke blev foretaget for cykling.

\subsection{Frekvensindhold af løb og cykling}
Databehandlingen af frekvensindholdet fra gang og løb påviser, at det største frekvensspektrum er placeret mellem 0~Hz og 45~Hz. Dette medfører, at samplingsfrekvensen vedrørende data fra accelerometeret kan bestemmes. \\
Databehandling af frekvensindholdet fra cykling påviser, at det største frekvensspektrum ligger mellem 0~Hz og 6~Hz. Dette medfører, at samplingsfrekvensen vedrørende data fra gyroskopets kan bestemmes. 

\subsection{Accelerometer karakteristika vedrørende gang og løb}
Databehandlingen vedrørende accelerometerets karakteristika af gang og løb resulterer i en sammenligning af dataene. Dataene fra gang viser to events, hvor peaks fremstår. Disse har en relativ kort afstand til hinanden efterfulgt af en længere pause, hvilket flere figurer i \secref{bevaegelse} viser som henholdsvis hælnedslag og tåafsæt. Ligeledes er disse forskellige events i løb, som også antages værende hælnedslag og tåafsæt. Der forekommer dog yderligere et harmonisk peak, som er betydeligt større end de andre events. Yderligere behandling af aktiviteternes data med viser algoritmer kan være nødvendig, men databehandlingen påviser, at signalkarakteristika for gang og løb kan bestemmes og heraf adskilles.\fxnote{hvis dette skal med skal er overvejes om man altid kan sige 0,43 sekunder, eller om man skal lave det relativt i forhold til tid (60/40)}

\subsection{Gyroskop karakteristika vedrørende gang, løb og cykling}
Databehandlingen af gyroskopets karakteristika vedrørende gang, løb og cykling resulterer i en sammenligning heraf. Resultatet af dette tyder på, at data fra et gyroskops z-akse tilhørende cykling tilnærmelsesvis kan afspejles som en sinus-bølge, samt at gang og løb antageligvis ikke kan forveksles heraf. Dette muliggør algoritmedesign med henblik på detektering af cykling. Det kan antages, at resultater fra cykling ved forskellige hastigheder påvirker signalet i en grad, hvor frekvens og amplitude ændres.

\section{Konklusion}
I pilotforsøget er aktiviteterne gang, løb og cykling undersøgt i en biomekanisk sammenhæng. Ud fra kalibreringen vurderes Shimmer3 til at måle korrekt i de forskellige akser. Derudover viser alle baselines at blive påvirket med mindre end 1\% vigende fra det forventede, hvormed det vurderes, at alle data kan sammenlignes, da Shimmer3 tilnærmelsesvis er placeret ens ved alle målinger for alle forsøgspersoner. \newline
Signalerne for gang og løb adskilles ved, at de maksimale målte amplituder for løb tilnærmelsesvis er dobbelt så stor som for gang, men ellers ser signalerne ensformige ud. Cykling målt med et gyroskop adskilles markant fra gang og løb, da cykling ikke har store peaks men i stedet er formet som en sinuslignende kurve. \newline
Signalernes udformning i forhold til placering har ikke en indflydelse på amplituden for gang. For løb stiger den positive amplitude imidlertid, jo mere distalt sensoren placeres, mens den stiger i negativ amplitude jo mere proximalt sensoren placeres. Hastighedsstigningen påvirkes på samme måde af placeringen som løb, mens amplituden ved cykling stort set ikke påvirkes efter placeringen. \newline
Frekvensspektrummet for gang og løb vælges ud for de laveste og højeste målte frekvenser, hvormed et frekvensspektrum på 0-45~Hz bestemmes. Frekvensspektrummet for cykling ligger på 0-6~Hz.\newline
Ud fra pilotforsøget vælges placering A som den mest optimale, da data ikke overskrider 16~g i en grad, der vil ødelægge signalet, og denne placering er den mest optimale i forhold til komfort. Derudover vælges et accelerometer med minimum 16~g og et gyroskop med minimum 320~dps.