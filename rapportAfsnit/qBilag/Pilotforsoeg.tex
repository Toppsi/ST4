\chapter{Pilotforsøg}
\textit{Dette bilag beskriver pilotforsøget, som er nødvendigt i forhold til testmåling og design af aktivitetsmåleren. Der undersøges tre forskellige placeringer af sensoren i tre forskellige aktivitetsformer på fire forsøgspersoner.}

\section{Teori}

\section{Formål}
For at kunne modificere og tilpasse softwaren til CY8CKIT-043 PSoC 4 M-Series Prototyping Kit er det nødvendigt at vide, hvordan forskellige aktivitetsformer påvirker sensoren. Målingerne skal undersøges for at kunne lave en algoritme, som kan få sensoren til at skelne imellem de pågældende aktivitetsformer. Derudover skal det bestemmes, hvor sensoren skal placeres på kroppen for mest optimalt udbytte. Derfor er formålet med pilotforsøget at undersøge følgende:
\begin{itemize}
	\item Hvordan påvirkes sensoren af gang, løb og cykling? 
	\item Hvor mange g ændrer sensorens målinger sig alt efter placering på kroppen?
\end{itemize}

\section{Materiale}
\begin{itemize}
	\item Løbebånd med justerbar hastighed.
	\item Motionscykel.
	\item Shimmer3 sensor.
	\item Computer med Multi Shimmer Sync for Windows.
\end{itemize}

\section{Fremgangsmåde}

\subsection{Databehandling}

\section{Resultater}

\section{Diskussion}

\section{Konklusion}

Formål: 
Hvordan påvirker gang, løb og cykling sensorene? 
Kan man se forskel på dataene i forhold til hvilken bevægelse der udføres.
Lav et forsøg som starter med langsom gang og så bliver hurtigere og hurtigere løb - for at se om der sker en markant ændring når man skifter fra gang til løb (i forhold til eks. G)
Hvor stort et stød (G) er der alt efter hvor sensoren påsættes ved gang, løb og cykling?
Ankel, midt på skinnebenet, under knæet.


Fremgangsmåde
Forskellige hastigheder inden for det at gå og løbe 
^samme med cykling
Placering af sensorer
